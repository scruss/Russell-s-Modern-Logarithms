% logtables.tex - CC BY-NC-SA - scruss.com - 2012-11-04

\documentclass[twoside,a4paper]{article}
\usepackage{longtable}
\usepackage{color}
\usepackage{colortbl}
\usepackage{graphicx}
\pagestyle{myheadings}
\usepackage[condensed,math]{iwona}
\usepackage[T1]{fontenc}
\usepackage{hyperref}

\hypersetup{
  pdftitle={Russell's Modern Logarithms},
  pdfauthor={Stewart C. Russell},
  pdfsubject={Logarithm and antilogarithm tables},
  pdfkeywords={logarithm} {antilogarithm} {multiplication} {division} {roots} {exponentiation}
}

\renewcommand{\tabcolsep}{3pt} % narrow table
\definecolor{bg}{RGB}{213,253,220} % alternate table row colour

\title{Modern Logarithms}
\author{Stewart C. Russell}

\begin{document}
\maketitle
\tableofcontents

\newpage
\thispagestyle{empty}
\section{Decimal Logarithms}
\newpage


% headings for log table
\makeatletter
\def\@oddhead{\thepage\qquad\textsc{Decimal Logarithms}\hfill \rightmark{} -- \leftmark}
\def\@evenhead{\rightmark{} -- \leftmark \hfill \textsc{Decimal Logarithms}\qquad\thepage}
\makeatother

\begin{longtable}[c]{|*{11}{c|}}
 & \textbf{0} & \textbf{1} & \textbf{2} & \textbf{3} & \textbf{4} & \textbf{5} & \textbf{6} & \textbf{7} & \textbf{8} & \textbf{9} \\
\hline \endhead
\hline \endfoot
%%%
 \textcolor{blue}{\textbf{1.00}}\markboth{1.00}{1.00} & .00000 & .00043 & .00087 & .00130 & .00173 & .00217 & .00260 & .00303 & .00346 & .00389 \\
\rowcolor{bg} \textbf{1.01}\markboth{1.01}{1.01} & .00432 & .00475 & .00518 & .00561 & .00604 & .00647 & .00689 & .00732 & .00775 & .00817 \\
 \textbf{1.02}\markboth{1.02}{1.02} & .00860 & .00903 & .00945 & .00988 & .01030 & .01072 & .01115 & .01157 & .01199 & .01242 \\
\rowcolor{bg} \textbf{1.03}\markboth{1.03}{1.03} & .01284 & .01326 & .01368 & .01410 & .01452 & .01494 & .01536 & .01578 & .01620 & .01662 \\
 \textbf{1.04}\markboth{1.04}{1.04} & .01703 & .01745 & .01787 & .01828 & .01870 & .01912 & .01953 & .01995 & .02036 & .02078 \\
\rowcolor{bg} \textbf{1.05}\markboth{1.05}{1.05} & .02119 & .02160 & .02202 & .02243 & .02284 & .02325 & .02366 & .02407 & .02449 & .02490 \\
 \textbf{1.06}\markboth{1.06}{1.06} & .02531 & .02572 & .02612 & .02653 & .02694 & .02735 & .02776 & .02816 & .02857 & .02898 \\
\rowcolor{bg} \textbf{1.07}\markboth{1.07}{1.07} & .02938 & .02979 & .03019 & .03060 & .03100 & .03141 & .03181 & .03222 & .03262 & .03302 \\
 \textbf{1.08}\markboth{1.08}{1.08} & .03342 & .03383 & .03423 & .03463 & .03503 & .03543 & .03583 & .03623 & .03663 & .03703 \\
\rowcolor{bg} \textbf{1.09}\markboth{1.09}{1.09} & .03743 & .03782 & .03822 & .03862 & .03902 & .03941 & .03981 & .04021 & .04060 & .04100 \\
 \textbf{1.10}\markboth{1.10}{1.10} & .04139 & .04179 & .04218 & .04258 & .04297 & .04336 & .04376 & .04415 & .04454 & .04493 \\
\rowcolor{bg} \textbf{1.11}\markboth{1.11}{1.11} & .04532 & .04571 & .04610 & .04650 & .04689 & .04727 & .04766 & .04805 & .04844 & .04883 \\
 \textbf{1.12}\markboth{1.12}{1.12} & .04922 & .04961 & .04999 & .05038 & .05077 & .05115 & .05154 & .05192 & .05231 & .05269 \\
\rowcolor{bg} \textbf{1.13}\markboth{1.13}{1.13} & .05308 & .05346 & .05385 & .05423 & .05461 & .05500 & .05538 & .05576 & .05614 & .05652 \\
 \textbf{1.14}\markboth{1.14}{1.14} & .05690 & .05729 & .05767 & .05805 & .05843 & .05881 & .05918 & .05956 & .05994 & .06032 \\
\rowcolor{bg} \textbf{1.15}\markboth{1.15}{1.15} & .06070 & .06108 & .06145 & .06183 & .06221 & .06258 & .06296 & .06333 & .06371 & .06408 \\
 \textbf{1.16}\markboth{1.16}{1.16} & .06446 & .06483 & .06521 & .06558 & .06595 & .06633 & .06670 & .06707 & .06744 & .06781 \\
\rowcolor{bg} \textbf{1.17}\markboth{1.17}{1.17} & .06819 & .06856 & .06893 & .06930 & .06967 & .07004 & .07041 & .07078 & .07115 & .07151 \\
 \textbf{1.18}\markboth{1.18}{1.18} & .07188 & .07225 & .07262 & .07298 & .07335 & .07372 & .07408 & .07445 & .07482 & .07518 \\
\rowcolor{bg} \textbf{1.19}\markboth{1.19}{1.19} & .07555 & .07591 & .07628 & .07664 & .07700 & .07737 & .07773 & .07809 & .07846 & .07882 \\
 \textbf{1.20}\markboth{1.20}{1.20} & .07918 & .07954 & .07990 & .08027 & .08063 & .08099 & .08135 & .08171 & .08207 & .08243 \\
\rowcolor{bg} \textbf{1.21}\markboth{1.21}{1.21} & .08279 & .08314 & .08350 & .08386 & .08422 & .08458 & .08493 & .08529 & .08565 & .08600 \\
 \textbf{1.22}\markboth{1.22}{1.22} & .08636 & .08672 & .08707 & .08743 & .08778 & .08814 & .08849 & .08884 & .08920 & .08955 \\
\rowcolor{bg} \textbf{1.23}\markboth{1.23}{1.23} & .08991 & .09026 & .09061 & .09096 & .09132 & .09167 & .09202 & .09237 & .09272 & .09307 \\
 \textbf{1.24}\markboth{1.24}{1.24} & .09342 & .09377 & .09412 & .09447 & .09482 & .09517 & .09552 & .09587 & .09621 & .09656 \\
\rowcolor{bg} \textbf{1.25}\markboth{1.25}{1.25} & .09691 & .09726 & .09760 & .09795 & .09830 & .09864 & .09899 & .09934 & .09968 & .10003 \\
 \textbf{1.26}\markboth{1.26}{1.26} & .10037 & .10072 & .10106 & .10140 & .10175 & .10209 & .10243 & .10278 & .10312 & .10346 \\
\rowcolor{bg} \textbf{1.27}\markboth{1.27}{1.27} & .10380 & .10415 & .10449 & .10483 & .10517 & .10551 & .10585 & .10619 & .10653 & .10687 \\
 \textbf{1.28}\markboth{1.28}{1.28} & .10721 & .10755 & .10789 & .10823 & .10857 & .10890 & .10924 & .10958 & .10992 & .11025 \\
\rowcolor{bg} \textbf{1.29}\markboth{1.29}{1.29} & .11059 & .11093 & .11126 & .11160 & .11193 & .11227 & .11261 & .11294 & .11327 & .11361 \\
 \textbf{1.30}\markboth{1.30}{1.30} & .11394 & .11428 & .11461 & .11494 & .11528 & .11561 & .11594 & .11628 & .11661 & .11694 \\
\rowcolor{bg} \textbf{1.31}\markboth{1.31}{1.31} & .11727 & .11760 & .11793 & .11826 & .11860 & .11893 & .11926 & .11959 & .11992 & .12024 \\
 \textbf{1.32}\markboth{1.32}{1.32} & .12057 & .12090 & .12123 & .12156 & .12189 & .12222 & .12254 & .12287 & .12320 & .12352 \\
\rowcolor{bg} \textbf{1.33}\markboth{1.33}{1.33} & .12385 & .12418 & .12450 & .12483 & .12516 & .12548 & .12581 & .12613 & .12646 & .12678 \\
 \textbf{1.34}\markboth{1.34}{1.34} & .12710 & .12743 & .12775 & .12808 & .12840 & .12872 & .12905 & .12937 & .12969 & .13001 \\
\rowcolor{bg} \textbf{1.35}\markboth{1.35}{1.35} & .13033 & .13066 & .13098 & .13130 & .13162 & .13194 & .13226 & .13258 & .13290 & .13322 \\
 \textbf{1.36}\markboth{1.36}{1.36} & .13354 & .13386 & .13418 & .13450 & .13481 & .13513 & .13545 & .13577 & .13609 & .13640 \\
\rowcolor{bg} \textbf{1.37}\markboth{1.37}{1.37} & .13672 & .13704 & .13735 & .13767 & .13799 & .13830 & .13862 & .13893 & .13925 & .13956 \\
 \textbf{1.38}\markboth{1.38}{1.38} & .13988 & .14019 & .14051 & .14082 & .14114 & .14145 & .14176 & .14208 & .14239 & .14270 \\
\rowcolor{bg} \textbf{1.39}\markboth{1.39}{1.39} & .14301 & .14333 & .14364 & .14395 & .14426 & .14457 & .14489 & .14520 & .14551 & .14582 \\
 \textbf{1.40}\markboth{1.40}{1.40} & .14613 & .14644 & .14675 & .14706 & .14737 & .14768 & .14799 & .14829 & .14860 & .14891 \\
\rowcolor{bg} \textbf{1.41}\markboth{1.41}{1.41} & .14922 & .14953 & .14983 & .15014 & .15045 & .15076 & .15106 & .15137 & .15168 & .15198 \\
 \textbf{1.42}\markboth{1.42}{1.42} & .15229 & .15259 & .15290 & .15320 & .15351 & .15381 & .15412 & .15442 & .15473 & .15503 \\
\rowcolor{bg} \textbf{1.43}\markboth{1.43}{1.43} & .15534 & .15564 & .15594 & .15625 & .15655 & .15685 & .15715 & .15746 & .15776 & .15806 \\
 \textbf{1.44}\markboth{1.44}{1.44} & .15836 & .15866 & .15897 & .15927 & .15957 & .15987 & .16017 & .16047 & .16077 & .16107 \\
\rowcolor{bg} \textbf{1.45}\markboth{1.45}{1.45} & .16137 & .16167 & .16197 & .16227 & .16256 & .16286 & .16316 & .16346 & .16376 & .16406 \\
 \textbf{1.46}\markboth{1.46}{1.46} & .16435 & .16465 & .16495 & .16524 & .16554 & .16584 & .16613 & .16643 & .16673 & .16702 \\
\rowcolor{bg} \textbf{1.47}\markboth{1.47}{1.47} & .16732 & .16761 & .16791 & .16820 & .16850 & .16879 & .16909 & .16938 & .16967 & .16997 \\
 \textbf{1.48}\markboth{1.48}{1.48} & .17026 & .17056 & .17085 & .17114 & .17143 & .17173 & .17202 & .17231 & .17260 & .17289 \\
\rowcolor{bg} \textbf{1.49}\markboth{1.49}{1.49} & .17319 & .17348 & .17377 & .17406 & .17435 & .17464 & .17493 & .17522 & .17551 & .17580 \\
 \textbf{1.50}\markboth{1.50}{1.50} & .17609 & .17638 & .17667 & .17696 & .17725 & .17754 & .17782 & .17811 & .17840 & .17869 \\
\rowcolor{bg} \textbf{1.51}\markboth{1.51}{1.51} & .17898 & .17926 & .17955 & .17984 & .18013 & .18041 & .18070 & .18099 & .18127 & .18156 \\
 \textbf{1.52}\markboth{1.52}{1.52} & .18184 & .18213 & .18241 & .18270 & .18298 & .18327 & .18355 & .18384 & .18412 & .18441 \\
\rowcolor{bg} \textbf{1.53}\markboth{1.53}{1.53} & .18469 & .18498 & .18526 & .18554 & .18583 & .18611 & .18639 & .18667 & .18696 & .18724 \\
 \textbf{1.54}\markboth{1.54}{1.54} & .18752 & .18780 & .18808 & .18837 & .18865 & .18893 & .18921 & .18949 & .18977 & .19005 \\
\rowcolor{bg} \textbf{1.55}\markboth{1.55}{1.55} & .19033 & .19061 & .19089 & .19117 & .19145 & .19173 & .19201 & .19229 & .19257 & .19285 \\
 \textbf{1.56}\markboth{1.56}{1.56} & .19312 & .19340 & .19368 & .19396 & .19424 & .19451 & .19479 & .19507 & .19535 & .19562 \\
\rowcolor{bg} \textbf{1.57}\markboth{1.57}{1.57} & .19590 & .19618 & .19645 & .19673 & .19700 & .19728 & .19756 & .19783 & .19811 & .19838 \\
 \textbf{1.58}\markboth{1.58}{1.58} & .19866 & .19893 & .19921 & .19948 & .19976 & .20003 & .20030 & .20058 & .20085 & .20112 \\
\rowcolor{bg} \textbf{1.59}\markboth{1.59}{1.59} & .20140 & .20167 & .20194 & .20222 & .20249 & .20276 & .20303 & .20330 & .20358 & .20385 \\
 \textbf{1.60}\markboth{1.60}{1.60} & .20412 & .20439 & .20466 & .20493 & .20520 & .20548 & .20575 & .20602 & .20629 & .20656 \\
\rowcolor{bg} \textbf{1.61}\markboth{1.61}{1.61} & .20683 & .20710 & .20737 & .20763 & .20790 & .20817 & .20844 & .20871 & .20898 & .20925 \\
 \textbf{1.62}\markboth{1.62}{1.62} & .20952 & .20978 & .21005 & .21032 & .21059 & .21085 & .21112 & .21139 & .21165 & .21192 \\
\rowcolor{bg} \textbf{1.63}\markboth{1.63}{1.63} & .21219 & .21245 & .21272 & .21299 & .21325 & .21352 & .21378 & .21405 & .21431 & .21458 \\
 \textbf{1.64}\markboth{1.64}{1.64} & .21484 & .21511 & .21537 & .21564 & .21590 & .21617 & .21643 & .21669 & .21696 & .21722 \\
\rowcolor{bg} \textbf{1.65}\markboth{1.65}{1.65} & .21748 & .21775 & .21801 & .21827 & .21854 & .21880 & .21906 & .21932 & .21958 & .21985 \\
 \textbf{1.66}\markboth{1.66}{1.66} & .22011 & .22037 & .22063 & .22089 & .22115 & .22141 & .22167 & .22194 & .22220 & .22246 \\
\rowcolor{bg} \textbf{1.67}\markboth{1.67}{1.67} & .22272 & .22298 & .22324 & .22350 & .22376 & .22401 & .22427 & .22453 & .22479 & .22505 \\
 \textbf{1.68}\markboth{1.68}{1.68} & .22531 & .22557 & .22583 & .22608 & .22634 & .22660 & .22686 & .22712 & .22737 & .22763 \\
\rowcolor{bg} \textbf{1.69}\markboth{1.69}{1.69} & .22789 & .22814 & .22840 & .22866 & .22891 & .22917 & .22943 & .22968 & .22994 & .23019 \\
 \textbf{1.70}\markboth{1.70}{1.70} & .23045 & .23070 & .23096 & .23121 & .23147 & .23172 & .23198 & .23223 & .23249 & .23274 \\
\rowcolor{bg} \textbf{1.71}\markboth{1.71}{1.71} & .23300 & .23325 & .23350 & .23376 & .23401 & .23426 & .23452 & .23477 & .23502 & .23528 \\
 \textbf{1.72}\markboth{1.72}{1.72} & .23553 & .23578 & .23603 & .23629 & .23654 & .23679 & .23704 & .23729 & .23754 & .23779 \\
\rowcolor{bg} \textbf{1.73}\markboth{1.73}{1.73} & .23805 & .23830 & .23855 & .23880 & .23905 & .23930 & .23955 & .23980 & .24005 & .24030 \\
 \textbf{1.74}\markboth{1.74}{1.74} & .24055 & .24080 & .24105 & .24130 & .24155 & .24180 & .24204 & .24229 & .24254 & .24279 \\
\rowcolor{bg} \textbf{1.75}\markboth{1.75}{1.75} & .24304 & .24329 & .24353 & .24378 & .24403 & .24428 & .24452 & .24477 & .24502 & .24527 \\
 \textbf{1.76}\markboth{1.76}{1.76} & .24551 & .24576 & .24601 & .24625 & .24650 & .24674 & .24699 & .24724 & .24748 & .24773 \\
\rowcolor{bg} \textbf{1.77}\markboth{1.77}{1.77} & .24797 & .24822 & .24846 & .24871 & .24895 & .24920 & .24944 & .24969 & .24993 & .25018 \\
 \textbf{1.78}\markboth{1.78}{1.78} & .25042 & .25066 & .25091 & .25115 & .25139 & .25164 & .25188 & .25212 & .25237 & .25261 \\
\rowcolor{bg} \textbf{1.79}\markboth{1.79}{1.79} & .25285 & .25310 & .25334 & .25358 & .25382 & .25406 & .25431 & .25455 & .25479 & .25503 \\
 \textbf{1.80}\markboth{1.80}{1.80} & .25527 & .25551 & .25575 & .25600 & .25624 & .25648 & .25672 & .25696 & .25720 & .25744 \\
\rowcolor{bg} \textbf{1.81}\markboth{1.81}{1.81} & .25768 & .25792 & .25816 & .25840 & .25864 & .25888 & .25912 & .25935 & .25959 & .25983 \\
 \textbf{1.82}\markboth{1.82}{1.82} & .26007 & .26031 & .26055 & .26079 & .26102 & .26126 & .26150 & .26174 & .26198 & .26221 \\
\rowcolor{bg} \textbf{1.83}\markboth{1.83}{1.83} & .26245 & .26269 & .26293 & .26316 & .26340 & .26364 & .26387 & .26411 & .26435 & .26458 \\
 \textbf{1.84}\markboth{1.84}{1.84} & .26482 & .26505 & .26529 & .26553 & .26576 & .26600 & .26623 & .26647 & .26670 & .26694 \\
\rowcolor{bg} \textbf{1.85}\markboth{1.85}{1.85} & .26717 & .26741 & .26764 & .26788 & .26811 & .26834 & .26858 & .26881 & .26905 & .26928 \\
 \textbf{1.86}\markboth{1.86}{1.86} & .26951 & .26975 & .26998 & .27021 & .27045 & .27068 & .27091 & .27114 & .27138 & .27161 \\
\rowcolor{bg} \textbf{1.87}\markboth{1.87}{1.87} & .27184 & .27207 & .27231 & .27254 & .27277 & .27300 & .27323 & .27346 & .27370 & .27393 \\
 \textbf{1.88}\markboth{1.88}{1.88} & .27416 & .27439 & .27462 & .27485 & .27508 & .27531 & .27554 & .27577 & .27600 & .27623 \\
\rowcolor{bg} \textbf{1.89}\markboth{1.89}{1.89} & .27646 & .27669 & .27692 & .27715 & .27738 & .27761 & .27784 & .27807 & .27830 & .27852 \\
 \textbf{1.90}\markboth{1.90}{1.90} & .27875 & .27898 & .27921 & .27944 & .27967 & .27989 & .28012 & .28035 & .28058 & .28081 \\
\rowcolor{bg} \textbf{1.91}\markboth{1.91}{1.91} & .28103 & .28126 & .28149 & .28171 & .28194 & .28217 & .28240 & .28262 & .28285 & .28307 \\
 \textbf{1.92}\markboth{1.92}{1.92} & .28330 & .28353 & .28375 & .28398 & .28421 & .28443 & .28466 & .28488 & .28511 & .28533 \\
\rowcolor{bg} \textbf{1.93}\markboth{1.93}{1.93} & .28556 & .28578 & .28601 & .28623 & .28646 & .28668 & .28691 & .28713 & .28735 & .28758 \\
 \textbf{1.94}\markboth{1.94}{1.94} & .28780 & .28803 & .28825 & .28847 & .28870 & .28892 & .28914 & .28937 & .28959 & .28981 \\
\rowcolor{bg} \textbf{1.95}\markboth{1.95}{1.95} & .29003 & .29026 & .29048 & .29070 & .29092 & .29115 & .29137 & .29159 & .29181 & .29203 \\
 \textbf{1.96}\markboth{1.96}{1.96} & .29226 & .29248 & .29270 & .29292 & .29314 & .29336 & .29358 & .29380 & .29403 & .29425 \\
\rowcolor{bg} \textbf{1.97}\markboth{1.97}{1.97} & .29447 & .29469 & .29491 & .29513 & .29535 & .29557 & .29579 & .29601 & .29623 & .29645 \\
 \textbf{1.98}\markboth{1.98}{1.98} & .29667 & .29688 & .29710 & .29732 & .29754 & .29776 & .29798 & .29820 & .29842 & .29863 \\
\rowcolor{bg} \textbf{1.99}\markboth{1.99}{1.99} & .29885 & .29907 & .29929 & .29951 & .29973 & .29994 & .30016 & .30038 & .30060 & .30081 \\
 \textcolor{blue}{\textbf{2.00}}\markboth{2.00}{2.00} & .30103 & .30125 & .30146 & .30168 & .30190 & .30211 & .30233 & .30255 & .30276 & .30298 \\
\rowcolor{bg} \textbf{2.01}\markboth{2.01}{2.01} & .30320 & .30341 & .30363 & .30384 & .30406 & .30428 & .30449 & .30471 & .30492 & .30514 \\
 \textbf{2.02}\markboth{2.02}{2.02} & .30535 & .30557 & .30578 & .30600 & .30621 & .30643 & .30664 & .30685 & .30707 & .30728 \\
\rowcolor{bg} \textbf{2.03}\markboth{2.03}{2.03} & .30750 & .30771 & .30792 & .30814 & .30835 & .30856 & .30878 & .30899 & .30920 & .30942 \\
 \textbf{2.04}\markboth{2.04}{2.04} & .30963 & .30984 & .31006 & .31027 & .31048 & .31069 & .31091 & .31112 & .31133 & .31154 \\
\rowcolor{bg} \textbf{2.05}\markboth{2.05}{2.05} & .31175 & .31197 & .31218 & .31239 & .31260 & .31281 & .31302 & .31323 & .31345 & .31366 \\
 \textbf{2.06}\markboth{2.06}{2.06} & .31387 & .31408 & .31429 & .31450 & .31471 & .31492 & .31513 & .31534 & .31555 & .31576 \\
\rowcolor{bg} \textbf{2.07}\markboth{2.07}{2.07} & .31597 & .31618 & .31639 & .31660 & .31681 & .31702 & .31723 & .31744 & .31765 & .31785 \\
 \textbf{2.08}\markboth{2.08}{2.08} & .31806 & .31827 & .31848 & .31869 & .31890 & .31911 & .31931 & .31952 & .31973 & .31994 \\
\rowcolor{bg} \textbf{2.09}\markboth{2.09}{2.09} & .32015 & .32035 & .32056 & .32077 & .32098 & .32118 & .32139 & .32160 & .32181 & .32201 \\
 \textbf{2.10}\markboth{2.10}{2.10} & .32222 & .32243 & .32263 & .32284 & .32305 & .32325 & .32346 & .32366 & .32387 & .32408 \\
\rowcolor{bg} \textbf{2.11}\markboth{2.11}{2.11} & .32428 & .32449 & .32469 & .32490 & .32510 & .32531 & .32552 & .32572 & .32593 & .32613 \\
 \textbf{2.12}\markboth{2.12}{2.12} & .32634 & .32654 & .32675 & .32695 & .32715 & .32736 & .32756 & .32777 & .32797 & .32818 \\
\rowcolor{bg} \textbf{2.13}\markboth{2.13}{2.13} & .32838 & .32858 & .32879 & .32899 & .32919 & .32940 & .32960 & .32980 & .33001 & .33021 \\
 \textbf{2.14}\markboth{2.14}{2.14} & .33041 & .33062 & .33082 & .33102 & .33122 & .33143 & .33163 & .33183 & .33203 & .33224 \\
\rowcolor{bg} \textbf{2.15}\markboth{2.15}{2.15} & .33244 & .33264 & .33284 & .33304 & .33325 & .33345 & .33365 & .33385 & .33405 & .33425 \\
 \textbf{2.16}\markboth{2.16}{2.16} & .33445 & .33465 & .33486 & .33506 & .33526 & .33546 & .33566 & .33586 & .33606 & .33626 \\
\rowcolor{bg} \textbf{2.17}\markboth{2.17}{2.17} & .33646 & .33666 & .33686 & .33706 & .33726 & .33746 & .33766 & .33786 & .33806 & .33826 \\
 \textbf{2.18}\markboth{2.18}{2.18} & .33846 & .33866 & .33885 & .33905 & .33925 & .33945 & .33965 & .33985 & .34005 & .34025 \\
\rowcolor{bg} \textbf{2.19}\markboth{2.19}{2.19} & .34044 & .34064 & .34084 & .34104 & .34124 & .34143 & .34163 & .34183 & .34203 & .34223 \\
 \textbf{2.20}\markboth{2.20}{2.20} & .34242 & .34262 & .34282 & .34301 & .34321 & .34341 & .34361 & .34380 & .34400 & .34420 \\
\rowcolor{bg} \textbf{2.21}\markboth{2.21}{2.21} & .34439 & .34459 & .34479 & .34498 & .34518 & .34537 & .34557 & .34577 & .34596 & .34616 \\
 \textbf{2.22}\markboth{2.22}{2.22} & .34635 & .34655 & .34674 & .34694 & .34713 & .34733 & .34753 & .34772 & .34792 & .34811 \\
\rowcolor{bg} \textbf{2.23}\markboth{2.23}{2.23} & .34830 & .34850 & .34869 & .34889 & .34908 & .34928 & .34947 & .34967 & .34986 & .35005 \\
 \textbf{2.24}\markboth{2.24}{2.24} & .35025 & .35044 & .35064 & .35083 & .35102 & .35122 & .35141 & .35160 & .35180 & .35199 \\
\rowcolor{bg} \textbf{2.25}\markboth{2.25}{2.25} & .35218 & .35238 & .35257 & .35276 & .35295 & .35315 & .35334 & .35353 & .35372 & .35392 \\
 \textbf{2.26}\markboth{2.26}{2.26} & .35411 & .35430 & .35449 & .35468 & .35488 & .35507 & .35526 & .35545 & .35564 & .35583 \\
\rowcolor{bg} \textbf{2.27}\markboth{2.27}{2.27} & .35603 & .35622 & .35641 & .35660 & .35679 & .35698 & .35717 & .35736 & .35755 & .35774 \\
 \textbf{2.28}\markboth{2.28}{2.28} & .35793 & .35813 & .35832 & .35851 & .35870 & .35889 & .35908 & .35927 & .35946 & .35965 \\
\rowcolor{bg} \textbf{2.29}\markboth{2.29}{2.29} & .35984 & .36003 & .36021 & .36040 & .36059 & .36078 & .36097 & .36116 & .36135 & .36154 \\
 \textbf{2.30}\markboth{2.30}{2.30} & .36173 & .36192 & .36211 & .36229 & .36248 & .36267 & .36286 & .36305 & .36324 & .36342 \\
\rowcolor{bg} \textbf{2.31}\markboth{2.31}{2.31} & .36361 & .36380 & .36399 & .36418 & .36436 & .36455 & .36474 & .36493 & .36511 & .36530 \\
 \textbf{2.32}\markboth{2.32}{2.32} & .36549 & .36568 & .36586 & .36605 & .36624 & .36642 & .36661 & .36680 & .36698 & .36717 \\
\rowcolor{bg} \textbf{2.33}\markboth{2.33}{2.33} & .36736 & .36754 & .36773 & .36791 & .36810 & .36829 & .36847 & .36866 & .36884 & .36903 \\
 \textbf{2.34}\markboth{2.34}{2.34} & .36922 & .36940 & .36959 & .36977 & .36996 & .37014 & .37033 & .37051 & .37070 & .37088 \\
\rowcolor{bg} \textbf{2.35}\markboth{2.35}{2.35} & .37107 & .37125 & .37144 & .37162 & .37181 & .37199 & .37218 & .37236 & .37254 & .37273 \\
 \textbf{2.36}\markboth{2.36}{2.36} & .37291 & .37310 & .37328 & .37346 & .37365 & .37383 & .37401 & .37420 & .37438 & .37457 \\
\rowcolor{bg} \textbf{2.37}\markboth{2.37}{2.37} & .37475 & .37493 & .37511 & .37530 & .37548 & .37566 & .37585 & .37603 & .37621 & .37639 \\
 \textbf{2.38}\markboth{2.38}{2.38} & .37658 & .37676 & .37694 & .37712 & .37731 & .37749 & .37767 & .37785 & .37803 & .37822 \\
\rowcolor{bg} \textbf{2.39}\markboth{2.39}{2.39} & .37840 & .37858 & .37876 & .37894 & .37912 & .37931 & .37949 & .37967 & .37985 & .38003 \\
 \textbf{2.40}\markboth{2.40}{2.40} & .38021 & .38039 & .38057 & .38075 & .38093 & .38112 & .38130 & .38148 & .38166 & .38184 \\
\rowcolor{bg} \textbf{2.41}\markboth{2.41}{2.41} & .38202 & .38220 & .38238 & .38256 & .38274 & .38292 & .38310 & .38328 & .38346 & .38364 \\
 \textbf{2.42}\markboth{2.42}{2.42} & .38382 & .38399 & .38417 & .38435 & .38453 & .38471 & .38489 & .38507 & .38525 & .38543 \\
\rowcolor{bg} \textbf{2.43}\markboth{2.43}{2.43} & .38561 & .38578 & .38596 & .38614 & .38632 & .38650 & .38668 & .38686 & .38703 & .38721 \\
 \textbf{2.44}\markboth{2.44}{2.44} & .38739 & .38757 & .38775 & .38792 & .38810 & .38828 & .38846 & .38863 & .38881 & .38899 \\
\rowcolor{bg} \textbf{2.45}\markboth{2.45}{2.45} & .38917 & .38934 & .38952 & .38970 & .38987 & .39005 & .39023 & .39041 & .39058 & .39076 \\
 \textbf{2.46}\markboth{2.46}{2.46} & .39094 & .39111 & .39129 & .39146 & .39164 & .39182 & .39199 & .39217 & .39235 & .39252 \\
\rowcolor{bg} \textbf{2.47}\markboth{2.47}{2.47} & .39270 & .39287 & .39305 & .39322 & .39340 & .39358 & .39375 & .39393 & .39410 & .39428 \\
 \textbf{2.48}\markboth{2.48}{2.48} & .39445 & .39463 & .39480 & .39498 & .39515 & .39533 & .39550 & .39568 & .39585 & .39602 \\
\rowcolor{bg} \textbf{2.49}\markboth{2.49}{2.49} & .39620 & .39637 & .39655 & .39672 & .39690 & .39707 & .39724 & .39742 & .39759 & .39777 \\
 \textbf{2.50}\markboth{2.50}{2.50} & .39794 & .39811 & .39829 & .39846 & .39863 & .39881 & .39898 & .39915 & .39933 & .39950 \\
\rowcolor{bg} \textbf{2.51}\markboth{2.51}{2.51} & .39967 & .39985 & .40002 & .40019 & .40037 & .40054 & .40071 & .40088 & .40106 & .40123 \\
 \textbf{2.52}\markboth{2.52}{2.52} & .40140 & .40157 & .40175 & .40192 & .40209 & .40226 & .40243 & .40261 & .40278 & .40295 \\
\rowcolor{bg} \textbf{2.53}\markboth{2.53}{2.53} & .40312 & .40329 & .40346 & .40364 & .40381 & .40398 & .40415 & .40432 & .40449 & .40466 \\
 \textbf{2.54}\markboth{2.54}{2.54} & .40483 & .40500 & .40518 & .40535 & .40552 & .40569 & .40586 & .40603 & .40620 & .40637 \\
\rowcolor{bg} \textbf{2.55}\markboth{2.55}{2.55} & .40654 & .40671 & .40688 & .40705 & .40722 & .40739 & .40756 & .40773 & .40790 & .40807 \\
 \textbf{2.56}\markboth{2.56}{2.56} & .40824 & .40841 & .40858 & .40875 & .40892 & .40909 & .40926 & .40943 & .40960 & .40976 \\
\rowcolor{bg} \textbf{2.57}\markboth{2.57}{2.57} & .40993 & .41010 & .41027 & .41044 & .41061 & .41078 & .41095 & .41111 & .41128 & .41145 \\
 \textbf{2.58}\markboth{2.58}{2.58} & .41162 & .41179 & .41196 & .41212 & .41229 & .41246 & .41263 & .41280 & .41296 & .41313 \\
\rowcolor{bg} \textbf{2.59}\markboth{2.59}{2.59} & .41330 & .41347 & .41363 & .41380 & .41397 & .41414 & .41430 & .41447 & .41464 & .41481 \\
 \textbf{2.60}\markboth{2.60}{2.60} & .41497 & .41514 & .41531 & .41547 & .41564 & .41581 & .41597 & .41614 & .41631 & .41647 \\
\rowcolor{bg} \textbf{2.61}\markboth{2.61}{2.61} & .41664 & .41681 & .41697 & .41714 & .41731 & .41747 & .41764 & .41780 & .41797 & .41814 \\
 \textbf{2.62}\markboth{2.62}{2.62} & .41830 & .41847 & .41863 & .41880 & .41896 & .41913 & .41929 & .41946 & .41963 & .41979 \\
\rowcolor{bg} \textbf{2.63}\markboth{2.63}{2.63} & .41996 & .42012 & .42029 & .42045 & .42062 & .42078 & .42095 & .42111 & .42127 & .42144 \\
 \textbf{2.64}\markboth{2.64}{2.64} & .42160 & .42177 & .42193 & .42210 & .42226 & .42243 & .42259 & .42275 & .42292 & .42308 \\
\rowcolor{bg} \textbf{2.65}\markboth{2.65}{2.65} & .42325 & .42341 & .42357 & .42374 & .42390 & .42406 & .42423 & .42439 & .42455 & .42472 \\
 \textbf{2.66}\markboth{2.66}{2.66} & .42488 & .42504 & .42521 & .42537 & .42553 & .42570 & .42586 & .42602 & .42619 & .42635 \\
\rowcolor{bg} \textbf{2.67}\markboth{2.67}{2.67} & .42651 & .42667 & .42684 & .42700 & .42716 & .42732 & .42749 & .42765 & .42781 & .42797 \\
 \textbf{2.68}\markboth{2.68}{2.68} & .42813 & .42830 & .42846 & .42862 & .42878 & .42894 & .42911 & .42927 & .42943 & .42959 \\
\rowcolor{bg} \textbf{2.69}\markboth{2.69}{2.69} & .42975 & .42991 & .43008 & .43024 & .43040 & .43056 & .43072 & .43088 & .43104 & .43120 \\
 \textbf{2.70}\markboth{2.70}{2.70} & .43136 & .43152 & .43169 & .43185 & .43201 & .43217 & .43233 & .43249 & .43265 & .43281 \\
\rowcolor{bg} \textbf{2.71}\markboth{2.71}{2.71} & .43297 & .43313 & .43329 & .43345 & .43361 & .43377 & .43393 & .43409 & .43425 & .43441 \\
 \textbf{2.72}\markboth{2.72}{2.72} & .43457 & .43473 & .43489 & .43505 & .43521 & .43537 & .43553 & .43569 & .43584 & .43600 \\
\rowcolor{bg} \textbf{2.73}\markboth{2.73}{2.73} & .43616 & .43632 & .43648 & .43664 & .43680 & .43696 & .43712 & .43727 & .43743 & .43759 \\
 \textbf{2.74}\markboth{2.74}{2.74} & .43775 & .43791 & .43807 & .43823 & .43838 & .43854 & .43870 & .43886 & .43902 & .43917 \\
\rowcolor{bg} \textbf{2.75}\markboth{2.75}{2.75} & .43933 & .43949 & .43965 & .43981 & .43996 & .44012 & .44028 & .44044 & .44059 & .44075 \\
 \textbf{2.76}\markboth{2.76}{2.76} & .44091 & .44107 & .44122 & .44138 & .44154 & .44170 & .44185 & .44201 & .44217 & .44232 \\
\rowcolor{bg} \textbf{2.77}\markboth{2.77}{2.77} & .44248 & .44264 & .44279 & .44295 & .44311 & .44326 & .44342 & .44358 & .44373 & .44389 \\
 \textbf{2.78}\markboth{2.78}{2.78} & .44404 & .44420 & .44436 & .44451 & .44467 & .44483 & .44498 & .44514 & .44529 & .44545 \\
\rowcolor{bg} \textbf{2.79}\markboth{2.79}{2.79} & .44560 & .44576 & .44592 & .44607 & .44623 & .44638 & .44654 & .44669 & .44685 & .44700 \\
 \textbf{2.80}\markboth{2.80}{2.80} & .44716 & .44731 & .44747 & .44762 & .44778 & .44793 & .44809 & .44824 & .44840 & .44855 \\
\rowcolor{bg} \textbf{2.81}\markboth{2.81}{2.81} & .44871 & .44886 & .44902 & .44917 & .44932 & .44948 & .44963 & .44979 & .44994 & .45010 \\
 \textbf{2.82}\markboth{2.82}{2.82} & .45025 & .45040 & .45056 & .45071 & .45086 & .45102 & .45117 & .45133 & .45148 & .45163 \\
\rowcolor{bg} \textbf{2.83}\markboth{2.83}{2.83} & .45179 & .45194 & .45209 & .45225 & .45240 & .45255 & .45271 & .45286 & .45301 & .45317 \\
 \textbf{2.84}\markboth{2.84}{2.84} & .45332 & .45347 & .45362 & .45378 & .45393 & .45408 & .45423 & .45439 & .45454 & .45469 \\
\rowcolor{bg} \textbf{2.85}\markboth{2.85}{2.85} & .45484 & .45500 & .45515 & .45530 & .45545 & .45561 & .45576 & .45591 & .45606 & .45621 \\
 \textbf{2.86}\markboth{2.86}{2.86} & .45637 & .45652 & .45667 & .45682 & .45697 & .45712 & .45728 & .45743 & .45758 & .45773 \\
\rowcolor{bg} \textbf{2.87}\markboth{2.87}{2.87} & .45788 & .45803 & .45818 & .45834 & .45849 & .45864 & .45879 & .45894 & .45909 & .45924 \\
 \textbf{2.88}\markboth{2.88}{2.88} & .45939 & .45954 & .45969 & .45984 & .46000 & .46015 & .46030 & .46045 & .46060 & .46075 \\
\rowcolor{bg} \textbf{2.89}\markboth{2.89}{2.89} & .46090 & .46105 & .46120 & .46135 & .46150 & .46165 & .46180 & .46195 & .46210 & .46225 \\
 \textbf{2.90}\markboth{2.90}{2.90} & .46240 & .46255 & .46270 & .46285 & .46300 & .46315 & .46330 & .46345 & .46359 & .46374 \\
\rowcolor{bg} \textbf{2.91}\markboth{2.91}{2.91} & .46389 & .46404 & .46419 & .46434 & .46449 & .46464 & .46479 & .46494 & .46509 & .46523 \\
 \textbf{2.92}\markboth{2.92}{2.92} & .46538 & .46553 & .46568 & .46583 & .46598 & .46613 & .46627 & .46642 & .46657 & .46672 \\
\rowcolor{bg} \textbf{2.93}\markboth{2.93}{2.93} & .46687 & .46702 & .46716 & .46731 & .46746 & .46761 & .46776 & .46790 & .46805 & .46820 \\
 \textbf{2.94}\markboth{2.94}{2.94} & .46835 & .46850 & .46864 & .46879 & .46894 & .46909 & .46923 & .46938 & .46953 & .46967 \\
\rowcolor{bg} \textbf{2.95}\markboth{2.95}{2.95} & .46982 & .46997 & .47012 & .47026 & .47041 & .47056 & .47070 & .47085 & .47100 & .47114 \\
 \textbf{2.96}\markboth{2.96}{2.96} & .47129 & .47144 & .47159 & .47173 & .47188 & .47202 & .47217 & .47232 & .47246 & .47261 \\
\rowcolor{bg} \textbf{2.97}\markboth{2.97}{2.97} & .47276 & .47290 & .47305 & .47319 & .47334 & .47349 & .47363 & .47378 & .47392 & .47407 \\
 \textbf{2.98}\markboth{2.98}{2.98} & .47422 & .47436 & .47451 & .47465 & .47480 & .47494 & .47509 & .47524 & .47538 & .47553 \\
\rowcolor{bg} \textbf{2.99}\markboth{2.99}{2.99} & .47567 & .47582 & .47596 & .47611 & .47625 & .47640 & .47654 & .47669 & .47683 & .47698 \\
 \textcolor{blue}{\textbf{3.00}}\markboth{3.00}{3.00} & .47712 & .47727 & .47741 & .47756 & .47770 & .47784 & .47799 & .47813 & .47828 & .47842 \\
\rowcolor{bg} \textbf{3.01}\markboth{3.01}{3.01} & .47857 & .47871 & .47885 & .47900 & .47914 & .47929 & .47943 & .47958 & .47972 & .47986 \\
 \textbf{3.02}\markboth{3.02}{3.02} & .48001 & .48015 & .48029 & .48044 & .48058 & .48073 & .48087 & .48101 & .48116 & .48130 \\
\rowcolor{bg} \textbf{3.03}\markboth{3.03}{3.03} & .48144 & .48159 & .48173 & .48187 & .48202 & .48216 & .48230 & .48244 & .48259 & .48273 \\
 \textbf{3.04}\markboth{3.04}{3.04} & .48287 & .48302 & .48316 & .48330 & .48344 & .48359 & .48373 & .48387 & .48401 & .48416 \\
\rowcolor{bg} \textbf{3.05}\markboth{3.05}{3.05} & .48430 & .48444 & .48458 & .48473 & .48487 & .48501 & .48515 & .48530 & .48544 & .48558 \\
 \textbf{3.06}\markboth{3.06}{3.06} & .48572 & .48586 & .48601 & .48615 & .48629 & .48643 & .48657 & .48671 & .48686 & .48700 \\
\rowcolor{bg} \textbf{3.07}\markboth{3.07}{3.07} & .48714 & .48728 & .48742 & .48756 & .48770 & .48785 & .48799 & .48813 & .48827 & .48841 \\
 \textbf{3.08}\markboth{3.08}{3.08} & .48855 & .48869 & .48883 & .48897 & .48911 & .48926 & .48940 & .48954 & .48968 & .48982 \\
\rowcolor{bg} \textbf{3.09}\markboth{3.09}{3.09} & .48996 & .49010 & .49024 & .49038 & .49052 & .49066 & .49080 & .49094 & .49108 & .49122 \\
 \textbf{3.10}\markboth{3.10}{3.10} & .49136 & .49150 & .49164 & .49178 & .49192 & .49206 & .49220 & .49234 & .49248 & .49262 \\
\rowcolor{bg} \textbf{3.11}\markboth{3.11}{3.11} & .49276 & .49290 & .49304 & .49318 & .49332 & .49346 & .49360 & .49374 & .49388 & .49402 \\
 \textbf{3.12}\markboth{3.12}{3.12} & .49415 & .49429 & .49443 & .49457 & .49471 & .49485 & .49499 & .49513 & .49527 & .49541 \\
\rowcolor{bg} \textbf{3.13}\markboth{3.13}{3.13} & .49554 & .49568 & .49582 & .49596 & .49610 & .49624 & .49638 & .49651 & .49665 & .49679 \\
 \textbf{3.14}\markboth{3.14}{3.14} & .49693 & .49707 & .49721 & .49734 & .49748 & .49762 & .49776 & .49790 & .49803 & .49817 \\
\rowcolor{bg} \textbf{3.15}\markboth{3.15}{3.15} & .49831 & .49845 & .49859 & .49872 & .49886 & .49900 & .49914 & .49927 & .49941 & .49955 \\
 \textbf{3.16}\markboth{3.16}{3.16} & .49969 & .49982 & .49996 & .50010 & .50024 & .50037 & .50051 & .50065 & .50079 & .50092 \\
\rowcolor{bg} \textbf{3.17}\markboth{3.17}{3.17} & .50106 & .50120 & .50133 & .50147 & .50161 & .50174 & .50188 & .50202 & .50215 & .50229 \\
 \textbf{3.18}\markboth{3.18}{3.18} & .50243 & .50256 & .50270 & .50284 & .50297 & .50311 & .50325 & .50338 & .50352 & .50365 \\
\rowcolor{bg} \textbf{3.19}\markboth{3.19}{3.19} & .50379 & .50393 & .50406 & .50420 & .50433 & .50447 & .50461 & .50474 & .50488 & .50501 \\
 \textbf{3.20}\markboth{3.20}{3.20} & .50515 & .50529 & .50542 & .50556 & .50569 & .50583 & .50596 & .50610 & .50623 & .50637 \\
\rowcolor{bg} \textbf{3.21}\markboth{3.21}{3.21} & .50651 & .50664 & .50678 & .50691 & .50705 & .50718 & .50732 & .50745 & .50759 & .50772 \\
 \textbf{3.22}\markboth{3.22}{3.22} & .50786 & .50799 & .50813 & .50826 & .50840 & .50853 & .50866 & .50880 & .50893 & .50907 \\
\rowcolor{bg} \textbf{3.23}\markboth{3.23}{3.23} & .50920 & .50934 & .50947 & .50961 & .50974 & .50987 & .51001 & .51014 & .51028 & .51041 \\
 \textbf{3.24}\markboth{3.24}{3.24} & .51055 & .51068 & .51081 & .51095 & .51108 & .51121 & .51135 & .51148 & .51162 & .51175 \\
\rowcolor{bg} \textbf{3.25}\markboth{3.25}{3.25} & .51188 & .51202 & .51215 & .51228 & .51242 & .51255 & .51268 & .51282 & .51295 & .51308 \\
 \textbf{3.26}\markboth{3.26}{3.26} & .51322 & .51335 & .51348 & .51362 & .51375 & .51388 & .51402 & .51415 & .51428 & .51441 \\
\rowcolor{bg} \textbf{3.27}\markboth{3.27}{3.27} & .51455 & .51468 & .51481 & .51495 & .51508 & .51521 & .51534 & .51548 & .51561 & .51574 \\
 \textbf{3.28}\markboth{3.28}{3.28} & .51587 & .51601 & .51614 & .51627 & .51640 & .51654 & .51667 & .51680 & .51693 & .51706 \\
\rowcolor{bg} \textbf{3.29}\markboth{3.29}{3.29} & .51720 & .51733 & .51746 & .51759 & .51772 & .51786 & .51799 & .51812 & .51825 & .51838 \\
 \textbf{3.30}\markboth{3.30}{3.30} & .51851 & .51865 & .51878 & .51891 & .51904 & .51917 & .51930 & .51943 & .51957 & .51970 \\
\rowcolor{bg} \textbf{3.31}\markboth{3.31}{3.31} & .51983 & .51996 & .52009 & .52022 & .52035 & .52048 & .52061 & .52075 & .52088 & .52101 \\
 \textbf{3.32}\markboth{3.32}{3.32} & .52114 & .52127 & .52140 & .52153 & .52166 & .52179 & .52192 & .52205 & .52218 & .52231 \\
\rowcolor{bg} \textbf{3.33}\markboth{3.33}{3.33} & .52244 & .52257 & .52270 & .52284 & .52297 & .52310 & .52323 & .52336 & .52349 & .52362 \\
 \textbf{3.34}\markboth{3.34}{3.34} & .52375 & .52388 & .52401 & .52414 & .52427 & .52440 & .52453 & .52466 & .52479 & .52492 \\
\rowcolor{bg} \textbf{3.35}\markboth{3.35}{3.35} & .52504 & .52517 & .52530 & .52543 & .52556 & .52569 & .52582 & .52595 & .52608 & .52621 \\
 \textbf{3.36}\markboth{3.36}{3.36} & .52634 & .52647 & .52660 & .52673 & .52686 & .52699 & .52711 & .52724 & .52737 & .52750 \\
\rowcolor{bg} \textbf{3.37}\markboth{3.37}{3.37} & .52763 & .52776 & .52789 & .52802 & .52815 & .52827 & .52840 & .52853 & .52866 & .52879 \\
 \textbf{3.38}\markboth{3.38}{3.38} & .52892 & .52905 & .52917 & .52930 & .52943 & .52956 & .52969 & .52982 & .52994 & .53007 \\
\rowcolor{bg} \textbf{3.39}\markboth{3.39}{3.39} & .53020 & .53033 & .53046 & .53058 & .53071 & .53084 & .53097 & .53110 & .53122 & .53135 \\
 \textbf{3.40}\markboth{3.40}{3.40} & .53148 & .53161 & .53173 & .53186 & .53199 & .53212 & .53224 & .53237 & .53250 & .53263 \\
\rowcolor{bg} \textbf{3.41}\markboth{3.41}{3.41} & .53275 & .53288 & .53301 & .53314 & .53326 & .53339 & .53352 & .53364 & .53377 & .53390 \\
 \textbf{3.42}\markboth{3.42}{3.42} & .53403 & .53415 & .53428 & .53441 & .53453 & .53466 & .53479 & .53491 & .53504 & .53517 \\
\rowcolor{bg} \textbf{3.43}\markboth{3.43}{3.43} & .53529 & .53542 & .53555 & .53567 & .53580 & .53593 & .53605 & .53618 & .53631 & .53643 \\
 \textbf{3.44}\markboth{3.44}{3.44} & .53656 & .53668 & .53681 & .53694 & .53706 & .53719 & .53732 & .53744 & .53757 & .53769 \\
\rowcolor{bg} \textbf{3.45}\markboth{3.45}{3.45} & .53782 & .53794 & .53807 & .53820 & .53832 & .53845 & .53857 & .53870 & .53882 & .53895 \\
 \textbf{3.46}\markboth{3.46}{3.46} & .53908 & .53920 & .53933 & .53945 & .53958 & .53970 & .53983 & .53995 & .54008 & .54020 \\
\rowcolor{bg} \textbf{3.47}\markboth{3.47}{3.47} & .54033 & .54045 & .54058 & .54070 & .54083 & .54095 & .54108 & .54120 & .54133 & .54145 \\
 \textbf{3.48}\markboth{3.48}{3.48} & .54158 & .54170 & .54183 & .54195 & .54208 & .54220 & .54233 & .54245 & .54258 & .54270 \\
\rowcolor{bg} \textbf{3.49}\markboth{3.49}{3.49} & .54283 & .54295 & .54307 & .54320 & .54332 & .54345 & .54357 & .54370 & .54382 & .54394 \\
 \textbf{3.50}\markboth{3.50}{3.50} & .54407 & .54419 & .54432 & .54444 & .54456 & .54469 & .54481 & .54494 & .54506 & .54518 \\
\rowcolor{bg} \textbf{3.51}\markboth{3.51}{3.51} & .54531 & .54543 & .54555 & .54568 & .54580 & .54593 & .54605 & .54617 & .54630 & .54642 \\
 \textbf{3.52}\markboth{3.52}{3.52} & .54654 & .54667 & .54679 & .54691 & .54704 & .54716 & .54728 & .54741 & .54753 & .54765 \\
\rowcolor{bg} \textbf{3.53}\markboth{3.53}{3.53} & .54777 & .54790 & .54802 & .54814 & .54827 & .54839 & .54851 & .54864 & .54876 & .54888 \\
 \textbf{3.54}\markboth{3.54}{3.54} & .54900 & .54913 & .54925 & .54937 & .54949 & .54962 & .54974 & .54986 & .54998 & .55011 \\
\rowcolor{bg} \textbf{3.55}\markboth{3.55}{3.55} & .55023 & .55035 & .55047 & .55060 & .55072 & .55084 & .55096 & .55108 & .55121 & .55133 \\
 \textbf{3.56}\markboth{3.56}{3.56} & .55145 & .55157 & .55169 & .55182 & .55194 & .55206 & .55218 & .55230 & .55242 & .55255 \\
\rowcolor{bg} \textbf{3.57}\markboth{3.57}{3.57} & .55267 & .55279 & .55291 & .55303 & .55315 & .55328 & .55340 & .55352 & .55364 & .55376 \\
 \textbf{3.58}\markboth{3.58}{3.58} & .55388 & .55400 & .55413 & .55425 & .55437 & .55449 & .55461 & .55473 & .55485 & .55497 \\
\rowcolor{bg} \textbf{3.59}\markboth{3.59}{3.59} & .55509 & .55522 & .55534 & .55546 & .55558 & .55570 & .55582 & .55594 & .55606 & .55618 \\
 \textbf{3.60}\markboth{3.60}{3.60} & .55630 & .55642 & .55654 & .55666 & .55678 & .55691 & .55703 & .55715 & .55727 & .55739 \\
\rowcolor{bg} \textbf{3.61}\markboth{3.61}{3.61} & .55751 & .55763 & .55775 & .55787 & .55799 & .55811 & .55823 & .55835 & .55847 & .55859 \\
 \textbf{3.62}\markboth{3.62}{3.62} & .55871 & .55883 & .55895 & .55907 & .55919 & .55931 & .55943 & .55955 & .55967 & .55979 \\
\rowcolor{bg} \textbf{3.63}\markboth{3.63}{3.63} & .55991 & .56003 & .56015 & .56027 & .56038 & .56050 & .56062 & .56074 & .56086 & .56098 \\
 \textbf{3.64}\markboth{3.64}{3.64} & .56110 & .56122 & .56134 & .56146 & .56158 & .56170 & .56182 & .56194 & .56205 & .56217 \\
\rowcolor{bg} \textbf{3.65}\markboth{3.65}{3.65} & .56229 & .56241 & .56253 & .56265 & .56277 & .56289 & .56301 & .56312 & .56324 & .56336 \\
 \textbf{3.66}\markboth{3.66}{3.66} & .56348 & .56360 & .56372 & .56384 & .56396 & .56407 & .56419 & .56431 & .56443 & .56455 \\
\rowcolor{bg} \textbf{3.67}\markboth{3.67}{3.67} & .56467 & .56478 & .56490 & .56502 & .56514 & .56526 & .56538 & .56549 & .56561 & .56573 \\
 \textbf{3.68}\markboth{3.68}{3.68} & .56585 & .56597 & .56608 & .56620 & .56632 & .56644 & .56656 & .56667 & .56679 & .56691 \\
\rowcolor{bg} \textbf{3.69}\markboth{3.69}{3.69} & .56703 & .56714 & .56726 & .56738 & .56750 & .56761 & .56773 & .56785 & .56797 & .56808 \\
 \textbf{3.70}\markboth{3.70}{3.70} & .56820 & .56832 & .56844 & .56855 & .56867 & .56879 & .56891 & .56902 & .56914 & .56926 \\
\rowcolor{bg} \textbf{3.71}\markboth{3.71}{3.71} & .56937 & .56949 & .56961 & .56972 & .56984 & .56996 & .57008 & .57019 & .57031 & .57043 \\
 \textbf{3.72}\markboth{3.72}{3.72} & .57054 & .57066 & .57078 & .57089 & .57101 & .57113 & .57124 & .57136 & .57148 & .57159 \\
\rowcolor{bg} \textbf{3.73}\markboth{3.73}{3.73} & .57171 & .57183 & .57194 & .57206 & .57217 & .57229 & .57241 & .57252 & .57264 & .57276 \\
 \textbf{3.74}\markboth{3.74}{3.74} & .57287 & .57299 & .57310 & .57322 & .57334 & .57345 & .57357 & .57368 & .57380 & .57392 \\
\rowcolor{bg} \textbf{3.75}\markboth{3.75}{3.75} & .57403 & .57415 & .57426 & .57438 & .57449 & .57461 & .57473 & .57484 & .57496 & .57507 \\
 \textbf{3.76}\markboth{3.76}{3.76} & .57519 & .57530 & .57542 & .57553 & .57565 & .57576 & .57588 & .57600 & .57611 & .57623 \\
\rowcolor{bg} \textbf{3.77}\markboth{3.77}{3.77} & .57634 & .57646 & .57657 & .57669 & .57680 & .57692 & .57703 & .57715 & .57726 & .57738 \\
 \textbf{3.78}\markboth{3.78}{3.78} & .57749 & .57761 & .57772 & .57784 & .57795 & .57807 & .57818 & .57830 & .57841 & .57852 \\
\rowcolor{bg} \textbf{3.79}\markboth{3.79}{3.79} & .57864 & .57875 & .57887 & .57898 & .57910 & .57921 & .57933 & .57944 & .57955 & .57967 \\
 \textbf{3.80}\markboth{3.80}{3.80} & .57978 & .57990 & .58001 & .58013 & .58024 & .58035 & .58047 & .58058 & .58070 & .58081 \\
\rowcolor{bg} \textbf{3.81}\markboth{3.81}{3.81} & .58092 & .58104 & .58115 & .58127 & .58138 & .58149 & .58161 & .58172 & .58184 & .58195 \\
 \textbf{3.82}\markboth{3.82}{3.82} & .58206 & .58218 & .58229 & .58240 & .58252 & .58263 & .58274 & .58286 & .58297 & .58309 \\
\rowcolor{bg} \textbf{3.83}\markboth{3.83}{3.83} & .58320 & .58331 & .58343 & .58354 & .58365 & .58377 & .58388 & .58399 & .58410 & .58422 \\
 \textbf{3.84}\markboth{3.84}{3.84} & .58433 & .58444 & .58456 & .58467 & .58478 & .58490 & .58501 & .58512 & .58524 & .58535 \\
\rowcolor{bg} \textbf{3.85}\markboth{3.85}{3.85} & .58546 & .58557 & .58569 & .58580 & .58591 & .58602 & .58614 & .58625 & .58636 & .58647 \\
 \textbf{3.86}\markboth{3.86}{3.86} & .58659 & .58670 & .58681 & .58692 & .58704 & .58715 & .58726 & .58737 & .58749 & .58760 \\
\rowcolor{bg} \textbf{3.87}\markboth{3.87}{3.87} & .58771 & .58782 & .58794 & .58805 & .58816 & .58827 & .58838 & .58850 & .58861 & .58872 \\
 \textbf{3.88}\markboth{3.88}{3.88} & .58883 & .58894 & .58906 & .58917 & .58928 & .58939 & .58950 & .58961 & .58973 & .58984 \\
\rowcolor{bg} \textbf{3.89}\markboth{3.89}{3.89} & .58995 & .59006 & .59017 & .59028 & .59040 & .59051 & .59062 & .59073 & .59084 & .59095 \\
 \textbf{3.90}\markboth{3.90}{3.90} & .59106 & .59118 & .59129 & .59140 & .59151 & .59162 & .59173 & .59184 & .59195 & .59207 \\
\rowcolor{bg} \textbf{3.91}\markboth{3.91}{3.91} & .59218 & .59229 & .59240 & .59251 & .59262 & .59273 & .59284 & .59295 & .59306 & .59318 \\
 \textbf{3.92}\markboth{3.92}{3.92} & .59329 & .59340 & .59351 & .59362 & .59373 & .59384 & .59395 & .59406 & .59417 & .59428 \\
\rowcolor{bg} \textbf{3.93}\markboth{3.93}{3.93} & .59439 & .59450 & .59461 & .59472 & .59483 & .59494 & .59506 & .59517 & .59528 & .59539 \\
 \textbf{3.94}\markboth{3.94}{3.94} & .59550 & .59561 & .59572 & .59583 & .59594 & .59605 & .59616 & .59627 & .59638 & .59649 \\
\rowcolor{bg} \textbf{3.95}\markboth{3.95}{3.95} & .59660 & .59671 & .59682 & .59693 & .59704 & .59715 & .59726 & .59737 & .59748 & .59759 \\
 \textbf{3.96}\markboth{3.96}{3.96} & .59770 & .59780 & .59791 & .59802 & .59813 & .59824 & .59835 & .59846 & .59857 & .59868 \\
\rowcolor{bg} \textbf{3.97}\markboth{3.97}{3.97} & .59879 & .59890 & .59901 & .59912 & .59923 & .59934 & .59945 & .59956 & .59966 & .59977 \\
 \textbf{3.98}\markboth{3.98}{3.98} & .59988 & .59999 & .60010 & .60021 & .60032 & .60043 & .60054 & .60065 & .60076 & .60086 \\
\rowcolor{bg} \textbf{3.99}\markboth{3.99}{3.99} & .60097 & .60108 & .60119 & .60130 & .60141 & .60152 & .60163 & .60173 & .60184 & .60195 \\
 \textcolor{blue}{\textbf{4.00}}\markboth{4.00}{4.00} & .60206 & .60217 & .60228 & .60239 & .60249 & .60260 & .60271 & .60282 & .60293 & .60304 \\
\rowcolor{bg} \textbf{4.01}\markboth{4.01}{4.01} & .60314 & .60325 & .60336 & .60347 & .60358 & .60369 & .60379 & .60390 & .60401 & .60412 \\
 \textbf{4.02}\markboth{4.02}{4.02} & .60423 & .60433 & .60444 & .60455 & .60466 & .60477 & .60487 & .60498 & .60509 & .60520 \\
\rowcolor{bg} \textbf{4.03}\markboth{4.03}{4.03} & .60531 & .60541 & .60552 & .60563 & .60574 & .60584 & .60595 & .60606 & .60617 & .60627 \\
 \textbf{4.04}\markboth{4.04}{4.04} & .60638 & .60649 & .60660 & .60670 & .60681 & .60692 & .60703 & .60713 & .60724 & .60735 \\
\rowcolor{bg} \textbf{4.05}\markboth{4.05}{4.05} & .60746 & .60756 & .60767 & .60778 & .60788 & .60799 & .60810 & .60821 & .60831 & .60842 \\
 \textbf{4.06}\markboth{4.06}{4.06} & .60853 & .60863 & .60874 & .60885 & .60895 & .60906 & .60917 & .60927 & .60938 & .60949 \\
\rowcolor{bg} \textbf{4.07}\markboth{4.07}{4.07} & .60959 & .60970 & .60981 & .60991 & .61002 & .61013 & .61023 & .61034 & .61045 & .61055 \\
 \textbf{4.08}\markboth{4.08}{4.08} & .61066 & .61077 & .61087 & .61098 & .61109 & .61119 & .61130 & .61140 & .61151 & .61162 \\
\rowcolor{bg} \textbf{4.09}\markboth{4.09}{4.09} & .61172 & .61183 & .61194 & .61204 & .61215 & .61225 & .61236 & .61247 & .61257 & .61268 \\
 \textbf{4.10}\markboth{4.10}{4.10} & .61278 & .61289 & .61300 & .61310 & .61321 & .61331 & .61342 & .61352 & .61363 & .61374 \\
\rowcolor{bg} \textbf{4.11}\markboth{4.11}{4.11} & .61384 & .61395 & .61405 & .61416 & .61426 & .61437 & .61448 & .61458 & .61469 & .61479 \\
 \textbf{4.12}\markboth{4.12}{4.12} & .61490 & .61500 & .61511 & .61521 & .61532 & .61542 & .61553 & .61563 & .61574 & .61584 \\
\rowcolor{bg} \textbf{4.13}\markboth{4.13}{4.13} & .61595 & .61606 & .61616 & .61627 & .61637 & .61648 & .61658 & .61669 & .61679 & .61690 \\
 \textbf{4.14}\markboth{4.14}{4.14} & .61700 & .61711 & .61721 & .61731 & .61742 & .61752 & .61763 & .61773 & .61784 & .61794 \\
\rowcolor{bg} \textbf{4.15}\markboth{4.15}{4.15} & .61805 & .61815 & .61826 & .61836 & .61847 & .61857 & .61868 & .61878 & .61888 & .61899 \\
 \textbf{4.16}\markboth{4.16}{4.16} & .61909 & .61920 & .61930 & .61941 & .61951 & .61962 & .61972 & .61982 & .61993 & .62003 \\
\rowcolor{bg} \textbf{4.17}\markboth{4.17}{4.17} & .62014 & .62024 & .62034 & .62045 & .62055 & .62066 & .62076 & .62086 & .62097 & .62107 \\
 \textbf{4.18}\markboth{4.18}{4.18} & .62118 & .62128 & .62138 & .62149 & .62159 & .62170 & .62180 & .62190 & .62201 & .62211 \\
\rowcolor{bg} \textbf{4.19}\markboth{4.19}{4.19} & .62221 & .62232 & .62242 & .62252 & .62263 & .62273 & .62284 & .62294 & .62304 & .62315 \\
 \textbf{4.20}\markboth{4.20}{4.20} & .62325 & .62335 & .62346 & .62356 & .62366 & .62377 & .62387 & .62397 & .62408 & .62418 \\
\rowcolor{bg} \textbf{4.21}\markboth{4.21}{4.21} & .62428 & .62439 & .62449 & .62459 & .62469 & .62480 & .62490 & .62500 & .62511 & .62521 \\
 \textbf{4.22}\markboth{4.22}{4.22} & .62531 & .62542 & .62552 & .62562 & .62572 & .62583 & .62593 & .62603 & .62613 & .62624 \\
\rowcolor{bg} \textbf{4.23}\markboth{4.23}{4.23} & .62634 & .62644 & .62655 & .62665 & .62675 & .62685 & .62696 & .62706 & .62716 & .62726 \\
 \textbf{4.24}\markboth{4.24}{4.24} & .62737 & .62747 & .62757 & .62767 & .62778 & .62788 & .62798 & .62808 & .62818 & .62829 \\
\rowcolor{bg} \textbf{4.25}\markboth{4.25}{4.25} & .62839 & .62849 & .62859 & .62870 & .62880 & .62890 & .62900 & .62910 & .62921 & .62931 \\
 \textbf{4.26}\markboth{4.26}{4.26} & .62941 & .62951 & .62961 & .62972 & .62982 & .62992 & .63002 & .63012 & .63022 & .63033 \\
\rowcolor{bg} \textbf{4.27}\markboth{4.27}{4.27} & .63043 & .63053 & .63063 & .63073 & .63083 & .63094 & .63104 & .63114 & .63124 & .63134 \\
 \textbf{4.28}\markboth{4.28}{4.28} & .63144 & .63155 & .63165 & .63175 & .63185 & .63195 & .63205 & .63215 & .63225 & .63236 \\
\rowcolor{bg} \textbf{4.29}\markboth{4.29}{4.29} & .63246 & .63256 & .63266 & .63276 & .63286 & .63296 & .63306 & .63317 & .63327 & .63337 \\
 \textbf{4.30}\markboth{4.30}{4.30} & .63347 & .63357 & .63367 & .63377 & .63387 & .63397 & .63407 & .63417 & .63428 & .63438 \\
\rowcolor{bg} \textbf{4.31}\markboth{4.31}{4.31} & .63448 & .63458 & .63468 & .63478 & .63488 & .63498 & .63508 & .63518 & .63528 & .63538 \\
 \textbf{4.32}\markboth{4.32}{4.32} & .63548 & .63558 & .63568 & .63579 & .63589 & .63599 & .63609 & .63619 & .63629 & .63639 \\
\rowcolor{bg} \textbf{4.33}\markboth{4.33}{4.33} & .63649 & .63659 & .63669 & .63679 & .63689 & .63699 & .63709 & .63719 & .63729 & .63739 \\
 \textbf{4.34}\markboth{4.34}{4.34} & .63749 & .63759 & .63769 & .63779 & .63789 & .63799 & .63809 & .63819 & .63829 & .63839 \\
\rowcolor{bg} \textbf{4.35}\markboth{4.35}{4.35} & .63849 & .63859 & .63869 & .63879 & .63889 & .63899 & .63909 & .63919 & .63929 & .63939 \\
 \textbf{4.36}\markboth{4.36}{4.36} & .63949 & .63959 & .63969 & .63979 & .63988 & .63998 & .64008 & .64018 & .64028 & .64038 \\
\rowcolor{bg} \textbf{4.37}\markboth{4.37}{4.37} & .64048 & .64058 & .64068 & .64078 & .64088 & .64098 & .64108 & .64118 & .64128 & .64137 \\
 \textbf{4.38}\markboth{4.38}{4.38} & .64147 & .64157 & .64167 & .64177 & .64187 & .64197 & .64207 & .64217 & .64227 & .64237 \\
\rowcolor{bg} \textbf{4.39}\markboth{4.39}{4.39} & .64246 & .64256 & .64266 & .64276 & .64286 & .64296 & .64306 & .64316 & .64326 & .64335 \\
 \textbf{4.40}\markboth{4.40}{4.40} & .64345 & .64355 & .64365 & .64375 & .64385 & .64395 & .64404 & .64414 & .64424 & .64434 \\
\rowcolor{bg} \textbf{4.41}\markboth{4.41}{4.41} & .64444 & .64454 & .64464 & .64473 & .64483 & .64493 & .64503 & .64513 & .64523 & .64532 \\
 \textbf{4.42}\markboth{4.42}{4.42} & .64542 & .64552 & .64562 & .64572 & .64582 & .64591 & .64601 & .64611 & .64621 & .64631 \\
\rowcolor{bg} \textbf{4.43}\markboth{4.43}{4.43} & .64640 & .64650 & .64660 & .64670 & .64680 & .64689 & .64699 & .64709 & .64719 & .64729 \\
 \textbf{4.44}\markboth{4.44}{4.44} & .64738 & .64748 & .64758 & .64768 & .64777 & .64787 & .64797 & .64807 & .64816 & .64826 \\
\rowcolor{bg} \textbf{4.45}\markboth{4.45}{4.45} & .64836 & .64846 & .64856 & .64865 & .64875 & .64885 & .64895 & .64904 & .64914 & .64924 \\
 \textbf{4.46}\markboth{4.46}{4.46} & .64933 & .64943 & .64953 & .64963 & .64972 & .64982 & .64992 & .65002 & .65011 & .65021 \\
\rowcolor{bg} \textbf{4.47}\markboth{4.47}{4.47} & .65031 & .65040 & .65050 & .65060 & .65070 & .65079 & .65089 & .65099 & .65108 & .65118 \\
 \textbf{4.48}\markboth{4.48}{4.48} & .65128 & .65137 & .65147 & .65157 & .65167 & .65176 & .65186 & .65196 & .65205 & .65215 \\
\rowcolor{bg} \textbf{4.49}\markboth{4.49}{4.49} & .65225 & .65234 & .65244 & .65254 & .65263 & .65273 & .65283 & .65292 & .65302 & .65312 \\
 \textbf{4.50}\markboth{4.50}{4.50} & .65321 & .65331 & .65341 & .65350 & .65360 & .65369 & .65379 & .65389 & .65398 & .65408 \\
\rowcolor{bg} \textbf{4.51}\markboth{4.51}{4.51} & .65418 & .65427 & .65437 & .65447 & .65456 & .65466 & .65475 & .65485 & .65495 & .65504 \\
 \textbf{4.52}\markboth{4.52}{4.52} & .65514 & .65523 & .65533 & .65543 & .65552 & .65562 & .65571 & .65581 & .65591 & .65600 \\
\rowcolor{bg} \textbf{4.53}\markboth{4.53}{4.53} & .65610 & .65619 & .65629 & .65639 & .65648 & .65658 & .65667 & .65677 & .65686 & .65696 \\
 \textbf{4.54}\markboth{4.54}{4.54} & .65706 & .65715 & .65725 & .65734 & .65744 & .65753 & .65763 & .65772 & .65782 & .65792 \\
\rowcolor{bg} \textbf{4.55}\markboth{4.55}{4.55} & .65801 & .65811 & .65820 & .65830 & .65839 & .65849 & .65858 & .65868 & .65877 & .65887 \\
 \textbf{4.56}\markboth{4.56}{4.56} & .65896 & .65906 & .65916 & .65925 & .65935 & .65944 & .65954 & .65963 & .65973 & .65982 \\
\rowcolor{bg} \textbf{4.57}\markboth{4.57}{4.57} & .65992 & .66001 & .66011 & .66020 & .66030 & .66039 & .66049 & .66058 & .66068 & .66077 \\
 \textbf{4.58}\markboth{4.58}{4.58} & .66087 & .66096 & .66106 & .66115 & .66124 & .66134 & .66143 & .66153 & .66162 & .66172 \\
\rowcolor{bg} \textbf{4.59}\markboth{4.59}{4.59} & .66181 & .66191 & .66200 & .66210 & .66219 & .66229 & .66238 & .66247 & .66257 & .66266 \\
 \textbf{4.60}\markboth{4.60}{4.60} & .66276 & .66285 & .66295 & .66304 & .66314 & .66323 & .66332 & .66342 & .66351 & .66361 \\
\rowcolor{bg} \textbf{4.61}\markboth{4.61}{4.61} & .66370 & .66380 & .66389 & .66398 & .66408 & .66417 & .66427 & .66436 & .66445 & .66455 \\
 \textbf{4.62}\markboth{4.62}{4.62} & .66464 & .66474 & .66483 & .66492 & .66502 & .66511 & .66521 & .66530 & .66539 & .66549 \\
\rowcolor{bg} \textbf{4.63}\markboth{4.63}{4.63} & .66558 & .66567 & .66577 & .66586 & .66596 & .66605 & .66614 & .66624 & .66633 & .66642 \\
 \textbf{4.64}\markboth{4.64}{4.64} & .66652 & .66661 & .66671 & .66680 & .66689 & .66699 & .66708 & .66717 & .66727 & .66736 \\
\rowcolor{bg} \textbf{4.65}\markboth{4.65}{4.65} & .66745 & .66755 & .66764 & .66773 & .66783 & .66792 & .66801 & .66811 & .66820 & .66829 \\
 \textbf{4.66}\markboth{4.66}{4.66} & .66839 & .66848 & .66857 & .66867 & .66876 & .66885 & .66894 & .66904 & .66913 & .66922 \\
\rowcolor{bg} \textbf{4.67}\markboth{4.67}{4.67} & .66932 & .66941 & .66950 & .66960 & .66969 & .66978 & .66987 & .66997 & .67006 & .67015 \\
 \textbf{4.68}\markboth{4.68}{4.68} & .67025 & .67034 & .67043 & .67052 & .67062 & .67071 & .67080 & .67089 & .67099 & .67108 \\
\rowcolor{bg} \textbf{4.69}\markboth{4.69}{4.69} & .67117 & .67127 & .67136 & .67145 & .67154 & .67164 & .67173 & .67182 & .67191 & .67201 \\
 \textbf{4.70}\markboth{4.70}{4.70} & .67210 & .67219 & .67228 & .67237 & .67247 & .67256 & .67265 & .67274 & .67284 & .67293 \\
\rowcolor{bg} \textbf{4.71}\markboth{4.71}{4.71} & .67302 & .67311 & .67321 & .67330 & .67339 & .67348 & .67357 & .67367 & .67376 & .67385 \\
 \textbf{4.72}\markboth{4.72}{4.72} & .67394 & .67403 & .67413 & .67422 & .67431 & .67440 & .67449 & .67459 & .67468 & .67477 \\
\rowcolor{bg} \textbf{4.73}\markboth{4.73}{4.73} & .67486 & .67495 & .67504 & .67514 & .67523 & .67532 & .67541 & .67550 & .67560 & .67569 \\
 \textbf{4.74}\markboth{4.74}{4.74} & .67578 & .67587 & .67596 & .67605 & .67614 & .67624 & .67633 & .67642 & .67651 & .67660 \\
\rowcolor{bg} \textbf{4.75}\markboth{4.75}{4.75} & .67669 & .67679 & .67688 & .67697 & .67706 & .67715 & .67724 & .67733 & .67742 & .67752 \\
 \textbf{4.76}\markboth{4.76}{4.76} & .67761 & .67770 & .67779 & .67788 & .67797 & .67806 & .67815 & .67825 & .67834 & .67843 \\
\rowcolor{bg} \textbf{4.77}\markboth{4.77}{4.77} & .67852 & .67861 & .67870 & .67879 & .67888 & .67897 & .67906 & .67916 & .67925 & .67934 \\
 \textbf{4.78}\markboth{4.78}{4.78} & .67943 & .67952 & .67961 & .67970 & .67979 & .67988 & .67997 & .68006 & .68015 & .68024 \\
\rowcolor{bg} \textbf{4.79}\markboth{4.79}{4.79} & .68034 & .68043 & .68052 & .68061 & .68070 & .68079 & .68088 & .68097 & .68106 & .68115 \\
 \textbf{4.80}\markboth{4.80}{4.80} & .68124 & .68133 & .68142 & .68151 & .68160 & .68169 & .68178 & .68187 & .68196 & .68205 \\
\rowcolor{bg} \textbf{4.81}\markboth{4.81}{4.81} & .68215 & .68224 & .68233 & .68242 & .68251 & .68260 & .68269 & .68278 & .68287 & .68296 \\
 \textbf{4.82}\markboth{4.82}{4.82} & .68305 & .68314 & .68323 & .68332 & .68341 & .68350 & .68359 & .68368 & .68377 & .68386 \\
\rowcolor{bg} \textbf{4.83}\markboth{4.83}{4.83} & .68395 & .68404 & .68413 & .68422 & .68431 & .68440 & .68449 & .68458 & .68467 & .68476 \\
 \textbf{4.84}\markboth{4.84}{4.84} & .68485 & .68494 & .68502 & .68511 & .68520 & .68529 & .68538 & .68547 & .68556 & .68565 \\
\rowcolor{bg} \textbf{4.85}\markboth{4.85}{4.85} & .68574 & .68583 & .68592 & .68601 & .68610 & .68619 & .68628 & .68637 & .68646 & .68655 \\
 \textbf{4.86}\markboth{4.86}{4.86} & .68664 & .68673 & .68681 & .68690 & .68699 & .68708 & .68717 & .68726 & .68735 & .68744 \\
\rowcolor{bg} \textbf{4.87}\markboth{4.87}{4.87} & .68753 & .68762 & .68771 & .68780 & .68789 & .68797 & .68806 & .68815 & .68824 & .68833 \\
 \textbf{4.88}\markboth{4.88}{4.88} & .68842 & .68851 & .68860 & .68869 & .68878 & .68886 & .68895 & .68904 & .68913 & .68922 \\
\rowcolor{bg} \textbf{4.89}\markboth{4.89}{4.89} & .68931 & .68940 & .68949 & .68958 & .68966 & .68975 & .68984 & .68993 & .69002 & .69011 \\
 \textbf{4.90}\markboth{4.90}{4.90} & .69020 & .69028 & .69037 & .69046 & .69055 & .69064 & .69073 & .69082 & .69090 & .69099 \\
\rowcolor{bg} \textbf{4.91}\markboth{4.91}{4.91} & .69108 & .69117 & .69126 & .69135 & .69144 & .69152 & .69161 & .69170 & .69179 & .69188 \\
 \textbf{4.92}\markboth{4.92}{4.92} & .69197 & .69205 & .69214 & .69223 & .69232 & .69241 & .69249 & .69258 & .69267 & .69276 \\
\rowcolor{bg} \textbf{4.93}\markboth{4.93}{4.93} & .69285 & .69294 & .69302 & .69311 & .69320 & .69329 & .69338 & .69346 & .69355 & .69364 \\
 \textbf{4.94}\markboth{4.94}{4.94} & .69373 & .69381 & .69390 & .69399 & .69408 & .69417 & .69425 & .69434 & .69443 & .69452 \\
\rowcolor{bg} \textbf{4.95}\markboth{4.95}{4.95} & .69461 & .69469 & .69478 & .69487 & .69496 & .69504 & .69513 & .69522 & .69531 & .69539 \\
 \textbf{4.96}\markboth{4.96}{4.96} & .69548 & .69557 & .69566 & .69574 & .69583 & .69592 & .69601 & .69609 & .69618 & .69627 \\
\rowcolor{bg} \textbf{4.97}\markboth{4.97}{4.97} & .69636 & .69644 & .69653 & .69662 & .69671 & .69679 & .69688 & .69697 & .69705 & .69714 \\
 \textbf{4.98}\markboth{4.98}{4.98} & .69723 & .69732 & .69740 & .69749 & .69758 & .69767 & .69775 & .69784 & .69793 & .69801 \\
\rowcolor{bg} \textbf{4.99}\markboth{4.99}{4.99} & .69810 & .69819 & .69827 & .69836 & .69845 & .69854 & .69862 & .69871 & .69880 & .69888 \\
 \textcolor{blue}{\textbf{5.00}}\markboth{5.00}{5.00} & .69897 & .69906 & .69914 & .69923 & .69932 & .69940 & .69949 & .69958 & .69966 & .69975 \\
\rowcolor{bg} \textbf{5.01}\markboth{5.01}{5.01} & .69984 & .69992 & .70001 & .70010 & .70018 & .70027 & .70036 & .70044 & .70053 & .70062 \\
 \textbf{5.02}\markboth{5.02}{5.02} & .70070 & .70079 & .70088 & .70096 & .70105 & .70114 & .70122 & .70131 & .70140 & .70148 \\
\rowcolor{bg} \textbf{5.03}\markboth{5.03}{5.03} & .70157 & .70165 & .70174 & .70183 & .70191 & .70200 & .70209 & .70217 & .70226 & .70234 \\
 \textbf{5.04}\markboth{5.04}{5.04} & .70243 & .70252 & .70260 & .70269 & .70278 & .70286 & .70295 & .70303 & .70312 & .70321 \\
\rowcolor{bg} \textbf{5.05}\markboth{5.05}{5.05} & .70329 & .70338 & .70346 & .70355 & .70364 & .70372 & .70381 & .70389 & .70398 & .70406 \\
 \textbf{5.06}\markboth{5.06}{5.06} & .70415 & .70424 & .70432 & .70441 & .70449 & .70458 & .70467 & .70475 & .70484 & .70492 \\
\rowcolor{bg} \textbf{5.07}\markboth{5.07}{5.07} & .70501 & .70509 & .70518 & .70526 & .70535 & .70544 & .70552 & .70561 & .70569 & .70578 \\
 \textbf{5.08}\markboth{5.08}{5.08} & .70586 & .70595 & .70603 & .70612 & .70621 & .70629 & .70638 & .70646 & .70655 & .70663 \\
\rowcolor{bg} \textbf{5.09}\markboth{5.09}{5.09} & .70672 & .70680 & .70689 & .70697 & .70706 & .70714 & .70723 & .70731 & .70740 & .70749 \\
 \textbf{5.10}\markboth{5.10}{5.10} & .70757 & .70766 & .70774 & .70783 & .70791 & .70800 & .70808 & .70817 & .70825 & .70834 \\
\rowcolor{bg} \textbf{5.11}\markboth{5.11}{5.11} & .70842 & .70851 & .70859 & .70868 & .70876 & .70885 & .70893 & .70902 & .70910 & .70919 \\
 \textbf{5.12}\markboth{5.12}{5.12} & .70927 & .70935 & .70944 & .70952 & .70961 & .70969 & .70978 & .70986 & .70995 & .71003 \\
\rowcolor{bg} \textbf{5.13}\markboth{5.13}{5.13} & .71012 & .71020 & .71029 & .71037 & .71046 & .71054 & .71063 & .71071 & .71079 & .71088 \\
 \textbf{5.14}\markboth{5.14}{5.14} & .71096 & .71105 & .71113 & .71122 & .71130 & .71139 & .71147 & .71155 & .71164 & .71172 \\
\rowcolor{bg} \textbf{5.15}\markboth{5.15}{5.15} & .71181 & .71189 & .71198 & .71206 & .71214 & .71223 & .71231 & .71240 & .71248 & .71257 \\
 \textbf{5.16}\markboth{5.16}{5.16} & .71265 & .71273 & .71282 & .71290 & .71299 & .71307 & .71315 & .71324 & .71332 & .71341 \\
\rowcolor{bg} \textbf{5.17}\markboth{5.17}{5.17} & .71349 & .71357 & .71366 & .71374 & .71383 & .71391 & .71399 & .71408 & .71416 & .71425 \\
 \textbf{5.18}\markboth{5.18}{5.18} & .71433 & .71441 & .71450 & .71458 & .71466 & .71475 & .71483 & .71492 & .71500 & .71508 \\
\rowcolor{bg} \textbf{5.19}\markboth{5.19}{5.19} & .71517 & .71525 & .71533 & .71542 & .71550 & .71559 & .71567 & .71575 & .71584 & .71592 \\
 \textbf{5.20}\markboth{5.20}{5.20} & .71600 & .71609 & .71617 & .71625 & .71634 & .71642 & .71650 & .71659 & .71667 & .71675 \\
\rowcolor{bg} \textbf{5.21}\markboth{5.21}{5.21} & .71684 & .71692 & .71700 & .71709 & .71717 & .71725 & .71734 & .71742 & .71750 & .71759 \\
 \textbf{5.22}\markboth{5.22}{5.22} & .71767 & .71775 & .71784 & .71792 & .71800 & .71809 & .71817 & .71825 & .71834 & .71842 \\
\rowcolor{bg} \textbf{5.23}\markboth{5.23}{5.23} & .71850 & .71858 & .71867 & .71875 & .71883 & .71892 & .71900 & .71908 & .71917 & .71925 \\
 \textbf{5.24}\markboth{5.24}{5.24} & .71933 & .71941 & .71950 & .71958 & .71966 & .71975 & .71983 & .71991 & .71999 & .72008 \\
\rowcolor{bg} \textbf{5.25}\markboth{5.25}{5.25} & .72016 & .72024 & .72032 & .72041 & .72049 & .72057 & .72066 & .72074 & .72082 & .72090 \\
 \textbf{5.26}\markboth{5.26}{5.26} & .72099 & .72107 & .72115 & .72123 & .72132 & .72140 & .72148 & .72156 & .72165 & .72173 \\
\rowcolor{bg} \textbf{5.27}\markboth{5.27}{5.27} & .72181 & .72189 & .72198 & .72206 & .72214 & .72222 & .72230 & .72239 & .72247 & .72255 \\
 \textbf{5.28}\markboth{5.28}{5.28} & .72263 & .72272 & .72280 & .72288 & .72296 & .72304 & .72313 & .72321 & .72329 & .72337 \\
\rowcolor{bg} \textbf{5.29}\markboth{5.29}{5.29} & .72346 & .72354 & .72362 & .72370 & .72378 & .72387 & .72395 & .72403 & .72411 & .72419 \\
 \textbf{5.30}\markboth{5.30}{5.30} & .72428 & .72436 & .72444 & .72452 & .72460 & .72469 & .72477 & .72485 & .72493 & .72501 \\
\rowcolor{bg} \textbf{5.31}\markboth{5.31}{5.31} & .72509 & .72518 & .72526 & .72534 & .72542 & .72550 & .72558 & .72567 & .72575 & .72583 \\
 \textbf{5.32}\markboth{5.32}{5.32} & .72591 & .72599 & .72607 & .72616 & .72624 & .72632 & .72640 & .72648 & .72656 & .72665 \\
\rowcolor{bg} \textbf{5.33}\markboth{5.33}{5.33} & .72673 & .72681 & .72689 & .72697 & .72705 & .72713 & .72722 & .72730 & .72738 & .72746 \\
 \textbf{5.34}\markboth{5.34}{5.34} & .72754 & .72762 & .72770 & .72779 & .72787 & .72795 & .72803 & .72811 & .72819 & .72827 \\
\rowcolor{bg} \textbf{5.35}\markboth{5.35}{5.35} & .72835 & .72843 & .72852 & .72860 & .72868 & .72876 & .72884 & .72892 & .72900 & .72908 \\
 \textbf{5.36}\markboth{5.36}{5.36} & .72916 & .72925 & .72933 & .72941 & .72949 & .72957 & .72965 & .72973 & .72981 & .72989 \\
\rowcolor{bg} \textbf{5.37}\markboth{5.37}{5.37} & .72997 & .73006 & .73014 & .73022 & .73030 & .73038 & .73046 & .73054 & .73062 & .73070 \\
 \textbf{5.38}\markboth{5.38}{5.38} & .73078 & .73086 & .73094 & .73102 & .73111 & .73119 & .73127 & .73135 & .73143 & .73151 \\
\rowcolor{bg} \textbf{5.39}\markboth{5.39}{5.39} & .73159 & .73167 & .73175 & .73183 & .73191 & .73199 & .73207 & .73215 & .73223 & .73231 \\
 \textbf{5.40}\markboth{5.40}{5.40} & .73239 & .73247 & .73255 & .73263 & .73272 & .73280 & .73288 & .73296 & .73304 & .73312 \\
\rowcolor{bg} \textbf{5.41}\markboth{5.41}{5.41} & .73320 & .73328 & .73336 & .73344 & .73352 & .73360 & .73368 & .73376 & .73384 & .73392 \\
 \textbf{5.42}\markboth{5.42}{5.42} & .73400 & .73408 & .73416 & .73424 & .73432 & .73440 & .73448 & .73456 & .73464 & .73472 \\
\rowcolor{bg} \textbf{5.43}\markboth{5.43}{5.43} & .73480 & .73488 & .73496 & .73504 & .73512 & .73520 & .73528 & .73536 & .73544 & .73552 \\
 \textbf{5.44}\markboth{5.44}{5.44} & .73560 & .73568 & .73576 & .73584 & .73592 & .73600 & .73608 & .73616 & .73624 & .73632 \\
\rowcolor{bg} \textbf{5.45}\markboth{5.45}{5.45} & .73640 & .73648 & .73656 & .73664 & .73672 & .73679 & .73687 & .73695 & .73703 & .73711 \\
 \textbf{5.46}\markboth{5.46}{5.46} & .73719 & .73727 & .73735 & .73743 & .73751 & .73759 & .73767 & .73775 & .73783 & .73791 \\
\rowcolor{bg} \textbf{5.47}\markboth{5.47}{5.47} & .73799 & .73807 & .73815 & .73823 & .73830 & .73838 & .73846 & .73854 & .73862 & .73870 \\
 \textbf{5.48}\markboth{5.48}{5.48} & .73878 & .73886 & .73894 & .73902 & .73910 & .73918 & .73926 & .73933 & .73941 & .73949 \\
\rowcolor{bg} \textbf{5.49}\markboth{5.49}{5.49} & .73957 & .73965 & .73973 & .73981 & .73989 & .73997 & .74005 & .74013 & .74020 & .74028 \\
 \textbf{5.50}\markboth{5.50}{5.50} & .74036 & .74044 & .74052 & .74060 & .74068 & .74076 & .74084 & .74092 & .74099 & .74107 \\
\rowcolor{bg} \textbf{5.51}\markboth{5.51}{5.51} & .74115 & .74123 & .74131 & .74139 & .74147 & .74155 & .74162 & .74170 & .74178 & .74186 \\
 \textbf{5.52}\markboth{5.52}{5.52} & .74194 & .74202 & .74210 & .74218 & .74225 & .74233 & .74241 & .74249 & .74257 & .74265 \\
\rowcolor{bg} \textbf{5.53}\markboth{5.53}{5.53} & .74273 & .74280 & .74288 & .74296 & .74304 & .74312 & .74320 & .74327 & .74335 & .74343 \\
 \textbf{5.54}\markboth{5.54}{5.54} & .74351 & .74359 & .74367 & .74374 & .74382 & .74390 & .74398 & .74406 & .74414 & .74421 \\
\rowcolor{bg} \textbf{5.55}\markboth{5.55}{5.55} & .74429 & .74437 & .74445 & .74453 & .74461 & .74468 & .74476 & .74484 & .74492 & .74500 \\
 \textbf{5.56}\markboth{5.56}{5.56} & .74507 & .74515 & .74523 & .74531 & .74539 & .74547 & .74554 & .74562 & .74570 & .74578 \\
\rowcolor{bg} \textbf{5.57}\markboth{5.57}{5.57} & .74586 & .74593 & .74601 & .74609 & .74617 & .74624 & .74632 & .74640 & .74648 & .74656 \\
 \textbf{5.58}\markboth{5.58}{5.58} & .74663 & .74671 & .74679 & .74687 & .74695 & .74702 & .74710 & .74718 & .74726 & .74733 \\
\rowcolor{bg} \textbf{5.59}\markboth{5.59}{5.59} & .74741 & .74749 & .74757 & .74764 & .74772 & .74780 & .74788 & .74796 & .74803 & .74811 \\
 \textbf{5.60}\markboth{5.60}{5.60} & .74819 & .74827 & .74834 & .74842 & .74850 & .74858 & .74865 & .74873 & .74881 & .74889 \\
\rowcolor{bg} \textbf{5.61}\markboth{5.61}{5.61} & .74896 & .74904 & .74912 & .74920 & .74927 & .74935 & .74943 & .74950 & .74958 & .74966 \\
 \textbf{5.62}\markboth{5.62}{5.62} & .74974 & .74981 & .74989 & .74997 & .75005 & .75012 & .75020 & .75028 & .75035 & .75043 \\
\rowcolor{bg} \textbf{5.63}\markboth{5.63}{5.63} & .75051 & .75059 & .75066 & .75074 & .75082 & .75089 & .75097 & .75105 & .75113 & .75120 \\
 \textbf{5.64}\markboth{5.64}{5.64} & .75128 & .75136 & .75143 & .75151 & .75159 & .75166 & .75174 & .75182 & .75189 & .75197 \\
\rowcolor{bg} \textbf{5.65}\markboth{5.65}{5.65} & .75205 & .75213 & .75220 & .75228 & .75236 & .75243 & .75251 & .75259 & .75266 & .75274 \\
 \textbf{5.66}\markboth{5.66}{5.66} & .75282 & .75289 & .75297 & .75305 & .75312 & .75320 & .75328 & .75335 & .75343 & .75351 \\
\rowcolor{bg} \textbf{5.67}\markboth{5.67}{5.67} & .75358 & .75366 & .75374 & .75381 & .75389 & .75397 & .75404 & .75412 & .75420 & .75427 \\
 \textbf{5.68}\markboth{5.68}{5.68} & .75435 & .75442 & .75450 & .75458 & .75465 & .75473 & .75481 & .75488 & .75496 & .75504 \\
\rowcolor{bg} \textbf{5.69}\markboth{5.69}{5.69} & .75511 & .75519 & .75526 & .75534 & .75542 & .75549 & .75557 & .75565 & .75572 & .75580 \\
 \textbf{5.70}\markboth{5.70}{5.70} & .75587 & .75595 & .75603 & .75610 & .75618 & .75626 & .75633 & .75641 & .75648 & .75656 \\
\rowcolor{bg} \textbf{5.71}\markboth{5.71}{5.71} & .75664 & .75671 & .75679 & .75686 & .75694 & .75702 & .75709 & .75717 & .75724 & .75732 \\
 \textbf{5.72}\markboth{5.72}{5.72} & .75740 & .75747 & .75755 & .75762 & .75770 & .75778 & .75785 & .75793 & .75800 & .75808 \\
\rowcolor{bg} \textbf{5.73}\markboth{5.73}{5.73} & .75815 & .75823 & .75831 & .75838 & .75846 & .75853 & .75861 & .75868 & .75876 & .75884 \\
 \textbf{5.74}\markboth{5.74}{5.74} & .75891 & .75899 & .75906 & .75914 & .75921 & .75929 & .75937 & .75944 & .75952 & .75959 \\
\rowcolor{bg} \textbf{5.75}\markboth{5.75}{5.75} & .75967 & .75974 & .75982 & .75989 & .75997 & .76005 & .76012 & .76020 & .76027 & .76035 \\
 \textbf{5.76}\markboth{5.76}{5.76} & .76042 & .76050 & .76057 & .76065 & .76072 & .76080 & .76087 & .76095 & .76103 & .76110 \\
\rowcolor{bg} \textbf{5.77}\markboth{5.77}{5.77} & .76118 & .76125 & .76133 & .76140 & .76148 & .76155 & .76163 & .76170 & .76178 & .76185 \\
 \textbf{5.78}\markboth{5.78}{5.78} & .76193 & .76200 & .76208 & .76215 & .76223 & .76230 & .76238 & .76245 & .76253 & .76260 \\
\rowcolor{bg} \textbf{5.79}\markboth{5.79}{5.79} & .76268 & .76275 & .76283 & .76290 & .76298 & .76305 & .76313 & .76320 & .76328 & .76335 \\
 \textbf{5.80}\markboth{5.80}{5.80} & .76343 & .76350 & .76358 & .76365 & .76373 & .76380 & .76388 & .76395 & .76403 & .76410 \\
\rowcolor{bg} \textbf{5.81}\markboth{5.81}{5.81} & .76418 & .76425 & .76433 & .76440 & .76448 & .76455 & .76462 & .76470 & .76477 & .76485 \\
 \textbf{5.82}\markboth{5.82}{5.82} & .76492 & .76500 & .76507 & .76515 & .76522 & .76530 & .76537 & .76545 & .76552 & .76559 \\
\rowcolor{bg} \textbf{5.83}\markboth{5.83}{5.83} & .76567 & .76574 & .76582 & .76589 & .76597 & .76604 & .76612 & .76619 & .76626 & .76634 \\
 \textbf{5.84}\markboth{5.84}{5.84} & .76641 & .76649 & .76656 & .76664 & .76671 & .76678 & .76686 & .76693 & .76701 & .76708 \\
\rowcolor{bg} \textbf{5.85}\markboth{5.85}{5.85} & .76716 & .76723 & .76730 & .76738 & .76745 & .76753 & .76760 & .76768 & .76775 & .76782 \\
 \textbf{5.86}\markboth{5.86}{5.86} & .76790 & .76797 & .76805 & .76812 & .76819 & .76827 & .76834 & .76842 & .76849 & .76856 \\
\rowcolor{bg} \textbf{5.87}\markboth{5.87}{5.87} & .76864 & .76871 & .76879 & .76886 & .76893 & .76901 & .76908 & .76916 & .76923 & .76930 \\
 \textbf{5.88}\markboth{5.88}{5.88} & .76938 & .76945 & .76953 & .76960 & .76967 & .76975 & .76982 & .76989 & .76997 & .77004 \\
\rowcolor{bg} \textbf{5.89}\markboth{5.89}{5.89} & .77012 & .77019 & .77026 & .77034 & .77041 & .77048 & .77056 & .77063 & .77070 & .77078 \\
 \textbf{5.90}\markboth{5.90}{5.90} & .77085 & .77093 & .77100 & .77107 & .77115 & .77122 & .77129 & .77137 & .77144 & .77151 \\
\rowcolor{bg} \textbf{5.91}\markboth{5.91}{5.91} & .77159 & .77166 & .77173 & .77181 & .77188 & .77195 & .77203 & .77210 & .77217 & .77225 \\
 \textbf{5.92}\markboth{5.92}{5.92} & .77232 & .77240 & .77247 & .77254 & .77262 & .77269 & .77276 & .77283 & .77291 & .77298 \\
\rowcolor{bg} \textbf{5.93}\markboth{5.93}{5.93} & .77305 & .77313 & .77320 & .77327 & .77335 & .77342 & .77349 & .77357 & .77364 & .77371 \\
 \textbf{5.94}\markboth{5.94}{5.94} & .77379 & .77386 & .77393 & .77401 & .77408 & .77415 & .77422 & .77430 & .77437 & .77444 \\
\rowcolor{bg} \textbf{5.95}\markboth{5.95}{5.95} & .77452 & .77459 & .77466 & .77474 & .77481 & .77488 & .77495 & .77503 & .77510 & .77517 \\
 \textbf{5.96}\markboth{5.96}{5.96} & .77525 & .77532 & .77539 & .77546 & .77554 & .77561 & .77568 & .77576 & .77583 & .77590 \\
\rowcolor{bg} \textbf{5.97}\markboth{5.97}{5.97} & .77597 & .77605 & .77612 & .77619 & .77627 & .77634 & .77641 & .77648 & .77656 & .77663 \\
 \textbf{5.98}\markboth{5.98}{5.98} & .77670 & .77677 & .77685 & .77692 & .77699 & .77706 & .77714 & .77721 & .77728 & .77735 \\
\rowcolor{bg} \textbf{5.99}\markboth{5.99}{5.99} & .77743 & .77750 & .77757 & .77764 & .77772 & .77779 & .77786 & .77793 & .77801 & .77808 \\
 \textcolor{blue}{\textbf{6.00}}\markboth{6.00}{6.00} & .77815 & .77822 & .77830 & .77837 & .77844 & .77851 & .77859 & .77866 & .77873 & .77880 \\
\rowcolor{bg} \textbf{6.01}\markboth{6.01}{6.01} & .77887 & .77895 & .77902 & .77909 & .77916 & .77924 & .77931 & .77938 & .77945 & .77952 \\
 \textbf{6.02}\markboth{6.02}{6.02} & .77960 & .77967 & .77974 & .77981 & .77988 & .77996 & .78003 & .78010 & .78017 & .78025 \\
\rowcolor{bg} \textbf{6.03}\markboth{6.03}{6.03} & .78032 & .78039 & .78046 & .78053 & .78061 & .78068 & .78075 & .78082 & .78089 & .78097 \\
 \textbf{6.04}\markboth{6.04}{6.04} & .78104 & .78111 & .78118 & .78125 & .78132 & .78140 & .78147 & .78154 & .78161 & .78168 \\
\rowcolor{bg} \textbf{6.05}\markboth{6.05}{6.05} & .78176 & .78183 & .78190 & .78197 & .78204 & .78211 & .78219 & .78226 & .78233 & .78240 \\
 \textbf{6.06}\markboth{6.06}{6.06} & .78247 & .78254 & .78262 & .78269 & .78276 & .78283 & .78290 & .78297 & .78305 & .78312 \\
\rowcolor{bg} \textbf{6.07}\markboth{6.07}{6.07} & .78319 & .78326 & .78333 & .78340 & .78347 & .78355 & .78362 & .78369 & .78376 & .78383 \\
 \textbf{6.08}\markboth{6.08}{6.08} & .78390 & .78398 & .78405 & .78412 & .78419 & .78426 & .78433 & .78440 & .78447 & .78455 \\
\rowcolor{bg} \textbf{6.09}\markboth{6.09}{6.09} & .78462 & .78469 & .78476 & .78483 & .78490 & .78497 & .78504 & .78512 & .78519 & .78526 \\
 \textbf{6.10}\markboth{6.10}{6.10} & .78533 & .78540 & .78547 & .78554 & .78561 & .78569 & .78576 & .78583 & .78590 & .78597 \\
\rowcolor{bg} \textbf{6.11}\markboth{6.11}{6.11} & .78604 & .78611 & .78618 & .78625 & .78633 & .78640 & .78647 & .78654 & .78661 & .78668 \\
 \textbf{6.12}\markboth{6.12}{6.12} & .78675 & .78682 & .78689 & .78696 & .78704 & .78711 & .78718 & .78725 & .78732 & .78739 \\
\rowcolor{bg} \textbf{6.13}\markboth{6.13}{6.13} & .78746 & .78753 & .78760 & .78767 & .78774 & .78781 & .78789 & .78796 & .78803 & .78810 \\
 \textbf{6.14}\markboth{6.14}{6.14} & .78817 & .78824 & .78831 & .78838 & .78845 & .78852 & .78859 & .78866 & .78873 & .78880 \\
\rowcolor{bg} \textbf{6.15}\markboth{6.15}{6.15} & .78888 & .78895 & .78902 & .78909 & .78916 & .78923 & .78930 & .78937 & .78944 & .78951 \\
 \textbf{6.16}\markboth{6.16}{6.16} & .78958 & .78965 & .78972 & .78979 & .78986 & .78993 & .79000 & .79007 & .79014 & .79021 \\
\rowcolor{bg} \textbf{6.17}\markboth{6.17}{6.17} & .79029 & .79036 & .79043 & .79050 & .79057 & .79064 & .79071 & .79078 & .79085 & .79092 \\
 \textbf{6.18}\markboth{6.18}{6.18} & .79099 & .79106 & .79113 & .79120 & .79127 & .79134 & .79141 & .79148 & .79155 & .79162 \\
\rowcolor{bg} \textbf{6.19}\markboth{6.19}{6.19} & .79169 & .79176 & .79183 & .79190 & .79197 & .79204 & .79211 & .79218 & .79225 & .79232 \\
 \textbf{6.20}\markboth{6.20}{6.20} & .79239 & .79246 & .79253 & .79260 & .79267 & .79274 & .79281 & .79288 & .79295 & .79302 \\
\rowcolor{bg} \textbf{6.21}\markboth{6.21}{6.21} & .79309 & .79316 & .79323 & .79330 & .79337 & .79344 & .79351 & .79358 & .79365 & .79372 \\
 \textbf{6.22}\markboth{6.22}{6.22} & .79379 & .79386 & .79393 & .79400 & .79407 & .79414 & .79421 & .79428 & .79435 & .79442 \\
\rowcolor{bg} \textbf{6.23}\markboth{6.23}{6.23} & .79449 & .79456 & .79463 & .79470 & .79477 & .79484 & .79491 & .79498 & .79505 & .79511 \\
 \textbf{6.24}\markboth{6.24}{6.24} & .79518 & .79525 & .79532 & .79539 & .79546 & .79553 & .79560 & .79567 & .79574 & .79581 \\
\rowcolor{bg} \textbf{6.25}\markboth{6.25}{6.25} & .79588 & .79595 & .79602 & .79609 & .79616 & .79623 & .79630 & .79637 & .79644 & .79650 \\
 \textbf{6.26}\markboth{6.26}{6.26} & .79657 & .79664 & .79671 & .79678 & .79685 & .79692 & .79699 & .79706 & .79713 & .79720 \\
\rowcolor{bg} \textbf{6.27}\markboth{6.27}{6.27} & .79727 & .79734 & .79741 & .79748 & .79754 & .79761 & .79768 & .79775 & .79782 & .79789 \\
 \textbf{6.28}\markboth{6.28}{6.28} & .79796 & .79803 & .79810 & .79817 & .79824 & .79831 & .79837 & .79844 & .79851 & .79858 \\
\rowcolor{bg} \textbf{6.29}\markboth{6.29}{6.29} & .79865 & .79872 & .79879 & .79886 & .79893 & .79900 & .79906 & .79913 & .79920 & .79927 \\
 \textbf{6.30}\markboth{6.30}{6.30} & .79934 & .79941 & .79948 & .79955 & .79962 & .79969 & .79975 & .79982 & .79989 & .79996 \\
\rowcolor{bg} \textbf{6.31}\markboth{6.31}{6.31} & .80003 & .80010 & .80017 & .80024 & .80030 & .80037 & .80044 & .80051 & .80058 & .80065 \\
 \textbf{6.32}\markboth{6.32}{6.32} & .80072 & .80079 & .80085 & .80092 & .80099 & .80106 & .80113 & .80120 & .80127 & .80134 \\
\rowcolor{bg} \textbf{6.33}\markboth{6.33}{6.33} & .80140 & .80147 & .80154 & .80161 & .80168 & .80175 & .80182 & .80188 & .80195 & .80202 \\
 \textbf{6.34}\markboth{6.34}{6.34} & .80209 & .80216 & .80223 & .80229 & .80236 & .80243 & .80250 & .80257 & .80264 & .80271 \\
\rowcolor{bg} \textbf{6.35}\markboth{6.35}{6.35} & .80277 & .80284 & .80291 & .80298 & .80305 & .80312 & .80318 & .80325 & .80332 & .80339 \\
 \textbf{6.36}\markboth{6.36}{6.36} & .80346 & .80353 & .80359 & .80366 & .80373 & .80380 & .80387 & .80393 & .80400 & .80407 \\
\rowcolor{bg} \textbf{6.37}\markboth{6.37}{6.37} & .80414 & .80421 & .80428 & .80434 & .80441 & .80448 & .80455 & .80462 & .80468 & .80475 \\
 \textbf{6.38}\markboth{6.38}{6.38} & .80482 & .80489 & .80496 & .80502 & .80509 & .80516 & .80523 & .80530 & .80536 & .80543 \\
\rowcolor{bg} \textbf{6.39}\markboth{6.39}{6.39} & .80550 & .80557 & .80564 & .80570 & .80577 & .80584 & .80591 & .80598 & .80604 & .80611 \\
 \textbf{6.40}\markboth{6.40}{6.40} & .80618 & .80625 & .80632 & .80638 & .80645 & .80652 & .80659 & .80665 & .80672 & .80679 \\
\rowcolor{bg} \textbf{6.41}\markboth{6.41}{6.41} & .80686 & .80693 & .80699 & .80706 & .80713 & .80720 & .80726 & .80733 & .80740 & .80747 \\
 \textbf{6.42}\markboth{6.42}{6.42} & .80754 & .80760 & .80767 & .80774 & .80781 & .80787 & .80794 & .80801 & .80808 & .80814 \\
\rowcolor{bg} \textbf{6.43}\markboth{6.43}{6.43} & .80821 & .80828 & .80835 & .80841 & .80848 & .80855 & .80862 & .80868 & .80875 & .80882 \\
 \textbf{6.44}\markboth{6.44}{6.44} & .80889 & .80895 & .80902 & .80909 & .80916 & .80922 & .80929 & .80936 & .80943 & .80949 \\
\rowcolor{bg} \textbf{6.45}\markboth{6.45}{6.45} & .80956 & .80963 & .80969 & .80976 & .80983 & .80990 & .80996 & .81003 & .81010 & .81017 \\
 \textbf{6.46}\markboth{6.46}{6.46} & .81023 & .81030 & .81037 & .81043 & .81050 & .81057 & .81064 & .81070 & .81077 & .81084 \\
\rowcolor{bg} \textbf{6.47}\markboth{6.47}{6.47} & .81090 & .81097 & .81104 & .81111 & .81117 & .81124 & .81131 & .81137 & .81144 & .81151 \\
 \textbf{6.48}\markboth{6.48}{6.48} & .81158 & .81164 & .81171 & .81178 & .81184 & .81191 & .81198 & .81204 & .81211 & .81218 \\
\rowcolor{bg} \textbf{6.49}\markboth{6.49}{6.49} & .81224 & .81231 & .81238 & .81245 & .81251 & .81258 & .81265 & .81271 & .81278 & .81285 \\
 \textbf{6.50}\markboth{6.50}{6.50} & .81291 & .81298 & .81305 & .81311 & .81318 & .81325 & .81331 & .81338 & .81345 & .81351 \\
\rowcolor{bg} \textbf{6.51}\markboth{6.51}{6.51} & .81358 & .81365 & .81371 & .81378 & .81385 & .81391 & .81398 & .81405 & .81411 & .81418 \\
 \textbf{6.52}\markboth{6.52}{6.52} & .81425 & .81431 & .81438 & .81445 & .81451 & .81458 & .81465 & .81471 & .81478 & .81485 \\
\rowcolor{bg} \textbf{6.53}\markboth{6.53}{6.53} & .81491 & .81498 & .81505 & .81511 & .81518 & .81525 & .81531 & .81538 & .81544 & .81551 \\
 \textbf{6.54}\markboth{6.54}{6.54} & .81558 & .81564 & .81571 & .81578 & .81584 & .81591 & .81598 & .81604 & .81611 & .81617 \\
\rowcolor{bg} \textbf{6.55}\markboth{6.55}{6.55} & .81624 & .81631 & .81637 & .81644 & .81651 & .81657 & .81664 & .81671 & .81677 & .81684 \\
 \textbf{6.56}\markboth{6.56}{6.56} & .81690 & .81697 & .81704 & .81710 & .81717 & .81723 & .81730 & .81737 & .81743 & .81750 \\
\rowcolor{bg} \textbf{6.57}\markboth{6.57}{6.57} & .81757 & .81763 & .81770 & .81776 & .81783 & .81790 & .81796 & .81803 & .81809 & .81816 \\
 \textbf{6.58}\markboth{6.58}{6.58} & .81823 & .81829 & .81836 & .81842 & .81849 & .81856 & .81862 & .81869 & .81875 & .81882 \\
\rowcolor{bg} \textbf{6.59}\markboth{6.59}{6.59} & .81889 & .81895 & .81902 & .81908 & .81915 & .81921 & .81928 & .81935 & .81941 & .81948 \\
 \textbf{6.60}\markboth{6.60}{6.60} & .81954 & .81961 & .81968 & .81974 & .81981 & .81987 & .81994 & .82000 & .82007 & .82014 \\
\rowcolor{bg} \textbf{6.61}\markboth{6.61}{6.61} & .82020 & .82027 & .82033 & .82040 & .82046 & .82053 & .82060 & .82066 & .82073 & .82079 \\
 \textbf{6.62}\markboth{6.62}{6.62} & .82086 & .82092 & .82099 & .82105 & .82112 & .82119 & .82125 & .82132 & .82138 & .82145 \\
\rowcolor{bg} \textbf{6.63}\markboth{6.63}{6.63} & .82151 & .82158 & .82164 & .82171 & .82178 & .82184 & .82191 & .82197 & .82204 & .82210 \\
 \textbf{6.64}\markboth{6.64}{6.64} & .82217 & .82223 & .82230 & .82236 & .82243 & .82249 & .82256 & .82263 & .82269 & .82276 \\
\rowcolor{bg} \textbf{6.65}\markboth{6.65}{6.65} & .82282 & .82289 & .82295 & .82302 & .82308 & .82315 & .82321 & .82328 & .82334 & .82341 \\
 \textbf{6.66}\markboth{6.66}{6.66} & .82347 & .82354 & .82360 & .82367 & .82373 & .82380 & .82387 & .82393 & .82400 & .82406 \\
\rowcolor{bg} \textbf{6.67}\markboth{6.67}{6.67} & .82413 & .82419 & .82426 & .82432 & .82439 & .82445 & .82452 & .82458 & .82465 & .82471 \\
 \textbf{6.68}\markboth{6.68}{6.68} & .82478 & .82484 & .82491 & .82497 & .82504 & .82510 & .82517 & .82523 & .82530 & .82536 \\
\rowcolor{bg} \textbf{6.69}\markboth{6.69}{6.69} & .82543 & .82549 & .82556 & .82562 & .82569 & .82575 & .82582 & .82588 & .82595 & .82601 \\
 \textbf{6.70}\markboth{6.70}{6.70} & .82607 & .82614 & .82620 & .82627 & .82633 & .82640 & .82646 & .82653 & .82659 & .82666 \\
\rowcolor{bg} \textbf{6.71}\markboth{6.71}{6.71} & .82672 & .82679 & .82685 & .82692 & .82698 & .82705 & .82711 & .82718 & .82724 & .82730 \\
 \textbf{6.72}\markboth{6.72}{6.72} & .82737 & .82743 & .82750 & .82756 & .82763 & .82769 & .82776 & .82782 & .82789 & .82795 \\
\rowcolor{bg} \textbf{6.73}\markboth{6.73}{6.73} & .82802 & .82808 & .82814 & .82821 & .82827 & .82834 & .82840 & .82847 & .82853 & .82860 \\
 \textbf{6.74}\markboth{6.74}{6.74} & .82866 & .82872 & .82879 & .82885 & .82892 & .82898 & .82905 & .82911 & .82918 & .82924 \\
\rowcolor{bg} \textbf{6.75}\markboth{6.75}{6.75} & .82930 & .82937 & .82943 & .82950 & .82956 & .82963 & .82969 & .82975 & .82982 & .82988 \\
 \textbf{6.76}\markboth{6.76}{6.76} & .82995 & .83001 & .83008 & .83014 & .83020 & .83027 & .83033 & .83040 & .83046 & .83052 \\
\rowcolor{bg} \textbf{6.77}\markboth{6.77}{6.77} & .83059 & .83065 & .83072 & .83078 & .83085 & .83091 & .83097 & .83104 & .83110 & .83117 \\
 \textbf{6.78}\markboth{6.78}{6.78} & .83123 & .83129 & .83136 & .83142 & .83149 & .83155 & .83161 & .83168 & .83174 & .83181 \\
\rowcolor{bg} \textbf{6.79}\markboth{6.79}{6.79} & .83187 & .83193 & .83200 & .83206 & .83213 & .83219 & .83225 & .83232 & .83238 & .83245 \\
 \textbf{6.80}\markboth{6.80}{6.80} & .83251 & .83257 & .83264 & .83270 & .83276 & .83283 & .83289 & .83296 & .83302 & .83308 \\
\rowcolor{bg} \textbf{6.81}\markboth{6.81}{6.81} & .83315 & .83321 & .83327 & .83334 & .83340 & .83347 & .83353 & .83359 & .83366 & .83372 \\
 \textbf{6.82}\markboth{6.82}{6.82} & .83378 & .83385 & .83391 & .83398 & .83404 & .83410 & .83417 & .83423 & .83429 & .83436 \\
\rowcolor{bg} \textbf{6.83}\markboth{6.83}{6.83} & .83442 & .83448 & .83455 & .83461 & .83467 & .83474 & .83480 & .83487 & .83493 & .83499 \\
 \textbf{6.84}\markboth{6.84}{6.84} & .83506 & .83512 & .83518 & .83525 & .83531 & .83537 & .83544 & .83550 & .83556 & .83563 \\
\rowcolor{bg} \textbf{6.85}\markboth{6.85}{6.85} & .83569 & .83575 & .83582 & .83588 & .83594 & .83601 & .83607 & .83613 & .83620 & .83626 \\
 \textbf{6.86}\markboth{6.86}{6.86} & .83632 & .83639 & .83645 & .83651 & .83658 & .83664 & .83670 & .83677 & .83683 & .83689 \\
\rowcolor{bg} \textbf{6.87}\markboth{6.87}{6.87} & .83696 & .83702 & .83708 & .83715 & .83721 & .83727 & .83734 & .83740 & .83746 & .83753 \\
 \textbf{6.88}\markboth{6.88}{6.88} & .83759 & .83765 & .83771 & .83778 & .83784 & .83790 & .83797 & .83803 & .83809 & .83816 \\
\rowcolor{bg} \textbf{6.89}\markboth{6.89}{6.89} & .83822 & .83828 & .83835 & .83841 & .83847 & .83853 & .83860 & .83866 & .83872 & .83879 \\
 \textbf{6.90}\markboth{6.90}{6.90} & .83885 & .83891 & .83897 & .83904 & .83910 & .83916 & .83923 & .83929 & .83935 & .83942 \\
\rowcolor{bg} \textbf{6.91}\markboth{6.91}{6.91} & .83948 & .83954 & .83960 & .83967 & .83973 & .83979 & .83985 & .83992 & .83998 & .84004 \\
 \textbf{6.92}\markboth{6.92}{6.92} & .84011 & .84017 & .84023 & .84029 & .84036 & .84042 & .84048 & .84055 & .84061 & .84067 \\
\rowcolor{bg} \textbf{6.93}\markboth{6.93}{6.93} & .84073 & .84080 & .84086 & .84092 & .84098 & .84105 & .84111 & .84117 & .84123 & .84130 \\
 \textbf{6.94}\markboth{6.94}{6.94} & .84136 & .84142 & .84148 & .84155 & .84161 & .84167 & .84173 & .84180 & .84186 & .84192 \\
\rowcolor{bg} \textbf{6.95}\markboth{6.95}{6.95} & .84198 & .84205 & .84211 & .84217 & .84223 & .84230 & .84236 & .84242 & .84248 & .84255 \\
 \textbf{6.96}\markboth{6.96}{6.96} & .84261 & .84267 & .84273 & .84280 & .84286 & .84292 & .84298 & .84305 & .84311 & .84317 \\
\rowcolor{bg} \textbf{6.97}\markboth{6.97}{6.97} & .84323 & .84330 & .84336 & .84342 & .84348 & .84354 & .84361 & .84367 & .84373 & .84379 \\
 \textbf{6.98}\markboth{6.98}{6.98} & .84386 & .84392 & .84398 & .84404 & .84410 & .84417 & .84423 & .84429 & .84435 & .84442 \\
\rowcolor{bg} \textbf{6.99}\markboth{6.99}{6.99} & .84448 & .84454 & .84460 & .84466 & .84473 & .84479 & .84485 & .84491 & .84497 & .84504 \\
 \textcolor{blue}{\textbf{7.00}}\markboth{7.00}{7.00} & .84510 & .84516 & .84522 & .84528 & .84535 & .84541 & .84547 & .84553 & .84559 & .84566 \\
\rowcolor{bg} \textbf{7.01}\markboth{7.01}{7.01} & .84572 & .84578 & .84584 & .84590 & .84597 & .84603 & .84609 & .84615 & .84621 & .84628 \\
 \textbf{7.02}\markboth{7.02}{7.02} & .84634 & .84640 & .84646 & .84652 & .84658 & .84665 & .84671 & .84677 & .84683 & .84689 \\
\rowcolor{bg} \textbf{7.03}\markboth{7.03}{7.03} & .84696 & .84702 & .84708 & .84714 & .84720 & .84726 & .84733 & .84739 & .84745 & .84751 \\
 \textbf{7.04}\markboth{7.04}{7.04} & .84757 & .84763 & .84770 & .84776 & .84782 & .84788 & .84794 & .84800 & .84807 & .84813 \\
\rowcolor{bg} \textbf{7.05}\markboth{7.05}{7.05} & .84819 & .84825 & .84831 & .84837 & .84844 & .84850 & .84856 & .84862 & .84868 & .84874 \\
 \textbf{7.06}\markboth{7.06}{7.06} & .84880 & .84887 & .84893 & .84899 & .84905 & .84911 & .84917 & .84924 & .84930 & .84936 \\
\rowcolor{bg} \textbf{7.07}\markboth{7.07}{7.07} & .84942 & .84948 & .84954 & .84960 & .84967 & .84973 & .84979 & .84985 & .84991 & .84997 \\
 \textbf{7.08}\markboth{7.08}{7.08} & .85003 & .85009 & .85016 & .85022 & .85028 & .85034 & .85040 & .85046 & .85052 & .85058 \\
\rowcolor{bg} \textbf{7.09}\markboth{7.09}{7.09} & .85065 & .85071 & .85077 & .85083 & .85089 & .85095 & .85101 & .85107 & .85114 & .85120 \\
 \textbf{7.10}\markboth{7.10}{7.10} & .85126 & .85132 & .85138 & .85144 & .85150 & .85156 & .85163 & .85169 & .85175 & .85181 \\
\rowcolor{bg} \textbf{7.11}\markboth{7.11}{7.11} & .85187 & .85193 & .85199 & .85205 & .85211 & .85217 & .85224 & .85230 & .85236 & .85242 \\
 \textbf{7.12}\markboth{7.12}{7.12} & .85248 & .85254 & .85260 & .85266 & .85272 & .85278 & .85285 & .85291 & .85297 & .85303 \\
\rowcolor{bg} \textbf{7.13}\markboth{7.13}{7.13} & .85309 & .85315 & .85321 & .85327 & .85333 & .85339 & .85345 & .85352 & .85358 & .85364 \\
 \textbf{7.14}\markboth{7.14}{7.14} & .85370 & .85376 & .85382 & .85388 & .85394 & .85400 & .85406 & .85412 & .85418 & .85425 \\
\rowcolor{bg} \textbf{7.15}\markboth{7.15}{7.15} & .85431 & .85437 & .85443 & .85449 & .85455 & .85461 & .85467 & .85473 & .85479 & .85485 \\
 \textbf{7.16}\markboth{7.16}{7.16} & .85491 & .85497 & .85503 & .85509 & .85516 & .85522 & .85528 & .85534 & .85540 & .85546 \\
\rowcolor{bg} \textbf{7.17}\markboth{7.17}{7.17} & .85552 & .85558 & .85564 & .85570 & .85576 & .85582 & .85588 & .85594 & .85600 & .85606 \\
 \textbf{7.18}\markboth{7.18}{7.18} & .85612 & .85618 & .85625 & .85631 & .85637 & .85643 & .85649 & .85655 & .85661 & .85667 \\
\rowcolor{bg} \textbf{7.19}\markboth{7.19}{7.19} & .85673 & .85679 & .85685 & .85691 & .85697 & .85703 & .85709 & .85715 & .85721 & .85727 \\
 \textbf{7.20}\markboth{7.20}{7.20} & .85733 & .85739 & .85745 & .85751 & .85757 & .85763 & .85769 & .85775 & .85781 & .85788 \\
\rowcolor{bg} \textbf{7.21}\markboth{7.21}{7.21} & .85794 & .85800 & .85806 & .85812 & .85818 & .85824 & .85830 & .85836 & .85842 & .85848 \\
 \textbf{7.22}\markboth{7.22}{7.22} & .85854 & .85860 & .85866 & .85872 & .85878 & .85884 & .85890 & .85896 & .85902 & .85908 \\
\rowcolor{bg} \textbf{7.23}\markboth{7.23}{7.23} & .85914 & .85920 & .85926 & .85932 & .85938 & .85944 & .85950 & .85956 & .85962 & .85968 \\
 \textbf{7.24}\markboth{7.24}{7.24} & .85974 & .85980 & .85986 & .85992 & .85998 & .86004 & .86010 & .86016 & .86022 & .86028 \\
\rowcolor{bg} \textbf{7.25}\markboth{7.25}{7.25} & .86034 & .86040 & .86046 & .86052 & .86058 & .86064 & .86070 & .86076 & .86082 & .86088 \\
 \textbf{7.26}\markboth{7.26}{7.26} & .86094 & .86100 & .86106 & .86112 & .86118 & .86124 & .86130 & .86136 & .86141 & .86147 \\
\rowcolor{bg} \textbf{7.27}\markboth{7.27}{7.27} & .86153 & .86159 & .86165 & .86171 & .86177 & .86183 & .86189 & .86195 & .86201 & .86207 \\
 \textbf{7.28}\markboth{7.28}{7.28} & .86213 & .86219 & .86225 & .86231 & .86237 & .86243 & .86249 & .86255 & .86261 & .86267 \\
\rowcolor{bg} \textbf{7.29}\markboth{7.29}{7.29} & .86273 & .86279 & .86285 & .86291 & .86297 & .86303 & .86308 & .86314 & .86320 & .86326 \\
 \textbf{7.30}\markboth{7.30}{7.30} & .86332 & .86338 & .86344 & .86350 & .86356 & .86362 & .86368 & .86374 & .86380 & .86386 \\
\rowcolor{bg} \textbf{7.31}\markboth{7.31}{7.31} & .86392 & .86398 & .86404 & .86410 & .86415 & .86421 & .86427 & .86433 & .86439 & .86445 \\
 \textbf{7.32}\markboth{7.32}{7.32} & .86451 & .86457 & .86463 & .86469 & .86475 & .86481 & .86487 & .86493 & .86499 & .86504 \\
\rowcolor{bg} \textbf{7.33}\markboth{7.33}{7.33} & .86510 & .86516 & .86522 & .86528 & .86534 & .86540 & .86546 & .86552 & .86558 & .86564 \\
 \textbf{7.34}\markboth{7.34}{7.34} & .86570 & .86576 & .86581 & .86587 & .86593 & .86599 & .86605 & .86611 & .86617 & .86623 \\
\rowcolor{bg} \textbf{7.35}\markboth{7.35}{7.35} & .86629 & .86635 & .86641 & .86646 & .86652 & .86658 & .86664 & .86670 & .86676 & .86682 \\
 \textbf{7.36}\markboth{7.36}{7.36} & .86688 & .86694 & .86700 & .86705 & .86711 & .86717 & .86723 & .86729 & .86735 & .86741 \\
\rowcolor{bg} \textbf{7.37}\markboth{7.37}{7.37} & .86747 & .86753 & .86759 & .86764 & .86770 & .86776 & .86782 & .86788 & .86794 & .86800 \\
 \textbf{7.38}\markboth{7.38}{7.38} & .86806 & .86812 & .86817 & .86823 & .86829 & .86835 & .86841 & .86847 & .86853 & .86859 \\
\rowcolor{bg} \textbf{7.39}\markboth{7.39}{7.39} & .86864 & .86870 & .86876 & .86882 & .86888 & .86894 & .86900 & .86906 & .86911 & .86917 \\
 \textbf{7.40}\markboth{7.40}{7.40} & .86923 & .86929 & .86935 & .86941 & .86947 & .86953 & .86958 & .86964 & .86970 & .86976 \\
\rowcolor{bg} \textbf{7.41}\markboth{7.41}{7.41} & .86982 & .86988 & .86994 & .86999 & .87005 & .87011 & .87017 & .87023 & .87029 & .87035 \\
 \textbf{7.42}\markboth{7.42}{7.42} & .87040 & .87046 & .87052 & .87058 & .87064 & .87070 & .87075 & .87081 & .87087 & .87093 \\
\rowcolor{bg} \textbf{7.43}\markboth{7.43}{7.43} & .87099 & .87105 & .87111 & .87116 & .87122 & .87128 & .87134 & .87140 & .87146 & .87151 \\
 \textbf{7.44}\markboth{7.44}{7.44} & .87157 & .87163 & .87169 & .87175 & .87181 & .87186 & .87192 & .87198 & .87204 & .87210 \\
\rowcolor{bg} \textbf{7.45}\markboth{7.45}{7.45} & .87216 & .87221 & .87227 & .87233 & .87239 & .87245 & .87251 & .87256 & .87262 & .87268 \\
 \textbf{7.46}\markboth{7.46}{7.46} & .87274 & .87280 & .87286 & .87291 & .87297 & .87303 & .87309 & .87315 & .87320 & .87326 \\
\rowcolor{bg} \textbf{7.47}\markboth{7.47}{7.47} & .87332 & .87338 & .87344 & .87349 & .87355 & .87361 & .87367 & .87373 & .87379 & .87384 \\
 \textbf{7.48}\markboth{7.48}{7.48} & .87390 & .87396 & .87402 & .87408 & .87413 & .87419 & .87425 & .87431 & .87437 & .87442 \\
\rowcolor{bg} \textbf{7.49}\markboth{7.49}{7.49} & .87448 & .87454 & .87460 & .87466 & .87471 & .87477 & .87483 & .87489 & .87495 & .87500 \\
 \textbf{7.50}\markboth{7.50}{7.50} & .87506 & .87512 & .87518 & .87523 & .87529 & .87535 & .87541 & .87547 & .87552 & .87558 \\
\rowcolor{bg} \textbf{7.51}\markboth{7.51}{7.51} & .87564 & .87570 & .87576 & .87581 & .87587 & .87593 & .87599 & .87604 & .87610 & .87616 \\
 \textbf{7.52}\markboth{7.52}{7.52} & .87622 & .87628 & .87633 & .87639 & .87645 & .87651 & .87656 & .87662 & .87668 & .87674 \\
\rowcolor{bg} \textbf{7.53}\markboth{7.53}{7.53} & .87679 & .87685 & .87691 & .87697 & .87703 & .87708 & .87714 & .87720 & .87726 & .87731 \\
 \textbf{7.54}\markboth{7.54}{7.54} & .87737 & .87743 & .87749 & .87754 & .87760 & .87766 & .87772 & .87777 & .87783 & .87789 \\
\rowcolor{bg} \textbf{7.55}\markboth{7.55}{7.55} & .87795 & .87800 & .87806 & .87812 & .87818 & .87823 & .87829 & .87835 & .87841 & .87846 \\
 \textbf{7.56}\markboth{7.56}{7.56} & .87852 & .87858 & .87864 & .87869 & .87875 & .87881 & .87887 & .87892 & .87898 & .87904 \\
\rowcolor{bg} \textbf{7.57}\markboth{7.57}{7.57} & .87910 & .87915 & .87921 & .87927 & .87933 & .87938 & .87944 & .87950 & .87955 & .87961 \\
 \textbf{7.58}\markboth{7.58}{7.58} & .87967 & .87973 & .87978 & .87984 & .87990 & .87996 & .88001 & .88007 & .88013 & .88018 \\
\rowcolor{bg} \textbf{7.59}\markboth{7.59}{7.59} & .88024 & .88030 & .88036 & .88041 & .88047 & .88053 & .88058 & .88064 & .88070 & .88076 \\
 \textbf{7.60}\markboth{7.60}{7.60} & .88081 & .88087 & .88093 & .88098 & .88104 & .88110 & .88116 & .88121 & .88127 & .88133 \\
\rowcolor{bg} \textbf{7.61}\markboth{7.61}{7.61} & .88138 & .88144 & .88150 & .88156 & .88161 & .88167 & .88173 & .88178 & .88184 & .88190 \\
 \textbf{7.62}\markboth{7.62}{7.62} & .88195 & .88201 & .88207 & .88213 & .88218 & .88224 & .88230 & .88235 & .88241 & .88247 \\
\rowcolor{bg} \textbf{7.63}\markboth{7.63}{7.63} & .88252 & .88258 & .88264 & .88270 & .88275 & .88281 & .88287 & .88292 & .88298 & .88304 \\
 \textbf{7.64}\markboth{7.64}{7.64} & .88309 & .88315 & .88321 & .88326 & .88332 & .88338 & .88343 & .88349 & .88355 & .88360 \\
\rowcolor{bg} \textbf{7.65}\markboth{7.65}{7.65} & .88366 & .88372 & .88377 & .88383 & .88389 & .88395 & .88400 & .88406 & .88412 & .88417 \\
 \textbf{7.66}\markboth{7.66}{7.66} & .88423 & .88429 & .88434 & .88440 & .88446 & .88451 & .88457 & .88463 & .88468 & .88474 \\
\rowcolor{bg} \textbf{7.67}\markboth{7.67}{7.67} & .88480 & .88485 & .88491 & .88497 & .88502 & .88508 & .88513 & .88519 & .88525 & .88530 \\
 \textbf{7.68}\markboth{7.68}{7.68} & .88536 & .88542 & .88547 & .88553 & .88559 & .88564 & .88570 & .88576 & .88581 & .88587 \\
\rowcolor{bg} \textbf{7.69}\markboth{7.69}{7.69} & .88593 & .88598 & .88604 & .88610 & .88615 & .88621 & .88627 & .88632 & .88638 & .88643 \\
 \textbf{7.70}\markboth{7.70}{7.70} & .88649 & .88655 & .88660 & .88666 & .88672 & .88677 & .88683 & .88689 & .88694 & .88700 \\
\rowcolor{bg} \textbf{7.71}\markboth{7.71}{7.71} & .88705 & .88711 & .88717 & .88722 & .88728 & .88734 & .88739 & .88745 & .88750 & .88756 \\
 \textbf{7.72}\markboth{7.72}{7.72} & .88762 & .88767 & .88773 & .88779 & .88784 & .88790 & .88795 & .88801 & .88807 & .88812 \\
\rowcolor{bg} \textbf{7.73}\markboth{7.73}{7.73} & .88818 & .88824 & .88829 & .88835 & .88840 & .88846 & .88852 & .88857 & .88863 & .88868 \\
 \textbf{7.74}\markboth{7.74}{7.74} & .88874 & .88880 & .88885 & .88891 & .88897 & .88902 & .88908 & .88913 & .88919 & .88925 \\
\rowcolor{bg} \textbf{7.75}\markboth{7.75}{7.75} & .88930 & .88936 & .88941 & .88947 & .88953 & .88958 & .88964 & .88969 & .88975 & .88981 \\
 \textbf{7.76}\markboth{7.76}{7.76} & .88986 & .88992 & .88997 & .89003 & .89009 & .89014 & .89020 & .89025 & .89031 & .89037 \\
\rowcolor{bg} \textbf{7.77}\markboth{7.77}{7.77} & .89042 & .89048 & .89053 & .89059 & .89064 & .89070 & .89076 & .89081 & .89087 & .89092 \\
 \textbf{7.78}\markboth{7.78}{7.78} & .89098 & .89104 & .89109 & .89115 & .89120 & .89126 & .89131 & .89137 & .89143 & .89148 \\
\rowcolor{bg} \textbf{7.79}\markboth{7.79}{7.79} & .89154 & .89159 & .89165 & .89170 & .89176 & .89182 & .89187 & .89193 & .89198 & .89204 \\
 \textbf{7.80}\markboth{7.80}{7.80} & .89209 & .89215 & .89221 & .89226 & .89232 & .89237 & .89243 & .89248 & .89254 & .89260 \\
\rowcolor{bg} \textbf{7.81}\markboth{7.81}{7.81} & .89265 & .89271 & .89276 & .89282 & .89287 & .89293 & .89298 & .89304 & .89310 & .89315 \\
 \textbf{7.82}\markboth{7.82}{7.82} & .89321 & .89326 & .89332 & .89337 & .89343 & .89348 & .89354 & .89360 & .89365 & .89371 \\
\rowcolor{bg} \textbf{7.83}\markboth{7.83}{7.83} & .89376 & .89382 & .89387 & .89393 & .89398 & .89404 & .89409 & .89415 & .89421 & .89426 \\
 \textbf{7.84}\markboth{7.84}{7.84} & .89432 & .89437 & .89443 & .89448 & .89454 & .89459 & .89465 & .89470 & .89476 & .89481 \\
\rowcolor{bg} \textbf{7.85}\markboth{7.85}{7.85} & .89487 & .89492 & .89498 & .89504 & .89509 & .89515 & .89520 & .89526 & .89531 & .89537 \\
 \textbf{7.86}\markboth{7.86}{7.86} & .89542 & .89548 & .89553 & .89559 & .89564 & .89570 & .89575 & .89581 & .89586 & .89592 \\
\rowcolor{bg} \textbf{7.87}\markboth{7.87}{7.87} & .89597 & .89603 & .89609 & .89614 & .89620 & .89625 & .89631 & .89636 & .89642 & .89647 \\
 \textbf{7.88}\markboth{7.88}{7.88} & .89653 & .89658 & .89664 & .89669 & .89675 & .89680 & .89686 & .89691 & .89697 & .89702 \\
\rowcolor{bg} \textbf{7.89}\markboth{7.89}{7.89} & .89708 & .89713 & .89719 & .89724 & .89730 & .89735 & .89741 & .89746 & .89752 & .89757 \\
 \textbf{7.90}\markboth{7.90}{7.90} & .89763 & .89768 & .89774 & .89779 & .89785 & .89790 & .89796 & .89801 & .89807 & .89812 \\
\rowcolor{bg} \textbf{7.91}\markboth{7.91}{7.91} & .89818 & .89823 & .89829 & .89834 & .89840 & .89845 & .89851 & .89856 & .89862 & .89867 \\
 \textbf{7.92}\markboth{7.92}{7.92} & .89873 & .89878 & .89883 & .89889 & .89894 & .89900 & .89905 & .89911 & .89916 & .89922 \\
\rowcolor{bg} \textbf{7.93}\markboth{7.93}{7.93} & .89927 & .89933 & .89938 & .89944 & .89949 & .89955 & .89960 & .89966 & .89971 & .89977 \\
 \textbf{7.94}\markboth{7.94}{7.94} & .89982 & .89988 & .89993 & .89998 & .90004 & .90009 & .90015 & .90020 & .90026 & .90031 \\
\rowcolor{bg} \textbf{7.95}\markboth{7.95}{7.95} & .90037 & .90042 & .90048 & .90053 & .90059 & .90064 & .90069 & .90075 & .90080 & .90086 \\
 \textbf{7.96}\markboth{7.96}{7.96} & .90091 & .90097 & .90102 & .90108 & .90113 & .90119 & .90124 & .90129 & .90135 & .90140 \\
\rowcolor{bg} \textbf{7.97}\markboth{7.97}{7.97} & .90146 & .90151 & .90157 & .90162 & .90168 & .90173 & .90179 & .90184 & .90189 & .90195 \\
 \textbf{7.98}\markboth{7.98}{7.98} & .90200 & .90206 & .90211 & .90217 & .90222 & .90227 & .90233 & .90238 & .90244 & .90249 \\
\rowcolor{bg} \textbf{7.99}\markboth{7.99}{7.99} & .90255 & .90260 & .90266 & .90271 & .90276 & .90282 & .90287 & .90293 & .90298 & .90304 \\
 \textcolor{blue}{\textbf{8.00}}\markboth{8.00}{8.00} & .90309 & .90314 & .90320 & .90325 & .90331 & .90336 & .90342 & .90347 & .90352 & .90358 \\
\rowcolor{bg} \textbf{8.01}\markboth{8.01}{8.01} & .90363 & .90369 & .90374 & .90380 & .90385 & .90390 & .90396 & .90401 & .90407 & .90412 \\
 \textbf{8.02}\markboth{8.02}{8.02} & .90417 & .90423 & .90428 & .90434 & .90439 & .90445 & .90450 & .90455 & .90461 & .90466 \\
\rowcolor{bg} \textbf{8.03}\markboth{8.03}{8.03} & .90472 & .90477 & .90482 & .90488 & .90493 & .90499 & .90504 & .90509 & .90515 & .90520 \\
 \textbf{8.04}\markboth{8.04}{8.04} & .90526 & .90531 & .90536 & .90542 & .90547 & .90553 & .90558 & .90563 & .90569 & .90574 \\
\rowcolor{bg} \textbf{8.05}\markboth{8.05}{8.05} & .90580 & .90585 & .90590 & .90596 & .90601 & .90607 & .90612 & .90617 & .90623 & .90628 \\
 \textbf{8.06}\markboth{8.06}{8.06} & .90634 & .90639 & .90644 & .90650 & .90655 & .90660 & .90666 & .90671 & .90677 & .90682 \\
\rowcolor{bg} \textbf{8.07}\markboth{8.07}{8.07} & .90687 & .90693 & .90698 & .90703 & .90709 & .90714 & .90720 & .90725 & .90730 & .90736 \\
 \textbf{8.08}\markboth{8.08}{8.08} & .90741 & .90747 & .90752 & .90757 & .90763 & .90768 & .90773 & .90779 & .90784 & .90789 \\
\rowcolor{bg} \textbf{8.09}\markboth{8.09}{8.09} & .90795 & .90800 & .90806 & .90811 & .90816 & .90822 & .90827 & .90832 & .90838 & .90843 \\
 \textbf{8.10}\markboth{8.10}{8.10} & .90849 & .90854 & .90859 & .90865 & .90870 & .90875 & .90881 & .90886 & .90891 & .90897 \\
\rowcolor{bg} \textbf{8.11}\markboth{8.11}{8.11} & .90902 & .90907 & .90913 & .90918 & .90924 & .90929 & .90934 & .90940 & .90945 & .90950 \\
 \textbf{8.12}\markboth{8.12}{8.12} & .90956 & .90961 & .90966 & .90972 & .90977 & .90982 & .90988 & .90993 & .90998 & .91004 \\
\rowcolor{bg} \textbf{8.13}\markboth{8.13}{8.13} & .91009 & .91014 & .91020 & .91025 & .91030 & .91036 & .91041 & .91046 & .91052 & .91057 \\
 \textbf{8.14}\markboth{8.14}{8.14} & .91062 & .91068 & .91073 & .91078 & .91084 & .91089 & .91094 & .91100 & .91105 & .91110 \\
\rowcolor{bg} \textbf{8.15}\markboth{8.15}{8.15} & .91116 & .91121 & .91126 & .91132 & .91137 & .91142 & .91148 & .91153 & .91158 & .91164 \\
 \textbf{8.16}\markboth{8.16}{8.16} & .91169 & .91174 & .91180 & .91185 & .91190 & .91196 & .91201 & .91206 & .91212 & .91217 \\
\rowcolor{bg} \textbf{8.17}\markboth{8.17}{8.17} & .91222 & .91228 & .91233 & .91238 & .91243 & .91249 & .91254 & .91259 & .91265 & .91270 \\
 \textbf{8.18}\markboth{8.18}{8.18} & .91275 & .91281 & .91286 & .91291 & .91297 & .91302 & .91307 & .91312 & .91318 & .91323 \\
\rowcolor{bg} \textbf{8.19}\markboth{8.19}{8.19} & .91328 & .91334 & .91339 & .91344 & .91350 & .91355 & .91360 & .91365 & .91371 & .91376 \\
 \textbf{8.20}\markboth{8.20}{8.20} & .91381 & .91387 & .91392 & .91397 & .91403 & .91408 & .91413 & .91418 & .91424 & .91429 \\
\rowcolor{bg} \textbf{8.21}\markboth{8.21}{8.21} & .91434 & .91440 & .91445 & .91450 & .91455 & .91461 & .91466 & .91471 & .91477 & .91482 \\
 \textbf{8.22}\markboth{8.22}{8.22} & .91487 & .91492 & .91498 & .91503 & .91508 & .91514 & .91519 & .91524 & .91529 & .91535 \\
\rowcolor{bg} \textbf{8.23}\markboth{8.23}{8.23} & .91540 & .91545 & .91551 & .91556 & .91561 & .91566 & .91572 & .91577 & .91582 & .91587 \\
 \textbf{8.24}\markboth{8.24}{8.24} & .91593 & .91598 & .91603 & .91609 & .91614 & .91619 & .91624 & .91630 & .91635 & .91640 \\
\rowcolor{bg} \textbf{8.25}\markboth{8.25}{8.25} & .91645 & .91651 & .91656 & .91661 & .91666 & .91672 & .91677 & .91682 & .91687 & .91693 \\
 \textbf{8.26}\markboth{8.26}{8.26} & .91698 & .91703 & .91709 & .91714 & .91719 & .91724 & .91730 & .91735 & .91740 & .91745 \\
\rowcolor{bg} \textbf{8.27}\markboth{8.27}{8.27} & .91751 & .91756 & .91761 & .91766 & .91772 & .91777 & .91782 & .91787 & .91793 & .91798 \\
 \textbf{8.28}\markboth{8.28}{8.28} & .91803 & .91808 & .91814 & .91819 & .91824 & .91829 & .91834 & .91840 & .91845 & .91850 \\
\rowcolor{bg} \textbf{8.29}\markboth{8.29}{8.29} & .91855 & .91861 & .91866 & .91871 & .91876 & .91882 & .91887 & .91892 & .91897 & .91903 \\
 \textbf{8.30}\markboth{8.30}{8.30} & .91908 & .91913 & .91918 & .91924 & .91929 & .91934 & .91939 & .91944 & .91950 & .91955 \\
\rowcolor{bg} \textbf{8.31}\markboth{8.31}{8.31} & .91960 & .91965 & .91971 & .91976 & .91981 & .91986 & .91991 & .91997 & .92002 & .92007 \\
 \textbf{8.32}\markboth{8.32}{8.32} & .92012 & .92018 & .92023 & .92028 & .92033 & .92038 & .92044 & .92049 & .92054 & .92059 \\
\rowcolor{bg} \textbf{8.33}\markboth{8.33}{8.33} & .92065 & .92070 & .92075 & .92080 & .92085 & .92091 & .92096 & .92101 & .92106 & .92111 \\
 \textbf{8.34}\markboth{8.34}{8.34} & .92117 & .92122 & .92127 & .92132 & .92137 & .92143 & .92148 & .92153 & .92158 & .92163 \\
\rowcolor{bg} \textbf{8.35}\markboth{8.35}{8.35} & .92169 & .92174 & .92179 & .92184 & .92189 & .92195 & .92200 & .92205 & .92210 & .92215 \\
 \textbf{8.36}\markboth{8.36}{8.36} & .92221 & .92226 & .92231 & .92236 & .92241 & .92247 & .92252 & .92257 & .92262 & .92267 \\
\rowcolor{bg} \textbf{8.37}\markboth{8.37}{8.37} & .92273 & .92278 & .92283 & .92288 & .92293 & .92298 & .92304 & .92309 & .92314 & .92319 \\
 \textbf{8.38}\markboth{8.38}{8.38} & .92324 & .92330 & .92335 & .92340 & .92345 & .92350 & .92355 & .92361 & .92366 & .92371 \\
\rowcolor{bg} \textbf{8.39}\markboth{8.39}{8.39} & .92376 & .92381 & .92387 & .92392 & .92397 & .92402 & .92407 & .92412 & .92418 & .92423 \\
 \textbf{8.40}\markboth{8.40}{8.40} & .92428 & .92433 & .92438 & .92443 & .92449 & .92454 & .92459 & .92464 & .92469 & .92474 \\
\rowcolor{bg} \textbf{8.41}\markboth{8.41}{8.41} & .92480 & .92485 & .92490 & .92495 & .92500 & .92505 & .92511 & .92516 & .92521 & .92526 \\
 \textbf{8.42}\markboth{8.42}{8.42} & .92531 & .92536 & .92542 & .92547 & .92552 & .92557 & .92562 & .92567 & .92572 & .92578 \\
\rowcolor{bg} \textbf{8.43}\markboth{8.43}{8.43} & .92583 & .92588 & .92593 & .92598 & .92603 & .92609 & .92614 & .92619 & .92624 & .92629 \\
 \textbf{8.44}\markboth{8.44}{8.44} & .92634 & .92639 & .92645 & .92650 & .92655 & .92660 & .92665 & .92670 & .92675 & .92681 \\
\rowcolor{bg} \textbf{8.45}\markboth{8.45}{8.45} & .92686 & .92691 & .92696 & .92701 & .92706 & .92711 & .92716 & .92722 & .92727 & .92732 \\
 \textbf{8.46}\markboth{8.46}{8.46} & .92737 & .92742 & .92747 & .92752 & .92758 & .92763 & .92768 & .92773 & .92778 & .92783 \\
\rowcolor{bg} \textbf{8.47}\markboth{8.47}{8.47} & .92788 & .92793 & .92799 & .92804 & .92809 & .92814 & .92819 & .92824 & .92829 & .92834 \\
 \textbf{8.48}\markboth{8.48}{8.48} & .92840 & .92845 & .92850 & .92855 & .92860 & .92865 & .92870 & .92875 & .92881 & .92886 \\
\rowcolor{bg} \textbf{8.49}\markboth{8.49}{8.49} & .92891 & .92896 & .92901 & .92906 & .92911 & .92916 & .92921 & .92927 & .92932 & .92937 \\
 \textbf{8.50}\markboth{8.50}{8.50} & .92942 & .92947 & .92952 & .92957 & .92962 & .92967 & .92973 & .92978 & .92983 & .92988 \\
\rowcolor{bg} \textbf{8.51}\markboth{8.51}{8.51} & .92993 & .92998 & .93003 & .93008 & .93013 & .93018 & .93024 & .93029 & .93034 & .93039 \\
 \textbf{8.52}\markboth{8.52}{8.52} & .93044 & .93049 & .93054 & .93059 & .93064 & .93069 & .93075 & .93080 & .93085 & .93090 \\
\rowcolor{bg} \textbf{8.53}\markboth{8.53}{8.53} & .93095 & .93100 & .93105 & .93110 & .93115 & .93120 & .93125 & .93131 & .93136 & .93141 \\
 \textbf{8.54}\markboth{8.54}{8.54} & .93146 & .93151 & .93156 & .93161 & .93166 & .93171 & .93176 & .93181 & .93186 & .93192 \\
\rowcolor{bg} \textbf{8.55}\markboth{8.55}{8.55} & .93197 & .93202 & .93207 & .93212 & .93217 & .93222 & .93227 & .93232 & .93237 & .93242 \\
 \textbf{8.56}\markboth{8.56}{8.56} & .93247 & .93252 & .93258 & .93263 & .93268 & .93273 & .93278 & .93283 & .93288 & .93293 \\
\rowcolor{bg} \textbf{8.57}\markboth{8.57}{8.57} & .93298 & .93303 & .93308 & .93313 & .93318 & .93323 & .93328 & .93334 & .93339 & .93344 \\
 \textbf{8.58}\markboth{8.58}{8.58} & .93349 & .93354 & .93359 & .93364 & .93369 & .93374 & .93379 & .93384 & .93389 & .93394 \\
\rowcolor{bg} \textbf{8.59}\markboth{8.59}{8.59} & .93399 & .93404 & .93409 & .93414 & .93420 & .93425 & .93430 & .93435 & .93440 & .93445 \\
 \textbf{8.60}\markboth{8.60}{8.60} & .93450 & .93455 & .93460 & .93465 & .93470 & .93475 & .93480 & .93485 & .93490 & .93495 \\
\rowcolor{bg} \textbf{8.61}\markboth{8.61}{8.61} & .93500 & .93505 & .93510 & .93515 & .93520 & .93526 & .93531 & .93536 & .93541 & .93546 \\
 \textbf{8.62}\markboth{8.62}{8.62} & .93551 & .93556 & .93561 & .93566 & .93571 & .93576 & .93581 & .93586 & .93591 & .93596 \\
\rowcolor{bg} \textbf{8.63}\markboth{8.63}{8.63} & .93601 & .93606 & .93611 & .93616 & .93621 & .93626 & .93631 & .93636 & .93641 & .93646 \\
 \textbf{8.64}\markboth{8.64}{8.64} & .93651 & .93656 & .93661 & .93666 & .93671 & .93676 & .93682 & .93687 & .93692 & .93697 \\
\rowcolor{bg} \textbf{8.65}\markboth{8.65}{8.65} & .93702 & .93707 & .93712 & .93717 & .93722 & .93727 & .93732 & .93737 & .93742 & .93747 \\
 \textbf{8.66}\markboth{8.66}{8.66} & .93752 & .93757 & .93762 & .93767 & .93772 & .93777 & .93782 & .93787 & .93792 & .93797 \\
\rowcolor{bg} \textbf{8.67}\markboth{8.67}{8.67} & .93802 & .93807 & .93812 & .93817 & .93822 & .93827 & .93832 & .93837 & .93842 & .93847 \\
 \textbf{8.68}\markboth{8.68}{8.68} & .93852 & .93857 & .93862 & .93867 & .93872 & .93877 & .93882 & .93887 & .93892 & .93897 \\
\rowcolor{bg} \textbf{8.69}\markboth{8.69}{8.69} & .93902 & .93907 & .93912 & .93917 & .93922 & .93927 & .93932 & .93937 & .93942 & .93947 \\
 \textbf{8.70}\markboth{8.70}{8.70} & .93952 & .93957 & .93962 & .93967 & .93972 & .93977 & .93982 & .93987 & .93992 & .93997 \\
\rowcolor{bg} \textbf{8.71}\markboth{8.71}{8.71} & .94002 & .94007 & .94012 & .94017 & .94022 & .94027 & .94032 & .94037 & .94042 & .94047 \\
 \textbf{8.72}\markboth{8.72}{8.72} & .94052 & .94057 & .94062 & .94067 & .94072 & .94077 & .94082 & .94086 & .94091 & .94096 \\
\rowcolor{bg} \textbf{8.73}\markboth{8.73}{8.73} & .94101 & .94106 & .94111 & .94116 & .94121 & .94126 & .94131 & .94136 & .94141 & .94146 \\
 \textbf{8.74}\markboth{8.74}{8.74} & .94151 & .94156 & .94161 & .94166 & .94171 & .94176 & .94181 & .94186 & .94191 & .94196 \\
\rowcolor{bg} \textbf{8.75}\markboth{8.75}{8.75} & .94201 & .94206 & .94211 & .94216 & .94221 & .94226 & .94231 & .94236 & .94240 & .94245 \\
 \textbf{8.76}\markboth{8.76}{8.76} & .94250 & .94255 & .94260 & .94265 & .94270 & .94275 & .94280 & .94285 & .94290 & .94295 \\
\rowcolor{bg} \textbf{8.77}\markboth{8.77}{8.77} & .94300 & .94305 & .94310 & .94315 & .94320 & .94325 & .94330 & .94335 & .94340 & .94345 \\
 \textbf{8.78}\markboth{8.78}{8.78} & .94349 & .94354 & .94359 & .94364 & .94369 & .94374 & .94379 & .94384 & .94389 & .94394 \\
\rowcolor{bg} \textbf{8.79}\markboth{8.79}{8.79} & .94399 & .94404 & .94409 & .94414 & .94419 & .94424 & .94429 & .94433 & .94438 & .94443 \\
 \textbf{8.80}\markboth{8.80}{8.80} & .94448 & .94453 & .94458 & .94463 & .94468 & .94473 & .94478 & .94483 & .94488 & .94493 \\
\rowcolor{bg} \textbf{8.81}\markboth{8.81}{8.81} & .94498 & .94503 & .94507 & .94512 & .94517 & .94522 & .94527 & .94532 & .94537 & .94542 \\
 \textbf{8.82}\markboth{8.82}{8.82} & .94547 & .94552 & .94557 & .94562 & .94567 & .94571 & .94576 & .94581 & .94586 & .94591 \\
\rowcolor{bg} \textbf{8.83}\markboth{8.83}{8.83} & .94596 & .94601 & .94606 & .94611 & .94616 & .94621 & .94626 & .94630 & .94635 & .94640 \\
 \textbf{8.84}\markboth{8.84}{8.84} & .94645 & .94650 & .94655 & .94660 & .94665 & .94670 & .94675 & .94680 & .94685 & .94689 \\
\rowcolor{bg} \textbf{8.85}\markboth{8.85}{8.85} & .94694 & .94699 & .94704 & .94709 & .94714 & .94719 & .94724 & .94729 & .94734 & .94738 \\
 \textbf{8.86}\markboth{8.86}{8.86} & .94743 & .94748 & .94753 & .94758 & .94763 & .94768 & .94773 & .94778 & .94783 & .94787 \\
\rowcolor{bg} \textbf{8.87}\markboth{8.87}{8.87} & .94792 & .94797 & .94802 & .94807 & .94812 & .94817 & .94822 & .94827 & .94832 & .94836 \\
 \textbf{8.88}\markboth{8.88}{8.88} & .94841 & .94846 & .94851 & .94856 & .94861 & .94866 & .94871 & .94876 & .94880 & .94885 \\
\rowcolor{bg} \textbf{8.89}\markboth{8.89}{8.89} & .94890 & .94895 & .94900 & .94905 & .94910 & .94915 & .94919 & .94924 & .94929 & .94934 \\
 \textbf{8.90}\markboth{8.90}{8.90} & .94939 & .94944 & .94949 & .94954 & .94959 & .94963 & .94968 & .94973 & .94978 & .94983 \\
\rowcolor{bg} \textbf{8.91}\markboth{8.91}{8.91} & .94988 & .94993 & .94998 & .95002 & .95007 & .95012 & .95017 & .95022 & .95027 & .95032 \\
 \textbf{8.92}\markboth{8.92}{8.92} & .95036 & .95041 & .95046 & .95051 & .95056 & .95061 & .95066 & .95071 & .95075 & .95080 \\
\rowcolor{bg} \textbf{8.93}\markboth{8.93}{8.93} & .95085 & .95090 & .95095 & .95100 & .95105 & .95109 & .95114 & .95119 & .95124 & .95129 \\
 \textbf{8.94}\markboth{8.94}{8.94} & .95134 & .95139 & .95143 & .95148 & .95153 & .95158 & .95163 & .95168 & .95173 & .95177 \\
\rowcolor{bg} \textbf{8.95}\markboth{8.95}{8.95} & .95182 & .95187 & .95192 & .95197 & .95202 & .95207 & .95211 & .95216 & .95221 & .95226 \\
 \textbf{8.96}\markboth{8.96}{8.96} & .95231 & .95236 & .95240 & .95245 & .95250 & .95255 & .95260 & .95265 & .95270 & .95274 \\
\rowcolor{bg} \textbf{8.97}\markboth{8.97}{8.97} & .95279 & .95284 & .95289 & .95294 & .95299 & .95303 & .95308 & .95313 & .95318 & .95323 \\
 \textbf{8.98}\markboth{8.98}{8.98} & .95328 & .95332 & .95337 & .95342 & .95347 & .95352 & .95357 & .95361 & .95366 & .95371 \\
\rowcolor{bg} \textbf{8.99}\markboth{8.99}{8.99} & .95376 & .95381 & .95386 & .95390 & .95395 & .95400 & .95405 & .95410 & .95415 & .95419 \\
 \textcolor{blue}{\textbf{9.00}}\markboth{9.00}{9.00} & .95424 & .95429 & .95434 & .95439 & .95444 & .95448 & .95453 & .95458 & .95463 & .95468 \\
\rowcolor{bg} \textbf{9.01}\markboth{9.01}{9.01} & .95472 & .95477 & .95482 & .95487 & .95492 & .95497 & .95501 & .95506 & .95511 & .95516 \\
 \textbf{9.02}\markboth{9.02}{9.02} & .95521 & .95525 & .95530 & .95535 & .95540 & .95545 & .95550 & .95554 & .95559 & .95564 \\
\rowcolor{bg} \textbf{9.03}\markboth{9.03}{9.03} & .95569 & .95574 & .95578 & .95583 & .95588 & .95593 & .95598 & .95602 & .95607 & .95612 \\
 \textbf{9.04}\markboth{9.04}{9.04} & .95617 & .95622 & .95626 & .95631 & .95636 & .95641 & .95646 & .95650 & .95655 & .95660 \\
\rowcolor{bg} \textbf{9.05}\markboth{9.05}{9.05} & .95665 & .95670 & .95674 & .95679 & .95684 & .95689 & .95694 & .95698 & .95703 & .95708 \\
 \textbf{9.06}\markboth{9.06}{9.06} & .95713 & .95718 & .95722 & .95727 & .95732 & .95737 & .95742 & .95746 & .95751 & .95756 \\
\rowcolor{bg} \textbf{9.07}\markboth{9.07}{9.07} & .95761 & .95766 & .95770 & .95775 & .95780 & .95785 & .95789 & .95794 & .95799 & .95804 \\
 \textbf{9.08}\markboth{9.08}{9.08} & .95809 & .95813 & .95818 & .95823 & .95828 & .95832 & .95837 & .95842 & .95847 & .95852 \\
\rowcolor{bg} \textbf{9.09}\markboth{9.09}{9.09} & .95856 & .95861 & .95866 & .95871 & .95875 & .95880 & .95885 & .95890 & .95895 & .95899 \\
 \textbf{9.10}\markboth{9.10}{9.10} & .95904 & .95909 & .95914 & .95918 & .95923 & .95928 & .95933 & .95938 & .95942 & .95947 \\
\rowcolor{bg} \textbf{9.11}\markboth{9.11}{9.11} & .95952 & .95957 & .95961 & .95966 & .95971 & .95976 & .95980 & .95985 & .95990 & .95995 \\
 \textbf{9.12}\markboth{9.12}{9.12} & .95999 & .96004 & .96009 & .96014 & .96019 & .96023 & .96028 & .96033 & .96038 & .96042 \\
\rowcolor{bg} \textbf{9.13}\markboth{9.13}{9.13} & .96047 & .96052 & .96057 & .96061 & .96066 & .96071 & .96076 & .96080 & .96085 & .96090 \\
 \textbf{9.14}\markboth{9.14}{9.14} & .96095 & .96099 & .96104 & .96109 & .96114 & .96118 & .96123 & .96128 & .96133 & .96137 \\
\rowcolor{bg} \textbf{9.15}\markboth{9.15}{9.15} & .96142 & .96147 & .96152 & .96156 & .96161 & .96166 & .96171 & .96175 & .96180 & .96185 \\
 \textbf{9.16}\markboth{9.16}{9.16} & .96190 & .96194 & .96199 & .96204 & .96209 & .96213 & .96218 & .96223 & .96227 & .96232 \\
\rowcolor{bg} \textbf{9.17}\markboth{9.17}{9.17} & .96237 & .96242 & .96246 & .96251 & .96256 & .96261 & .96265 & .96270 & .96275 & .96280 \\
 \textbf{9.18}\markboth{9.18}{9.18} & .96284 & .96289 & .96294 & .96298 & .96303 & .96308 & .96313 & .96317 & .96322 & .96327 \\
\rowcolor{bg} \textbf{9.19}\markboth{9.19}{9.19} & .96332 & .96336 & .96341 & .96346 & .96350 & .96355 & .96360 & .96365 & .96369 & .96374 \\
 \textbf{9.20}\markboth{9.20}{9.20} & .96379 & .96384 & .96388 & .96393 & .96398 & .96402 & .96407 & .96412 & .96417 & .96421 \\
\rowcolor{bg} \textbf{9.21}\markboth{9.21}{9.21} & .96426 & .96431 & .96435 & .96440 & .96445 & .96450 & .96454 & .96459 & .96464 & .96468 \\
 \textbf{9.22}\markboth{9.22}{9.22} & .96473 & .96478 & .96483 & .96487 & .96492 & .96497 & .96501 & .96506 & .96511 & .96515 \\
\rowcolor{bg} \textbf{9.23}\markboth{9.23}{9.23} & .96520 & .96525 & .96530 & .96534 & .96539 & .96544 & .96548 & .96553 & .96558 & .96562 \\
 \textbf{9.24}\markboth{9.24}{9.24} & .96567 & .96572 & .96577 & .96581 & .96586 & .96591 & .96595 & .96600 & .96605 & .96609 \\
\rowcolor{bg} \textbf{9.25}\markboth{9.25}{9.25} & .96614 & .96619 & .96624 & .96628 & .96633 & .96638 & .96642 & .96647 & .96652 & .96656 \\
 \textbf{9.26}\markboth{9.26}{9.26} & .96661 & .96666 & .96670 & .96675 & .96680 & .96685 & .96689 & .96694 & .96699 & .96703 \\
\rowcolor{bg} \textbf{9.27}\markboth{9.27}{9.27} & .96708 & .96713 & .96717 & .96722 & .96727 & .96731 & .96736 & .96741 & .96745 & .96750 \\
 \textbf{9.28}\markboth{9.28}{9.28} & .96755 & .96759 & .96764 & .96769 & .96774 & .96778 & .96783 & .96788 & .96792 & .96797 \\
\rowcolor{bg} \textbf{9.29}\markboth{9.29}{9.29} & .96802 & .96806 & .96811 & .96816 & .96820 & .96825 & .96830 & .96834 & .96839 & .96844 \\
 \textbf{9.30}\markboth{9.30}{9.30} & .96848 & .96853 & .96858 & .96862 & .96867 & .96872 & .96876 & .96881 & .96886 & .96890 \\
\rowcolor{bg} \textbf{9.31}\markboth{9.31}{9.31} & .96895 & .96900 & .96904 & .96909 & .96914 & .96918 & .96923 & .96928 & .96932 & .96937 \\
 \textbf{9.32}\markboth{9.32}{9.32} & .96942 & .96946 & .96951 & .96956 & .96960 & .96965 & .96970 & .96974 & .96979 & .96984 \\
\rowcolor{bg} \textbf{9.33}\markboth{9.33}{9.33} & .96988 & .96993 & .96997 & .97002 & .97007 & .97011 & .97016 & .97021 & .97025 & .97030 \\
 \textbf{9.34}\markboth{9.34}{9.34} & .97035 & .97039 & .97044 & .97049 & .97053 & .97058 & .97063 & .97067 & .97072 & .97077 \\
\rowcolor{bg} \textbf{9.35}\markboth{9.35}{9.35} & .97081 & .97086 & .97090 & .97095 & .97100 & .97104 & .97109 & .97114 & .97118 & .97123 \\
 \textbf{9.36}\markboth{9.36}{9.36} & .97128 & .97132 & .97137 & .97142 & .97146 & .97151 & .97155 & .97160 & .97165 & .97169 \\
\rowcolor{bg} \textbf{9.37}\markboth{9.37}{9.37} & .97174 & .97179 & .97183 & .97188 & .97192 & .97197 & .97202 & .97206 & .97211 & .97216 \\
 \textbf{9.38}\markboth{9.38}{9.38} & .97220 & .97225 & .97230 & .97234 & .97239 & .97243 & .97248 & .97253 & .97257 & .97262 \\
\rowcolor{bg} \textbf{9.39}\markboth{9.39}{9.39} & .97267 & .97271 & .97276 & .97280 & .97285 & .97290 & .97294 & .97299 & .97304 & .97308 \\
 \textbf{9.40}\markboth{9.40}{9.40} & .97313 & .97317 & .97322 & .97327 & .97331 & .97336 & .97340 & .97345 & .97350 & .97354 \\
\rowcolor{bg} \textbf{9.41}\markboth{9.41}{9.41} & .97359 & .97364 & .97368 & .97373 & .97377 & .97382 & .97387 & .97391 & .97396 & .97400 \\
 \textbf{9.42}\markboth{9.42}{9.42} & .97405 & .97410 & .97414 & .97419 & .97424 & .97428 & .97433 & .97437 & .97442 & .97447 \\
\rowcolor{bg} \textbf{9.43}\markboth{9.43}{9.43} & .97451 & .97456 & .97460 & .97465 & .97470 & .97474 & .97479 & .97483 & .97488 & .97493 \\
 \textbf{9.44}\markboth{9.44}{9.44} & .97497 & .97502 & .97506 & .97511 & .97516 & .97520 & .97525 & .97529 & .97534 & .97539 \\
\rowcolor{bg} \textbf{9.45}\markboth{9.45}{9.45} & .97543 & .97548 & .97552 & .97557 & .97562 & .97566 & .97571 & .97575 & .97580 & .97585 \\
 \textbf{9.46}\markboth{9.46}{9.46} & .97589 & .97594 & .97598 & .97603 & .97607 & .97612 & .97617 & .97621 & .97626 & .97630 \\
\rowcolor{bg} \textbf{9.47}\markboth{9.47}{9.47} & .97635 & .97640 & .97644 & .97649 & .97653 & .97658 & .97663 & .97667 & .97672 & .97676 \\
 \textbf{9.48}\markboth{9.48}{9.48} & .97681 & .97685 & .97690 & .97695 & .97699 & .97704 & .97708 & .97713 & .97717 & .97722 \\
\rowcolor{bg} \textbf{9.49}\markboth{9.49}{9.49} & .97727 & .97731 & .97736 & .97740 & .97745 & .97749 & .97754 & .97759 & .97763 & .97768 \\
 \textbf{9.50}\markboth{9.50}{9.50} & .97772 & .97777 & .97782 & .97786 & .97791 & .97795 & .97800 & .97804 & .97809 & .97813 \\
\rowcolor{bg} \textbf{9.51}\markboth{9.51}{9.51} & .97818 & .97823 & .97827 & .97832 & .97836 & .97841 & .97845 & .97850 & .97855 & .97859 \\
 \textbf{9.52}\markboth{9.52}{9.52} & .97864 & .97868 & .97873 & .97877 & .97882 & .97886 & .97891 & .97896 & .97900 & .97905 \\
\rowcolor{bg} \textbf{9.53}\markboth{9.53}{9.53} & .97909 & .97914 & .97918 & .97923 & .97928 & .97932 & .97937 & .97941 & .97946 & .97950 \\
 \textbf{9.54}\markboth{9.54}{9.54} & .97955 & .97959 & .97964 & .97968 & .97973 & .97978 & .97982 & .97987 & .97991 & .97996 \\
\rowcolor{bg} \textbf{9.55}\markboth{9.55}{9.55} & .98000 & .98005 & .98009 & .98014 & .98019 & .98023 & .98028 & .98032 & .98037 & .98041 \\
 \textbf{9.56}\markboth{9.56}{9.56} & .98046 & .98050 & .98055 & .98059 & .98064 & .98068 & .98073 & .98078 & .98082 & .98087 \\
\rowcolor{bg} \textbf{9.57}\markboth{9.57}{9.57} & .98091 & .98096 & .98100 & .98105 & .98109 & .98114 & .98118 & .98123 & .98127 & .98132 \\
 \textbf{9.58}\markboth{9.58}{9.58} & .98137 & .98141 & .98146 & .98150 & .98155 & .98159 & .98164 & .98168 & .98173 & .98177 \\
\rowcolor{bg} \textbf{9.59}\markboth{9.59}{9.59} & .98182 & .98186 & .98191 & .98195 & .98200 & .98204 & .98209 & .98214 & .98218 & .98223 \\
 \textbf{9.60}\markboth{9.60}{9.60} & .98227 & .98232 & .98236 & .98241 & .98245 & .98250 & .98254 & .98259 & .98263 & .98268 \\
\rowcolor{bg} \textbf{9.61}\markboth{9.61}{9.61} & .98272 & .98277 & .98281 & .98286 & .98290 & .98295 & .98299 & .98304 & .98308 & .98313 \\
 \textbf{9.62}\markboth{9.62}{9.62} & .98318 & .98322 & .98327 & .98331 & .98336 & .98340 & .98345 & .98349 & .98354 & .98358 \\
\rowcolor{bg} \textbf{9.63}\markboth{9.63}{9.63} & .98363 & .98367 & .98372 & .98376 & .98381 & .98385 & .98390 & .98394 & .98399 & .98403 \\
 \textbf{9.64}\markboth{9.64}{9.64} & .98408 & .98412 & .98417 & .98421 & .98426 & .98430 & .98435 & .98439 & .98444 & .98448 \\
\rowcolor{bg} \textbf{9.65}\markboth{9.65}{9.65} & .98453 & .98457 & .98462 & .98466 & .98471 & .98475 & .98480 & .98484 & .98489 & .98493 \\
 \textbf{9.66}\markboth{9.66}{9.66} & .98498 & .98502 & .98507 & .98511 & .98516 & .98520 & .98525 & .98529 & .98534 & .98538 \\
\rowcolor{bg} \textbf{9.67}\markboth{9.67}{9.67} & .98543 & .98547 & .98552 & .98556 & .98561 & .98565 & .98570 & .98574 & .98579 & .98583 \\
 \textbf{9.68}\markboth{9.68}{9.68} & .98588 & .98592 & .98597 & .98601 & .98605 & .98610 & .98614 & .98619 & .98623 & .98628 \\
\rowcolor{bg} \textbf{9.69}\markboth{9.69}{9.69} & .98632 & .98637 & .98641 & .98646 & .98650 & .98655 & .98659 & .98664 & .98668 & .98673 \\
 \textbf{9.70}\markboth{9.70}{9.70} & .98677 & .98682 & .98686 & .98691 & .98695 & .98700 & .98704 & .98709 & .98713 & .98717 \\
\rowcolor{bg} \textbf{9.71}\markboth{9.71}{9.71} & .98722 & .98726 & .98731 & .98735 & .98740 & .98744 & .98749 & .98753 & .98758 & .98762 \\
 \textbf{9.72}\markboth{9.72}{9.72} & .98767 & .98771 & .98776 & .98780 & .98784 & .98789 & .98793 & .98798 & .98802 & .98807 \\
\rowcolor{bg} \textbf{9.73}\markboth{9.73}{9.73} & .98811 & .98816 & .98820 & .98825 & .98829 & .98834 & .98838 & .98843 & .98847 & .98851 \\
 \textbf{9.74}\markboth{9.74}{9.74} & .98856 & .98860 & .98865 & .98869 & .98874 & .98878 & .98883 & .98887 & .98892 & .98896 \\
\rowcolor{bg} \textbf{9.75}\markboth{9.75}{9.75} & .98900 & .98905 & .98909 & .98914 & .98918 & .98923 & .98927 & .98932 & .98936 & .98941 \\
 \textbf{9.76}\markboth{9.76}{9.76} & .98945 & .98949 & .98954 & .98958 & .98963 & .98967 & .98972 & .98976 & .98981 & .98985 \\
\rowcolor{bg} \textbf{9.77}\markboth{9.77}{9.77} & .98989 & .98994 & .98998 & .99003 & .99007 & .99012 & .99016 & .99021 & .99025 & .99029 \\
 \textbf{9.78}\markboth{9.78}{9.78} & .99034 & .99038 & .99043 & .99047 & .99052 & .99056 & .99061 & .99065 & .99069 & .99074 \\
\rowcolor{bg} \textbf{9.79}\markboth{9.79}{9.79} & .99078 & .99083 & .99087 & .99092 & .99096 & .99100 & .99105 & .99109 & .99114 & .99118 \\
 \textbf{9.80}\markboth{9.80}{9.80} & .99123 & .99127 & .99131 & .99136 & .99140 & .99145 & .99149 & .99154 & .99158 & .99162 \\
\rowcolor{bg} \textbf{9.81}\markboth{9.81}{9.81} & .99167 & .99171 & .99176 & .99180 & .99185 & .99189 & .99193 & .99198 & .99202 & .99207 \\
 \textbf{9.82}\markboth{9.82}{9.82} & .99211 & .99216 & .99220 & .99224 & .99229 & .99233 & .99238 & .99242 & .99247 & .99251 \\
\rowcolor{bg} \textbf{9.83}\markboth{9.83}{9.83} & .99255 & .99260 & .99264 & .99269 & .99273 & .99277 & .99282 & .99286 & .99291 & .99295 \\
 \textbf{9.84}\markboth{9.84}{9.84} & .99300 & .99304 & .99308 & .99313 & .99317 & .99322 & .99326 & .99330 & .99335 & .99339 \\
\rowcolor{bg} \textbf{9.85}\markboth{9.85}{9.85} & .99344 & .99348 & .99352 & .99357 & .99361 & .99366 & .99370 & .99374 & .99379 & .99383 \\
 \textbf{9.86}\markboth{9.86}{9.86} & .99388 & .99392 & .99396 & .99401 & .99405 & .99410 & .99414 & .99419 & .99423 & .99427 \\
\rowcolor{bg} \textbf{9.87}\markboth{9.87}{9.87} & .99432 & .99436 & .99441 & .99445 & .99449 & .99454 & .99458 & .99463 & .99467 & .99471 \\
 \textbf{9.88}\markboth{9.88}{9.88} & .99476 & .99480 & .99484 & .99489 & .99493 & .99498 & .99502 & .99506 & .99511 & .99515 \\
\rowcolor{bg} \textbf{9.89}\markboth{9.89}{9.89} & .99520 & .99524 & .99528 & .99533 & .99537 & .99542 & .99546 & .99550 & .99555 & .99559 \\
 \textbf{9.90}\markboth{9.90}{9.90} & .99564 & .99568 & .99572 & .99577 & .99581 & .99585 & .99590 & .99594 & .99599 & .99603 \\
\rowcolor{bg} \textbf{9.91}\markboth{9.91}{9.91} & .99607 & .99612 & .99616 & .99621 & .99625 & .99629 & .99634 & .99638 & .99642 & .99647 \\
 \textbf{9.92}\markboth{9.92}{9.92} & .99651 & .99656 & .99660 & .99664 & .99669 & .99673 & .99677 & .99682 & .99686 & .99691 \\
\rowcolor{bg} \textbf{9.93}\markboth{9.93}{9.93} & .99695 & .99699 & .99704 & .99708 & .99712 & .99717 & .99721 & .99726 & .99730 & .99734 \\
 \textbf{9.94}\markboth{9.94}{9.94} & .99739 & .99743 & .99747 & .99752 & .99756 & .99760 & .99765 & .99769 & .99774 & .99778 \\
\rowcolor{bg} \textbf{9.95}\markboth{9.95}{9.95} & .99782 & .99787 & .99791 & .99795 & .99800 & .99804 & .99808 & .99813 & .99817 & .99822 \\
 \textbf{9.96}\markboth{9.96}{9.96} & .99826 & .99830 & .99835 & .99839 & .99843 & .99848 & .99852 & .99856 & .99861 & .99865 \\
\rowcolor{bg} \textbf{9.97}\markboth{9.97}{9.97} & .99870 & .99874 & .99878 & .99883 & .99887 & .99891 & .99896 & .99900 & .99904 & .99909 \\
 \textbf{9.98}\markboth{9.98}{9.98} & .99913 & .99917 & .99922 & .99926 & .99930 & .99935 & .99939 & .99944 & .99948 & .99952 \\
\rowcolor{bg} \textbf{9.99}\markboth{9.99}{9.99} & .99957 & .99961 & .99965 & .99970 & .99974 & .99978 & .99983 & .99987 & .99991 & .99996 \\

\end{longtable}

\newpage
\thispagestyle{empty}
\section{Decimal Anti-Logarithms}
\newpage

% headings for antilog table
\makeatletter
\def\@oddhead{\thepage\qquad\textsc{Decimal Anti-Logarithms}\hfill \rightmark{} -- \leftmark}
\def\@evenhead{\rightmark{} -- \leftmark\hfill \textsc{Decimal Anti-Logarithms}\qquad\thepage}
\makeatother

\begin{longtable}[c]{|*{11}{c|}}
 & \textbf{0} & \textbf{1} & \textbf{2} & \textbf{3} & \textbf{4} & \textbf{5} & \textbf{6} & \textbf{7} & \textbf{8} & \textbf{9} \\
\hline \endhead
\hline \endfoot
%%%
 \textcolor{blue}{\textbf{.000}}\markboth{0.000}{0.000} & 1.0000 & 1.0002 & 1.0005 & 1.0007 & 1.0009 & 1.0012 & 1.0014 & 1.0016 & 1.0018 & 1.0021 \\
\rowcolor{bg} \textbf{.001}\markboth{0.001}{0.001} & 1.0023 & 1.0025 & 1.0028 & 1.0030 & 1.0032 & 1.0035 & 1.0037 & 1.0039 & 1.0042 & 1.0044 \\
 \textbf{.002}\markboth{0.002}{0.002} & 1.0046 & 1.0048 & 1.0051 & 1.0053 & 1.0055 & 1.0058 & 1.0060 & 1.0062 & 1.0065 & 1.0067 \\
\rowcolor{bg} \textbf{.003}\markboth{0.003}{0.003} & 1.0069 & 1.0072 & 1.0074 & 1.0076 & 1.0079 & 1.0081 & 1.0083 & 1.0086 & 1.0088 & 1.0090 \\
 \textbf{.004}\markboth{0.004}{0.004} & 1.0093 & 1.0095 & 1.0097 & 1.0100 & 1.0102 & 1.0104 & 1.0106 & 1.0109 & 1.0111 & 1.0113 \\
\rowcolor{bg} \textbf{.005}\markboth{0.005}{0.005} & 1.0116 & 1.0118 & 1.0120 & 1.0123 & 1.0125 & 1.0127 & 1.0130 & 1.0132 & 1.0134 & 1.0137 \\
 \textbf{.006}\markboth{0.006}{0.006} & 1.0139 & 1.0141 & 1.0144 & 1.0146 & 1.0148 & 1.0151 & 1.0153 & 1.0155 & 1.0158 & 1.0160 \\
\rowcolor{bg} \textbf{.007}\markboth{0.007}{0.007} & 1.0162 & 1.0165 & 1.0167 & 1.0170 & 1.0172 & 1.0174 & 1.0177 & 1.0179 & 1.0181 & 1.0184 \\
 \textbf{.008}\markboth{0.008}{0.008} & 1.0186 & 1.0188 & 1.0191 & 1.0193 & 1.0195 & 1.0198 & 1.0200 & 1.0202 & 1.0205 & 1.0207 \\
\rowcolor{bg} \textbf{.009}\markboth{0.009}{0.009} & 1.0209 & 1.0212 & 1.0214 & 1.0216 & 1.0219 & 1.0221 & 1.0224 & 1.0226 & 1.0228 & 1.0231 \\
 \textbf{.010}\markboth{0.010}{0.010} & 1.0233 & 1.0235 & 1.0238 & 1.0240 & 1.0242 & 1.0245 & 1.0247 & 1.0249 & 1.0252 & 1.0254 \\
\rowcolor{bg} \textbf{.011}\markboth{0.011}{0.011} & 1.0257 & 1.0259 & 1.0261 & 1.0264 & 1.0266 & 1.0268 & 1.0271 & 1.0273 & 1.0275 & 1.0278 \\
 \textbf{.012}\markboth{0.012}{0.012} & 1.0280 & 1.0283 & 1.0285 & 1.0287 & 1.0290 & 1.0292 & 1.0294 & 1.0297 & 1.0299 & 1.0301 \\
\rowcolor{bg} \textbf{.013}\markboth{0.013}{0.013} & 1.0304 & 1.0306 & 1.0309 & 1.0311 & 1.0313 & 1.0316 & 1.0318 & 1.0320 & 1.0323 & 1.0325 \\
 \textbf{.014}\markboth{0.014}{0.014} & 1.0328 & 1.0330 & 1.0332 & 1.0335 & 1.0337 & 1.0340 & 1.0342 & 1.0344 & 1.0347 & 1.0349 \\
\rowcolor{bg} \textbf{.015}\markboth{0.015}{0.015} & 1.0351 & 1.0354 & 1.0356 & 1.0359 & 1.0361 & 1.0363 & 1.0366 & 1.0368 & 1.0371 & 1.0373 \\
 \textbf{.016}\markboth{0.016}{0.016} & 1.0375 & 1.0378 & 1.0380 & 1.0382 & 1.0385 & 1.0387 & 1.0390 & 1.0392 & 1.0394 & 1.0397 \\
\rowcolor{bg} \textbf{.017}\markboth{0.017}{0.017} & 1.0399 & 1.0402 & 1.0404 & 1.0406 & 1.0409 & 1.0411 & 1.0414 & 1.0416 & 1.0418 & 1.0421 \\
 \textbf{.018}\markboth{0.018}{0.018} & 1.0423 & 1.0426 & 1.0428 & 1.0430 & 1.0433 & 1.0435 & 1.0438 & 1.0440 & 1.0442 & 1.0445 \\
\rowcolor{bg} \textbf{.019}\markboth{0.019}{0.019} & 1.0447 & 1.0450 & 1.0452 & 1.0454 & 1.0457 & 1.0459 & 1.0462 & 1.0464 & 1.0466 & 1.0469 \\
 \textbf{.020}\markboth{0.020}{0.020} & 1.0471 & 1.0474 & 1.0476 & 1.0479 & 1.0481 & 1.0483 & 1.0486 & 1.0488 & 1.0491 & 1.0493 \\
\rowcolor{bg} \textbf{.021}\markboth{0.021}{0.021} & 1.0495 & 1.0498 & 1.0500 & 1.0503 & 1.0505 & 1.0508 & 1.0510 & 1.0512 & 1.0515 & 1.0517 \\
 \textbf{.022}\markboth{0.022}{0.022} & 1.0520 & 1.0522 & 1.0524 & 1.0527 & 1.0529 & 1.0532 & 1.0534 & 1.0537 & 1.0539 & 1.0541 \\
\rowcolor{bg} \textbf{.023}\markboth{0.023}{0.023} & 1.0544 & 1.0546 & 1.0549 & 1.0551 & 1.0554 & 1.0556 & 1.0558 & 1.0561 & 1.0563 & 1.0566 \\
 \textbf{.024}\markboth{0.024}{0.024} & 1.0568 & 1.0571 & 1.0573 & 1.0575 & 1.0578 & 1.0580 & 1.0583 & 1.0585 & 1.0588 & 1.0590 \\
\rowcolor{bg} \textbf{.025}\markboth{0.025}{0.025} & 1.0593 & 1.0595 & 1.0597 & 1.0600 & 1.0602 & 1.0605 & 1.0607 & 1.0610 & 1.0612 & 1.0615 \\
 \textbf{.026}\markboth{0.026}{0.026} & 1.0617 & 1.0619 & 1.0622 & 1.0624 & 1.0627 & 1.0629 & 1.0632 & 1.0634 & 1.0637 & 1.0639 \\
\rowcolor{bg} \textbf{.027}\markboth{0.027}{0.027} & 1.0641 & 1.0644 & 1.0646 & 1.0649 & 1.0651 & 1.0654 & 1.0656 & 1.0659 & 1.0661 & 1.0664 \\
 \textbf{.028}\markboth{0.028}{0.028} & 1.0666 & 1.0668 & 1.0671 & 1.0673 & 1.0676 & 1.0678 & 1.0681 & 1.0683 & 1.0686 & 1.0688 \\
\rowcolor{bg} \textbf{.029}\markboth{0.029}{0.029} & 1.0691 & 1.0693 & 1.0695 & 1.0698 & 1.0700 & 1.0703 & 1.0705 & 1.0708 & 1.0710 & 1.0713 \\
 \textbf{.030}\markboth{0.030}{0.030} & 1.0715 & 1.0718 & 1.0720 & 1.0723 & 1.0725 & 1.0728 & 1.0730 & 1.0732 & 1.0735 & 1.0737 \\
\rowcolor{bg} \textbf{.031}\markboth{0.031}{0.031} & 1.0740 & 1.0742 & 1.0745 & 1.0747 & 1.0750 & 1.0752 & 1.0755 & 1.0757 & 1.0760 & 1.0762 \\
 \textbf{.032}\markboth{0.032}{0.032} & 1.0765 & 1.0767 & 1.0770 & 1.0772 & 1.0775 & 1.0777 & 1.0780 & 1.0782 & 1.0784 & 1.0787 \\
\rowcolor{bg} \textbf{.033}\markboth{0.033}{0.033} & 1.0789 & 1.0792 & 1.0794 & 1.0797 & 1.0799 & 1.0802 & 1.0804 & 1.0807 & 1.0809 & 1.0812 \\
 \textbf{.034}\markboth{0.034}{0.034} & 1.0814 & 1.0817 & 1.0819 & 1.0822 & 1.0824 & 1.0827 & 1.0829 & 1.0832 & 1.0834 & 1.0837 \\
\rowcolor{bg} \textbf{.035}\markboth{0.035}{0.035} & 1.0839 & 1.0842 & 1.0844 & 1.0847 & 1.0849 & 1.0852 & 1.0854 & 1.0857 & 1.0859 & 1.0862 \\
 \textbf{.036}\markboth{0.036}{0.036} & 1.0864 & 1.0867 & 1.0869 & 1.0872 & 1.0874 & 1.0877 & 1.0879 & 1.0882 & 1.0884 & 1.0887 \\
\rowcolor{bg} \textbf{.037}\markboth{0.037}{0.037} & 1.0889 & 1.0892 & 1.0894 & 1.0897 & 1.0899 & 1.0902 & 1.0904 & 1.0907 & 1.0909 & 1.0912 \\
 \textbf{.038}\markboth{0.038}{0.038} & 1.0914 & 1.0917 & 1.0919 & 1.0922 & 1.0924 & 1.0927 & 1.0929 & 1.0932 & 1.0935 & 1.0937 \\
\rowcolor{bg} \textbf{.039}\markboth{0.039}{0.039} & 1.0940 & 1.0942 & 1.0945 & 1.0947 & 1.0950 & 1.0952 & 1.0955 & 1.0957 & 1.0960 & 1.0962 \\
 \textbf{.040}\markboth{0.040}{0.040} & 1.0965 & 1.0967 & 1.0970 & 1.0972 & 1.0975 & 1.0977 & 1.0980 & 1.0982 & 1.0985 & 1.0988 \\
\rowcolor{bg} \textbf{.041}\markboth{0.041}{0.041} & 1.0990 & 1.0993 & 1.0995 & 1.0998 & 1.1000 & 1.1003 & 1.1005 & 1.1008 & 1.1010 & 1.1013 \\
 \textbf{.042}\markboth{0.042}{0.042} & 1.1015 & 1.1018 & 1.1020 & 1.1023 & 1.1026 & 1.1028 & 1.1031 & 1.1033 & 1.1036 & 1.1038 \\
\rowcolor{bg} \textbf{.043}\markboth{0.043}{0.043} & 1.1041 & 1.1043 & 1.1046 & 1.1048 & 1.1051 & 1.1054 & 1.1056 & 1.1059 & 1.1061 & 1.1064 \\
 \textbf{.044}\markboth{0.044}{0.044} & 1.1066 & 1.1069 & 1.1071 & 1.1074 & 1.1076 & 1.1079 & 1.1082 & 1.1084 & 1.1087 & 1.1089 \\
\rowcolor{bg} \textbf{.045}\markboth{0.045}{0.045} & 1.1092 & 1.1094 & 1.1097 & 1.1099 & 1.1102 & 1.1105 & 1.1107 & 1.1110 & 1.1112 & 1.1115 \\
 \textbf{.046}\markboth{0.046}{0.046} & 1.1117 & 1.1120 & 1.1122 & 1.1125 & 1.1128 & 1.1130 & 1.1133 & 1.1135 & 1.1138 & 1.1140 \\
\rowcolor{bg} \textbf{.047}\markboth{0.047}{0.047} & 1.1143 & 1.1146 & 1.1148 & 1.1151 & 1.1153 & 1.1156 & 1.1158 & 1.1161 & 1.1163 & 1.1166 \\
 \textbf{.048}\markboth{0.048}{0.048} & 1.1169 & 1.1171 & 1.1174 & 1.1176 & 1.1179 & 1.1181 & 1.1184 & 1.1187 & 1.1189 & 1.1192 \\
\rowcolor{bg} \textbf{.049}\markboth{0.049}{0.049} & 1.1194 & 1.1197 & 1.1200 & 1.1202 & 1.1205 & 1.1207 & 1.1210 & 1.1212 & 1.1215 & 1.1218 \\
 \textbf{.050}\markboth{0.050}{0.050} & 1.1220 & 1.1223 & 1.1225 & 1.1228 & 1.1231 & 1.1233 & 1.1236 & 1.1238 & 1.1241 & 1.1243 \\
\rowcolor{bg} \textbf{.051}\markboth{0.051}{0.051} & 1.1246 & 1.1249 & 1.1251 & 1.1254 & 1.1256 & 1.1259 & 1.1262 & 1.1264 & 1.1267 & 1.1269 \\
 \textbf{.052}\markboth{0.052}{0.052} & 1.1272 & 1.1275 & 1.1277 & 1.1280 & 1.1282 & 1.1285 & 1.1288 & 1.1290 & 1.1293 & 1.1295 \\
\rowcolor{bg} \textbf{.053}\markboth{0.053}{0.053} & 1.1298 & 1.1301 & 1.1303 & 1.1306 & 1.1308 & 1.1311 & 1.1314 & 1.1316 & 1.1319 & 1.1321 \\
 \textbf{.054}\markboth{0.054}{0.054} & 1.1324 & 1.1327 & 1.1329 & 1.1332 & 1.1334 & 1.1337 & 1.1340 & 1.1342 & 1.1345 & 1.1347 \\
\rowcolor{bg} \textbf{.055}\markboth{0.055}{0.055} & 1.1350 & 1.1353 & 1.1355 & 1.1358 & 1.1361 & 1.1363 & 1.1366 & 1.1368 & 1.1371 & 1.1374 \\
 \textbf{.056}\markboth{0.056}{0.056} & 1.1376 & 1.1379 & 1.1382 & 1.1384 & 1.1387 & 1.1389 & 1.1392 & 1.1395 & 1.1397 & 1.1400 \\
\rowcolor{bg} \textbf{.057}\markboth{0.057}{0.057} & 1.1402 & 1.1405 & 1.1408 & 1.1410 & 1.1413 & 1.1416 & 1.1418 & 1.1421 & 1.1424 & 1.1426 \\
 \textbf{.058}\markboth{0.058}{0.058} & 1.1429 & 1.1431 & 1.1434 & 1.1437 & 1.1439 & 1.1442 & 1.1445 & 1.1447 & 1.1450 & 1.1452 \\
\rowcolor{bg} \textbf{.059}\markboth{0.059}{0.059} & 1.1455 & 1.1458 & 1.1460 & 1.1463 & 1.1466 & 1.1468 & 1.1471 & 1.1474 & 1.1476 & 1.1479 \\
 \textbf{.060}\markboth{0.060}{0.060} & 1.1482 & 1.1484 & 1.1487 & 1.1489 & 1.1492 & 1.1495 & 1.1497 & 1.1500 & 1.1503 & 1.1505 \\
\rowcolor{bg} \textbf{.061}\markboth{0.061}{0.061} & 1.1508 & 1.1511 & 1.1513 & 1.1516 & 1.1519 & 1.1521 & 1.1524 & 1.1527 & 1.1529 & 1.1532 \\
 \textbf{.062}\markboth{0.062}{0.062} & 1.1535 & 1.1537 & 1.1540 & 1.1543 & 1.1545 & 1.1548 & 1.1550 & 1.1553 & 1.1556 & 1.1558 \\
\rowcolor{bg} \textbf{.063}\markboth{0.063}{0.063} & 1.1561 & 1.1564 & 1.1566 & 1.1569 & 1.1572 & 1.1574 & 1.1577 & 1.1580 & 1.1582 & 1.1585 \\
 \textbf{.064}\markboth{0.064}{0.064} & 1.1588 & 1.1590 & 1.1593 & 1.1596 & 1.1598 & 1.1601 & 1.1604 & 1.1606 & 1.1609 & 1.1612 \\
\rowcolor{bg} \textbf{.065}\markboth{0.065}{0.065} & 1.1614 & 1.1617 & 1.1620 & 1.1623 & 1.1625 & 1.1628 & 1.1631 & 1.1633 & 1.1636 & 1.1639 \\
 \textbf{.066}\markboth{0.066}{0.066} & 1.1641 & 1.1644 & 1.1647 & 1.1649 & 1.1652 & 1.1655 & 1.1657 & 1.1660 & 1.1663 & 1.1665 \\
\rowcolor{bg} \textbf{.067}\markboth{0.067}{0.067} & 1.1668 & 1.1671 & 1.1673 & 1.1676 & 1.1679 & 1.1682 & 1.1684 & 1.1687 & 1.1690 & 1.1692 \\
 \textbf{.068}\markboth{0.068}{0.068} & 1.1695 & 1.1698 & 1.1700 & 1.1703 & 1.1706 & 1.1708 & 1.1711 & 1.1714 & 1.1717 & 1.1719 \\
\rowcolor{bg} \textbf{.069}\markboth{0.069}{0.069} & 1.1722 & 1.1725 & 1.1727 & 1.1730 & 1.1733 & 1.1735 & 1.1738 & 1.1741 & 1.1744 & 1.1746 \\
 \textbf{.070}\markboth{0.070}{0.070} & 1.1749 & 1.1752 & 1.1754 & 1.1757 & 1.1760 & 1.1763 & 1.1765 & 1.1768 & 1.1771 & 1.1773 \\
\rowcolor{bg} \textbf{.071}\markboth{0.071}{0.071} & 1.1776 & 1.1779 & 1.1781 & 1.1784 & 1.1787 & 1.1790 & 1.1792 & 1.1795 & 1.1798 & 1.1800 \\
 \textbf{.072}\markboth{0.072}{0.072} & 1.1803 & 1.1806 & 1.1809 & 1.1811 & 1.1814 & 1.1817 & 1.1820 & 1.1822 & 1.1825 & 1.1828 \\
\rowcolor{bg} \textbf{.073}\markboth{0.073}{0.073} & 1.1830 & 1.1833 & 1.1836 & 1.1839 & 1.1841 & 1.1844 & 1.1847 & 1.1849 & 1.1852 & 1.1855 \\
 \textbf{.074}\markboth{0.074}{0.074} & 1.1858 & 1.1860 & 1.1863 & 1.1866 & 1.1869 & 1.1871 & 1.1874 & 1.1877 & 1.1880 & 1.1882 \\
\rowcolor{bg} \textbf{.075}\markboth{0.075}{0.075} & 1.1885 & 1.1888 & 1.1890 & 1.1893 & 1.1896 & 1.1899 & 1.1901 & 1.1904 & 1.1907 & 1.1910 \\
 \textbf{.076}\markboth{0.076}{0.076} & 1.1912 & 1.1915 & 1.1918 & 1.1921 & 1.1923 & 1.1926 & 1.1929 & 1.1932 & 1.1934 & 1.1937 \\
\rowcolor{bg} \textbf{.077}\markboth{0.077}{0.077} & 1.1940 & 1.1943 & 1.1945 & 1.1948 & 1.1951 & 1.1954 & 1.1956 & 1.1959 & 1.1962 & 1.1965 \\
 \textbf{.078}\markboth{0.078}{0.078} & 1.1967 & 1.1970 & 1.1973 & 1.1976 & 1.1978 & 1.1981 & 1.1984 & 1.1987 & 1.1989 & 1.1992 \\
\rowcolor{bg} \textbf{.079}\markboth{0.079}{0.079} & 1.1995 & 1.1998 & 1.2001 & 1.2003 & 1.2006 & 1.2009 & 1.2012 & 1.2014 & 1.2017 & 1.2020 \\
 \textbf{.080}\markboth{0.080}{0.080} & 1.2023 & 1.2025 & 1.2028 & 1.2031 & 1.2034 & 1.2036 & 1.2039 & 1.2042 & 1.2045 & 1.2048 \\
\rowcolor{bg} \textbf{.081}\markboth{0.081}{0.081} & 1.2050 & 1.2053 & 1.2056 & 1.2059 & 1.2061 & 1.2064 & 1.2067 & 1.2070 & 1.2073 & 1.2075 \\
 \textbf{.082}\markboth{0.082}{0.082} & 1.2078 & 1.2081 & 1.2084 & 1.2086 & 1.2089 & 1.2092 & 1.2095 & 1.2098 & 1.2100 & 1.2103 \\
\rowcolor{bg} \textbf{.083}\markboth{0.083}{0.083} & 1.2106 & 1.2109 & 1.2112 & 1.2114 & 1.2117 & 1.2120 & 1.2123 & 1.2126 & 1.2128 & 1.2131 \\
 \textbf{.084}\markboth{0.084}{0.084} & 1.2134 & 1.2137 & 1.2139 & 1.2142 & 1.2145 & 1.2148 & 1.2151 & 1.2153 & 1.2156 & 1.2159 \\
\rowcolor{bg} \textbf{.085}\markboth{0.085}{0.085} & 1.2162 & 1.2165 & 1.2167 & 1.2170 & 1.2173 & 1.2176 & 1.2179 & 1.2181 & 1.2184 & 1.2187 \\
 \textbf{.086}\markboth{0.086}{0.086} & 1.2190 & 1.2193 & 1.2196 & 1.2198 & 1.2201 & 1.2204 & 1.2207 & 1.2210 & 1.2212 & 1.2215 \\
\rowcolor{bg} \textbf{.087}\markboth{0.087}{0.087} & 1.2218 & 1.2221 & 1.2224 & 1.2226 & 1.2229 & 1.2232 & 1.2235 & 1.2238 & 1.2241 & 1.2243 \\
 \textbf{.088}\markboth{0.088}{0.088} & 1.2246 & 1.2249 & 1.2252 & 1.2255 & 1.2257 & 1.2260 & 1.2263 & 1.2266 & 1.2269 & 1.2272 \\
\rowcolor{bg} \textbf{.089}\markboth{0.089}{0.089} & 1.2274 & 1.2277 & 1.2280 & 1.2283 & 1.2286 & 1.2289 & 1.2291 & 1.2294 & 1.2297 & 1.2300 \\
 \textbf{.090}\markboth{0.090}{0.090} & 1.2303 & 1.2306 & 1.2308 & 1.2311 & 1.2314 & 1.2317 & 1.2320 & 1.2323 & 1.2325 & 1.2328 \\
\rowcolor{bg} \textbf{.091}\markboth{0.091}{0.091} & 1.2331 & 1.2334 & 1.2337 & 1.2340 & 1.2342 & 1.2345 & 1.2348 & 1.2351 & 1.2354 & 1.2357 \\
 \textbf{.092}\markboth{0.092}{0.092} & 1.2359 & 1.2362 & 1.2365 & 1.2368 & 1.2371 & 1.2374 & 1.2377 & 1.2379 & 1.2382 & 1.2385 \\
\rowcolor{bg} \textbf{.093}\markboth{0.093}{0.093} & 1.2388 & 1.2391 & 1.2394 & 1.2397 & 1.2399 & 1.2402 & 1.2405 & 1.2408 & 1.2411 & 1.2414 \\
 \textbf{.094}\markboth{0.094}{0.094} & 1.2417 & 1.2419 & 1.2422 & 1.2425 & 1.2428 & 1.2431 & 1.2434 & 1.2437 & 1.2439 & 1.2442 \\
\rowcolor{bg} \textbf{.095}\markboth{0.095}{0.095} & 1.2445 & 1.2448 & 1.2451 & 1.2454 & 1.2457 & 1.2459 & 1.2462 & 1.2465 & 1.2468 & 1.2471 \\
 \textbf{.096}\markboth{0.096}{0.096} & 1.2474 & 1.2477 & 1.2480 & 1.2482 & 1.2485 & 1.2488 & 1.2491 & 1.2494 & 1.2497 & 1.2500 \\
\rowcolor{bg} \textbf{.097}\markboth{0.097}{0.097} & 1.2503 & 1.2505 & 1.2508 & 1.2511 & 1.2514 & 1.2517 & 1.2520 & 1.2523 & 1.2526 & 1.2529 \\
 \textbf{.098}\markboth{0.098}{0.098} & 1.2531 & 1.2534 & 1.2537 & 1.2540 & 1.2543 & 1.2546 & 1.2549 & 1.2552 & 1.2555 & 1.2557 \\
\rowcolor{bg} \textbf{.099}\markboth{0.099}{0.099} & 1.2560 & 1.2563 & 1.2566 & 1.2569 & 1.2572 & 1.2575 & 1.2578 & 1.2581 & 1.2583 & 1.2586 \\
 \textcolor{blue}{\textbf{.100}}\markboth{0.100}{0.100} & 1.2589 & 1.2592 & 1.2595 & 1.2598 & 1.2601 & 1.2604 & 1.2607 & 1.2610 & 1.2612 & 1.2615 \\
\rowcolor{bg} \textbf{.101}\markboth{0.101}{0.101} & 1.2618 & 1.2621 & 1.2624 & 1.2627 & 1.2630 & 1.2633 & 1.2636 & 1.2639 & 1.2642 & 1.2644 \\
 \textbf{.102}\markboth{0.102}{0.102} & 1.2647 & 1.2650 & 1.2653 & 1.2656 & 1.2659 & 1.2662 & 1.2665 & 1.2668 & 1.2671 & 1.2674 \\
\rowcolor{bg} \textbf{.103}\markboth{0.103}{0.103} & 1.2677 & 1.2679 & 1.2682 & 1.2685 & 1.2688 & 1.2691 & 1.2694 & 1.2697 & 1.2700 & 1.2703 \\
 \textbf{.104}\markboth{0.104}{0.104} & 1.2706 & 1.2709 & 1.2712 & 1.2715 & 1.2717 & 1.2720 & 1.2723 & 1.2726 & 1.2729 & 1.2732 \\
\rowcolor{bg} \textbf{.105}\markboth{0.105}{0.105} & 1.2735 & 1.2738 & 1.2741 & 1.2744 & 1.2747 & 1.2750 & 1.2753 & 1.2756 & 1.2759 & 1.2761 \\
 \textbf{.106}\markboth{0.106}{0.106} & 1.2764 & 1.2767 & 1.2770 & 1.2773 & 1.2776 & 1.2779 & 1.2782 & 1.2785 & 1.2788 & 1.2791 \\
\rowcolor{bg} \textbf{.107}\markboth{0.107}{0.107} & 1.2794 & 1.2797 & 1.2800 & 1.2803 & 1.2806 & 1.2809 & 1.2812 & 1.2814 & 1.2817 & 1.2820 \\
 \textbf{.108}\markboth{0.108}{0.108} & 1.2823 & 1.2826 & 1.2829 & 1.2832 & 1.2835 & 1.2838 & 1.2841 & 1.2844 & 1.2847 & 1.2850 \\
\rowcolor{bg} \textbf{.109}\markboth{0.109}{0.109} & 1.2853 & 1.2856 & 1.2859 & 1.2862 & 1.2865 & 1.2868 & 1.2871 & 1.2874 & 1.2877 & 1.2880 \\
 \textbf{.110}\markboth{0.110}{0.110} & 1.2882 & 1.2885 & 1.2888 & 1.2891 & 1.2894 & 1.2897 & 1.2900 & 1.2903 & 1.2906 & 1.2909 \\
\rowcolor{bg} \textbf{.111}\markboth{0.111}{0.111} & 1.2912 & 1.2915 & 1.2918 & 1.2921 & 1.2924 & 1.2927 & 1.2930 & 1.2933 & 1.2936 & 1.2939 \\
 \textbf{.112}\markboth{0.112}{0.112} & 1.2942 & 1.2945 & 1.2948 & 1.2951 & 1.2954 & 1.2957 & 1.2960 & 1.2963 & 1.2966 & 1.2969 \\
\rowcolor{bg} \textbf{.113}\markboth{0.113}{0.113} & 1.2972 & 1.2975 & 1.2978 & 1.2981 & 1.2984 & 1.2987 & 1.2990 & 1.2993 & 1.2996 & 1.2999 \\
 \textbf{.114}\markboth{0.114}{0.114} & 1.3002 & 1.3005 & 1.3008 & 1.3011 & 1.3014 & 1.3017 & 1.3020 & 1.3023 & 1.3026 & 1.3029 \\
\rowcolor{bg} \textbf{.115}\markboth{0.115}{0.115} & 1.3032 & 1.3035 & 1.3038 & 1.3041 & 1.3044 & 1.3047 & 1.3050 & 1.3053 & 1.3056 & 1.3059 \\
 \textbf{.116}\markboth{0.116}{0.116} & 1.3062 & 1.3065 & 1.3068 & 1.3071 & 1.3074 & 1.3077 & 1.3080 & 1.3083 & 1.3086 & 1.3089 \\
\rowcolor{bg} \textbf{.117}\markboth{0.117}{0.117} & 1.3092 & 1.3095 & 1.3098 & 1.3101 & 1.3104 & 1.3107 & 1.3110 & 1.3113 & 1.3116 & 1.3119 \\
 \textbf{.118}\markboth{0.118}{0.118} & 1.3122 & 1.3125 & 1.3128 & 1.3131 & 1.3134 & 1.3137 & 1.3140 & 1.3143 & 1.3146 & 1.3149 \\
\rowcolor{bg} \textbf{.119}\markboth{0.119}{0.119} & 1.3152 & 1.3155 & 1.3158 & 1.3161 & 1.3164 & 1.3167 & 1.3170 & 1.3173 & 1.3176 & 1.3180 \\
 \textbf{.120}\markboth{0.120}{0.120} & 1.3183 & 1.3186 & 1.3189 & 1.3192 & 1.3195 & 1.3198 & 1.3201 & 1.3204 & 1.3207 & 1.3210 \\
\rowcolor{bg} \textbf{.121}\markboth{0.121}{0.121} & 1.3213 & 1.3216 & 1.3219 & 1.3222 & 1.3225 & 1.3228 & 1.3231 & 1.3234 & 1.3237 & 1.3240 \\
 \textbf{.122}\markboth{0.122}{0.122} & 1.3243 & 1.3246 & 1.3250 & 1.3253 & 1.3256 & 1.3259 & 1.3262 & 1.3265 & 1.3268 & 1.3271 \\
\rowcolor{bg} \textbf{.123}\markboth{0.123}{0.123} & 1.3274 & 1.3277 & 1.3280 & 1.3283 & 1.3286 & 1.3289 & 1.3292 & 1.3295 & 1.3298 & 1.3301 \\
 \textbf{.124}\markboth{0.124}{0.124} & 1.3305 & 1.3308 & 1.3311 & 1.3314 & 1.3317 & 1.3320 & 1.3323 & 1.3326 & 1.3329 & 1.3332 \\
\rowcolor{bg} \textbf{.125}\markboth{0.125}{0.125} & 1.3335 & 1.3338 & 1.3341 & 1.3344 & 1.3348 & 1.3351 & 1.3354 & 1.3357 & 1.3360 & 1.3363 \\
 \textbf{.126}\markboth{0.126}{0.126} & 1.3366 & 1.3369 & 1.3372 & 1.3375 & 1.3378 & 1.3381 & 1.3384 & 1.3388 & 1.3391 & 1.3394 \\
\rowcolor{bg} \textbf{.127}\markboth{0.127}{0.127} & 1.3397 & 1.3400 & 1.3403 & 1.3406 & 1.3409 & 1.3412 & 1.3415 & 1.3418 & 1.3421 & 1.3425 \\
 \textbf{.128}\markboth{0.128}{0.128} & 1.3428 & 1.3431 & 1.3434 & 1.3437 & 1.3440 & 1.3443 & 1.3446 & 1.3449 & 1.3452 & 1.3456 \\
\rowcolor{bg} \textbf{.129}\markboth{0.129}{0.129} & 1.3459 & 1.3462 & 1.3465 & 1.3468 & 1.3471 & 1.3474 & 1.3477 & 1.3480 & 1.3483 & 1.3487 \\
 \textbf{.130}\markboth{0.130}{0.130} & 1.3490 & 1.3493 & 1.3496 & 1.3499 & 1.3502 & 1.3505 & 1.3508 & 1.3511 & 1.3515 & 1.3518 \\
\rowcolor{bg} \textbf{.131}\markboth{0.131}{0.131} & 1.3521 & 1.3524 & 1.3527 & 1.3530 & 1.3533 & 1.3536 & 1.3539 & 1.3543 & 1.3546 & 1.3549 \\
 \textbf{.132}\markboth{0.132}{0.132} & 1.3552 & 1.3555 & 1.3558 & 1.3561 & 1.3564 & 1.3568 & 1.3571 & 1.3574 & 1.3577 & 1.3580 \\
\rowcolor{bg} \textbf{.133}\markboth{0.133}{0.133} & 1.3583 & 1.3586 & 1.3589 & 1.3593 & 1.3596 & 1.3599 & 1.3602 & 1.3605 & 1.3608 & 1.3611 \\
 \textbf{.134}\markboth{0.134}{0.134} & 1.3614 & 1.3618 & 1.3621 & 1.3624 & 1.3627 & 1.3630 & 1.3633 & 1.3636 & 1.3640 & 1.3643 \\
\rowcolor{bg} \textbf{.135}\markboth{0.135}{0.135} & 1.3646 & 1.3649 & 1.3652 & 1.3655 & 1.3658 & 1.3662 & 1.3665 & 1.3668 & 1.3671 & 1.3674 \\
 \textbf{.136}\markboth{0.136}{0.136} & 1.3677 & 1.3680 & 1.3684 & 1.3687 & 1.3690 & 1.3693 & 1.3696 & 1.3699 & 1.3703 & 1.3706 \\
\rowcolor{bg} \textbf{.137}\markboth{0.137}{0.137} & 1.3709 & 1.3712 & 1.3715 & 1.3718 & 1.3721 & 1.3725 & 1.3728 & 1.3731 & 1.3734 & 1.3737 \\
 \textbf{.138}\markboth{0.138}{0.138} & 1.3740 & 1.3744 & 1.3747 & 1.3750 & 1.3753 & 1.3756 & 1.3759 & 1.3763 & 1.3766 & 1.3769 \\
\rowcolor{bg} \textbf{.139}\markboth{0.139}{0.139} & 1.3772 & 1.3775 & 1.3778 & 1.3782 & 1.3785 & 1.3788 & 1.3791 & 1.3794 & 1.3797 & 1.3801 \\
 \textbf{.140}\markboth{0.140}{0.140} & 1.3804 & 1.3807 & 1.3810 & 1.3813 & 1.3817 & 1.3820 & 1.3823 & 1.3826 & 1.3829 & 1.3832 \\
\rowcolor{bg} \textbf{.141}\markboth{0.141}{0.141} & 1.3836 & 1.3839 & 1.3842 & 1.3845 & 1.3848 & 1.3852 & 1.3855 & 1.3858 & 1.3861 & 1.3864 \\
 \textbf{.142}\markboth{0.142}{0.142} & 1.3868 & 1.3871 & 1.3874 & 1.3877 & 1.3880 & 1.3884 & 1.3887 & 1.3890 & 1.3893 & 1.3896 \\
\rowcolor{bg} \textbf{.143}\markboth{0.143}{0.143} & 1.3900 & 1.3903 & 1.3906 & 1.3909 & 1.3912 & 1.3916 & 1.3919 & 1.3922 & 1.3925 & 1.3928 \\
 \textbf{.144}\markboth{0.144}{0.144} & 1.3932 & 1.3935 & 1.3938 & 1.3941 & 1.3944 & 1.3948 & 1.3951 & 1.3954 & 1.3957 & 1.3960 \\
\rowcolor{bg} \textbf{.145}\markboth{0.145}{0.145} & 1.3964 & 1.3967 & 1.3970 & 1.3973 & 1.3977 & 1.3980 & 1.3983 & 1.3986 & 1.3989 & 1.3993 \\
 \textbf{.146}\markboth{0.146}{0.146} & 1.3996 & 1.3999 & 1.4002 & 1.4006 & 1.4009 & 1.4012 & 1.4015 & 1.4018 & 1.4022 & 1.4025 \\
\rowcolor{bg} \textbf{.147}\markboth{0.147}{0.147} & 1.4028 & 1.4031 & 1.4035 & 1.4038 & 1.4041 & 1.4044 & 1.4048 & 1.4051 & 1.4054 & 1.4057 \\
 \textbf{.148}\markboth{0.148}{0.148} & 1.4060 & 1.4064 & 1.4067 & 1.4070 & 1.4073 & 1.4077 & 1.4080 & 1.4083 & 1.4086 & 1.4090 \\
\rowcolor{bg} \textbf{.149}\markboth{0.149}{0.149} & 1.4093 & 1.4096 & 1.4099 & 1.4103 & 1.4106 & 1.4109 & 1.4112 & 1.4116 & 1.4119 & 1.4122 \\
 \textbf{.150}\markboth{0.150}{0.150} & 1.4125 & 1.4129 & 1.4132 & 1.4135 & 1.4138 & 1.4142 & 1.4145 & 1.4148 & 1.4151 & 1.4155 \\
\rowcolor{bg} \textbf{.151}\markboth{0.151}{0.151} & 1.4158 & 1.4161 & 1.4164 & 1.4168 & 1.4171 & 1.4174 & 1.4178 & 1.4181 & 1.4184 & 1.4187 \\
 \textbf{.152}\markboth{0.152}{0.152} & 1.4191 & 1.4194 & 1.4197 & 1.4200 & 1.4204 & 1.4207 & 1.4210 & 1.4213 & 1.4217 & 1.4220 \\
\rowcolor{bg} \textbf{.153}\markboth{0.153}{0.153} & 1.4223 & 1.4227 & 1.4230 & 1.4233 & 1.4236 & 1.4240 & 1.4243 & 1.4246 & 1.4250 & 1.4253 \\
 \textbf{.154}\markboth{0.154}{0.154} & 1.4256 & 1.4259 & 1.4263 & 1.4266 & 1.4269 & 1.4272 & 1.4276 & 1.4279 & 1.4282 & 1.4286 \\
\rowcolor{bg} \textbf{.155}\markboth{0.155}{0.155} & 1.4289 & 1.4292 & 1.4296 & 1.4299 & 1.4302 & 1.4305 & 1.4309 & 1.4312 & 1.4315 & 1.4319 \\
 \textbf{.156}\markboth{0.156}{0.156} & 1.4322 & 1.4325 & 1.4328 & 1.4332 & 1.4335 & 1.4338 & 1.4342 & 1.4345 & 1.4348 & 1.4352 \\
\rowcolor{bg} \textbf{.157}\markboth{0.157}{0.157} & 1.4355 & 1.4358 & 1.4362 & 1.4365 & 1.4368 & 1.4371 & 1.4375 & 1.4378 & 1.4381 & 1.4385 \\
 \textbf{.158}\markboth{0.158}{0.158} & 1.4388 & 1.4391 & 1.4395 & 1.4398 & 1.4401 & 1.4405 & 1.4408 & 1.4411 & 1.4415 & 1.4418 \\
\rowcolor{bg} \textbf{.159}\markboth{0.159}{0.159} & 1.4421 & 1.4424 & 1.4428 & 1.4431 & 1.4434 & 1.4438 & 1.4441 & 1.4444 & 1.4448 & 1.4451 \\
 \textbf{.160}\markboth{0.160}{0.160} & 1.4454 & 1.4458 & 1.4461 & 1.4464 & 1.4468 & 1.4471 & 1.4474 & 1.4478 & 1.4481 & 1.4484 \\
\rowcolor{bg} \textbf{.161}\markboth{0.161}{0.161} & 1.4488 & 1.4491 & 1.4494 & 1.4498 & 1.4501 & 1.4504 & 1.4508 & 1.4511 & 1.4514 & 1.4518 \\
 \textbf{.162}\markboth{0.162}{0.162} & 1.4521 & 1.4524 & 1.4528 & 1.4531 & 1.4534 & 1.4538 & 1.4541 & 1.4545 & 1.4548 & 1.4551 \\
\rowcolor{bg} \textbf{.163}\markboth{0.163}{0.163} & 1.4555 & 1.4558 & 1.4561 & 1.4565 & 1.4568 & 1.4571 & 1.4575 & 1.4578 & 1.4581 & 1.4585 \\
 \textbf{.164}\markboth{0.164}{0.164} & 1.4588 & 1.4592 & 1.4595 & 1.4598 & 1.4602 & 1.4605 & 1.4608 & 1.4612 & 1.4615 & 1.4618 \\
\rowcolor{bg} \textbf{.165}\markboth{0.165}{0.165} & 1.4622 & 1.4625 & 1.4629 & 1.4632 & 1.4635 & 1.4639 & 1.4642 & 1.4645 & 1.4649 & 1.4652 \\
 \textbf{.166}\markboth{0.166}{0.166} & 1.4655 & 1.4659 & 1.4662 & 1.4666 & 1.4669 & 1.4672 & 1.4676 & 1.4679 & 1.4682 & 1.4686 \\
\rowcolor{bg} \textbf{.167}\markboth{0.167}{0.167} & 1.4689 & 1.4693 & 1.4696 & 1.4699 & 1.4703 & 1.4706 & 1.4710 & 1.4713 & 1.4716 & 1.4720 \\
 \textbf{.168}\markboth{0.168}{0.168} & 1.4723 & 1.4727 & 1.4730 & 1.4733 & 1.4737 & 1.4740 & 1.4743 & 1.4747 & 1.4750 & 1.4754 \\
\rowcolor{bg} \textbf{.169}\markboth{0.169}{0.169} & 1.4757 & 1.4760 & 1.4764 & 1.4767 & 1.4771 & 1.4774 & 1.4777 & 1.4781 & 1.4784 & 1.4788 \\
 \textbf{.170}\markboth{0.170}{0.170} & 1.4791 & 1.4794 & 1.4798 & 1.4801 & 1.4805 & 1.4808 & 1.4812 & 1.4815 & 1.4818 & 1.4822 \\
\rowcolor{bg} \textbf{.171}\markboth{0.171}{0.171} & 1.4825 & 1.4829 & 1.4832 & 1.4835 & 1.4839 & 1.4842 & 1.4846 & 1.4849 & 1.4853 & 1.4856 \\
 \textbf{.172}\markboth{0.172}{0.172} & 1.4859 & 1.4863 & 1.4866 & 1.4870 & 1.4873 & 1.4876 & 1.4880 & 1.4883 & 1.4887 & 1.4890 \\
\rowcolor{bg} \textbf{.173}\markboth{0.173}{0.173} & 1.4894 & 1.4897 & 1.4900 & 1.4904 & 1.4907 & 1.4911 & 1.4914 & 1.4918 & 1.4921 & 1.4925 \\
 \textbf{.174}\markboth{0.174}{0.174} & 1.4928 & 1.4931 & 1.4935 & 1.4938 & 1.4942 & 1.4945 & 1.4949 & 1.4952 & 1.4955 & 1.4959 \\
\rowcolor{bg} \textbf{.175}\markboth{0.175}{0.175} & 1.4962 & 1.4966 & 1.4969 & 1.4973 & 1.4976 & 1.4980 & 1.4983 & 1.4986 & 1.4990 & 1.4993 \\
 \textbf{.176}\markboth{0.176}{0.176} & 1.4997 & 1.5000 & 1.5004 & 1.5007 & 1.5011 & 1.5014 & 1.5018 & 1.5021 & 1.5024 & 1.5028 \\
\rowcolor{bg} \textbf{.177}\markboth{0.177}{0.177} & 1.5031 & 1.5035 & 1.5038 & 1.5042 & 1.5045 & 1.5049 & 1.5052 & 1.5056 & 1.5059 & 1.5063 \\
 \textbf{.178}\markboth{0.178}{0.178} & 1.5066 & 1.5070 & 1.5073 & 1.5076 & 1.5080 & 1.5083 & 1.5087 & 1.5090 & 1.5094 & 1.5097 \\
\rowcolor{bg} \textbf{.179}\markboth{0.179}{0.179} & 1.5101 & 1.5104 & 1.5108 & 1.5111 & 1.5115 & 1.5118 & 1.5122 & 1.5125 & 1.5129 & 1.5132 \\
 \textbf{.180}\markboth{0.180}{0.180} & 1.5136 & 1.5139 & 1.5143 & 1.5146 & 1.5150 & 1.5153 & 1.5157 & 1.5160 & 1.5164 & 1.5167 \\
\rowcolor{bg} \textbf{.181}\markboth{0.181}{0.181} & 1.5171 & 1.5174 & 1.5177 & 1.5181 & 1.5184 & 1.5188 & 1.5191 & 1.5195 & 1.5198 & 1.5202 \\
 \textbf{.182}\markboth{0.182}{0.182} & 1.5205 & 1.5209 & 1.5212 & 1.5216 & 1.5219 & 1.5223 & 1.5226 & 1.5230 & 1.5234 & 1.5237 \\
\rowcolor{bg} \textbf{.183}\markboth{0.183}{0.183} & 1.5241 & 1.5244 & 1.5248 & 1.5251 & 1.5255 & 1.5258 & 1.5262 & 1.5265 & 1.5269 & 1.5272 \\
 \textbf{.184}\markboth{0.184}{0.184} & 1.5276 & 1.5279 & 1.5283 & 1.5286 & 1.5290 & 1.5293 & 1.5297 & 1.5300 & 1.5304 & 1.5307 \\
\rowcolor{bg} \textbf{.185}\markboth{0.185}{0.185} & 1.5311 & 1.5314 & 1.5318 & 1.5321 & 1.5325 & 1.5329 & 1.5332 & 1.5336 & 1.5339 & 1.5343 \\
 \textbf{.186}\markboth{0.186}{0.186} & 1.5346 & 1.5350 & 1.5353 & 1.5357 & 1.5360 & 1.5364 & 1.5367 & 1.5371 & 1.5374 & 1.5378 \\
\rowcolor{bg} \textbf{.187}\markboth{0.187}{0.187} & 1.5382 & 1.5385 & 1.5389 & 1.5392 & 1.5396 & 1.5399 & 1.5403 & 1.5406 & 1.5410 & 1.5413 \\
 \textbf{.188}\markboth{0.188}{0.188} & 1.5417 & 1.5421 & 1.5424 & 1.5428 & 1.5431 & 1.5435 & 1.5438 & 1.5442 & 1.5445 & 1.5449 \\
\rowcolor{bg} \textbf{.189}\markboth{0.189}{0.189} & 1.5453 & 1.5456 & 1.5460 & 1.5463 & 1.5467 & 1.5470 & 1.5474 & 1.5477 & 1.5481 & 1.5485 \\
 \textbf{.190}\markboth{0.190}{0.190} & 1.5488 & 1.5492 & 1.5495 & 1.5499 & 1.5502 & 1.5506 & 1.5510 & 1.5513 & 1.5517 & 1.5520 \\
\rowcolor{bg} \textbf{.191}\markboth{0.191}{0.191} & 1.5524 & 1.5527 & 1.5531 & 1.5535 & 1.5538 & 1.5542 & 1.5545 & 1.5549 & 1.5552 & 1.5556 \\
 \textbf{.192}\markboth{0.192}{0.192} & 1.5560 & 1.5563 & 1.5567 & 1.5570 & 1.5574 & 1.5578 & 1.5581 & 1.5585 & 1.5588 & 1.5592 \\
\rowcolor{bg} \textbf{.193}\markboth{0.193}{0.193} & 1.5596 & 1.5599 & 1.5603 & 1.5606 & 1.5610 & 1.5613 & 1.5617 & 1.5621 & 1.5624 & 1.5628 \\
 \textbf{.194}\markboth{0.194}{0.194} & 1.5631 & 1.5635 & 1.5639 & 1.5642 & 1.5646 & 1.5649 & 1.5653 & 1.5657 & 1.5660 & 1.5664 \\
\rowcolor{bg} \textbf{.195}\markboth{0.195}{0.195} & 1.5668 & 1.5671 & 1.5675 & 1.5678 & 1.5682 & 1.5686 & 1.5689 & 1.5693 & 1.5696 & 1.5700 \\
 \textbf{.196}\markboth{0.196}{0.196} & 1.5704 & 1.5707 & 1.5711 & 1.5714 & 1.5718 & 1.5722 & 1.5725 & 1.5729 & 1.5733 & 1.5736 \\
\rowcolor{bg} \textbf{.197}\markboth{0.197}{0.197} & 1.5740 & 1.5743 & 1.5747 & 1.5751 & 1.5754 & 1.5758 & 1.5762 & 1.5765 & 1.5769 & 1.5772 \\
 \textbf{.198}\markboth{0.198}{0.198} & 1.5776 & 1.5780 & 1.5783 & 1.5787 & 1.5791 & 1.5794 & 1.5798 & 1.5802 & 1.5805 & 1.5809 \\
\rowcolor{bg} \textbf{.199}\markboth{0.199}{0.199} & 1.5812 & 1.5816 & 1.5820 & 1.5823 & 1.5827 & 1.5831 & 1.5834 & 1.5838 & 1.5842 & 1.5845 \\
 \textcolor{blue}{\textbf{.200}}\markboth{0.200}{0.200} & 1.5849 & 1.5853 & 1.5856 & 1.5860 & 1.5864 & 1.5867 & 1.5871 & 1.5874 & 1.5878 & 1.5882 \\
\rowcolor{bg} \textbf{.201}\markboth{0.201}{0.201} & 1.5885 & 1.5889 & 1.5893 & 1.5896 & 1.5900 & 1.5904 & 1.5907 & 1.5911 & 1.5915 & 1.5918 \\
 \textbf{.202}\markboth{0.202}{0.202} & 1.5922 & 1.5926 & 1.5929 & 1.5933 & 1.5937 & 1.5940 & 1.5944 & 1.5948 & 1.5951 & 1.5955 \\
\rowcolor{bg} \textbf{.203}\markboth{0.203}{0.203} & 1.5959 & 1.5962 & 1.5966 & 1.5970 & 1.5973 & 1.5977 & 1.5981 & 1.5985 & 1.5988 & 1.5992 \\
 \textbf{.204}\markboth{0.204}{0.204} & 1.5996 & 1.5999 & 1.6003 & 1.6007 & 1.6010 & 1.6014 & 1.6018 & 1.6021 & 1.6025 & 1.6029 \\
\rowcolor{bg} \textbf{.205}\markboth{0.205}{0.205} & 1.6032 & 1.6036 & 1.6040 & 1.6044 & 1.6047 & 1.6051 & 1.6055 & 1.6058 & 1.6062 & 1.6066 \\
 \textbf{.206}\markboth{0.206}{0.206} & 1.6069 & 1.6073 & 1.6077 & 1.6081 & 1.6084 & 1.6088 & 1.6092 & 1.6095 & 1.6099 & 1.6103 \\
\rowcolor{bg} \textbf{.207}\markboth{0.207}{0.207} & 1.6106 & 1.6110 & 1.6114 & 1.6118 & 1.6121 & 1.6125 & 1.6129 & 1.6132 & 1.6136 & 1.6140 \\
 \textbf{.208}\markboth{0.208}{0.208} & 1.6144 & 1.6147 & 1.6151 & 1.6155 & 1.6158 & 1.6162 & 1.6166 & 1.6170 & 1.6173 & 1.6177 \\
\rowcolor{bg} \textbf{.209}\markboth{0.209}{0.209} & 1.6181 & 1.6185 & 1.6188 & 1.6192 & 1.6196 & 1.6199 & 1.6203 & 1.6207 & 1.6211 & 1.6214 \\
 \textbf{.210}\markboth{0.210}{0.210} & 1.6218 & 1.6222 & 1.6226 & 1.6229 & 1.6233 & 1.6237 & 1.6241 & 1.6244 & 1.6248 & 1.6252 \\
\rowcolor{bg} \textbf{.211}\markboth{0.211}{0.211} & 1.6255 & 1.6259 & 1.6263 & 1.6267 & 1.6270 & 1.6274 & 1.6278 & 1.6282 & 1.6285 & 1.6289 \\
 \textbf{.212}\markboth{0.212}{0.212} & 1.6293 & 1.6297 & 1.6300 & 1.6304 & 1.6308 & 1.6312 & 1.6315 & 1.6319 & 1.6323 & 1.6327 \\
\rowcolor{bg} \textbf{.213}\markboth{0.213}{0.213} & 1.6331 & 1.6334 & 1.6338 & 1.6342 & 1.6346 & 1.6349 & 1.6353 & 1.6357 & 1.6361 & 1.6364 \\
 \textbf{.214}\markboth{0.214}{0.214} & 1.6368 & 1.6372 & 1.6376 & 1.6379 & 1.6383 & 1.6387 & 1.6391 & 1.6395 & 1.6398 & 1.6402 \\
\rowcolor{bg} \textbf{.215}\markboth{0.215}{0.215} & 1.6406 & 1.6410 & 1.6413 & 1.6417 & 1.6421 & 1.6425 & 1.6429 & 1.6432 & 1.6436 & 1.6440 \\
 \textbf{.216}\markboth{0.216}{0.216} & 1.6444 & 1.6448 & 1.6451 & 1.6455 & 1.6459 & 1.6463 & 1.6466 & 1.6470 & 1.6474 & 1.6478 \\
\rowcolor{bg} \textbf{.217}\markboth{0.217}{0.217} & 1.6482 & 1.6485 & 1.6489 & 1.6493 & 1.6497 & 1.6501 & 1.6504 & 1.6508 & 1.6512 & 1.6516 \\
 \textbf{.218}\markboth{0.218}{0.218} & 1.6520 & 1.6523 & 1.6527 & 1.6531 & 1.6535 & 1.6539 & 1.6542 & 1.6546 & 1.6550 & 1.6554 \\
\rowcolor{bg} \textbf{.219}\markboth{0.219}{0.219} & 1.6558 & 1.6562 & 1.6565 & 1.6569 & 1.6573 & 1.6577 & 1.6581 & 1.6584 & 1.6588 & 1.6592 \\
 \textbf{.220}\markboth{0.220}{0.220} & 1.6596 & 1.6600 & 1.6604 & 1.6607 & 1.6611 & 1.6615 & 1.6619 & 1.6623 & 1.6626 & 1.6630 \\
\rowcolor{bg} \textbf{.221}\markboth{0.221}{0.221} & 1.6634 & 1.6638 & 1.6642 & 1.6646 & 1.6649 & 1.6653 & 1.6657 & 1.6661 & 1.6665 & 1.6669 \\
 \textbf{.222}\markboth{0.222}{0.222} & 1.6672 & 1.6676 & 1.6680 & 1.6684 & 1.6688 & 1.6692 & 1.6696 & 1.6699 & 1.6703 & 1.6707 \\
\rowcolor{bg} \textbf{.223}\markboth{0.223}{0.223} & 1.6711 & 1.6715 & 1.6719 & 1.6722 & 1.6726 & 1.6730 & 1.6734 & 1.6738 & 1.6742 & 1.6746 \\
 \textbf{.224}\markboth{0.224}{0.224} & 1.6749 & 1.6753 & 1.6757 & 1.6761 & 1.6765 & 1.6769 & 1.6773 & 1.6776 & 1.6780 & 1.6784 \\
\rowcolor{bg} \textbf{.225}\markboth{0.225}{0.225} & 1.6788 & 1.6792 & 1.6796 & 1.6800 & 1.6804 & 1.6807 & 1.6811 & 1.6815 & 1.6819 & 1.6823 \\
 \textbf{.226}\markboth{0.226}{0.226} & 1.6827 & 1.6831 & 1.6834 & 1.6838 & 1.6842 & 1.6846 & 1.6850 & 1.6854 & 1.6858 & 1.6862 \\
\rowcolor{bg} \textbf{.227}\markboth{0.227}{0.227} & 1.6866 & 1.6869 & 1.6873 & 1.6877 & 1.6881 & 1.6885 & 1.6889 & 1.6893 & 1.6897 & 1.6901 \\
 \textbf{.228}\markboth{0.228}{0.228} & 1.6904 & 1.6908 & 1.6912 & 1.6916 & 1.6920 & 1.6924 & 1.6928 & 1.6932 & 1.6936 & 1.6939 \\
\rowcolor{bg} \textbf{.229}\markboth{0.229}{0.229} & 1.6943 & 1.6947 & 1.6951 & 1.6955 & 1.6959 & 1.6963 & 1.6967 & 1.6971 & 1.6975 & 1.6979 \\
 \textbf{.230}\markboth{0.230}{0.230} & 1.6982 & 1.6986 & 1.6990 & 1.6994 & 1.6998 & 1.7002 & 1.7006 & 1.7010 & 1.7014 & 1.7018 \\
\rowcolor{bg} \textbf{.231}\markboth{0.231}{0.231} & 1.7022 & 1.7026 & 1.7029 & 1.7033 & 1.7037 & 1.7041 & 1.7045 & 1.7049 & 1.7053 & 1.7057 \\
 \textbf{.232}\markboth{0.232}{0.232} & 1.7061 & 1.7065 & 1.7069 & 1.7073 & 1.7077 & 1.7080 & 1.7084 & 1.7088 & 1.7092 & 1.7096 \\
\rowcolor{bg} \textbf{.233}\markboth{0.233}{0.233} & 1.7100 & 1.7104 & 1.7108 & 1.7112 & 1.7116 & 1.7120 & 1.7124 & 1.7128 & 1.7132 & 1.7136 \\
 \textbf{.234}\markboth{0.234}{0.234} & 1.7140 & 1.7144 & 1.7147 & 1.7151 & 1.7155 & 1.7159 & 1.7163 & 1.7167 & 1.7171 & 1.7175 \\
\rowcolor{bg} \textbf{.235}\markboth{0.235}{0.235} & 1.7179 & 1.7183 & 1.7187 & 1.7191 & 1.7195 & 1.7199 & 1.7203 & 1.7207 & 1.7211 & 1.7215 \\
 \textbf{.236}\markboth{0.236}{0.236} & 1.7219 & 1.7223 & 1.7227 & 1.7231 & 1.7235 & 1.7239 & 1.7242 & 1.7246 & 1.7250 & 1.7254 \\
\rowcolor{bg} \textbf{.237}\markboth{0.237}{0.237} & 1.7258 & 1.7262 & 1.7266 & 1.7270 & 1.7274 & 1.7278 & 1.7282 & 1.7286 & 1.7290 & 1.7294 \\
 \textbf{.238}\markboth{0.238}{0.238} & 1.7298 & 1.7302 & 1.7306 & 1.7310 & 1.7314 & 1.7318 & 1.7322 & 1.7326 & 1.7330 & 1.7334 \\
\rowcolor{bg} \textbf{.239}\markboth{0.239}{0.239} & 1.7338 & 1.7342 & 1.7346 & 1.7350 & 1.7354 & 1.7358 & 1.7362 & 1.7366 & 1.7370 & 1.7374 \\
 \textbf{.240}\markboth{0.240}{0.240} & 1.7378 & 1.7382 & 1.7386 & 1.7390 & 1.7394 & 1.7398 & 1.7402 & 1.7406 & 1.7410 & 1.7414 \\
\rowcolor{bg} \textbf{.241}\markboth{0.241}{0.241} & 1.7418 & 1.7422 & 1.7426 & 1.7430 & 1.7434 & 1.7438 & 1.7442 & 1.7446 & 1.7450 & 1.7454 \\
 \textbf{.242}\markboth{0.242}{0.242} & 1.7458 & 1.7462 & 1.7466 & 1.7470 & 1.7474 & 1.7478 & 1.7482 & 1.7486 & 1.7490 & 1.7494 \\
\rowcolor{bg} \textbf{.243}\markboth{0.243}{0.243} & 1.7498 & 1.7502 & 1.7507 & 1.7511 & 1.7515 & 1.7519 & 1.7523 & 1.7527 & 1.7531 & 1.7535 \\
 \textbf{.244}\markboth{0.244}{0.244} & 1.7539 & 1.7543 & 1.7547 & 1.7551 & 1.7555 & 1.7559 & 1.7563 & 1.7567 & 1.7571 & 1.7575 \\
\rowcolor{bg} \textbf{.245}\markboth{0.245}{0.245} & 1.7579 & 1.7583 & 1.7587 & 1.7591 & 1.7595 & 1.7599 & 1.7604 & 1.7608 & 1.7612 & 1.7616 \\
 \textbf{.246}\markboth{0.246}{0.246} & 1.7620 & 1.7624 & 1.7628 & 1.7632 & 1.7636 & 1.7640 & 1.7644 & 1.7648 & 1.7652 & 1.7656 \\
\rowcolor{bg} \textbf{.247}\markboth{0.247}{0.247} & 1.7660 & 1.7664 & 1.7669 & 1.7673 & 1.7677 & 1.7681 & 1.7685 & 1.7689 & 1.7693 & 1.7697 \\
 \textbf{.248}\markboth{0.248}{0.248} & 1.7701 & 1.7705 & 1.7709 & 1.7713 & 1.7717 & 1.7721 & 1.7726 & 1.7730 & 1.7734 & 1.7738 \\
\rowcolor{bg} \textbf{.249}\markboth{0.249}{0.249} & 1.7742 & 1.7746 & 1.7750 & 1.7754 & 1.7758 & 1.7762 & 1.7766 & 1.7771 & 1.7775 & 1.7779 \\
 \textbf{.250}\markboth{0.250}{0.250} & 1.7783 & 1.7787 & 1.7791 & 1.7795 & 1.7799 & 1.7803 & 1.7807 & 1.7811 & 1.7816 & 1.7820 \\
\rowcolor{bg} \textbf{.251}\markboth{0.251}{0.251} & 1.7824 & 1.7828 & 1.7832 & 1.7836 & 1.7840 & 1.7844 & 1.7848 & 1.7853 & 1.7857 & 1.7861 \\
 \textbf{.252}\markboth{0.252}{0.252} & 1.7865 & 1.7869 & 1.7873 & 1.7877 & 1.7881 & 1.7885 & 1.7890 & 1.7894 & 1.7898 & 1.7902 \\
\rowcolor{bg} \textbf{.253}\markboth{0.253}{0.253} & 1.7906 & 1.7910 & 1.7914 & 1.7918 & 1.7923 & 1.7927 & 1.7931 & 1.7935 & 1.7939 & 1.7943 \\
 \textbf{.254}\markboth{0.254}{0.254} & 1.7947 & 1.7951 & 1.7956 & 1.7960 & 1.7964 & 1.7968 & 1.7972 & 1.7976 & 1.7980 & 1.7985 \\
\rowcolor{bg} \textbf{.255}\markboth{0.255}{0.255} & 1.7989 & 1.7993 & 1.7997 & 1.8001 & 1.8005 & 1.8009 & 1.8014 & 1.8018 & 1.8022 & 1.8026 \\
 \textbf{.256}\markboth{0.256}{0.256} & 1.8030 & 1.8034 & 1.8038 & 1.8043 & 1.8047 & 1.8051 & 1.8055 & 1.8059 & 1.8063 & 1.8068 \\
\rowcolor{bg} \textbf{.257}\markboth{0.257}{0.257} & 1.8072 & 1.8076 & 1.8080 & 1.8084 & 1.8088 & 1.8093 & 1.8097 & 1.8101 & 1.8105 & 1.8109 \\
 \textbf{.258}\markboth{0.258}{0.258} & 1.8113 & 1.8118 & 1.8122 & 1.8126 & 1.8130 & 1.8134 & 1.8138 & 1.8143 & 1.8147 & 1.8151 \\
\rowcolor{bg} \textbf{.259}\markboth{0.259}{0.259} & 1.8155 & 1.8159 & 1.8164 & 1.8168 & 1.8172 & 1.8176 & 1.8180 & 1.8184 & 1.8189 & 1.8193 \\
 \textbf{.260}\markboth{0.260}{0.260} & 1.8197 & 1.8201 & 1.8205 & 1.8210 & 1.8214 & 1.8218 & 1.8222 & 1.8226 & 1.8231 & 1.8235 \\
\rowcolor{bg} \textbf{.261}\markboth{0.261}{0.261} & 1.8239 & 1.8243 & 1.8247 & 1.8252 & 1.8256 & 1.8260 & 1.8264 & 1.8268 & 1.8273 & 1.8277 \\
 \textbf{.262}\markboth{0.262}{0.262} & 1.8281 & 1.8285 & 1.8289 & 1.8294 & 1.8298 & 1.8302 & 1.8306 & 1.8310 & 1.8315 & 1.8319 \\
\rowcolor{bg} \textbf{.263}\markboth{0.263}{0.263} & 1.8323 & 1.8327 & 1.8332 & 1.8336 & 1.8340 & 1.8344 & 1.8348 & 1.8353 & 1.8357 & 1.8361 \\
 \textbf{.264}\markboth{0.264}{0.264} & 1.8365 & 1.8370 & 1.8374 & 1.8378 & 1.8382 & 1.8387 & 1.8391 & 1.8395 & 1.8399 & 1.8403 \\
\rowcolor{bg} \textbf{.265}\markboth{0.265}{0.265} & 1.8408 & 1.8412 & 1.8416 & 1.8420 & 1.8425 & 1.8429 & 1.8433 & 1.8437 & 1.8442 & 1.8446 \\
 \textbf{.266}\markboth{0.266}{0.266} & 1.8450 & 1.8454 & 1.8459 & 1.8463 & 1.8467 & 1.8471 & 1.8476 & 1.8480 & 1.8484 & 1.8488 \\
\rowcolor{bg} \textbf{.267}\markboth{0.267}{0.267} & 1.8493 & 1.8497 & 1.8501 & 1.8505 & 1.8510 & 1.8514 & 1.8518 & 1.8523 & 1.8527 & 1.8531 \\
 \textbf{.268}\markboth{0.268}{0.268} & 1.8535 & 1.8540 & 1.8544 & 1.8548 & 1.8552 & 1.8557 & 1.8561 & 1.8565 & 1.8569 & 1.8574 \\
\rowcolor{bg} \textbf{.269}\markboth{0.269}{0.269} & 1.8578 & 1.8582 & 1.8587 & 1.8591 & 1.8595 & 1.8599 & 1.8604 & 1.8608 & 1.8612 & 1.8617 \\
 \textbf{.270}\markboth{0.270}{0.270} & 1.8621 & 1.8625 & 1.8629 & 1.8634 & 1.8638 & 1.8642 & 1.8647 & 1.8651 & 1.8655 & 1.8659 \\
\rowcolor{bg} \textbf{.271}\markboth{0.271}{0.271} & 1.8664 & 1.8668 & 1.8672 & 1.8677 & 1.8681 & 1.8685 & 1.8690 & 1.8694 & 1.8698 & 1.8703 \\
 \textbf{.272}\markboth{0.272}{0.272} & 1.8707 & 1.8711 & 1.8715 & 1.8720 & 1.8724 & 1.8728 & 1.8733 & 1.8737 & 1.8741 & 1.8746 \\
\rowcolor{bg} \textbf{.273}\markboth{0.273}{0.273} & 1.8750 & 1.8754 & 1.8759 & 1.8763 & 1.8767 & 1.8772 & 1.8776 & 1.8780 & 1.8785 & 1.8789 \\
 \textbf{.274}\markboth{0.274}{0.274} & 1.8793 & 1.8797 & 1.8802 & 1.8806 & 1.8810 & 1.8815 & 1.8819 & 1.8823 & 1.8828 & 1.8832 \\
\rowcolor{bg} \textbf{.275}\markboth{0.275}{0.275} & 1.8836 & 1.8841 & 1.8845 & 1.8850 & 1.8854 & 1.8858 & 1.8863 & 1.8867 & 1.8871 & 1.8876 \\
 \textbf{.276}\markboth{0.276}{0.276} & 1.8880 & 1.8884 & 1.8889 & 1.8893 & 1.8897 & 1.8902 & 1.8906 & 1.8910 & 1.8915 & 1.8919 \\
\rowcolor{bg} \textbf{.277}\markboth{0.277}{0.277} & 1.8923 & 1.8928 & 1.8932 & 1.8937 & 1.8941 & 1.8945 & 1.8950 & 1.8954 & 1.8958 & 1.8963 \\
 \textbf{.278}\markboth{0.278}{0.278} & 1.8967 & 1.8971 & 1.8976 & 1.8980 & 1.8985 & 1.8989 & 1.8993 & 1.8998 & 1.9002 & 1.9006 \\
\rowcolor{bg} \textbf{.279}\markboth{0.279}{0.279} & 1.9011 & 1.9015 & 1.9020 & 1.9024 & 1.9028 & 1.9033 & 1.9037 & 1.9041 & 1.9046 & 1.9050 \\
 \textbf{.280}\markboth{0.280}{0.280} & 1.9055 & 1.9059 & 1.9063 & 1.9068 & 1.9072 & 1.9077 & 1.9081 & 1.9085 & 1.9090 & 1.9094 \\
\rowcolor{bg} \textbf{.281}\markboth{0.281}{0.281} & 1.9099 & 1.9103 & 1.9107 & 1.9112 & 1.9116 & 1.9121 & 1.9125 & 1.9129 & 1.9134 & 1.9138 \\
 \textbf{.282}\markboth{0.282}{0.282} & 1.9143 & 1.9147 & 1.9151 & 1.9156 & 1.9160 & 1.9165 & 1.9169 & 1.9173 & 1.9178 & 1.9182 \\
\rowcolor{bg} \textbf{.283}\markboth{0.283}{0.283} & 1.9187 & 1.9191 & 1.9196 & 1.9200 & 1.9204 & 1.9209 & 1.9213 & 1.9218 & 1.9222 & 1.9226 \\
 \textbf{.284}\markboth{0.284}{0.284} & 1.9231 & 1.9235 & 1.9240 & 1.9244 & 1.9249 & 1.9253 & 1.9258 & 1.9262 & 1.9266 & 1.9271 \\
\rowcolor{bg} \textbf{.285}\markboth{0.285}{0.285} & 1.9275 & 1.9280 & 1.9284 & 1.9289 & 1.9293 & 1.9297 & 1.9302 & 1.9306 & 1.9311 & 1.9315 \\
 \textbf{.286}\markboth{0.286}{0.286} & 1.9320 & 1.9324 & 1.9329 & 1.9333 & 1.9337 & 1.9342 & 1.9346 & 1.9351 & 1.9355 & 1.9360 \\
\rowcolor{bg} \textbf{.287}\markboth{0.287}{0.287} & 1.9364 & 1.9369 & 1.9373 & 1.9378 & 1.9382 & 1.9387 & 1.9391 & 1.9395 & 1.9400 & 1.9404 \\
 \textbf{.288}\markboth{0.288}{0.288} & 1.9409 & 1.9413 & 1.9418 & 1.9422 & 1.9427 & 1.9431 & 1.9436 & 1.9440 & 1.9445 & 1.9449 \\
\rowcolor{bg} \textbf{.289}\markboth{0.289}{0.289} & 1.9454 & 1.9458 & 1.9463 & 1.9467 & 1.9472 & 1.9476 & 1.9480 & 1.9485 & 1.9489 & 1.9494 \\
 \textbf{.290}\markboth{0.290}{0.290} & 1.9498 & 1.9503 & 1.9507 & 1.9512 & 1.9516 & 1.9521 & 1.9525 & 1.9530 & 1.9534 & 1.9539 \\
\rowcolor{bg} \textbf{.291}\markboth{0.291}{0.291} & 1.9543 & 1.9548 & 1.9552 & 1.9557 & 1.9561 & 1.9566 & 1.9570 & 1.9575 & 1.9579 & 1.9584 \\
 \textbf{.292}\markboth{0.292}{0.292} & 1.9588 & 1.9593 & 1.9597 & 1.9602 & 1.9606 & 1.9611 & 1.9616 & 1.9620 & 1.9625 & 1.9629 \\
\rowcolor{bg} \textbf{.293}\markboth{0.293}{0.293} & 1.9634 & 1.9638 & 1.9643 & 1.9647 & 1.9652 & 1.9656 & 1.9661 & 1.9665 & 1.9670 & 1.9674 \\
 \textbf{.294}\markboth{0.294}{0.294} & 1.9679 & 1.9683 & 1.9688 & 1.9692 & 1.9697 & 1.9702 & 1.9706 & 1.9711 & 1.9715 & 1.9720 \\
\rowcolor{bg} \textbf{.295}\markboth{0.295}{0.295} & 1.9724 & 1.9729 & 1.9733 & 1.9738 & 1.9742 & 1.9747 & 1.9751 & 1.9756 & 1.9761 & 1.9765 \\
 \textbf{.296}\markboth{0.296}{0.296} & 1.9770 & 1.9774 & 1.9779 & 1.9783 & 1.9788 & 1.9792 & 1.9797 & 1.9802 & 1.9806 & 1.9811 \\
\rowcolor{bg} \textbf{.297}\markboth{0.297}{0.297} & 1.9815 & 1.9820 & 1.9824 & 1.9829 & 1.9834 & 1.9838 & 1.9843 & 1.9847 & 1.9852 & 1.9856 \\
 \textbf{.298}\markboth{0.298}{0.298} & 1.9861 & 1.9866 & 1.9870 & 1.9875 & 1.9879 & 1.9884 & 1.9888 & 1.9893 & 1.9898 & 1.9902 \\
\rowcolor{bg} \textbf{.299}\markboth{0.299}{0.299} & 1.9907 & 1.9911 & 1.9916 & 1.9920 & 1.9925 & 1.9930 & 1.9934 & 1.9939 & 1.9943 & 1.9948 \\
 \textcolor{blue}{\textbf{.300}}\markboth{0.300}{0.300} & 1.9953 & 1.9957 & 1.9962 & 1.9966 & 1.9971 & 1.9976 & 1.9980 & 1.9985 & 1.9989 & 1.9994 \\
\rowcolor{bg} \textbf{.301}\markboth{0.301}{0.301} & 1.9999 & 2.0003 & 2.0008 & 2.0012 & 2.0017 & 2.0022 & 2.0026 & 2.0031 & 2.0035 & 2.0040 \\
 \textbf{.302}\markboth{0.302}{0.302} & 2.0045 & 2.0049 & 2.0054 & 2.0059 & 2.0063 & 2.0068 & 2.0072 & 2.0077 & 2.0082 & 2.0086 \\
\rowcolor{bg} \textbf{.303}\markboth{0.303}{0.303} & 2.0091 & 2.0096 & 2.0100 & 2.0105 & 2.0109 & 2.0114 & 2.0119 & 2.0123 & 2.0128 & 2.0133 \\
 \textbf{.304}\markboth{0.304}{0.304} & 2.0137 & 2.0142 & 2.0147 & 2.0151 & 2.0156 & 2.0160 & 2.0165 & 2.0170 & 2.0174 & 2.0179 \\
\rowcolor{bg} \textbf{.305}\markboth{0.305}{0.305} & 2.0184 & 2.0188 & 2.0193 & 2.0198 & 2.0202 & 2.0207 & 2.0212 & 2.0216 & 2.0221 & 2.0226 \\
 \textbf{.306}\markboth{0.306}{0.306} & 2.0230 & 2.0235 & 2.0240 & 2.0244 & 2.0249 & 2.0253 & 2.0258 & 2.0263 & 2.0267 & 2.0272 \\
\rowcolor{bg} \textbf{.307}\markboth{0.307}{0.307} & 2.0277 & 2.0281 & 2.0286 & 2.0291 & 2.0296 & 2.0300 & 2.0305 & 2.0310 & 2.0314 & 2.0319 \\
 \textbf{.308}\markboth{0.308}{0.308} & 2.0324 & 2.0328 & 2.0333 & 2.0338 & 2.0342 & 2.0347 & 2.0352 & 2.0356 & 2.0361 & 2.0366 \\
\rowcolor{bg} \textbf{.309}\markboth{0.309}{0.309} & 2.0370 & 2.0375 & 2.0380 & 2.0384 & 2.0389 & 2.0394 & 2.0399 & 2.0403 & 2.0408 & 2.0413 \\
 \textbf{.310}\markboth{0.310}{0.310} & 2.0417 & 2.0422 & 2.0427 & 2.0431 & 2.0436 & 2.0441 & 2.0446 & 2.0450 & 2.0455 & 2.0460 \\
\rowcolor{bg} \textbf{.311}\markboth{0.311}{0.311} & 2.0464 & 2.0469 & 2.0474 & 2.0479 & 2.0483 & 2.0488 & 2.0493 & 2.0497 & 2.0502 & 2.0507 \\
 \textbf{.312}\markboth{0.312}{0.312} & 2.0512 & 2.0516 & 2.0521 & 2.0526 & 2.0531 & 2.0535 & 2.0540 & 2.0545 & 2.0549 & 2.0554 \\
\rowcolor{bg} \textbf{.313}\markboth{0.313}{0.313} & 2.0559 & 2.0564 & 2.0568 & 2.0573 & 2.0578 & 2.0583 & 2.0587 & 2.0592 & 2.0597 & 2.0602 \\
 \textbf{.314}\markboth{0.314}{0.314} & 2.0606 & 2.0611 & 2.0616 & 2.0621 & 2.0625 & 2.0630 & 2.0635 & 2.0640 & 2.0644 & 2.0649 \\
\rowcolor{bg} \textbf{.315}\markboth{0.315}{0.315} & 2.0654 & 2.0659 & 2.0663 & 2.0668 & 2.0673 & 2.0678 & 2.0682 & 2.0687 & 2.0692 & 2.0697 \\
 \textbf{.316}\markboth{0.316}{0.316} & 2.0701 & 2.0706 & 2.0711 & 2.0716 & 2.0720 & 2.0725 & 2.0730 & 2.0735 & 2.0740 & 2.0744 \\
\rowcolor{bg} \textbf{.317}\markboth{0.317}{0.317} & 2.0749 & 2.0754 & 2.0759 & 2.0763 & 2.0768 & 2.0773 & 2.0778 & 2.0783 & 2.0787 & 2.0792 \\
 \textbf{.318}\markboth{0.318}{0.318} & 2.0797 & 2.0802 & 2.0807 & 2.0811 & 2.0816 & 2.0821 & 2.0826 & 2.0831 & 2.0835 & 2.0840 \\
\rowcolor{bg} \textbf{.319}\markboth{0.319}{0.319} & 2.0845 & 2.0850 & 2.0855 & 2.0859 & 2.0864 & 2.0869 & 2.0874 & 2.0879 & 2.0883 & 2.0888 \\
 \textbf{.320}\markboth{0.320}{0.320} & 2.0893 & 2.0898 & 2.0903 & 2.0907 & 2.0912 & 2.0917 & 2.0922 & 2.0927 & 2.0931 & 2.0936 \\
\rowcolor{bg} \textbf{.321}\markboth{0.321}{0.321} & 2.0941 & 2.0946 & 2.0951 & 2.0956 & 2.0960 & 2.0965 & 2.0970 & 2.0975 & 2.0980 & 2.0985 \\
 \textbf{.322}\markboth{0.322}{0.322} & 2.0989 & 2.0994 & 2.0999 & 2.1004 & 2.1009 & 2.1014 & 2.1018 & 2.1023 & 2.1028 & 2.1033 \\
\rowcolor{bg} \textbf{.323}\markboth{0.323}{0.323} & 2.1038 & 2.1043 & 2.1047 & 2.1052 & 2.1057 & 2.1062 & 2.1067 & 2.1072 & 2.1077 & 2.1081 \\
 \textbf{.324}\markboth{0.324}{0.324} & 2.1086 & 2.1091 & 2.1096 & 2.1101 & 2.1106 & 2.1111 & 2.1115 & 2.1120 & 2.1125 & 2.1130 \\
\rowcolor{bg} \textbf{.325}\markboth{0.325}{0.325} & 2.1135 & 2.1140 & 2.1145 & 2.1149 & 2.1154 & 2.1159 & 2.1164 & 2.1169 & 2.1174 & 2.1179 \\
 \textbf{.326}\markboth{0.326}{0.326} & 2.1184 & 2.1188 & 2.1193 & 2.1198 & 2.1203 & 2.1208 & 2.1213 & 2.1218 & 2.1223 & 2.1228 \\
\rowcolor{bg} \textbf{.327}\markboth{0.327}{0.327} & 2.1232 & 2.1237 & 2.1242 & 2.1247 & 2.1252 & 2.1257 & 2.1262 & 2.1267 & 2.1272 & 2.1276 \\
 \textbf{.328}\markboth{0.328}{0.328} & 2.1281 & 2.1286 & 2.1291 & 2.1296 & 2.1301 & 2.1306 & 2.1311 & 2.1316 & 2.1321 & 2.1326 \\
\rowcolor{bg} \textbf{.329}\markboth{0.329}{0.329} & 2.1330 & 2.1335 & 2.1340 & 2.1345 & 2.1350 & 2.1355 & 2.1360 & 2.1365 & 2.1370 & 2.1375 \\
 \textbf{.330}\markboth{0.330}{0.330} & 2.1380 & 2.1385 & 2.1389 & 2.1394 & 2.1399 & 2.1404 & 2.1409 & 2.1414 & 2.1419 & 2.1424 \\
\rowcolor{bg} \textbf{.331}\markboth{0.331}{0.331} & 2.1429 & 2.1434 & 2.1439 & 2.1444 & 2.1449 & 2.1454 & 2.1459 & 2.1463 & 2.1468 & 2.1473 \\
 \textbf{.332}\markboth{0.332}{0.332} & 2.1478 & 2.1483 & 2.1488 & 2.1493 & 2.1498 & 2.1503 & 2.1508 & 2.1513 & 2.1518 & 2.1523 \\
\rowcolor{bg} \textbf{.333}\markboth{0.333}{0.333} & 2.1528 & 2.1533 & 2.1538 & 2.1543 & 2.1548 & 2.1553 & 2.1558 & 2.1563 & 2.1568 & 2.1572 \\
 \textbf{.334}\markboth{0.334}{0.334} & 2.1577 & 2.1582 & 2.1587 & 2.1592 & 2.1597 & 2.1602 & 2.1607 & 2.1612 & 2.1617 & 2.1622 \\
\rowcolor{bg} \textbf{.335}\markboth{0.335}{0.335} & 2.1627 & 2.1632 & 2.1637 & 2.1642 & 2.1647 & 2.1652 & 2.1657 & 2.1662 & 2.1667 & 2.1672 \\
 \textbf{.336}\markboth{0.336}{0.336} & 2.1677 & 2.1682 & 2.1687 & 2.1692 & 2.1697 & 2.1702 & 2.1707 & 2.1712 & 2.1717 & 2.1722 \\
\rowcolor{bg} \textbf{.337}\markboth{0.337}{0.337} & 2.1727 & 2.1732 & 2.1737 & 2.1742 & 2.1747 & 2.1752 & 2.1757 & 2.1762 & 2.1767 & 2.1772 \\
 \textbf{.338}\markboth{0.338}{0.338} & 2.1777 & 2.1782 & 2.1787 & 2.1792 & 2.1797 & 2.1802 & 2.1807 & 2.1812 & 2.1817 & 2.1822 \\
\rowcolor{bg} \textbf{.339}\markboth{0.339}{0.339} & 2.1827 & 2.1832 & 2.1837 & 2.1842 & 2.1847 & 2.1852 & 2.1857 & 2.1863 & 2.1868 & 2.1873 \\
 \textbf{.340}\markboth{0.340}{0.340} & 2.1878 & 2.1883 & 2.1888 & 2.1893 & 2.1898 & 2.1903 & 2.1908 & 2.1913 & 2.1918 & 2.1923 \\
\rowcolor{bg} \textbf{.341}\markboth{0.341}{0.341} & 2.1928 & 2.1933 & 2.1938 & 2.1943 & 2.1948 & 2.1953 & 2.1958 & 2.1963 & 2.1968 & 2.1974 \\
 \textbf{.342}\markboth{0.342}{0.342} & 2.1979 & 2.1984 & 2.1989 & 2.1994 & 2.1999 & 2.2004 & 2.2009 & 2.2014 & 2.2019 & 2.2024 \\
\rowcolor{bg} \textbf{.343}\markboth{0.343}{0.343} & 2.2029 & 2.2034 & 2.2039 & 2.2044 & 2.2050 & 2.2055 & 2.2060 & 2.2065 & 2.2070 & 2.2075 \\
 \textbf{.344}\markboth{0.344}{0.344} & 2.2080 & 2.2085 & 2.2090 & 2.2095 & 2.2100 & 2.2105 & 2.2111 & 2.2116 & 2.2121 & 2.2126 \\
\rowcolor{bg} \textbf{.345}\markboth{0.345}{0.345} & 2.2131 & 2.2136 & 2.2141 & 2.2146 & 2.2151 & 2.2156 & 2.2162 & 2.2167 & 2.2172 & 2.2177 \\
 \textbf{.346}\markboth{0.346}{0.346} & 2.2182 & 2.2187 & 2.2192 & 2.2197 & 2.2202 & 2.2208 & 2.2213 & 2.2218 & 2.2223 & 2.2228 \\
\rowcolor{bg} \textbf{.347}\markboth{0.347}{0.347} & 2.2233 & 2.2238 & 2.2243 & 2.2248 & 2.2254 & 2.2259 & 2.2264 & 2.2269 & 2.2274 & 2.2279 \\
 \textbf{.348}\markboth{0.348}{0.348} & 2.2284 & 2.2289 & 2.2295 & 2.2300 & 2.2305 & 2.2310 & 2.2315 & 2.2320 & 2.2325 & 2.2331 \\
\rowcolor{bg} \textbf{.349}\markboth{0.349}{0.349} & 2.2336 & 2.2341 & 2.2346 & 2.2351 & 2.2356 & 2.2361 & 2.2367 & 2.2372 & 2.2377 & 2.2382 \\
 \textbf{.350}\markboth{0.350}{0.350} & 2.2387 & 2.2392 & 2.2398 & 2.2403 & 2.2408 & 2.2413 & 2.2418 & 2.2423 & 2.2428 & 2.2434 \\
\rowcolor{bg} \textbf{.351}\markboth{0.351}{0.351} & 2.2439 & 2.2444 & 2.2449 & 2.2454 & 2.2459 & 2.2465 & 2.2470 & 2.2475 & 2.2480 & 2.2485 \\
 \textbf{.352}\markboth{0.352}{0.352} & 2.2491 & 2.2496 & 2.2501 & 2.2506 & 2.2511 & 2.2516 & 2.2522 & 2.2527 & 2.2532 & 2.2537 \\
\rowcolor{bg} \textbf{.353}\markboth{0.353}{0.353} & 2.2542 & 2.2548 & 2.2553 & 2.2558 & 2.2563 & 2.2568 & 2.2574 & 2.2579 & 2.2584 & 2.2589 \\
 \textbf{.354}\markboth{0.354}{0.354} & 2.2594 & 2.2600 & 2.2605 & 2.2610 & 2.2615 & 2.2620 & 2.2626 & 2.2631 & 2.2636 & 2.2641 \\
\rowcolor{bg} \textbf{.355}\markboth{0.355}{0.355} & 2.2646 & 2.2652 & 2.2657 & 2.2662 & 2.2667 & 2.2673 & 2.2678 & 2.2683 & 2.2688 & 2.2693 \\
 \textbf{.356}\markboth{0.356}{0.356} & 2.2699 & 2.2704 & 2.2709 & 2.2714 & 2.2720 & 2.2725 & 2.2730 & 2.2735 & 2.2740 & 2.2746 \\
\rowcolor{bg} \textbf{.357}\markboth{0.357}{0.357} & 2.2751 & 2.2756 & 2.2761 & 2.2767 & 2.2772 & 2.2777 & 2.2782 & 2.2788 & 2.2793 & 2.2798 \\
 \textbf{.358}\markboth{0.358}{0.358} & 2.2803 & 2.2809 & 2.2814 & 2.2819 & 2.2824 & 2.2830 & 2.2835 & 2.2840 & 2.2845 & 2.2851 \\
\rowcolor{bg} \textbf{.359}\markboth{0.359}{0.359} & 2.2856 & 2.2861 & 2.2867 & 2.2872 & 2.2877 & 2.2882 & 2.2888 & 2.2893 & 2.2898 & 2.2903 \\
 \textbf{.360}\markboth{0.360}{0.360} & 2.2909 & 2.2914 & 2.2919 & 2.2925 & 2.2930 & 2.2935 & 2.2940 & 2.2946 & 2.2951 & 2.2956 \\
\rowcolor{bg} \textbf{.361}\markboth{0.361}{0.361} & 2.2961 & 2.2967 & 2.2972 & 2.2977 & 2.2983 & 2.2988 & 2.2993 & 2.2999 & 2.3004 & 2.3009 \\
 \textbf{.362}\markboth{0.362}{0.362} & 2.3014 & 2.3020 & 2.3025 & 2.3030 & 2.3036 & 2.3041 & 2.3046 & 2.3052 & 2.3057 & 2.3062 \\
\rowcolor{bg} \textbf{.363}\markboth{0.363}{0.363} & 2.3067 & 2.3073 & 2.3078 & 2.3083 & 2.3089 & 2.3094 & 2.3099 & 2.3105 & 2.3110 & 2.3115 \\
 \textbf{.364}\markboth{0.364}{0.364} & 2.3121 & 2.3126 & 2.3131 & 2.3137 & 2.3142 & 2.3147 & 2.3153 & 2.3158 & 2.3163 & 2.3169 \\
\rowcolor{bg} \textbf{.365}\markboth{0.365}{0.365} & 2.3174 & 2.3179 & 2.3185 & 2.3190 & 2.3195 & 2.3201 & 2.3206 & 2.3211 & 2.3217 & 2.3222 \\
 \textbf{.366}\markboth{0.366}{0.366} & 2.3227 & 2.3233 & 2.3238 & 2.3243 & 2.3249 & 2.3254 & 2.3259 & 2.3265 & 2.3270 & 2.3276 \\
\rowcolor{bg} \textbf{.367}\markboth{0.367}{0.367} & 2.3281 & 2.3286 & 2.3292 & 2.3297 & 2.3302 & 2.3308 & 2.3313 & 2.3318 & 2.3324 & 2.3329 \\
 \textbf{.368}\markboth{0.368}{0.368} & 2.3335 & 2.3340 & 2.3345 & 2.3351 & 2.3356 & 2.3361 & 2.3367 & 2.3372 & 2.3378 & 2.3383 \\
\rowcolor{bg} \textbf{.369}\markboth{0.369}{0.369} & 2.3388 & 2.3394 & 2.3399 & 2.3405 & 2.3410 & 2.3415 & 2.3421 & 2.3426 & 2.3431 & 2.3437 \\
 \textbf{.370}\markboth{0.370}{0.370} & 2.3442 & 2.3448 & 2.3453 & 2.3458 & 2.3464 & 2.3469 & 2.3475 & 2.3480 & 2.3486 & 2.3491 \\
\rowcolor{bg} \textbf{.371}\markboth{0.371}{0.371} & 2.3496 & 2.3502 & 2.3507 & 2.3513 & 2.3518 & 2.3523 & 2.3529 & 2.3534 & 2.3540 & 2.3545 \\
 \textbf{.372}\markboth{0.372}{0.372} & 2.3550 & 2.3556 & 2.3561 & 2.3567 & 2.3572 & 2.3578 & 2.3583 & 2.3588 & 2.3594 & 2.3599 \\
\rowcolor{bg} \textbf{.373}\markboth{0.373}{0.373} & 2.3605 & 2.3610 & 2.3616 & 2.3621 & 2.3627 & 2.3632 & 2.3637 & 2.3643 & 2.3648 & 2.3654 \\
 \textbf{.374}\markboth{0.374}{0.374} & 2.3659 & 2.3665 & 2.3670 & 2.3676 & 2.3681 & 2.3686 & 2.3692 & 2.3697 & 2.3703 & 2.3708 \\
\rowcolor{bg} \textbf{.375}\markboth{0.375}{0.375} & 2.3714 & 2.3719 & 2.3725 & 2.3730 & 2.3736 & 2.3741 & 2.3747 & 2.3752 & 2.3757 & 2.3763 \\
 \textbf{.376}\markboth{0.376}{0.376} & 2.3768 & 2.3774 & 2.3779 & 2.3785 & 2.3790 & 2.3796 & 2.3801 & 2.3807 & 2.3812 & 2.3818 \\
\rowcolor{bg} \textbf{.377}\markboth{0.377}{0.377} & 2.3823 & 2.3829 & 2.3834 & 2.3840 & 2.3845 & 2.3851 & 2.3856 & 2.3862 & 2.3867 & 2.3873 \\
 \textbf{.378}\markboth{0.378}{0.378} & 2.3878 & 2.3884 & 2.3889 & 2.3895 & 2.3900 & 2.3906 & 2.3911 & 2.3917 & 2.3922 & 2.3928 \\
\rowcolor{bg} \textbf{.379}\markboth{0.379}{0.379} & 2.3933 & 2.3939 & 2.3944 & 2.3950 & 2.3955 & 2.3961 & 2.3966 & 2.3972 & 2.3977 & 2.3983 \\
 \textbf{.380}\markboth{0.380}{0.380} & 2.3988 & 2.3994 & 2.3999 & 2.4005 & 2.4010 & 2.4016 & 2.4021 & 2.4027 & 2.4033 & 2.4038 \\
\rowcolor{bg} \textbf{.381}\markboth{0.381}{0.381} & 2.4044 & 2.4049 & 2.4055 & 2.4060 & 2.4066 & 2.4071 & 2.4077 & 2.4082 & 2.4088 & 2.4094 \\
 \textbf{.382}\markboth{0.382}{0.382} & 2.4099 & 2.4105 & 2.4110 & 2.4116 & 2.4121 & 2.4127 & 2.4132 & 2.4138 & 2.4143 & 2.4149 \\
\rowcolor{bg} \textbf{.383}\markboth{0.383}{0.383} & 2.4155 & 2.4160 & 2.4166 & 2.4171 & 2.4177 & 2.4182 & 2.4188 & 2.4194 & 2.4199 & 2.4205 \\
 \textbf{.384}\markboth{0.384}{0.384} & 2.4210 & 2.4216 & 2.4221 & 2.4227 & 2.4233 & 2.4238 & 2.4244 & 2.4249 & 2.4255 & 2.4261 \\
\rowcolor{bg} \textbf{.385}\markboth{0.385}{0.385} & 2.4266 & 2.4272 & 2.4277 & 2.4283 & 2.4288 & 2.4294 & 2.4300 & 2.4305 & 2.4311 & 2.4316 \\
 \textbf{.386}\markboth{0.386}{0.386} & 2.4322 & 2.4328 & 2.4333 & 2.4339 & 2.4344 & 2.4350 & 2.4356 & 2.4361 & 2.4367 & 2.4372 \\
\rowcolor{bg} \textbf{.387}\markboth{0.387}{0.387} & 2.4378 & 2.4384 & 2.4389 & 2.4395 & 2.4401 & 2.4406 & 2.4412 & 2.4417 & 2.4423 & 2.4429 \\
 \textbf{.388}\markboth{0.388}{0.388} & 2.4434 & 2.4440 & 2.4446 & 2.4451 & 2.4457 & 2.4462 & 2.4468 & 2.4474 & 2.4479 & 2.4485 \\
\rowcolor{bg} \textbf{.389}\markboth{0.389}{0.389} & 2.4491 & 2.4496 & 2.4502 & 2.4508 & 2.4513 & 2.4519 & 2.4524 & 2.4530 & 2.4536 & 2.4541 \\
 \textbf{.390}\markboth{0.390}{0.390} & 2.4547 & 2.4553 & 2.4558 & 2.4564 & 2.4570 & 2.4575 & 2.4581 & 2.4587 & 2.4592 & 2.4598 \\
\rowcolor{bg} \textbf{.391}\markboth{0.391}{0.391} & 2.4604 & 2.4609 & 2.4615 & 2.4621 & 2.4626 & 2.4632 & 2.4638 & 2.4643 & 2.4649 & 2.4655 \\
 \textbf{.392}\markboth{0.392}{0.392} & 2.4660 & 2.4666 & 2.4672 & 2.4677 & 2.4683 & 2.4689 & 2.4694 & 2.4700 & 2.4706 & 2.4712 \\
\rowcolor{bg} \textbf{.393}\markboth{0.393}{0.393} & 2.4717 & 2.4723 & 2.4729 & 2.4734 & 2.4740 & 2.4746 & 2.4751 & 2.4757 & 2.4763 & 2.4769 \\
 \textbf{.394}\markboth{0.394}{0.394} & 2.4774 & 2.4780 & 2.4786 & 2.4791 & 2.4797 & 2.4803 & 2.4808 & 2.4814 & 2.4820 & 2.4826 \\
\rowcolor{bg} \textbf{.395}\markboth{0.395}{0.395} & 2.4831 & 2.4837 & 2.4843 & 2.4848 & 2.4854 & 2.4860 & 2.4866 & 2.4871 & 2.4877 & 2.4883 \\
 \textbf{.396}\markboth{0.396}{0.396} & 2.4889 & 2.4894 & 2.4900 & 2.4906 & 2.4912 & 2.4917 & 2.4923 & 2.4929 & 2.4934 & 2.4940 \\
\rowcolor{bg} \textbf{.397}\markboth{0.397}{0.397} & 2.4946 & 2.4952 & 2.4957 & 2.4963 & 2.4969 & 2.4975 & 2.4980 & 2.4986 & 2.4992 & 2.4998 \\
 \textbf{.398}\markboth{0.398}{0.398} & 2.5003 & 2.5009 & 2.5015 & 2.5021 & 2.5026 & 2.5032 & 2.5038 & 2.5044 & 2.5050 & 2.5055 \\
\rowcolor{bg} \textbf{.399}\markboth{0.399}{0.399} & 2.5061 & 2.5067 & 2.5073 & 2.5078 & 2.5084 & 2.5090 & 2.5096 & 2.5102 & 2.5107 & 2.5113 \\
 \textcolor{blue}{\textbf{.400}}\markboth{0.400}{0.400} & 2.5119 & 2.5125 & 2.5130 & 2.5136 & 2.5142 & 2.5148 & 2.5154 & 2.5159 & 2.5165 & 2.5171 \\
\rowcolor{bg} \textbf{.401}\markboth{0.401}{0.401} & 2.5177 & 2.5183 & 2.5188 & 2.5194 & 2.5200 & 2.5206 & 2.5212 & 2.5217 & 2.5223 & 2.5229 \\
 \textbf{.402}\markboth{0.402}{0.402} & 2.5235 & 2.5241 & 2.5246 & 2.5252 & 2.5258 & 2.5264 & 2.5270 & 2.5276 & 2.5281 & 2.5287 \\
\rowcolor{bg} \textbf{.403}\markboth{0.403}{0.403} & 2.5293 & 2.5299 & 2.5305 & 2.5310 & 2.5316 & 2.5322 & 2.5328 & 2.5334 & 2.5340 & 2.5345 \\
 \textbf{.404}\markboth{0.404}{0.404} & 2.5351 & 2.5357 & 2.5363 & 2.5369 & 2.5375 & 2.5380 & 2.5386 & 2.5392 & 2.5398 & 2.5404 \\
\rowcolor{bg} \textbf{.405}\markboth{0.405}{0.405} & 2.5410 & 2.5416 & 2.5421 & 2.5427 & 2.5433 & 2.5439 & 2.5445 & 2.5451 & 2.5457 & 2.5462 \\
 \textbf{.406}\markboth{0.406}{0.406} & 2.5468 & 2.5474 & 2.5480 & 2.5486 & 2.5492 & 2.5498 & 2.5504 & 2.5509 & 2.5515 & 2.5521 \\
\rowcolor{bg} \textbf{.407}\markboth{0.407}{0.407} & 2.5527 & 2.5533 & 2.5539 & 2.5545 & 2.5551 & 2.5556 & 2.5562 & 2.5568 & 2.5574 & 2.5580 \\
 \textbf{.408}\markboth{0.408}{0.408} & 2.5586 & 2.5592 & 2.5598 & 2.5604 & 2.5609 & 2.5615 & 2.5621 & 2.5627 & 2.5633 & 2.5639 \\
\rowcolor{bg} \textbf{.409}\markboth{0.409}{0.409} & 2.5645 & 2.5651 & 2.5657 & 2.5663 & 2.5668 & 2.5674 & 2.5680 & 2.5686 & 2.5692 & 2.5698 \\
 \textbf{.410}\markboth{0.410}{0.410} & 2.5704 & 2.5710 & 2.5716 & 2.5722 & 2.5728 & 2.5734 & 2.5739 & 2.5745 & 2.5751 & 2.5757 \\
\rowcolor{bg} \textbf{.411}\markboth{0.411}{0.411} & 2.5763 & 2.5769 & 2.5775 & 2.5781 & 2.5787 & 2.5793 & 2.5799 & 2.5805 & 2.5811 & 2.5817 \\
 \textbf{.412}\markboth{0.412}{0.412} & 2.5823 & 2.5829 & 2.5834 & 2.5840 & 2.5846 & 2.5852 & 2.5858 & 2.5864 & 2.5870 & 2.5876 \\
\rowcolor{bg} \textbf{.413}\markboth{0.413}{0.413} & 2.5882 & 2.5888 & 2.5894 & 2.5900 & 2.5906 & 2.5912 & 2.5918 & 2.5924 & 2.5930 & 2.5936 \\
 \textbf{.414}\markboth{0.414}{0.414} & 2.5942 & 2.5948 & 2.5954 & 2.5960 & 2.5966 & 2.5972 & 2.5978 & 2.5984 & 2.5990 & 2.5996 \\
\rowcolor{bg} \textbf{.415}\markboth{0.415}{0.415} & 2.6002 & 2.6008 & 2.6014 & 2.6020 & 2.6026 & 2.6032 & 2.6038 & 2.6044 & 2.6050 & 2.6056 \\
 \textbf{.416}\markboth{0.416}{0.416} & 2.6062 & 2.6068 & 2.6074 & 2.6080 & 2.6086 & 2.6092 & 2.6098 & 2.6104 & 2.6110 & 2.6116 \\
\rowcolor{bg} \textbf{.417}\markboth{0.417}{0.417} & 2.6122 & 2.6128 & 2.6134 & 2.6140 & 2.6146 & 2.6152 & 2.6158 & 2.6164 & 2.6170 & 2.6176 \\
 \textbf{.418}\markboth{0.418}{0.418} & 2.6182 & 2.6188 & 2.6194 & 2.6200 & 2.6206 & 2.6212 & 2.6218 & 2.6224 & 2.6230 & 2.6236 \\
\rowcolor{bg} \textbf{.419}\markboth{0.419}{0.419} & 2.6242 & 2.6248 & 2.6254 & 2.6260 & 2.6266 & 2.6272 & 2.6278 & 2.6285 & 2.6291 & 2.6297 \\
 \textbf{.420}\markboth{0.420}{0.420} & 2.6303 & 2.6309 & 2.6315 & 2.6321 & 2.6327 & 2.6333 & 2.6339 & 2.6345 & 2.6351 & 2.6357 \\
\rowcolor{bg} \textbf{.421}\markboth{0.421}{0.421} & 2.6363 & 2.6369 & 2.6375 & 2.6382 & 2.6388 & 2.6394 & 2.6400 & 2.6406 & 2.6412 & 2.6418 \\
 \textbf{.422}\markboth{0.422}{0.422} & 2.6424 & 2.6430 & 2.6436 & 2.6442 & 2.6448 & 2.6455 & 2.6461 & 2.6467 & 2.6473 & 2.6479 \\
\rowcolor{bg} \textbf{.423}\markboth{0.423}{0.423} & 2.6485 & 2.6491 & 2.6497 & 2.6503 & 2.6509 & 2.6516 & 2.6522 & 2.6528 & 2.6534 & 2.6540 \\
 \textbf{.424}\markboth{0.424}{0.424} & 2.6546 & 2.6552 & 2.6558 & 2.6564 & 2.6571 & 2.6577 & 2.6583 & 2.6589 & 2.6595 & 2.6601 \\
\rowcolor{bg} \textbf{.425}\markboth{0.425}{0.425} & 2.6607 & 2.6613 & 2.6620 & 2.6626 & 2.6632 & 2.6638 & 2.6644 & 2.6650 & 2.6656 & 2.6662 \\
 \textbf{.426}\markboth{0.426}{0.426} & 2.6669 & 2.6675 & 2.6681 & 2.6687 & 2.6693 & 2.6699 & 2.6705 & 2.6712 & 2.6718 & 2.6724 \\
\rowcolor{bg} \textbf{.427}\markboth{0.427}{0.427} & 2.6730 & 2.6736 & 2.6742 & 2.6749 & 2.6755 & 2.6761 & 2.6767 & 2.6773 & 2.6779 & 2.6786 \\
 \textbf{.428}\markboth{0.428}{0.428} & 2.6792 & 2.6798 & 2.6804 & 2.6810 & 2.6816 & 2.6823 & 2.6829 & 2.6835 & 2.6841 & 2.6847 \\
\rowcolor{bg} \textbf{.429}\markboth{0.429}{0.429} & 2.6853 & 2.6860 & 2.6866 & 2.6872 & 2.6878 & 2.6884 & 2.6891 & 2.6897 & 2.6903 & 2.6909 \\
 \textbf{.430}\markboth{0.430}{0.430} & 2.6915 & 2.6922 & 2.6928 & 2.6934 & 2.6940 & 2.6946 & 2.6953 & 2.6959 & 2.6965 & 2.6971 \\
\rowcolor{bg} \textbf{.431}\markboth{0.431}{0.431} & 2.6977 & 2.6984 & 2.6990 & 2.6996 & 2.7002 & 2.7008 & 2.7015 & 2.7021 & 2.7027 & 2.7033 \\
 \textbf{.432}\markboth{0.432}{0.432} & 2.7040 & 2.7046 & 2.7052 & 2.7058 & 2.7064 & 2.7071 & 2.7077 & 2.7083 & 2.7089 & 2.7096 \\
\rowcolor{bg} \textbf{.433}\markboth{0.433}{0.433} & 2.7102 & 2.7108 & 2.7114 & 2.7121 & 2.7127 & 2.7133 & 2.7139 & 2.7146 & 2.7152 & 2.7158 \\
 \textbf{.434}\markboth{0.434}{0.434} & 2.7164 & 2.7171 & 2.7177 & 2.7183 & 2.7189 & 2.7196 & 2.7202 & 2.7208 & 2.7214 & 2.7221 \\
\rowcolor{bg} \textbf{.435}\markboth{0.435}{0.435} & 2.7227 & 2.7233 & 2.7240 & 2.7246 & 2.7252 & 2.7258 & 2.7265 & 2.7271 & 2.7277 & 2.7283 \\
 \textbf{.436}\markboth{0.436}{0.436} & 2.7290 & 2.7296 & 2.7302 & 2.7309 & 2.7315 & 2.7321 & 2.7328 & 2.7334 & 2.7340 & 2.7346 \\
\rowcolor{bg} \textbf{.437}\markboth{0.437}{0.437} & 2.7353 & 2.7359 & 2.7365 & 2.7372 & 2.7378 & 2.7384 & 2.7391 & 2.7397 & 2.7403 & 2.7409 \\
 \textbf{.438}\markboth{0.438}{0.438} & 2.7416 & 2.7422 & 2.7428 & 2.7435 & 2.7441 & 2.7447 & 2.7454 & 2.7460 & 2.7466 & 2.7473 \\
\rowcolor{bg} \textbf{.439}\markboth{0.439}{0.439} & 2.7479 & 2.7485 & 2.7492 & 2.7498 & 2.7504 & 2.7511 & 2.7517 & 2.7523 & 2.7530 & 2.7536 \\
 \textbf{.440}\markboth{0.440}{0.440} & 2.7542 & 2.7549 & 2.7555 & 2.7561 & 2.7568 & 2.7574 & 2.7580 & 2.7587 & 2.7593 & 2.7599 \\
\rowcolor{bg} \textbf{.441}\markboth{0.441}{0.441} & 2.7606 & 2.7612 & 2.7618 & 2.7625 & 2.7631 & 2.7638 & 2.7644 & 2.7650 & 2.7657 & 2.7663 \\
 \textbf{.442}\markboth{0.442}{0.442} & 2.7669 & 2.7676 & 2.7682 & 2.7689 & 2.7695 & 2.7701 & 2.7708 & 2.7714 & 2.7720 & 2.7727 \\
\rowcolor{bg} \textbf{.443}\markboth{0.443}{0.443} & 2.7733 & 2.7740 & 2.7746 & 2.7752 & 2.7759 & 2.7765 & 2.7772 & 2.7778 & 2.7784 & 2.7791 \\
 \textbf{.444}\markboth{0.444}{0.444} & 2.7797 & 2.7804 & 2.7810 & 2.7816 & 2.7823 & 2.7829 & 2.7836 & 2.7842 & 2.7848 & 2.7855 \\
\rowcolor{bg} \textbf{.445}\markboth{0.445}{0.445} & 2.7861 & 2.7868 & 2.7874 & 2.7880 & 2.7887 & 2.7893 & 2.7900 & 2.7906 & 2.7913 & 2.7919 \\
 \textbf{.446}\markboth{0.446}{0.446} & 2.7925 & 2.7932 & 2.7938 & 2.7945 & 2.7951 & 2.7958 & 2.7964 & 2.7970 & 2.7977 & 2.7983 \\
\rowcolor{bg} \textbf{.447}\markboth{0.447}{0.447} & 2.7990 & 2.7996 & 2.8003 & 2.8009 & 2.8016 & 2.8022 & 2.8029 & 2.8035 & 2.8041 & 2.8048 \\
 \textbf{.448}\markboth{0.448}{0.448} & 2.8054 & 2.8061 & 2.8067 & 2.8074 & 2.8080 & 2.8087 & 2.8093 & 2.8100 & 2.8106 & 2.8113 \\
\rowcolor{bg} \textbf{.449}\markboth{0.449}{0.449} & 2.8119 & 2.8125 & 2.8132 & 2.8138 & 2.8145 & 2.8151 & 2.8158 & 2.8164 & 2.8171 & 2.8177 \\
 \textbf{.450}\markboth{0.450}{0.450} & 2.8184 & 2.8190 & 2.8197 & 2.8203 & 2.8210 & 2.8216 & 2.8223 & 2.8229 & 2.8236 & 2.8242 \\
\rowcolor{bg} \textbf{.451}\markboth{0.451}{0.451} & 2.8249 & 2.8255 & 2.8262 & 2.8268 & 2.8275 & 2.8281 & 2.8288 & 2.8294 & 2.8301 & 2.8307 \\
 \textbf{.452}\markboth{0.452}{0.452} & 2.8314 & 2.8320 & 2.8327 & 2.8333 & 2.8340 & 2.8347 & 2.8353 & 2.8360 & 2.8366 & 2.8373 \\
\rowcolor{bg} \textbf{.453}\markboth{0.453}{0.453} & 2.8379 & 2.8386 & 2.8392 & 2.8399 & 2.8405 & 2.8412 & 2.8418 & 2.8425 & 2.8432 & 2.8438 \\
 \textbf{.454}\markboth{0.454}{0.454} & 2.8445 & 2.8451 & 2.8458 & 2.8464 & 2.8471 & 2.8477 & 2.8484 & 2.8490 & 2.8497 & 2.8504 \\
\rowcolor{bg} \textbf{.455}\markboth{0.455}{0.455} & 2.8510 & 2.8517 & 2.8523 & 2.8530 & 2.8536 & 2.8543 & 2.8550 & 2.8556 & 2.8563 & 2.8569 \\
 \textbf{.456}\markboth{0.456}{0.456} & 2.8576 & 2.8582 & 2.8589 & 2.8596 & 2.8602 & 2.8609 & 2.8615 & 2.8622 & 2.8629 & 2.8635 \\
\rowcolor{bg} \textbf{.457}\markboth{0.457}{0.457} & 2.8642 & 2.8648 & 2.8655 & 2.8662 & 2.8668 & 2.8675 & 2.8681 & 2.8688 & 2.8695 & 2.8701 \\
 \textbf{.458}\markboth{0.458}{0.458} & 2.8708 & 2.8714 & 2.8721 & 2.8728 & 2.8734 & 2.8741 & 2.8747 & 2.8754 & 2.8761 & 2.8767 \\
\rowcolor{bg} \textbf{.459}\markboth{0.459}{0.459} & 2.8774 & 2.8781 & 2.8787 & 2.8794 & 2.8800 & 2.8807 & 2.8814 & 2.8820 & 2.8827 & 2.8834 \\
 \textbf{.460}\markboth{0.460}{0.460} & 2.8840 & 2.8847 & 2.8854 & 2.8860 & 2.8867 & 2.8874 & 2.8880 & 2.8887 & 2.8893 & 2.8900 \\
\rowcolor{bg} \textbf{.461}\markboth{0.461}{0.461} & 2.8907 & 2.8913 & 2.8920 & 2.8927 & 2.8933 & 2.8940 & 2.8947 & 2.8953 & 2.8960 & 2.8967 \\
 \textbf{.462}\markboth{0.462}{0.462} & 2.8973 & 2.8980 & 2.8987 & 2.8993 & 2.9000 & 2.9007 & 2.9013 & 2.9020 & 2.9027 & 2.9034 \\
\rowcolor{bg} \textbf{.463}\markboth{0.463}{0.463} & 2.9040 & 2.9047 & 2.9054 & 2.9060 & 2.9067 & 2.9074 & 2.9080 & 2.9087 & 2.9094 & 2.9100 \\
 \textbf{.464}\markboth{0.464}{0.464} & 2.9107 & 2.9114 & 2.9121 & 2.9127 & 2.9134 & 2.9141 & 2.9147 & 2.9154 & 2.9161 & 2.9168 \\
\rowcolor{bg} \textbf{.465}\markboth{0.465}{0.465} & 2.9174 & 2.9181 & 2.9188 & 2.9194 & 2.9201 & 2.9208 & 2.9215 & 2.9221 & 2.9228 & 2.9235 \\
 \textbf{.466}\markboth{0.466}{0.466} & 2.9242 & 2.9248 & 2.9255 & 2.9262 & 2.9268 & 2.9275 & 2.9282 & 2.9289 & 2.9295 & 2.9302 \\
\rowcolor{bg} \textbf{.467}\markboth{0.467}{0.467} & 2.9309 & 2.9316 & 2.9322 & 2.9329 & 2.9336 & 2.9343 & 2.9349 & 2.9356 & 2.9363 & 2.9370 \\
 \textbf{.468}\markboth{0.468}{0.468} & 2.9376 & 2.9383 & 2.9390 & 2.9397 & 2.9404 & 2.9410 & 2.9417 & 2.9424 & 2.9431 & 2.9437 \\
\rowcolor{bg} \textbf{.469}\markboth{0.469}{0.469} & 2.9444 & 2.9451 & 2.9458 & 2.9465 & 2.9471 & 2.9478 & 2.9485 & 2.9492 & 2.9499 & 2.9505 \\
 \textbf{.470}\markboth{0.470}{0.470} & 2.9512 & 2.9519 & 2.9526 & 2.9532 & 2.9539 & 2.9546 & 2.9553 & 2.9560 & 2.9567 & 2.9573 \\
\rowcolor{bg} \textbf{.471}\markboth{0.471}{0.471} & 2.9580 & 2.9587 & 2.9594 & 2.9601 & 2.9607 & 2.9614 & 2.9621 & 2.9628 & 2.9635 & 2.9641 \\
 \textbf{.472}\markboth{0.472}{0.472} & 2.9648 & 2.9655 & 2.9662 & 2.9669 & 2.9676 & 2.9682 & 2.9689 & 2.9696 & 2.9703 & 2.9710 \\
\rowcolor{bg} \textbf{.473}\markboth{0.473}{0.473} & 2.9717 & 2.9724 & 2.9730 & 2.9737 & 2.9744 & 2.9751 & 2.9758 & 2.9765 & 2.9771 & 2.9778 \\
 \textbf{.474}\markboth{0.474}{0.474} & 2.9785 & 2.9792 & 2.9799 & 2.9806 & 2.9813 & 2.9819 & 2.9826 & 2.9833 & 2.9840 & 2.9847 \\
\rowcolor{bg} \textbf{.475}\markboth{0.475}{0.475} & 2.9854 & 2.9861 & 2.9868 & 2.9874 & 2.9881 & 2.9888 & 2.9895 & 2.9902 & 2.9909 & 2.9916 \\
 \textbf{.476}\markboth{0.476}{0.476} & 2.9923 & 2.9930 & 2.9936 & 2.9943 & 2.9950 & 2.9957 & 2.9964 & 2.9971 & 2.9978 & 2.9985 \\
\rowcolor{bg} \textbf{.477}\markboth{0.477}{0.477} & 2.9992 & 2.9999 & 3.0005 & 3.0012 & 3.0019 & 3.0026 & 3.0033 & 3.0040 & 3.0047 & 3.0054 \\
 \textbf{.478}\markboth{0.478}{0.478} & 3.0061 & 3.0068 & 3.0075 & 3.0082 & 3.0088 & 3.0095 & 3.0102 & 3.0109 & 3.0116 & 3.0123 \\
\rowcolor{bg} \textbf{.479}\markboth{0.479}{0.479} & 3.0130 & 3.0137 & 3.0144 & 3.0151 & 3.0158 & 3.0165 & 3.0172 & 3.0179 & 3.0186 & 3.0193 \\
 \textbf{.480}\markboth{0.480}{0.480} & 3.0200 & 3.0206 & 3.0213 & 3.0220 & 3.0227 & 3.0234 & 3.0241 & 3.0248 & 3.0255 & 3.0262 \\
\rowcolor{bg} \textbf{.481}\markboth{0.481}{0.481} & 3.0269 & 3.0276 & 3.0283 & 3.0290 & 3.0297 & 3.0304 & 3.0311 & 3.0318 & 3.0325 & 3.0332 \\
 \textbf{.482}\markboth{0.482}{0.482} & 3.0339 & 3.0346 & 3.0353 & 3.0360 & 3.0367 & 3.0374 & 3.0381 & 3.0388 & 3.0395 & 3.0402 \\
\rowcolor{bg} \textbf{.483}\markboth{0.483}{0.483} & 3.0409 & 3.0416 & 3.0423 & 3.0430 & 3.0437 & 3.0444 & 3.0451 & 3.0458 & 3.0465 & 3.0472 \\
 \textbf{.484}\markboth{0.484}{0.484} & 3.0479 & 3.0486 & 3.0493 & 3.0500 & 3.0507 & 3.0514 & 3.0521 & 3.0528 & 3.0535 & 3.0542 \\
\rowcolor{bg} \textbf{.485}\markboth{0.485}{0.485} & 3.0549 & 3.0556 & 3.0563 & 3.0570 & 3.0577 & 3.0584 & 3.0591 & 3.0598 & 3.0606 & 3.0613 \\
 \textbf{.486}\markboth{0.486}{0.486} & 3.0620 & 3.0627 & 3.0634 & 3.0641 & 3.0648 & 3.0655 & 3.0662 & 3.0669 & 3.0676 & 3.0683 \\
\rowcolor{bg} \textbf{.487}\markboth{0.487}{0.487} & 3.0690 & 3.0697 & 3.0704 & 3.0711 & 3.0718 & 3.0726 & 3.0733 & 3.0740 & 3.0747 & 3.0754 \\
 \textbf{.488}\markboth{0.488}{0.488} & 3.0761 & 3.0768 & 3.0775 & 3.0782 & 3.0789 & 3.0796 & 3.0803 & 3.0811 & 3.0818 & 3.0825 \\
\rowcolor{bg} \textbf{.489}\markboth{0.489}{0.489} & 3.0832 & 3.0839 & 3.0846 & 3.0853 & 3.0860 & 3.0867 & 3.0875 & 3.0882 & 3.0889 & 3.0896 \\
 \textbf{.490}\markboth{0.490}{0.490} & 3.0903 & 3.0910 & 3.0917 & 3.0924 & 3.0931 & 3.0939 & 3.0946 & 3.0953 & 3.0960 & 3.0967 \\
\rowcolor{bg} \textbf{.491}\markboth{0.491}{0.491} & 3.0974 & 3.0981 & 3.0988 & 3.0996 & 3.1003 & 3.1010 & 3.1017 & 3.1024 & 3.1031 & 3.1038 \\
 \textbf{.492}\markboth{0.492}{0.492} & 3.1046 & 3.1053 & 3.1060 & 3.1067 & 3.1074 & 3.1081 & 3.1089 & 3.1096 & 3.1103 & 3.1110 \\
\rowcolor{bg} \textbf{.493}\markboth{0.493}{0.493} & 3.1117 & 3.1124 & 3.1131 & 3.1139 & 3.1146 & 3.1153 & 3.1160 & 3.1167 & 3.1175 & 3.1182 \\
 \textbf{.494}\markboth{0.494}{0.494} & 3.1189 & 3.1196 & 3.1203 & 3.1210 & 3.1218 & 3.1225 & 3.1232 & 3.1239 & 3.1246 & 3.1254 \\
\rowcolor{bg} \textbf{.495}\markboth{0.495}{0.495} & 3.1261 & 3.1268 & 3.1275 & 3.1282 & 3.1290 & 3.1297 & 3.1304 & 3.1311 & 3.1318 & 3.1326 \\
 \textbf{.496}\markboth{0.496}{0.496} & 3.1333 & 3.1340 & 3.1347 & 3.1355 & 3.1362 & 3.1369 & 3.1376 & 3.1383 & 3.1391 & 3.1398 \\
\rowcolor{bg} \textbf{.497}\markboth{0.497}{0.497} & 3.1405 & 3.1412 & 3.1420 & 3.1427 & 3.1434 & 3.1441 & 3.1449 & 3.1456 & 3.1463 & 3.1470 \\
 \textbf{.498}\markboth{0.498}{0.498} & 3.1477 & 3.1485 & 3.1492 & 3.1499 & 3.1506 & 3.1514 & 3.1521 & 3.1528 & 3.1536 & 3.1543 \\
\rowcolor{bg} \textbf{.499}\markboth{0.499}{0.499} & 3.1550 & 3.1557 & 3.1565 & 3.1572 & 3.1579 & 3.1586 & 3.1594 & 3.1601 & 3.1608 & 3.1615 \\
 \textcolor{blue}{\textbf{.500}}\markboth{0.500}{0.500} & 3.1623 & 3.1630 & 3.1637 & 3.1645 & 3.1652 & 3.1659 & 3.1666 & 3.1674 & 3.1681 & 3.1688 \\
\rowcolor{bg} \textbf{.501}\markboth{0.501}{0.501} & 3.1696 & 3.1703 & 3.1710 & 3.1718 & 3.1725 & 3.1732 & 3.1739 & 3.1747 & 3.1754 & 3.1761 \\
 \textbf{.502}\markboth{0.502}{0.502} & 3.1769 & 3.1776 & 3.1783 & 3.1791 & 3.1798 & 3.1805 & 3.1813 & 3.1820 & 3.1827 & 3.1835 \\
\rowcolor{bg} \textbf{.503}\markboth{0.503}{0.503} & 3.1842 & 3.1849 & 3.1857 & 3.1864 & 3.1871 & 3.1879 & 3.1886 & 3.1893 & 3.1901 & 3.1908 \\
 \textbf{.504}\markboth{0.504}{0.504} & 3.1915 & 3.1923 & 3.1930 & 3.1937 & 3.1945 & 3.1952 & 3.1960 & 3.1967 & 3.1974 & 3.1982 \\
\rowcolor{bg} \textbf{.505}\markboth{0.505}{0.505} & 3.1989 & 3.1996 & 3.2004 & 3.2011 & 3.2018 & 3.2026 & 3.2033 & 3.2041 & 3.2048 & 3.2055 \\
 \textbf{.506}\markboth{0.506}{0.506} & 3.2063 & 3.2070 & 3.2077 & 3.2085 & 3.2092 & 3.2100 & 3.2107 & 3.2114 & 3.2122 & 3.2129 \\
\rowcolor{bg} \textbf{.507}\markboth{0.507}{0.507} & 3.2137 & 3.2144 & 3.2151 & 3.2159 & 3.2166 & 3.2174 & 3.2181 & 3.2188 & 3.2196 & 3.2203 \\
 \textbf{.508}\markboth{0.508}{0.508} & 3.2211 & 3.2218 & 3.2226 & 3.2233 & 3.2240 & 3.2248 & 3.2255 & 3.2263 & 3.2270 & 3.2278 \\
\rowcolor{bg} \textbf{.509}\markboth{0.509}{0.509} & 3.2285 & 3.2292 & 3.2300 & 3.2307 & 3.2315 & 3.2322 & 3.2330 & 3.2337 & 3.2344 & 3.2352 \\
 \textbf{.510}\markboth{0.510}{0.510} & 3.2359 & 3.2367 & 3.2374 & 3.2382 & 3.2389 & 3.2397 & 3.2404 & 3.2412 & 3.2419 & 3.2426 \\
\rowcolor{bg} \textbf{.511}\markboth{0.511}{0.511} & 3.2434 & 3.2441 & 3.2449 & 3.2456 & 3.2464 & 3.2471 & 3.2479 & 3.2486 & 3.2494 & 3.2501 \\
 \textbf{.512}\markboth{0.512}{0.512} & 3.2509 & 3.2516 & 3.2524 & 3.2531 & 3.2539 & 3.2546 & 3.2554 & 3.2561 & 3.2569 & 3.2576 \\
\rowcolor{bg} \textbf{.513}\markboth{0.513}{0.513} & 3.2584 & 3.2591 & 3.2599 & 3.2606 & 3.2614 & 3.2621 & 3.2629 & 3.2636 & 3.2644 & 3.2651 \\
 \textbf{.514}\markboth{0.514}{0.514} & 3.2659 & 3.2666 & 3.2674 & 3.2681 & 3.2689 & 3.2696 & 3.2704 & 3.2711 & 3.2719 & 3.2727 \\
\rowcolor{bg} \textbf{.515}\markboth{0.515}{0.515} & 3.2734 & 3.2742 & 3.2749 & 3.2757 & 3.2764 & 3.2772 & 3.2779 & 3.2787 & 3.2794 & 3.2802 \\
 \textbf{.516}\markboth{0.516}{0.516} & 3.2810 & 3.2817 & 3.2825 & 3.2832 & 3.2840 & 3.2847 & 3.2855 & 3.2862 & 3.2870 & 3.2878 \\
\rowcolor{bg} \textbf{.517}\markboth{0.517}{0.517} & 3.2885 & 3.2893 & 3.2900 & 3.2908 & 3.2915 & 3.2923 & 3.2931 & 3.2938 & 3.2946 & 3.2953 \\
 \textbf{.518}\markboth{0.518}{0.518} & 3.2961 & 3.2969 & 3.2976 & 3.2984 & 3.2991 & 3.2999 & 3.3007 & 3.3014 & 3.3022 & 3.3029 \\
\rowcolor{bg} \textbf{.519}\markboth{0.519}{0.519} & 3.3037 & 3.3045 & 3.3052 & 3.3060 & 3.3067 & 3.3075 & 3.3083 & 3.3090 & 3.3098 & 3.3105 \\
 \textbf{.520}\markboth{0.520}{0.520} & 3.3113 & 3.3121 & 3.3128 & 3.3136 & 3.3144 & 3.3151 & 3.3159 & 3.3167 & 3.3174 & 3.3182 \\
\rowcolor{bg} \textbf{.521}\markboth{0.521}{0.521} & 3.3189 & 3.3197 & 3.3205 & 3.3212 & 3.3220 & 3.3228 & 3.3235 & 3.3243 & 3.3251 & 3.3258 \\
 \textbf{.522}\markboth{0.522}{0.522} & 3.3266 & 3.3274 & 3.3281 & 3.3289 & 3.3297 & 3.3304 & 3.3312 & 3.3320 & 3.3327 & 3.3335 \\
\rowcolor{bg} \textbf{.523}\markboth{0.523}{0.523} & 3.3343 & 3.3350 & 3.3358 & 3.3366 & 3.3373 & 3.3381 & 3.3389 & 3.3396 & 3.3404 & 3.3412 \\
 \textbf{.524}\markboth{0.524}{0.524} & 3.3420 & 3.3427 & 3.3435 & 3.3443 & 3.3450 & 3.3458 & 3.3466 & 3.3473 & 3.3481 & 3.3489 \\
\rowcolor{bg} \textbf{.525}\markboth{0.525}{0.525} & 3.3497 & 3.3504 & 3.3512 & 3.3520 & 3.3527 & 3.3535 & 3.3543 & 3.3551 & 3.3558 & 3.3566 \\
 \textbf{.526}\markboth{0.526}{0.526} & 3.3574 & 3.3581 & 3.3589 & 3.3597 & 3.3605 & 3.3612 & 3.3620 & 3.3628 & 3.3636 & 3.3643 \\
\rowcolor{bg} \textbf{.527}\markboth{0.527}{0.527} & 3.3651 & 3.3659 & 3.3667 & 3.3674 & 3.3682 & 3.3690 & 3.3698 & 3.3705 & 3.3713 & 3.3721 \\
 \textbf{.528}\markboth{0.528}{0.528} & 3.3729 & 3.3736 & 3.3744 & 3.3752 & 3.3760 & 3.3768 & 3.3775 & 3.3783 & 3.3791 & 3.3799 \\
\rowcolor{bg} \textbf{.529}\markboth{0.529}{0.529} & 3.3806 & 3.3814 & 3.3822 & 3.3830 & 3.3838 & 3.3845 & 3.3853 & 3.3861 & 3.3869 & 3.3877 \\
 \textbf{.530}\markboth{0.530}{0.530} & 3.3884 & 3.3892 & 3.3900 & 3.3908 & 3.3916 & 3.3923 & 3.3931 & 3.3939 & 3.3947 & 3.3955 \\
\rowcolor{bg} \textbf{.531}\markboth{0.531}{0.531} & 3.3963 & 3.3970 & 3.3978 & 3.3986 & 3.3994 & 3.4002 & 3.4009 & 3.4017 & 3.4025 & 3.4033 \\
 \textbf{.532}\markboth{0.532}{0.532} & 3.4041 & 3.4049 & 3.4056 & 3.4064 & 3.4072 & 3.4080 & 3.4088 & 3.4096 & 3.4104 & 3.4111 \\
\rowcolor{bg} \textbf{.533}\markboth{0.533}{0.533} & 3.4119 & 3.4127 & 3.4135 & 3.4143 & 3.4151 & 3.4159 & 3.4166 & 3.4174 & 3.4182 & 3.4190 \\
 \textbf{.534}\markboth{0.534}{0.534} & 3.4198 & 3.4206 & 3.4214 & 3.4222 & 3.4229 & 3.4237 & 3.4245 & 3.4253 & 3.4261 & 3.4269 \\
\rowcolor{bg} \textbf{.535}\markboth{0.535}{0.535} & 3.4277 & 3.4285 & 3.4293 & 3.4300 & 3.4308 & 3.4316 & 3.4324 & 3.4332 & 3.4340 & 3.4348 \\
 \textbf{.536}\markboth{0.536}{0.536} & 3.4356 & 3.4364 & 3.4372 & 3.4380 & 3.4387 & 3.4395 & 3.4403 & 3.4411 & 3.4419 & 3.4427 \\
\rowcolor{bg} \textbf{.537}\markboth{0.537}{0.537} & 3.4435 & 3.4443 & 3.4451 & 3.4459 & 3.4467 & 3.4475 & 3.4483 & 3.4491 & 3.4498 & 3.4506 \\
 \textbf{.538}\markboth{0.538}{0.538} & 3.4514 & 3.4522 & 3.4530 & 3.4538 & 3.4546 & 3.4554 & 3.4562 & 3.4570 & 3.4578 & 3.4586 \\
\rowcolor{bg} \textbf{.539}\markboth{0.539}{0.539} & 3.4594 & 3.4602 & 3.4610 & 3.4618 & 3.4626 & 3.4634 & 3.4642 & 3.4650 & 3.4658 & 3.4666 \\
 \textbf{.540}\markboth{0.540}{0.540} & 3.4674 & 3.4682 & 3.4690 & 3.4698 & 3.4706 & 3.4714 & 3.4722 & 3.4730 & 3.4738 & 3.4746 \\
\rowcolor{bg} \textbf{.541}\markboth{0.541}{0.541} & 3.4754 & 3.4762 & 3.4770 & 3.4778 & 3.4786 & 3.4794 & 3.4802 & 3.4810 & 3.4818 & 3.4826 \\
 \textbf{.542}\markboth{0.542}{0.542} & 3.4834 & 3.4842 & 3.4850 & 3.4858 & 3.4866 & 3.4874 & 3.4882 & 3.4890 & 3.4898 & 3.4906 \\
\rowcolor{bg} \textbf{.543}\markboth{0.543}{0.543} & 3.4914 & 3.4922 & 3.4930 & 3.4938 & 3.4946 & 3.4954 & 3.4962 & 3.4970 & 3.4978 & 3.4986 \\
 \textbf{.544}\markboth{0.544}{0.544} & 3.4995 & 3.5003 & 3.5011 & 3.5019 & 3.5027 & 3.5035 & 3.5043 & 3.5051 & 3.5059 & 3.5067 \\
\rowcolor{bg} \textbf{.545}\markboth{0.545}{0.545} & 3.5075 & 3.5083 & 3.5091 & 3.5099 & 3.5108 & 3.5116 & 3.5124 & 3.5132 & 3.5140 & 3.5148 \\
 \textbf{.546}\markboth{0.546}{0.546} & 3.5156 & 3.5164 & 3.5172 & 3.5180 & 3.5188 & 3.5197 & 3.5205 & 3.5213 & 3.5221 & 3.5229 \\
\rowcolor{bg} \textbf{.547}\markboth{0.547}{0.547} & 3.5237 & 3.5245 & 3.5253 & 3.5261 & 3.5270 & 3.5278 & 3.5286 & 3.5294 & 3.5302 & 3.5310 \\
 \textbf{.548}\markboth{0.548}{0.548} & 3.5318 & 3.5326 & 3.5335 & 3.5343 & 3.5351 & 3.5359 & 3.5367 & 3.5375 & 3.5383 & 3.5392 \\
\rowcolor{bg} \textbf{.549}\markboth{0.549}{0.549} & 3.5400 & 3.5408 & 3.5416 & 3.5424 & 3.5432 & 3.5441 & 3.5449 & 3.5457 & 3.5465 & 3.5473 \\
 \textbf{.550}\markboth{0.550}{0.550} & 3.5481 & 3.5490 & 3.5498 & 3.5506 & 3.5514 & 3.5522 & 3.5530 & 3.5539 & 3.5547 & 3.5555 \\
\rowcolor{bg} \textbf{.551}\markboth{0.551}{0.551} & 3.5563 & 3.5571 & 3.5580 & 3.5588 & 3.5596 & 3.5604 & 3.5612 & 3.5620 & 3.5629 & 3.5637 \\
 \textbf{.552}\markboth{0.552}{0.552} & 3.5645 & 3.5653 & 3.5662 & 3.5670 & 3.5678 & 3.5686 & 3.5694 & 3.5703 & 3.5711 & 3.5719 \\
\rowcolor{bg} \textbf{.553}\markboth{0.553}{0.553} & 3.5727 & 3.5736 & 3.5744 & 3.5752 & 3.5760 & 3.5768 & 3.5777 & 3.5785 & 3.5793 & 3.5801 \\
 \textbf{.554}\markboth{0.554}{0.554} & 3.5810 & 3.5818 & 3.5826 & 3.5834 & 3.5843 & 3.5851 & 3.5859 & 3.5867 & 3.5876 & 3.5884 \\
\rowcolor{bg} \textbf{.555}\markboth{0.555}{0.555} & 3.5892 & 3.5900 & 3.5909 & 3.5917 & 3.5925 & 3.5934 & 3.5942 & 3.5950 & 3.5958 & 3.5967 \\
 \textbf{.556}\markboth{0.556}{0.556} & 3.5975 & 3.5983 & 3.5992 & 3.6000 & 3.6008 & 3.6016 & 3.6025 & 3.6033 & 3.6041 & 3.6050 \\
\rowcolor{bg} \textbf{.557}\markboth{0.557}{0.557} & 3.6058 & 3.6066 & 3.6074 & 3.6083 & 3.6091 & 3.6099 & 3.6108 & 3.6116 & 3.6124 & 3.6133 \\
 \textbf{.558}\markboth{0.558}{0.558} & 3.6141 & 3.6149 & 3.6158 & 3.6166 & 3.6174 & 3.6183 & 3.6191 & 3.6199 & 3.6208 & 3.6216 \\
\rowcolor{bg} \textbf{.559}\markboth{0.559}{0.559} & 3.6224 & 3.6233 & 3.6241 & 3.6249 & 3.6258 & 3.6266 & 3.6274 & 3.6283 & 3.6291 & 3.6299 \\
 \textbf{.560}\markboth{0.560}{0.560} & 3.6308 & 3.6316 & 3.6325 & 3.6333 & 3.6341 & 3.6350 & 3.6358 & 3.6366 & 3.6375 & 3.6383 \\
\rowcolor{bg} \textbf{.561}\markboth{0.561}{0.561} & 3.6392 & 3.6400 & 3.6408 & 3.6417 & 3.6425 & 3.6433 & 3.6442 & 3.6450 & 3.6459 & 3.6467 \\
 \textbf{.562}\markboth{0.562}{0.562} & 3.6475 & 3.6484 & 3.6492 & 3.6501 & 3.6509 & 3.6517 & 3.6526 & 3.6534 & 3.6543 & 3.6551 \\
\rowcolor{bg} \textbf{.563}\markboth{0.563}{0.563} & 3.6559 & 3.6568 & 3.6576 & 3.6585 & 3.6593 & 3.6602 & 3.6610 & 3.6618 & 3.6627 & 3.6635 \\
 \textbf{.564}\markboth{0.564}{0.564} & 3.6644 & 3.6652 & 3.6661 & 3.6669 & 3.6678 & 3.6686 & 3.6694 & 3.6703 & 3.6711 & 3.6720 \\
\rowcolor{bg} \textbf{.565}\markboth{0.565}{0.565} & 3.6728 & 3.6737 & 3.6745 & 3.6754 & 3.6762 & 3.6771 & 3.6779 & 3.6787 & 3.6796 & 3.6804 \\
 \textbf{.566}\markboth{0.566}{0.566} & 3.6813 & 3.6821 & 3.6830 & 3.6838 & 3.6847 & 3.6855 & 3.6864 & 3.6872 & 3.6881 & 3.6889 \\
\rowcolor{bg} \textbf{.567}\markboth{0.567}{0.567} & 3.6898 & 3.6906 & 3.6915 & 3.6923 & 3.6932 & 3.6940 & 3.6949 & 3.6957 & 3.6966 & 3.6974 \\
 \textbf{.568}\markboth{0.568}{0.568} & 3.6983 & 3.6991 & 3.7000 & 3.7008 & 3.7017 & 3.7025 & 3.7034 & 3.7042 & 3.7051 & 3.7060 \\
\rowcolor{bg} \textbf{.569}\markboth{0.569}{0.569} & 3.7068 & 3.7077 & 3.7085 & 3.7094 & 3.7102 & 3.7111 & 3.7119 & 3.7128 & 3.7136 & 3.7145 \\
 \textbf{.570}\markboth{0.570}{0.570} & 3.7154 & 3.7162 & 3.7171 & 3.7179 & 3.7188 & 3.7196 & 3.7205 & 3.7213 & 3.7222 & 3.7231 \\
\rowcolor{bg} \textbf{.571}\markboth{0.571}{0.571} & 3.7239 & 3.7248 & 3.7256 & 3.7265 & 3.7273 & 3.7282 & 3.7291 & 3.7299 & 3.7308 & 3.7316 \\
 \textbf{.572}\markboth{0.572}{0.572} & 3.7325 & 3.7334 & 3.7342 & 3.7351 & 3.7359 & 3.7368 & 3.7377 & 3.7385 & 3.7394 & 3.7402 \\
\rowcolor{bg} \textbf{.573}\markboth{0.573}{0.573} & 3.7411 & 3.7420 & 3.7428 & 3.7437 & 3.7446 & 3.7454 & 3.7463 & 3.7471 & 3.7480 & 3.7489 \\
 \textbf{.574}\markboth{0.574}{0.574} & 3.7497 & 3.7506 & 3.7515 & 3.7523 & 3.7532 & 3.7540 & 3.7549 & 3.7558 & 3.7566 & 3.7575 \\
\rowcolor{bg} \textbf{.575}\markboth{0.575}{0.575} & 3.7584 & 3.7592 & 3.7601 & 3.7610 & 3.7618 & 3.7627 & 3.7636 & 3.7644 & 3.7653 & 3.7662 \\
 \textbf{.576}\markboth{0.576}{0.576} & 3.7670 & 3.7679 & 3.7688 & 3.7696 & 3.7705 & 3.7714 & 3.7722 & 3.7731 & 3.7740 & 3.7749 \\
\rowcolor{bg} \textbf{.577}\markboth{0.577}{0.577} & 3.7757 & 3.7766 & 3.7775 & 3.7783 & 3.7792 & 3.7801 & 3.7809 & 3.7818 & 3.7827 & 3.7836 \\
 \textbf{.578}\markboth{0.578}{0.578} & 3.7844 & 3.7853 & 3.7862 & 3.7870 & 3.7879 & 3.7888 & 3.7897 & 3.7905 & 3.7914 & 3.7923 \\
\rowcolor{bg} \textbf{.579}\markboth{0.579}{0.579} & 3.7931 & 3.7940 & 3.7949 & 3.7958 & 3.7966 & 3.7975 & 3.7984 & 3.7993 & 3.8001 & 3.8010 \\
 \textbf{.580}\markboth{0.580}{0.580} & 3.8019 & 3.8028 & 3.8036 & 3.8045 & 3.8054 & 3.8063 & 3.8072 & 3.8080 & 3.8089 & 3.8098 \\
\rowcolor{bg} \textbf{.581}\markboth{0.581}{0.581} & 3.8107 & 3.8115 & 3.8124 & 3.8133 & 3.8142 & 3.8150 & 3.8159 & 3.8168 & 3.8177 & 3.8186 \\
 \textbf{.582}\markboth{0.582}{0.582} & 3.8194 & 3.8203 & 3.8212 & 3.8221 & 3.8230 & 3.8238 & 3.8247 & 3.8256 & 3.8265 & 3.8274 \\
\rowcolor{bg} \textbf{.583}\markboth{0.583}{0.583} & 3.8282 & 3.8291 & 3.8300 & 3.8309 & 3.8318 & 3.8327 & 3.8335 & 3.8344 & 3.8353 & 3.8362 \\
 \textbf{.584}\markboth{0.584}{0.584} & 3.8371 & 3.8380 & 3.8388 & 3.8397 & 3.8406 & 3.8415 & 3.8424 & 3.8433 & 3.8441 & 3.8450 \\
\rowcolor{bg} \textbf{.585}\markboth{0.585}{0.585} & 3.8459 & 3.8468 & 3.8477 & 3.8486 & 3.8495 & 3.8503 & 3.8512 & 3.8521 & 3.8530 & 3.8539 \\
 \textbf{.586}\markboth{0.586}{0.586} & 3.8548 & 3.8557 & 3.8566 & 3.8574 & 3.8583 & 3.8592 & 3.8601 & 3.8610 & 3.8619 & 3.8628 \\
\rowcolor{bg} \textbf{.587}\markboth{0.587}{0.587} & 3.8637 & 3.8646 & 3.8654 & 3.8663 & 3.8672 & 3.8681 & 3.8690 & 3.8699 & 3.8708 & 3.8717 \\
 \textbf{.588}\markboth{0.588}{0.588} & 3.8726 & 3.8735 & 3.8744 & 3.8753 & 3.8761 & 3.8770 & 3.8779 & 3.8788 & 3.8797 & 3.8806 \\
\rowcolor{bg} \textbf{.589}\markboth{0.589}{0.589} & 3.8815 & 3.8824 & 3.8833 & 3.8842 & 3.8851 & 3.8860 & 3.8869 & 3.8878 & 3.8887 & 3.8896 \\
 \textbf{.590}\markboth{0.590}{0.590} & 3.8905 & 3.8913 & 3.8922 & 3.8931 & 3.8940 & 3.8949 & 3.8958 & 3.8967 & 3.8976 & 3.8985 \\
\rowcolor{bg} \textbf{.591}\markboth{0.591}{0.591} & 3.8994 & 3.9003 & 3.9012 & 3.9021 & 3.9030 & 3.9039 & 3.9048 & 3.9057 & 3.9066 & 3.9075 \\
 \textbf{.592}\markboth{0.592}{0.592} & 3.9084 & 3.9093 & 3.9102 & 3.9111 & 3.9120 & 3.9129 & 3.9138 & 3.9147 & 3.9156 & 3.9165 \\
\rowcolor{bg} \textbf{.593}\markboth{0.593}{0.593} & 3.9174 & 3.9183 & 3.9192 & 3.9201 & 3.9210 & 3.9219 & 3.9228 & 3.9237 & 3.9246 & 3.9255 \\
 \textbf{.594}\markboth{0.594}{0.594} & 3.9264 & 3.9274 & 3.9283 & 3.9292 & 3.9301 & 3.9310 & 3.9319 & 3.9328 & 3.9337 & 3.9346 \\
\rowcolor{bg} \textbf{.595}\markboth{0.595}{0.595} & 3.9355 & 3.9364 & 3.9373 & 3.9382 & 3.9391 & 3.9400 & 3.9409 & 3.9418 & 3.9428 & 3.9437 \\
 \textbf{.596}\markboth{0.596}{0.596} & 3.9446 & 3.9455 & 3.9464 & 3.9473 & 3.9482 & 3.9491 & 3.9500 & 3.9509 & 3.9518 & 3.9528 \\
\rowcolor{bg} \textbf{.597}\markboth{0.597}{0.597} & 3.9537 & 3.9546 & 3.9555 & 3.9564 & 3.9573 & 3.9582 & 3.9591 & 3.9600 & 3.9610 & 3.9619 \\
 \textbf{.598}\markboth{0.598}{0.598} & 3.9628 & 3.9637 & 3.9646 & 3.9655 & 3.9664 & 3.9673 & 3.9683 & 3.9692 & 3.9701 & 3.9710 \\
\rowcolor{bg} \textbf{.599}\markboth{0.599}{0.599} & 3.9719 & 3.9728 & 3.9737 & 3.9747 & 3.9756 & 3.9765 & 3.9774 & 3.9783 & 3.9792 & 3.9802 \\
 \textcolor{blue}{\textbf{.600}}\markboth{0.600}{0.600} & 3.9811 & 3.9820 & 3.9829 & 3.9838 & 3.9847 & 3.9857 & 3.9866 & 3.9875 & 3.9884 & 3.9893 \\
\rowcolor{bg} \textbf{.601}\markboth{0.601}{0.601} & 3.9902 & 3.9912 & 3.9921 & 3.9930 & 3.9939 & 3.9948 & 3.9958 & 3.9967 & 3.9976 & 3.9985 \\
 \textbf{.602}\markboth{0.602}{0.602} & 3.9994 & 4.0004 & 4.0013 & 4.0022 & 4.0031 & 4.0041 & 4.0050 & 4.0059 & 4.0068 & 4.0077 \\
\rowcolor{bg} \textbf{.603}\markboth{0.603}{0.603} & 4.0087 & 4.0096 & 4.0105 & 4.0114 & 4.0124 & 4.0133 & 4.0142 & 4.0151 & 4.0161 & 4.0170 \\
 \textbf{.604}\markboth{0.604}{0.604} & 4.0179 & 4.0188 & 4.0198 & 4.0207 & 4.0216 & 4.0225 & 4.0235 & 4.0244 & 4.0253 & 4.0262 \\
\rowcolor{bg} \textbf{.605}\markboth{0.605}{0.605} & 4.0272 & 4.0281 & 4.0290 & 4.0300 & 4.0309 & 4.0318 & 4.0327 & 4.0337 & 4.0346 & 4.0355 \\
 \textbf{.606}\markboth{0.606}{0.606} & 4.0365 & 4.0374 & 4.0383 & 4.0392 & 4.0402 & 4.0411 & 4.0420 & 4.0430 & 4.0439 & 4.0448 \\
\rowcolor{bg} \textbf{.607}\markboth{0.607}{0.607} & 4.0458 & 4.0467 & 4.0476 & 4.0486 & 4.0495 & 4.0504 & 4.0514 & 4.0523 & 4.0532 & 4.0542 \\
 \textbf{.608}\markboth{0.608}{0.608} & 4.0551 & 4.0560 & 4.0570 & 4.0579 & 4.0588 & 4.0598 & 4.0607 & 4.0616 & 4.0626 & 4.0635 \\
\rowcolor{bg} \textbf{.609}\markboth{0.609}{0.609} & 4.0644 & 4.0654 & 4.0663 & 4.0672 & 4.0682 & 4.0691 & 4.0701 & 4.0710 & 4.0719 & 4.0729 \\
 \textbf{.610}\markboth{0.610}{0.610} & 4.0738 & 4.0747 & 4.0757 & 4.0766 & 4.0776 & 4.0785 & 4.0794 & 4.0804 & 4.0813 & 4.0823 \\
\rowcolor{bg} \textbf{.611}\markboth{0.611}{0.611} & 4.0832 & 4.0841 & 4.0851 & 4.0860 & 4.0870 & 4.0879 & 4.0888 & 4.0898 & 4.0907 & 4.0917 \\
 \textbf{.612}\markboth{0.612}{0.612} & 4.0926 & 4.0935 & 4.0945 & 4.0954 & 4.0964 & 4.0973 & 4.0983 & 4.0992 & 4.1002 & 4.1011 \\
\rowcolor{bg} \textbf{.613}\markboth{0.613}{0.613} & 4.1020 & 4.1030 & 4.1039 & 4.1049 & 4.1058 & 4.1068 & 4.1077 & 4.1087 & 4.1096 & 4.1106 \\
 \textbf{.614}\markboth{0.614}{0.614} & 4.1115 & 4.1124 & 4.1134 & 4.1143 & 4.1153 & 4.1162 & 4.1172 & 4.1181 & 4.1191 & 4.1200 \\
\rowcolor{bg} \textbf{.615}\markboth{0.615}{0.615} & 4.1210 & 4.1219 & 4.1229 & 4.1238 & 4.1248 & 4.1257 & 4.1267 & 4.1276 & 4.1286 & 4.1295 \\
 \textbf{.616}\markboth{0.616}{0.616} & 4.1305 & 4.1314 & 4.1324 & 4.1333 & 4.1343 & 4.1352 & 4.1362 & 4.1371 & 4.1381 & 4.1390 \\
\rowcolor{bg} \textbf{.617}\markboth{0.617}{0.617} & 4.1400 & 4.1410 & 4.1419 & 4.1429 & 4.1438 & 4.1448 & 4.1457 & 4.1467 & 4.1476 & 4.1486 \\
 \textbf{.618}\markboth{0.618}{0.618} & 4.1495 & 4.1505 & 4.1515 & 4.1524 & 4.1534 & 4.1543 & 4.1553 & 4.1562 & 4.1572 & 4.1581 \\
\rowcolor{bg} \textbf{.619}\markboth{0.619}{0.619} & 4.1591 & 4.1601 & 4.1610 & 4.1620 & 4.1629 & 4.1639 & 4.1649 & 4.1658 & 4.1668 & 4.1677 \\
 \textbf{.620}\markboth{0.620}{0.620} & 4.1687 & 4.1697 & 4.1706 & 4.1716 & 4.1725 & 4.1735 & 4.1745 & 4.1754 & 4.1764 & 4.1773 \\
\rowcolor{bg} \textbf{.621}\markboth{0.621}{0.621} & 4.1783 & 4.1793 & 4.1802 & 4.1812 & 4.1822 & 4.1831 & 4.1841 & 4.1850 & 4.1860 & 4.1870 \\
 \textbf{.622}\markboth{0.622}{0.622} & 4.1879 & 4.1889 & 4.1899 & 4.1908 & 4.1918 & 4.1928 & 4.1937 & 4.1947 & 4.1957 & 4.1966 \\
\rowcolor{bg} \textbf{.623}\markboth{0.623}{0.623} & 4.1976 & 4.1986 & 4.1995 & 4.2005 & 4.2015 & 4.2024 & 4.2034 & 4.2044 & 4.2053 & 4.2063 \\
 \textbf{.624}\markboth{0.624}{0.624} & 4.2073 & 4.2082 & 4.2092 & 4.2102 & 4.2111 & 4.2121 & 4.2131 & 4.2141 & 4.2150 & 4.2160 \\
\rowcolor{bg} \textbf{.625}\markboth{0.625}{0.625} & 4.2170 & 4.2179 & 4.2189 & 4.2199 & 4.2209 & 4.2218 & 4.2228 & 4.2238 & 4.2247 & 4.2257 \\
 \textbf{.626}\markboth{0.626}{0.626} & 4.2267 & 4.2277 & 4.2286 & 4.2296 & 4.2306 & 4.2316 & 4.2325 & 4.2335 & 4.2345 & 4.2355 \\
\rowcolor{bg} \textbf{.627}\markboth{0.627}{0.627} & 4.2364 & 4.2374 & 4.2384 & 4.2394 & 4.2403 & 4.2413 & 4.2423 & 4.2433 & 4.2442 & 4.2452 \\
 \textbf{.628}\markboth{0.628}{0.628} & 4.2462 & 4.2472 & 4.2482 & 4.2491 & 4.2501 & 4.2511 & 4.2521 & 4.2530 & 4.2540 & 4.2550 \\
\rowcolor{bg} \textbf{.629}\markboth{0.629}{0.629} & 4.2560 & 4.2570 & 4.2579 & 4.2589 & 4.2599 & 4.2609 & 4.2619 & 4.2628 & 4.2638 & 4.2648 \\
 \textbf{.630}\markboth{0.630}{0.630} & 4.2658 & 4.2668 & 4.2678 & 4.2687 & 4.2697 & 4.2707 & 4.2717 & 4.2727 & 4.2737 & 4.2746 \\
\rowcolor{bg} \textbf{.631}\markboth{0.631}{0.631} & 4.2756 & 4.2766 & 4.2776 & 4.2786 & 4.2796 & 4.2806 & 4.2815 & 4.2825 & 4.2835 & 4.2845 \\
 \textbf{.632}\markboth{0.632}{0.632} & 4.2855 & 4.2865 & 4.2875 & 4.2884 & 4.2894 & 4.2904 & 4.2914 & 4.2924 & 4.2934 & 4.2944 \\
\rowcolor{bg} \textbf{.633}\markboth{0.633}{0.633} & 4.2954 & 4.2964 & 4.2973 & 4.2983 & 4.2993 & 4.3003 & 4.3013 & 4.3023 & 4.3033 & 4.3043 \\
 \textbf{.634}\markboth{0.634}{0.634} & 4.3053 & 4.3063 & 4.3072 & 4.3082 & 4.3092 & 4.3102 & 4.3112 & 4.3122 & 4.3132 & 4.3142 \\
\rowcolor{bg} \textbf{.635}\markboth{0.635}{0.635} & 4.3152 & 4.3162 & 4.3172 & 4.3182 & 4.3192 & 4.3202 & 4.3212 & 4.3222 & 4.3231 & 4.3241 \\
 \textbf{.636}\markboth{0.636}{0.636} & 4.3251 & 4.3261 & 4.3271 & 4.3281 & 4.3291 & 4.3301 & 4.3311 & 4.3321 & 4.3331 & 4.3341 \\
\rowcolor{bg} \textbf{.637}\markboth{0.637}{0.637} & 4.3351 & 4.3361 & 4.3371 & 4.3381 & 4.3391 & 4.3401 & 4.3411 & 4.3421 & 4.3431 & 4.3441 \\
 \textbf{.638}\markboth{0.638}{0.638} & 4.3451 & 4.3461 & 4.3471 & 4.3481 & 4.3491 & 4.3501 & 4.3511 & 4.3521 & 4.3531 & 4.3541 \\
\rowcolor{bg} \textbf{.639}\markboth{0.639}{0.639} & 4.3551 & 4.3561 & 4.3571 & 4.3581 & 4.3591 & 4.3601 & 4.3611 & 4.3621 & 4.3631 & 4.3642 \\
 \textbf{.640}\markboth{0.640}{0.640} & 4.3652 & 4.3662 & 4.3672 & 4.3682 & 4.3692 & 4.3702 & 4.3712 & 4.3722 & 4.3732 & 4.3742 \\
\rowcolor{bg} \textbf{.641}\markboth{0.641}{0.641} & 4.3752 & 4.3762 & 4.3772 & 4.3782 & 4.3793 & 4.3803 & 4.3813 & 4.3823 & 4.3833 & 4.3843 \\
 \textbf{.642}\markboth{0.642}{0.642} & 4.3853 & 4.3863 & 4.3873 & 4.3883 & 4.3893 & 4.3904 & 4.3914 & 4.3924 & 4.3934 & 4.3944 \\
\rowcolor{bg} \textbf{.643}\markboth{0.643}{0.643} & 4.3954 & 4.3964 & 4.3974 & 4.3985 & 4.3995 & 4.4005 & 4.4015 & 4.4025 & 4.4035 & 4.4045 \\
 \textbf{.644}\markboth{0.644}{0.644} & 4.4055 & 4.4066 & 4.4076 & 4.4086 & 4.4096 & 4.4106 & 4.4116 & 4.4127 & 4.4137 & 4.4147 \\
\rowcolor{bg} \textbf{.645}\markboth{0.645}{0.645} & 4.4157 & 4.4167 & 4.4177 & 4.4188 & 4.4198 & 4.4208 & 4.4218 & 4.4228 & 4.4238 & 4.4249 \\
 \textbf{.646}\markboth{0.646}{0.646} & 4.4259 & 4.4269 & 4.4279 & 4.4289 & 4.4300 & 4.4310 & 4.4320 & 4.4330 & 4.4340 & 4.4351 \\
\rowcolor{bg} \textbf{.647}\markboth{0.647}{0.647} & 4.4361 & 4.4371 & 4.4381 & 4.4392 & 4.4402 & 4.4412 & 4.4422 & 4.4432 & 4.4443 & 4.4453 \\
 \textbf{.648}\markboth{0.648}{0.648} & 4.4463 & 4.4473 & 4.4484 & 4.4494 & 4.4504 & 4.4514 & 4.4525 & 4.4535 & 4.4545 & 4.4555 \\
\rowcolor{bg} \textbf{.649}\markboth{0.649}{0.649} & 4.4566 & 4.4576 & 4.4586 & 4.4596 & 4.4607 & 4.4617 & 4.4627 & 4.4638 & 4.4648 & 4.4658 \\
 \textbf{.650}\markboth{0.650}{0.650} & 4.4668 & 4.4679 & 4.4689 & 4.4699 & 4.4710 & 4.4720 & 4.4730 & 4.4740 & 4.4751 & 4.4761 \\
\rowcolor{bg} \textbf{.651}\markboth{0.651}{0.651} & 4.4771 & 4.4782 & 4.4792 & 4.4802 & 4.4813 & 4.4823 & 4.4833 & 4.4844 & 4.4854 & 4.4864 \\
 \textbf{.652}\markboth{0.652}{0.652} & 4.4875 & 4.4885 & 4.4895 & 4.4906 & 4.4916 & 4.4926 & 4.4937 & 4.4947 & 4.4957 & 4.4968 \\
\rowcolor{bg} \textbf{.653}\markboth{0.653}{0.653} & 4.4978 & 4.4988 & 4.4999 & 4.5009 & 4.5019 & 4.5030 & 4.5040 & 4.5051 & 4.5061 & 4.5071 \\
 \textbf{.654}\markboth{0.654}{0.654} & 4.5082 & 4.5092 & 4.5102 & 4.5113 & 4.5123 & 4.5134 & 4.5144 & 4.5154 & 4.5165 & 4.5175 \\
\rowcolor{bg} \textbf{.655}\markboth{0.655}{0.655} & 4.5186 & 4.5196 & 4.5206 & 4.5217 & 4.5227 & 4.5238 & 4.5248 & 4.5258 & 4.5269 & 4.5279 \\
 \textbf{.656}\markboth{0.656}{0.656} & 4.5290 & 4.5300 & 4.5311 & 4.5321 & 4.5331 & 4.5342 & 4.5352 & 4.5363 & 4.5373 & 4.5384 \\
\rowcolor{bg} \textbf{.657}\markboth{0.657}{0.657} & 4.5394 & 4.5405 & 4.5415 & 4.5426 & 4.5436 & 4.5446 & 4.5457 & 4.5467 & 4.5478 & 4.5488 \\
 \textbf{.658}\markboth{0.658}{0.658} & 4.5499 & 4.5509 & 4.5520 & 4.5530 & 4.5541 & 4.5551 & 4.5562 & 4.5572 & 4.5583 & 4.5593 \\
\rowcolor{bg} \textbf{.659}\markboth{0.659}{0.659} & 4.5604 & 4.5614 & 4.5625 & 4.5635 & 4.5646 & 4.5656 & 4.5667 & 4.5677 & 4.5688 & 4.5698 \\
 \textbf{.660}\markboth{0.660}{0.660} & 4.5709 & 4.5719 & 4.5730 & 4.5740 & 4.5751 & 4.5761 & 4.5772 & 4.5783 & 4.5793 & 4.5804 \\
\rowcolor{bg} \textbf{.661}\markboth{0.661}{0.661} & 4.5814 & 4.5825 & 4.5835 & 4.5846 & 4.5856 & 4.5867 & 4.5878 & 4.5888 & 4.5899 & 4.5909 \\
 \textbf{.662}\markboth{0.662}{0.662} & 4.5920 & 4.5930 & 4.5941 & 4.5952 & 4.5962 & 4.5973 & 4.5983 & 4.5994 & 4.6004 & 4.6015 \\
\rowcolor{bg} \textbf{.663}\markboth{0.663}{0.663} & 4.6026 & 4.6036 & 4.6047 & 4.6057 & 4.6068 & 4.6079 & 4.6089 & 4.6100 & 4.6111 & 4.6121 \\
 \textbf{.664}\markboth{0.664}{0.664} & 4.6132 & 4.6142 & 4.6153 & 4.6164 & 4.6174 & 4.6185 & 4.6196 & 4.6206 & 4.6217 & 4.6227 \\
\rowcolor{bg} \textbf{.665}\markboth{0.665}{0.665} & 4.6238 & 4.6249 & 4.6259 & 4.6270 & 4.6281 & 4.6291 & 4.6302 & 4.6313 & 4.6323 & 4.6334 \\
 \textbf{.666}\markboth{0.666}{0.666} & 4.6345 & 4.6355 & 4.6366 & 4.6377 & 4.6387 & 4.6398 & 4.6409 & 4.6419 & 4.6430 & 4.6441 \\
\rowcolor{bg} \textbf{.667}\markboth{0.667}{0.667} & 4.6452 & 4.6462 & 4.6473 & 4.6484 & 4.6494 & 4.6505 & 4.6516 & 4.6526 & 4.6537 & 4.6548 \\
 \textbf{.668}\markboth{0.668}{0.668} & 4.6559 & 4.6569 & 4.6580 & 4.6591 & 4.6602 & 4.6612 & 4.6623 & 4.6634 & 4.6644 & 4.6655 \\
\rowcolor{bg} \textbf{.669}\markboth{0.669}{0.669} & 4.6666 & 4.6677 & 4.6687 & 4.6698 & 4.6709 & 4.6720 & 4.6730 & 4.6741 & 4.6752 & 4.6763 \\
 \textbf{.670}\markboth{0.670}{0.670} & 4.6774 & 4.6784 & 4.6795 & 4.6806 & 4.6817 & 4.6827 & 4.6838 & 4.6849 & 4.6860 & 4.6871 \\
\rowcolor{bg} \textbf{.671}\markboth{0.671}{0.671} & 4.6881 & 4.6892 & 4.6903 & 4.6914 & 4.6925 & 4.6935 & 4.6946 & 4.6957 & 4.6968 & 4.6979 \\
 \textbf{.672}\markboth{0.672}{0.672} & 4.6989 & 4.7000 & 4.7011 & 4.7022 & 4.7033 & 4.7044 & 4.7054 & 4.7065 & 4.7076 & 4.7087 \\
\rowcolor{bg} \textbf{.673}\markboth{0.673}{0.673} & 4.7098 & 4.7109 & 4.7119 & 4.7130 & 4.7141 & 4.7152 & 4.7163 & 4.7174 & 4.7185 & 4.7195 \\
 \textbf{.674}\markboth{0.674}{0.674} & 4.7206 & 4.7217 & 4.7228 & 4.7239 & 4.7250 & 4.7261 & 4.7272 & 4.7282 & 4.7293 & 4.7304 \\
\rowcolor{bg} \textbf{.675}\markboth{0.675}{0.675} & 4.7315 & 4.7326 & 4.7337 & 4.7348 & 4.7359 & 4.7370 & 4.7381 & 4.7391 & 4.7402 & 4.7413 \\
 \textbf{.676}\markboth{0.676}{0.676} & 4.7424 & 4.7435 & 4.7446 & 4.7457 & 4.7468 & 4.7479 & 4.7490 & 4.7501 & 4.7512 & 4.7523 \\
\rowcolor{bg} \textbf{.677}\markboth{0.677}{0.677} & 4.7534 & 4.7544 & 4.7555 & 4.7566 & 4.7577 & 4.7588 & 4.7599 & 4.7610 & 4.7621 & 4.7632 \\
 \textbf{.678}\markboth{0.678}{0.678} & 4.7643 & 4.7654 & 4.7665 & 4.7676 & 4.7687 & 4.7698 & 4.7709 & 4.7720 & 4.7731 & 4.7742 \\
\rowcolor{bg} \textbf{.679}\markboth{0.679}{0.679} & 4.7753 & 4.7764 & 4.7775 & 4.7786 & 4.7797 & 4.7808 & 4.7819 & 4.7830 & 4.7841 & 4.7852 \\
 \textbf{.680}\markboth{0.680}{0.680} & 4.7863 & 4.7874 & 4.7885 & 4.7896 & 4.7907 & 4.7918 & 4.7929 & 4.7940 & 4.7951 & 4.7962 \\
\rowcolor{bg} \textbf{.681}\markboth{0.681}{0.681} & 4.7973 & 4.7984 & 4.7995 & 4.8006 & 4.8018 & 4.8029 & 4.8040 & 4.8051 & 4.8062 & 4.8073 \\
 \textbf{.682}\markboth{0.682}{0.682} & 4.8084 & 4.8095 & 4.8106 & 4.8117 & 4.8128 & 4.8139 & 4.8150 & 4.8161 & 4.8173 & 4.8184 \\
\rowcolor{bg} \textbf{.683}\markboth{0.683}{0.683} & 4.8195 & 4.8206 & 4.8217 & 4.8228 & 4.8239 & 4.8250 & 4.8261 & 4.8273 & 4.8284 & 4.8295 \\
 \textbf{.684}\markboth{0.684}{0.684} & 4.8306 & 4.8317 & 4.8328 & 4.8339 & 4.8350 & 4.8362 & 4.8373 & 4.8384 & 4.8395 & 4.8406 \\
\rowcolor{bg} \textbf{.685}\markboth{0.685}{0.685} & 4.8417 & 4.8428 & 4.8440 & 4.8451 & 4.8462 & 4.8473 & 4.8484 & 4.8495 & 4.8507 & 4.8518 \\
 \textbf{.686}\markboth{0.686}{0.686} & 4.8529 & 4.8540 & 4.8551 & 4.8562 & 4.8574 & 4.8585 & 4.8596 & 4.8607 & 4.8618 & 4.8630 \\
\rowcolor{bg} \textbf{.687}\markboth{0.687}{0.687} & 4.8641 & 4.8652 & 4.8663 & 4.8674 & 4.8686 & 4.8697 & 4.8708 & 4.8719 & 4.8730 & 4.8742 \\
 \textbf{.688}\markboth{0.688}{0.688} & 4.8753 & 4.8764 & 4.8775 & 4.8787 & 4.8798 & 4.8809 & 4.8820 & 4.8831 & 4.8843 & 4.8854 \\
\rowcolor{bg} \textbf{.689}\markboth{0.689}{0.689} & 4.8865 & 4.8876 & 4.8888 & 4.8899 & 4.8910 & 4.8922 & 4.8933 & 4.8944 & 4.8955 & 4.8967 \\
 \textbf{.690}\markboth{0.690}{0.690} & 4.8978 & 4.8989 & 4.9000 & 4.9012 & 4.9023 & 4.9034 & 4.9046 & 4.9057 & 4.9068 & 4.9079 \\
\rowcolor{bg} \textbf{.691}\markboth{0.691}{0.691} & 4.9091 & 4.9102 & 4.9113 & 4.9125 & 4.9136 & 4.9147 & 4.9159 & 4.9170 & 4.9181 & 4.9193 \\
 \textbf{.692}\markboth{0.692}{0.692} & 4.9204 & 4.9215 & 4.9227 & 4.9238 & 4.9249 & 4.9261 & 4.9272 & 4.9283 & 4.9295 & 4.9306 \\
\rowcolor{bg} \textbf{.693}\markboth{0.693}{0.693} & 4.9317 & 4.9329 & 4.9340 & 4.9351 & 4.9363 & 4.9374 & 4.9386 & 4.9397 & 4.9408 & 4.9420 \\
 \textbf{.694}\markboth{0.694}{0.694} & 4.9431 & 4.9442 & 4.9454 & 4.9465 & 4.9477 & 4.9488 & 4.9499 & 4.9511 & 4.9522 & 4.9534 \\
\rowcolor{bg} \textbf{.695}\markboth{0.695}{0.695} & 4.9545 & 4.9556 & 4.9568 & 4.9579 & 4.9591 & 4.9602 & 4.9614 & 4.9625 & 4.9636 & 4.9648 \\
 \textbf{.696}\markboth{0.696}{0.696} & 4.9659 & 4.9671 & 4.9682 & 4.9694 & 4.9705 & 4.9716 & 4.9728 & 4.9739 & 4.9751 & 4.9762 \\
\rowcolor{bg} \textbf{.697}\markboth{0.697}{0.697} & 4.9774 & 4.9785 & 4.9797 & 4.9808 & 4.9820 & 4.9831 & 4.9843 & 4.9854 & 4.9865 & 4.9877 \\
 \textbf{.698}\markboth{0.698}{0.698} & 4.9888 & 4.9900 & 4.9911 & 4.9923 & 4.9934 & 4.9946 & 4.9957 & 4.9969 & 4.9980 & 4.9992 \\
\rowcolor{bg} \textbf{.699}\markboth{0.699}{0.699} & 5.0003 & 5.0015 & 5.0026 & 5.0038 & 5.0050 & 5.0061 & 5.0073 & 5.0084 & 5.0096 & 5.0107 \\
 \textcolor{blue}{\textbf{.700}}\markboth{0.700}{0.700} & 5.0119 & 5.0130 & 5.0142 & 5.0153 & 5.0165 & 5.0176 & 5.0188 & 5.0200 & 5.0211 & 5.0223 \\
\rowcolor{bg} \textbf{.701}\markboth{0.701}{0.701} & 5.0234 & 5.0246 & 5.0257 & 5.0269 & 5.0281 & 5.0292 & 5.0304 & 5.0315 & 5.0327 & 5.0338 \\
 \textbf{.702}\markboth{0.702}{0.702} & 5.0350 & 5.0362 & 5.0373 & 5.0385 & 5.0396 & 5.0408 & 5.0420 & 5.0431 & 5.0443 & 5.0455 \\
\rowcolor{bg} \textbf{.703}\markboth{0.703}{0.703} & 5.0466 & 5.0478 & 5.0489 & 5.0501 & 5.0513 & 5.0524 & 5.0536 & 5.0548 & 5.0559 & 5.0571 \\
 \textbf{.704}\markboth{0.704}{0.704} & 5.0582 & 5.0594 & 5.0606 & 5.0617 & 5.0629 & 5.0641 & 5.0652 & 5.0664 & 5.0676 & 5.0687 \\
\rowcolor{bg} \textbf{.705}\markboth{0.705}{0.705} & 5.0699 & 5.0711 & 5.0722 & 5.0734 & 5.0746 & 5.0757 & 5.0769 & 5.0781 & 5.0793 & 5.0804 \\
 \textbf{.706}\markboth{0.706}{0.706} & 5.0816 & 5.0828 & 5.0839 & 5.0851 & 5.0863 & 5.0874 & 5.0886 & 5.0898 & 5.0910 & 5.0921 \\
\rowcolor{bg} \textbf{.707}\markboth{0.707}{0.707} & 5.0933 & 5.0945 & 5.0957 & 5.0968 & 5.0980 & 5.0992 & 5.1004 & 5.1015 & 5.1027 & 5.1039 \\
 \textbf{.708}\markboth{0.708}{0.708} & 5.1050 & 5.1062 & 5.1074 & 5.1086 & 5.1098 & 5.1109 & 5.1121 & 5.1133 & 5.1145 & 5.1156 \\
\rowcolor{bg} \textbf{.709}\markboth{0.709}{0.709} & 5.1168 & 5.1180 & 5.1192 & 5.1204 & 5.1215 & 5.1227 & 5.1239 & 5.1251 & 5.1263 & 5.1274 \\
 \textbf{.710}\markboth{0.710}{0.710} & 5.1286 & 5.1298 & 5.1310 & 5.1322 & 5.1333 & 5.1345 & 5.1357 & 5.1369 & 5.1381 & 5.1393 \\
\rowcolor{bg} \textbf{.711}\markboth{0.711}{0.711} & 5.1404 & 5.1416 & 5.1428 & 5.1440 & 5.1452 & 5.1464 & 5.1475 & 5.1487 & 5.1499 & 5.1511 \\
 \textbf{.712}\markboth{0.712}{0.712} & 5.1523 & 5.1535 & 5.1547 & 5.1558 & 5.1570 & 5.1582 & 5.1594 & 5.1606 & 5.1618 & 5.1630 \\
\rowcolor{bg} \textbf{.713}\markboth{0.713}{0.713} & 5.1642 & 5.1654 & 5.1665 & 5.1677 & 5.1689 & 5.1701 & 5.1713 & 5.1725 & 5.1737 & 5.1749 \\
 \textbf{.714}\markboth{0.714}{0.714} & 5.1761 & 5.1773 & 5.1785 & 5.1796 & 5.1808 & 5.1820 & 5.1832 & 5.1844 & 5.1856 & 5.1868 \\
\rowcolor{bg} \textbf{.715}\markboth{0.715}{0.715} & 5.1880 & 5.1892 & 5.1904 & 5.1916 & 5.1928 & 5.1940 & 5.1952 & 5.1964 & 5.1976 & 5.1988 \\
 \textbf{.716}\markboth{0.716}{0.716} & 5.2000 & 5.2012 & 5.2024 & 5.2036 & 5.2048 & 5.2060 & 5.2071 & 5.2083 & 5.2095 & 5.2107 \\
\rowcolor{bg} \textbf{.717}\markboth{0.717}{0.717} & 5.2119 & 5.2131 & 5.2143 & 5.2155 & 5.2167 & 5.2180 & 5.2192 & 5.2204 & 5.2216 & 5.2228 \\
 \textbf{.718}\markboth{0.718}{0.718} & 5.2240 & 5.2252 & 5.2264 & 5.2276 & 5.2288 & 5.2300 & 5.2312 & 5.2324 & 5.2336 & 5.2348 \\
\rowcolor{bg} \textbf{.719}\markboth{0.719}{0.719} & 5.2360 & 5.2372 & 5.2384 & 5.2396 & 5.2408 & 5.2420 & 5.2432 & 5.2445 & 5.2457 & 5.2469 \\
 \textbf{.720}\markboth{0.720}{0.720} & 5.2481 & 5.2493 & 5.2505 & 5.2517 & 5.2529 & 5.2541 & 5.2553 & 5.2565 & 5.2578 & 5.2590 \\
\rowcolor{bg} \textbf{.721}\markboth{0.721}{0.721} & 5.2602 & 5.2614 & 5.2626 & 5.2638 & 5.2650 & 5.2662 & 5.2674 & 5.2687 & 5.2699 & 5.2711 \\
 \textbf{.722}\markboth{0.722}{0.722} & 5.2723 & 5.2735 & 5.2747 & 5.2759 & 5.2772 & 5.2784 & 5.2796 & 5.2808 & 5.2820 & 5.2832 \\
\rowcolor{bg} \textbf{.723}\markboth{0.723}{0.723} & 5.2845 & 5.2857 & 5.2869 & 5.2881 & 5.2893 & 5.2905 & 5.2918 & 5.2930 & 5.2942 & 5.2954 \\
 \textbf{.724}\markboth{0.724}{0.724} & 5.2966 & 5.2979 & 5.2991 & 5.3003 & 5.3015 & 5.3027 & 5.3040 & 5.3052 & 5.3064 & 5.3076 \\
\rowcolor{bg} \textbf{.725}\markboth{0.725}{0.725} & 5.3088 & 5.3101 & 5.3113 & 5.3125 & 5.3137 & 5.3150 & 5.3162 & 5.3174 & 5.3186 & 5.3199 \\
 \textbf{.726}\markboth{0.726}{0.726} & 5.3211 & 5.3223 & 5.3235 & 5.3248 & 5.3260 & 5.3272 & 5.3284 & 5.3297 & 5.3309 & 5.3321 \\
\rowcolor{bg} \textbf{.727}\markboth{0.727}{0.727} & 5.3333 & 5.3346 & 5.3358 & 5.3370 & 5.3383 & 5.3395 & 5.3407 & 5.3420 & 5.3432 & 5.3444 \\
 \textbf{.728}\markboth{0.728}{0.728} & 5.3456 & 5.3469 & 5.3481 & 5.3493 & 5.3506 & 5.3518 & 5.3530 & 5.3543 & 5.3555 & 5.3567 \\
\rowcolor{bg} \textbf{.729}\markboth{0.729}{0.729} & 5.3580 & 5.3592 & 5.3604 & 5.3617 & 5.3629 & 5.3641 & 5.3654 & 5.3666 & 5.3678 & 5.3691 \\
 \textbf{.730}\markboth{0.730}{0.730} & 5.3703 & 5.3716 & 5.3728 & 5.3740 & 5.3753 & 5.3765 & 5.3777 & 5.3790 & 5.3802 & 5.3815 \\
\rowcolor{bg} \textbf{.731}\markboth{0.731}{0.731} & 5.3827 & 5.3839 & 5.3852 & 5.3864 & 5.3877 & 5.3889 & 5.3901 & 5.3914 & 5.3926 & 5.3939 \\
 \textbf{.732}\markboth{0.732}{0.732} & 5.3951 & 5.3963 & 5.3976 & 5.3988 & 5.4001 & 5.4013 & 5.4026 & 5.4038 & 5.4051 & 5.4063 \\
\rowcolor{bg} \textbf{.733}\markboth{0.733}{0.733} & 5.4075 & 5.4088 & 5.4100 & 5.4113 & 5.4125 & 5.4138 & 5.4150 & 5.4163 & 5.4175 & 5.4188 \\
 \textbf{.734}\markboth{0.734}{0.734} & 5.4200 & 5.4213 & 5.4225 & 5.4238 & 5.4250 & 5.4263 & 5.4275 & 5.4288 & 5.4300 & 5.4313 \\
\rowcolor{bg} \textbf{.735}\markboth{0.735}{0.735} & 5.4325 & 5.4338 & 5.4350 & 5.4363 & 5.4375 & 5.4388 & 5.4400 & 5.4413 & 5.4425 & 5.4438 \\
 \textbf{.736}\markboth{0.736}{0.736} & 5.4450 & 5.4463 & 5.4475 & 5.4488 & 5.4500 & 5.4513 & 5.4526 & 5.4538 & 5.4551 & 5.4563 \\
\rowcolor{bg} \textbf{.737}\markboth{0.737}{0.737} & 5.4576 & 5.4588 & 5.4601 & 5.4613 & 5.4626 & 5.4639 & 5.4651 & 5.4664 & 5.4676 & 5.4689 \\
 \textbf{.738}\markboth{0.738}{0.738} & 5.4702 & 5.4714 & 5.4727 & 5.4739 & 5.4752 & 5.4765 & 5.4777 & 5.4790 & 5.4802 & 5.4815 \\
\rowcolor{bg} \textbf{.739}\markboth{0.739}{0.739} & 5.4828 & 5.4840 & 5.4853 & 5.4866 & 5.4878 & 5.4891 & 5.4903 & 5.4916 & 5.4929 & 5.4941 \\
 \textbf{.740}\markboth{0.740}{0.740} & 5.4954 & 5.4967 & 5.4979 & 5.4992 & 5.5005 & 5.5017 & 5.5030 & 5.5043 & 5.5055 & 5.5068 \\
\rowcolor{bg} \textbf{.741}\markboth{0.741}{0.741} & 5.5081 & 5.5093 & 5.5106 & 5.5119 & 5.5132 & 5.5144 & 5.5157 & 5.5170 & 5.5182 & 5.5195 \\
 \textbf{.742}\markboth{0.742}{0.742} & 5.5208 & 5.5220 & 5.5233 & 5.5246 & 5.5259 & 5.5271 & 5.5284 & 5.5297 & 5.5310 & 5.5322 \\
\rowcolor{bg} \textbf{.743}\markboth{0.743}{0.743} & 5.5335 & 5.5348 & 5.5360 & 5.5373 & 5.5386 & 5.5399 & 5.5412 & 5.5424 & 5.5437 & 5.5450 \\
 \textbf{.744}\markboth{0.744}{0.744} & 5.5463 & 5.5475 & 5.5488 & 5.5501 & 5.5514 & 5.5526 & 5.5539 & 5.5552 & 5.5565 & 5.5578 \\
\rowcolor{bg} \textbf{.745}\markboth{0.745}{0.745} & 5.5590 & 5.5603 & 5.5616 & 5.5629 & 5.5642 & 5.5654 & 5.5667 & 5.5680 & 5.5693 & 5.5706 \\
 \textbf{.746}\markboth{0.746}{0.746} & 5.5719 & 5.5731 & 5.5744 & 5.5757 & 5.5770 & 5.5783 & 5.5796 & 5.5808 & 5.5821 & 5.5834 \\
\rowcolor{bg} \textbf{.747}\markboth{0.747}{0.747} & 5.5847 & 5.5860 & 5.5873 & 5.5886 & 5.5898 & 5.5911 & 5.5924 & 5.5937 & 5.5950 & 5.5963 \\
 \textbf{.748}\markboth{0.748}{0.748} & 5.5976 & 5.5989 & 5.6002 & 5.6014 & 5.6027 & 5.6040 & 5.6053 & 5.6066 & 5.6079 & 5.6092 \\
\rowcolor{bg} \textbf{.749}\markboth{0.749}{0.749} & 5.6105 & 5.6118 & 5.6131 & 5.6144 & 5.6156 & 5.6169 & 5.6182 & 5.6195 & 5.6208 & 5.6221 \\
 \textbf{.750}\markboth{0.750}{0.750} & 5.6234 & 5.6247 & 5.6260 & 5.6273 & 5.6286 & 5.6299 & 5.6312 & 5.6325 & 5.6338 & 5.6351 \\
\rowcolor{bg} \textbf{.751}\markboth{0.751}{0.751} & 5.6364 & 5.6377 & 5.6390 & 5.6403 & 5.6416 & 5.6429 & 5.6442 & 5.6455 & 5.6468 & 5.6481 \\
 \textbf{.752}\markboth{0.752}{0.752} & 5.6494 & 5.6507 & 5.6520 & 5.6533 & 5.6546 & 5.6559 & 5.6572 & 5.6585 & 5.6598 & 5.6611 \\
\rowcolor{bg} \textbf{.753}\markboth{0.753}{0.753} & 5.6624 & 5.6637 & 5.6650 & 5.6663 & 5.6676 & 5.6689 & 5.6702 & 5.6715 & 5.6728 & 5.6741 \\
 \textbf{.754}\markboth{0.754}{0.754} & 5.6754 & 5.6768 & 5.6781 & 5.6794 & 5.6807 & 5.6820 & 5.6833 & 5.6846 & 5.6859 & 5.6872 \\
\rowcolor{bg} \textbf{.755}\markboth{0.755}{0.755} & 5.6885 & 5.6898 & 5.6911 & 5.6925 & 5.6938 & 5.6951 & 5.6964 & 5.6977 & 5.6990 & 5.7003 \\
 \textbf{.756}\markboth{0.756}{0.756} & 5.7016 & 5.7030 & 5.7043 & 5.7056 & 5.7069 & 5.7082 & 5.7095 & 5.7108 & 5.7122 & 5.7135 \\
\rowcolor{bg} \textbf{.757}\markboth{0.757}{0.757} & 5.7148 & 5.7161 & 5.7174 & 5.7187 & 5.7201 & 5.7214 & 5.7227 & 5.7240 & 5.7253 & 5.7266 \\
 \textbf{.758}\markboth{0.758}{0.758} & 5.7280 & 5.7293 & 5.7306 & 5.7319 & 5.7332 & 5.7346 & 5.7359 & 5.7372 & 5.7385 & 5.7398 \\
\rowcolor{bg} \textbf{.759}\markboth{0.759}{0.759} & 5.7412 & 5.7425 & 5.7438 & 5.7451 & 5.7465 & 5.7478 & 5.7491 & 5.7504 & 5.7517 & 5.7531 \\
 \textbf{.760}\markboth{0.760}{0.760} & 5.7544 & 5.7557 & 5.7570 & 5.7584 & 5.7597 & 5.7610 & 5.7624 & 5.7637 & 5.7650 & 5.7663 \\
\rowcolor{bg} \textbf{.761}\markboth{0.761}{0.761} & 5.7677 & 5.7690 & 5.7703 & 5.7717 & 5.7730 & 5.7743 & 5.7756 & 5.7770 & 5.7783 & 5.7796 \\
 \textbf{.762}\markboth{0.762}{0.762} & 5.7810 & 5.7823 & 5.7836 & 5.7850 & 5.7863 & 5.7876 & 5.7890 & 5.7903 & 5.7916 & 5.7930 \\
\rowcolor{bg} \textbf{.763}\markboth{0.763}{0.763} & 5.7943 & 5.7956 & 5.7970 & 5.7983 & 5.7996 & 5.8010 & 5.8023 & 5.8036 & 5.8050 & 5.8063 \\
 \textbf{.764}\markboth{0.764}{0.764} & 5.8076 & 5.8090 & 5.8103 & 5.8117 & 5.8130 & 5.8143 & 5.8157 & 5.8170 & 5.8184 & 5.8197 \\
\rowcolor{bg} \textbf{.765}\markboth{0.765}{0.765} & 5.8210 & 5.8224 & 5.8237 & 5.8251 & 5.8264 & 5.8277 & 5.8291 & 5.8304 & 5.8318 & 5.8331 \\
 \textbf{.766}\markboth{0.766}{0.766} & 5.8345 & 5.8358 & 5.8371 & 5.8385 & 5.8398 & 5.8412 & 5.8425 & 5.8439 & 5.8452 & 5.8466 \\
\rowcolor{bg} \textbf{.767}\markboth{0.767}{0.767} & 5.8479 & 5.8492 & 5.8506 & 5.8519 & 5.8533 & 5.8546 & 5.8560 & 5.8573 & 5.8587 & 5.8600 \\
 \textbf{.768}\markboth{0.768}{0.768} & 5.8614 & 5.8627 & 5.8641 & 5.8654 & 5.8668 & 5.8681 & 5.8695 & 5.8708 & 5.8722 & 5.8735 \\
\rowcolor{bg} \textbf{.769}\markboth{0.769}{0.769} & 5.8749 & 5.8762 & 5.8776 & 5.8790 & 5.8803 & 5.8817 & 5.8830 & 5.8844 & 5.8857 & 5.8871 \\
 \textbf{.770}\markboth{0.770}{0.770} & 5.8884 & 5.8898 & 5.8911 & 5.8925 & 5.8939 & 5.8952 & 5.8966 & 5.8979 & 5.8993 & 5.9007 \\
\rowcolor{bg} \textbf{.771}\markboth{0.771}{0.771} & 5.9020 & 5.9034 & 5.9047 & 5.9061 & 5.9074 & 5.9088 & 5.9102 & 5.9115 & 5.9129 & 5.9143 \\
 \textbf{.772}\markboth{0.772}{0.772} & 5.9156 & 5.9170 & 5.9183 & 5.9197 & 5.9211 & 5.9224 & 5.9238 & 5.9252 & 5.9265 & 5.9279 \\
\rowcolor{bg} \textbf{.773}\markboth{0.773}{0.773} & 5.9293 & 5.9306 & 5.9320 & 5.9334 & 5.9347 & 5.9361 & 5.9375 & 5.9388 & 5.9402 & 5.9416 \\
 \textbf{.774}\markboth{0.774}{0.774} & 5.9429 & 5.9443 & 5.9457 & 5.9470 & 5.9484 & 5.9498 & 5.9511 & 5.9525 & 5.9539 & 5.9553 \\
\rowcolor{bg} \textbf{.775}\markboth{0.775}{0.775} & 5.9566 & 5.9580 & 5.9594 & 5.9607 & 5.9621 & 5.9635 & 5.9649 & 5.9662 & 5.9676 & 5.9690 \\
 \textbf{.776}\markboth{0.776}{0.776} & 5.9704 & 5.9717 & 5.9731 & 5.9745 & 5.9759 & 5.9772 & 5.9786 & 5.9800 & 5.9814 & 5.9827 \\
\rowcolor{bg} \textbf{.777}\markboth{0.777}{0.777} & 5.9841 & 5.9855 & 5.9869 & 5.9883 & 5.9896 & 5.9910 & 5.9924 & 5.9938 & 5.9951 & 5.9965 \\
 \textbf{.778}\markboth{0.778}{0.778} & 5.9979 & 5.9993 & 6.0007 & 6.0021 & 6.0034 & 6.0048 & 6.0062 & 6.0076 & 6.0090 & 6.0104 \\
\rowcolor{bg} \textbf{.779}\markboth{0.779}{0.779} & 6.0117 & 6.0131 & 6.0145 & 6.0159 & 6.0173 & 6.0187 & 6.0200 & 6.0214 & 6.0228 & 6.0242 \\
 \textbf{.780}\markboth{0.780}{0.780} & 6.0256 & 6.0270 & 6.0284 & 6.0298 & 6.0311 & 6.0325 & 6.0339 & 6.0353 & 6.0367 & 6.0381 \\
\rowcolor{bg} \textbf{.781}\markboth{0.781}{0.781} & 6.0395 & 6.0409 & 6.0423 & 6.0437 & 6.0451 & 6.0464 & 6.0478 & 6.0492 & 6.0506 & 6.0520 \\
 \textbf{.782}\markboth{0.782}{0.782} & 6.0534 & 6.0548 & 6.0562 & 6.0576 & 6.0590 & 6.0604 & 6.0618 & 6.0632 & 6.0646 & 6.0660 \\
\rowcolor{bg} \textbf{.783}\markboth{0.783}{0.783} & 6.0674 & 6.0688 & 6.0702 & 6.0716 & 6.0730 & 6.0744 & 6.0758 & 6.0772 & 6.0786 & 6.0799 \\
 \textbf{.784}\markboth{0.784}{0.784} & 6.0814 & 6.0828 & 6.0842 & 6.0856 & 6.0870 & 6.0884 & 6.0898 & 6.0912 & 6.0926 & 6.0940 \\
\rowcolor{bg} \textbf{.785}\markboth{0.785}{0.785} & 6.0954 & 6.0968 & 6.0982 & 6.0996 & 6.1010 & 6.1024 & 6.1038 & 6.1052 & 6.1066 & 6.1080 \\
 \textbf{.786}\markboth{0.786}{0.786} & 6.1094 & 6.1108 & 6.1122 & 6.1136 & 6.1150 & 6.1165 & 6.1179 & 6.1193 & 6.1207 & 6.1221 \\
\rowcolor{bg} \textbf{.787}\markboth{0.787}{0.787} & 6.1235 & 6.1249 & 6.1263 & 6.1277 & 6.1291 & 6.1306 & 6.1320 & 6.1334 & 6.1348 & 6.1362 \\
 \textbf{.788}\markboth{0.788}{0.788} & 6.1376 & 6.1390 & 6.1404 & 6.1419 & 6.1433 & 6.1447 & 6.1461 & 6.1475 & 6.1489 & 6.1504 \\
\rowcolor{bg} \textbf{.789}\markboth{0.789}{0.789} & 6.1518 & 6.1532 & 6.1546 & 6.1560 & 6.1574 & 6.1589 & 6.1603 & 6.1617 & 6.1631 & 6.1645 \\
 \textbf{.790}\markboth{0.790}{0.790} & 6.1660 & 6.1674 & 6.1688 & 6.1702 & 6.1716 & 6.1731 & 6.1745 & 6.1759 & 6.1773 & 6.1787 \\
\rowcolor{bg} \textbf{.791}\markboth{0.791}{0.791} & 6.1802 & 6.1816 & 6.1830 & 6.1844 & 6.1859 & 6.1873 & 6.1887 & 6.1901 & 6.1916 & 6.1930 \\
 \textbf{.792}\markboth{0.792}{0.792} & 6.1944 & 6.1958 & 6.1973 & 6.1987 & 6.2001 & 6.2015 & 6.2030 & 6.2044 & 6.2058 & 6.2073 \\
\rowcolor{bg} \textbf{.793}\markboth{0.793}{0.793} & 6.2087 & 6.2101 & 6.2116 & 6.2130 & 6.2144 & 6.2158 & 6.2173 & 6.2187 & 6.2201 & 6.2216 \\
 \textbf{.794}\markboth{0.794}{0.794} & 6.2230 & 6.2244 & 6.2259 & 6.2273 & 6.2287 & 6.2302 & 6.2316 & 6.2330 & 6.2345 & 6.2359 \\
\rowcolor{bg} \textbf{.795}\markboth{0.795}{0.795} & 6.2373 & 6.2388 & 6.2402 & 6.2417 & 6.2431 & 6.2445 & 6.2460 & 6.2474 & 6.2488 & 6.2503 \\
 \textbf{.796}\markboth{0.796}{0.796} & 6.2517 & 6.2532 & 6.2546 & 6.2560 & 6.2575 & 6.2589 & 6.2604 & 6.2618 & 6.2633 & 6.2647 \\
\rowcolor{bg} \textbf{.797}\markboth{0.797}{0.797} & 6.2661 & 6.2676 & 6.2690 & 6.2705 & 6.2719 & 6.2734 & 6.2748 & 6.2762 & 6.2777 & 6.2791 \\
 \textbf{.798}\markboth{0.798}{0.798} & 6.2806 & 6.2820 & 6.2835 & 6.2849 & 6.2864 & 6.2878 & 6.2893 & 6.2907 & 6.2922 & 6.2936 \\
\rowcolor{bg} \textbf{.799}\markboth{0.799}{0.799} & 6.2951 & 6.2965 & 6.2980 & 6.2994 & 6.3009 & 6.3023 & 6.3038 & 6.3052 & 6.3067 & 6.3081 \\
 \textcolor{blue}{\textbf{.800}}\markboth{0.800}{0.800} & 6.3096 & 6.3110 & 6.3125 & 6.3139 & 6.3154 & 6.3168 & 6.3183 & 6.3198 & 6.3212 & 6.3227 \\
\rowcolor{bg} \textbf{.801}\markboth{0.801}{0.801} & 6.3241 & 6.3256 & 6.3270 & 6.3285 & 6.3299 & 6.3314 & 6.3329 & 6.3343 & 6.3358 & 6.3372 \\
 \textbf{.802}\markboth{0.802}{0.802} & 6.3387 & 6.3402 & 6.3416 & 6.3431 & 6.3445 & 6.3460 & 6.3475 & 6.3489 & 6.3504 & 6.3518 \\
\rowcolor{bg} \textbf{.803}\markboth{0.803}{0.803} & 6.3533 & 6.3548 & 6.3562 & 6.3577 & 6.3592 & 6.3606 & 6.3621 & 6.3636 & 6.3650 & 6.3665 \\
 \textbf{.804}\markboth{0.804}{0.804} & 6.3680 & 6.3694 & 6.3709 & 6.3724 & 6.3738 & 6.3753 & 6.3768 & 6.3782 & 6.3797 & 6.3812 \\
\rowcolor{bg} \textbf{.805}\markboth{0.805}{0.805} & 6.3826 & 6.3841 & 6.3856 & 6.3870 & 6.3885 & 6.3900 & 6.3915 & 6.3929 & 6.3944 & 6.3959 \\
 \textbf{.806}\markboth{0.806}{0.806} & 6.3973 & 6.3988 & 6.4003 & 6.4018 & 6.4032 & 6.4047 & 6.4062 & 6.4077 & 6.4091 & 6.4106 \\
\rowcolor{bg} \textbf{.807}\markboth{0.807}{0.807} & 6.4121 & 6.4136 & 6.4150 & 6.4165 & 6.4180 & 6.4195 & 6.4210 & 6.4224 & 6.4239 & 6.4254 \\
 \textbf{.808}\markboth{0.808}{0.808} & 6.4269 & 6.4284 & 6.4298 & 6.4313 & 6.4328 & 6.4343 & 6.4358 & 6.4372 & 6.4387 & 6.4402 \\
\rowcolor{bg} \textbf{.809}\markboth{0.809}{0.809} & 6.4417 & 6.4432 & 6.4447 & 6.4461 & 6.4476 & 6.4491 & 6.4506 & 6.4521 & 6.4536 & 6.4551 \\
 \textbf{.810}\markboth{0.810}{0.810} & 6.4565 & 6.4580 & 6.4595 & 6.4610 & 6.4625 & 6.4640 & 6.4655 & 6.4670 & 6.4684 & 6.4699 \\
\rowcolor{bg} \textbf{.811}\markboth{0.811}{0.811} & 6.4714 & 6.4729 & 6.4744 & 6.4759 & 6.4774 & 6.4789 & 6.4804 & 6.4819 & 6.4834 & 6.4849 \\
 \textbf{.812}\markboth{0.812}{0.812} & 6.4863 & 6.4878 & 6.4893 & 6.4908 & 6.4923 & 6.4938 & 6.4953 & 6.4968 & 6.4983 & 6.4998 \\
\rowcolor{bg} \textbf{.813}\markboth{0.813}{0.813} & 6.5013 & 6.5028 & 6.5043 & 6.5058 & 6.5073 & 6.5088 & 6.5103 & 6.5118 & 6.5133 & 6.5148 \\
 \textbf{.814}\markboth{0.814}{0.814} & 6.5163 & 6.5178 & 6.5193 & 6.5208 & 6.5223 & 6.5238 & 6.5253 & 6.5268 & 6.5283 & 6.5298 \\
\rowcolor{bg} \textbf{.815}\markboth{0.815}{0.815} & 6.5313 & 6.5328 & 6.5343 & 6.5358 & 6.5373 & 6.5388 & 6.5403 & 6.5418 & 6.5433 & 6.5449 \\
 \textbf{.816}\markboth{0.816}{0.816} & 6.5464 & 6.5479 & 6.5494 & 6.5509 & 6.5524 & 6.5539 & 6.5554 & 6.5569 & 6.5584 & 6.5599 \\
\rowcolor{bg} \textbf{.817}\markboth{0.817}{0.817} & 6.5615 & 6.5630 & 6.5645 & 6.5660 & 6.5675 & 6.5690 & 6.5705 & 6.5720 & 6.5736 & 6.5751 \\
 \textbf{.818}\markboth{0.818}{0.818} & 6.5766 & 6.5781 & 6.5796 & 6.5811 & 6.5826 & 6.5842 & 6.5857 & 6.5872 & 6.5887 & 6.5902 \\
\rowcolor{bg} \textbf{.819}\markboth{0.819}{0.819} & 6.5917 & 6.5933 & 6.5948 & 6.5963 & 6.5978 & 6.5993 & 6.6009 & 6.6024 & 6.6039 & 6.6054 \\
 \textbf{.820}\markboth{0.820}{0.820} & 6.6069 & 6.6085 & 6.6100 & 6.6115 & 6.6130 & 6.6145 & 6.6161 & 6.6176 & 6.6191 & 6.6206 \\
\rowcolor{bg} \textbf{.821}\markboth{0.821}{0.821} & 6.6222 & 6.6237 & 6.6252 & 6.6267 & 6.6283 & 6.6298 & 6.6313 & 6.6328 & 6.6344 & 6.6359 \\
 \textbf{.822}\markboth{0.822}{0.822} & 6.6374 & 6.6390 & 6.6405 & 6.6420 & 6.6435 & 6.6451 & 6.6466 & 6.6481 & 6.6497 & 6.6512 \\
\rowcolor{bg} \textbf{.823}\markboth{0.823}{0.823} & 6.6527 & 6.6543 & 6.6558 & 6.6573 & 6.6589 & 6.6604 & 6.6619 & 6.6635 & 6.6650 & 6.6665 \\
 \textbf{.824}\markboth{0.824}{0.824} & 6.6681 & 6.6696 & 6.6711 & 6.6727 & 6.6742 & 6.6757 & 6.6773 & 6.6788 & 6.6804 & 6.6819 \\
\rowcolor{bg} \textbf{.825}\markboth{0.825}{0.825} & 6.6834 & 6.6850 & 6.6865 & 6.6881 & 6.6896 & 6.6911 & 6.6927 & 6.6942 & 6.6958 & 6.6973 \\
 \textbf{.826}\markboth{0.826}{0.826} & 6.6988 & 6.7004 & 6.7019 & 6.7035 & 6.7050 & 6.7066 & 6.7081 & 6.7097 & 6.7112 & 6.7127 \\
\rowcolor{bg} \textbf{.827}\markboth{0.827}{0.827} & 6.7143 & 6.7158 & 6.7174 & 6.7189 & 6.7205 & 6.7220 & 6.7236 & 6.7251 & 6.7267 & 6.7282 \\
 \textbf{.828}\markboth{0.828}{0.828} & 6.7298 & 6.7313 & 6.7329 & 6.7344 & 6.7360 & 6.7375 & 6.7391 & 6.7406 & 6.7422 & 6.7437 \\
\rowcolor{bg} \textbf{.829}\markboth{0.829}{0.829} & 6.7453 & 6.7468 & 6.7484 & 6.7499 & 6.7515 & 6.7531 & 6.7546 & 6.7562 & 6.7577 & 6.7593 \\
 \textbf{.830}\markboth{0.830}{0.830} & 6.7608 & 6.7624 & 6.7639 & 6.7655 & 6.7671 & 6.7686 & 6.7702 & 6.7717 & 6.7733 & 6.7749 \\
\rowcolor{bg} \textbf{.831}\markboth{0.831}{0.831} & 6.7764 & 6.7780 & 6.7795 & 6.7811 & 6.7827 & 6.7842 & 6.7858 & 6.7873 & 6.7889 & 6.7905 \\
 \textbf{.832}\markboth{0.832}{0.832} & 6.7920 & 6.7936 & 6.7952 & 6.7967 & 6.7983 & 6.7999 & 6.8014 & 6.8030 & 6.8046 & 6.8061 \\
\rowcolor{bg} \textbf{.833}\markboth{0.833}{0.833} & 6.8077 & 6.8093 & 6.8108 & 6.8124 & 6.8140 & 6.8155 & 6.8171 & 6.8187 & 6.8202 & 6.8218 \\
 \textbf{.834}\markboth{0.834}{0.834} & 6.8234 & 6.8250 & 6.8265 & 6.8281 & 6.8297 & 6.8312 & 6.8328 & 6.8344 & 6.8360 & 6.8375 \\
\rowcolor{bg} \textbf{.835}\markboth{0.835}{0.835} & 6.8391 & 6.8407 & 6.8423 & 6.8438 & 6.8454 & 6.8470 & 6.8486 & 6.8501 & 6.8517 & 6.8533 \\
 \textbf{.836}\markboth{0.836}{0.836} & 6.8549 & 6.8565 & 6.8580 & 6.8596 & 6.8612 & 6.8628 & 6.8644 & 6.8659 & 6.8675 & 6.8691 \\
\rowcolor{bg} \textbf{.837}\markboth{0.837}{0.837} & 6.8707 & 6.8723 & 6.8738 & 6.8754 & 6.8770 & 6.8786 & 6.8802 & 6.8818 & 6.8834 & 6.8849 \\
 \textbf{.838}\markboth{0.838}{0.838} & 6.8865 & 6.8881 & 6.8897 & 6.8913 & 6.8929 & 6.8945 & 6.8960 & 6.8976 & 6.8992 & 6.9008 \\
\rowcolor{bg} \textbf{.839}\markboth{0.839}{0.839} & 6.9024 & 6.9040 & 6.9056 & 6.9072 & 6.9088 & 6.9103 & 6.9119 & 6.9135 & 6.9151 & 6.9167 \\
 \textbf{.840}\markboth{0.840}{0.840} & 6.9183 & 6.9199 & 6.9215 & 6.9231 & 6.9247 & 6.9263 & 6.9279 & 6.9295 & 6.9311 & 6.9327 \\
\rowcolor{bg} \textbf{.841}\markboth{0.841}{0.841} & 6.9343 & 6.9359 & 6.9375 & 6.9390 & 6.9406 & 6.9422 & 6.9438 & 6.9454 & 6.9470 & 6.9486 \\
 \textbf{.842}\markboth{0.842}{0.842} & 6.9502 & 6.9518 & 6.9534 & 6.9550 & 6.9566 & 6.9582 & 6.9599 & 6.9615 & 6.9631 & 6.9647 \\
\rowcolor{bg} \textbf{.843}\markboth{0.843}{0.843} & 6.9663 & 6.9679 & 6.9695 & 6.9711 & 6.9727 & 6.9743 & 6.9759 & 6.9775 & 6.9791 & 6.9807 \\
 \textbf{.844}\markboth{0.844}{0.844} & 6.9823 & 6.9839 & 6.9855 & 6.9871 & 6.9888 & 6.9904 & 6.9920 & 6.9936 & 6.9952 & 6.9968 \\
\rowcolor{bg} \textbf{.845}\markboth{0.845}{0.845} & 6.9984 & 7.0000 & 7.0016 & 7.0033 & 7.0049 & 7.0065 & 7.0081 & 7.0097 & 7.0113 & 7.0129 \\
 \textbf{.846}\markboth{0.846}{0.846} & 7.0146 & 7.0162 & 7.0178 & 7.0194 & 7.0210 & 7.0226 & 7.0243 & 7.0259 & 7.0275 & 7.0291 \\
\rowcolor{bg} \textbf{.847}\markboth{0.847}{0.847} & 7.0307 & 7.0323 & 7.0340 & 7.0356 & 7.0372 & 7.0388 & 7.0404 & 7.0421 & 7.0437 & 7.0453 \\
 \textbf{.848}\markboth{0.848}{0.848} & 7.0469 & 7.0486 & 7.0502 & 7.0518 & 7.0534 & 7.0550 & 7.0567 & 7.0583 & 7.0599 & 7.0615 \\
\rowcolor{bg} \textbf{.849}\markboth{0.849}{0.849} & 7.0632 & 7.0648 & 7.0664 & 7.0681 & 7.0697 & 7.0713 & 7.0729 & 7.0746 & 7.0762 & 7.0778 \\
 \textbf{.850}\markboth{0.850}{0.850} & 7.0795 & 7.0811 & 7.0827 & 7.0843 & 7.0860 & 7.0876 & 7.0892 & 7.0909 & 7.0925 & 7.0941 \\
\rowcolor{bg} \textbf{.851}\markboth{0.851}{0.851} & 7.0958 & 7.0974 & 7.0990 & 7.1007 & 7.1023 & 7.1040 & 7.1056 & 7.1072 & 7.1089 & 7.1105 \\
 \textbf{.852}\markboth{0.852}{0.852} & 7.1121 & 7.1138 & 7.1154 & 7.1170 & 7.1187 & 7.1203 & 7.1220 & 7.1236 & 7.1252 & 7.1269 \\
\rowcolor{bg} \textbf{.853}\markboth{0.853}{0.853} & 7.1285 & 7.1302 & 7.1318 & 7.1335 & 7.1351 & 7.1367 & 7.1384 & 7.1400 & 7.1417 & 7.1433 \\
 \textbf{.854}\markboth{0.854}{0.854} & 7.1450 & 7.1466 & 7.1483 & 7.1499 & 7.1515 & 7.1532 & 7.1548 & 7.1565 & 7.1581 & 7.1598 \\
\rowcolor{bg} \textbf{.855}\markboth{0.855}{0.855} & 7.1614 & 7.1631 & 7.1647 & 7.1664 & 7.1680 & 7.1697 & 7.1713 & 7.1730 & 7.1746 & 7.1763 \\
 \textbf{.856}\markboth{0.856}{0.856} & 7.1779 & 7.1796 & 7.1812 & 7.1829 & 7.1846 & 7.1862 & 7.1879 & 7.1895 & 7.1912 & 7.1928 \\
\rowcolor{bg} \textbf{.857}\markboth{0.857}{0.857} & 7.1945 & 7.1961 & 7.1978 & 7.1995 & 7.2011 & 7.2028 & 7.2044 & 7.2061 & 7.2078 & 7.2094 \\
 \textbf{.858}\markboth{0.858}{0.858} & 7.2111 & 7.2127 & 7.2144 & 7.2161 & 7.2177 & 7.2194 & 7.2210 & 7.2227 & 7.2244 & 7.2260 \\
\rowcolor{bg} \textbf{.859}\markboth{0.859}{0.859} & 7.2277 & 7.2294 & 7.2310 & 7.2327 & 7.2344 & 7.2360 & 7.2377 & 7.2394 & 7.2410 & 7.2427 \\
 \textbf{.860}\markboth{0.860}{0.860} & 7.2444 & 7.2460 & 7.2477 & 7.2494 & 7.2510 & 7.2527 & 7.2544 & 7.2560 & 7.2577 & 7.2594 \\
\rowcolor{bg} \textbf{.861}\markboth{0.861}{0.861} & 7.2611 & 7.2627 & 7.2644 & 7.2661 & 7.2678 & 7.2694 & 7.2711 & 7.2728 & 7.2744 & 7.2761 \\
 \textbf{.862}\markboth{0.862}{0.862} & 7.2778 & 7.2795 & 7.2812 & 7.2828 & 7.2845 & 7.2862 & 7.2879 & 7.2895 & 7.2912 & 7.2929 \\
\rowcolor{bg} \textbf{.863}\markboth{0.863}{0.863} & 7.2946 & 7.2963 & 7.2979 & 7.2996 & 7.3013 & 7.3030 & 7.3047 & 7.3063 & 7.3080 & 7.3097 \\
 \textbf{.864}\markboth{0.864}{0.864} & 7.3114 & 7.3131 & 7.3148 & 7.3164 & 7.3181 & 7.3198 & 7.3215 & 7.3232 & 7.3249 & 7.3266 \\
\rowcolor{bg} \textbf{.865}\markboth{0.865}{0.865} & 7.3282 & 7.3299 & 7.3316 & 7.3333 & 7.3350 & 7.3367 & 7.3384 & 7.3401 & 7.3418 & 7.3434 \\
 \textbf{.866}\markboth{0.866}{0.866} & 7.3451 & 7.3468 & 7.3485 & 7.3502 & 7.3519 & 7.3536 & 7.3553 & 7.3570 & 7.3587 & 7.3604 \\
\rowcolor{bg} \textbf{.867}\markboth{0.867}{0.867} & 7.3621 & 7.3638 & 7.3655 & 7.3672 & 7.3689 & 7.3706 & 7.3722 & 7.3739 & 7.3756 & 7.3773 \\
 \textbf{.868}\markboth{0.868}{0.868} & 7.3790 & 7.3807 & 7.3824 & 7.3841 & 7.3858 & 7.3875 & 7.3892 & 7.3909 & 7.3926 & 7.3943 \\
\rowcolor{bg} \textbf{.869}\markboth{0.869}{0.869} & 7.3961 & 7.3978 & 7.3995 & 7.4012 & 7.4029 & 7.4046 & 7.4063 & 7.4080 & 7.4097 & 7.4114 \\
 \textbf{.870}\markboth{0.870}{0.870} & 7.4131 & 7.4148 & 7.4165 & 7.4182 & 7.4199 & 7.4216 & 7.4234 & 7.4251 & 7.4268 & 7.4285 \\
\rowcolor{bg} \textbf{.871}\markboth{0.871}{0.871} & 7.4302 & 7.4319 & 7.4336 & 7.4353 & 7.4370 & 7.4388 & 7.4405 & 7.4422 & 7.4439 & 7.4456 \\
 \textbf{.872}\markboth{0.872}{0.872} & 7.4473 & 7.4490 & 7.4508 & 7.4525 & 7.4542 & 7.4559 & 7.4576 & 7.4593 & 7.4611 & 7.4628 \\
\rowcolor{bg} \textbf{.873}\markboth{0.873}{0.873} & 7.4645 & 7.4662 & 7.4679 & 7.4696 & 7.4714 & 7.4731 & 7.4748 & 7.4765 & 7.4783 & 7.4800 \\
 \textbf{.874}\markboth{0.874}{0.874} & 7.4817 & 7.4834 & 7.4851 & 7.4869 & 7.4886 & 7.4903 & 7.4920 & 7.4938 & 7.4955 & 7.4972 \\
\rowcolor{bg} \textbf{.875}\markboth{0.875}{0.875} & 7.4989 & 7.5007 & 7.5024 & 7.5041 & 7.5059 & 7.5076 & 7.5093 & 7.5110 & 7.5128 & 7.5145 \\
 \textbf{.876}\markboth{0.876}{0.876} & 7.5162 & 7.5180 & 7.5197 & 7.5214 & 7.5232 & 7.5249 & 7.5266 & 7.5284 & 7.5301 & 7.5318 \\
\rowcolor{bg} \textbf{.877}\markboth{0.877}{0.877} & 7.5336 & 7.5353 & 7.5370 & 7.5388 & 7.5405 & 7.5422 & 7.5440 & 7.5457 & 7.5474 & 7.5492 \\
 \textbf{.878}\markboth{0.878}{0.878} & 7.5509 & 7.5527 & 7.5544 & 7.5561 & 7.5579 & 7.5596 & 7.5614 & 7.5631 & 7.5648 & 7.5666 \\
\rowcolor{bg} \textbf{.879}\markboth{0.879}{0.879} & 7.5683 & 7.5701 & 7.5718 & 7.5736 & 7.5753 & 7.5770 & 7.5788 & 7.5805 & 7.5823 & 7.5840 \\
 \textbf{.880}\markboth{0.880}{0.880} & 7.5858 & 7.5875 & 7.5893 & 7.5910 & 7.5928 & 7.5945 & 7.5963 & 7.5980 & 7.5998 & 7.6015 \\
\rowcolor{bg} \textbf{.881}\markboth{0.881}{0.881} & 7.6033 & 7.6050 & 7.6068 & 7.6085 & 7.6103 & 7.6120 & 7.6138 & 7.6155 & 7.6173 & 7.6190 \\
 \textbf{.882}\markboth{0.882}{0.882} & 7.6208 & 7.6225 & 7.6243 & 7.6261 & 7.6278 & 7.6296 & 7.6313 & 7.6331 & 7.6348 & 7.6366 \\
\rowcolor{bg} \textbf{.883}\markboth{0.883}{0.883} & 7.6384 & 7.6401 & 7.6419 & 7.6436 & 7.6454 & 7.6472 & 7.6489 & 7.6507 & 7.6524 & 7.6542 \\
 \textbf{.884}\markboth{0.884}{0.884} & 7.6560 & 7.6577 & 7.6595 & 7.6613 & 7.6630 & 7.6648 & 7.6666 & 7.6683 & 7.6701 & 7.6718 \\
\rowcolor{bg} \textbf{.885}\markboth{0.885}{0.885} & 7.6736 & 7.6754 & 7.6771 & 7.6789 & 7.6807 & 7.6825 & 7.6842 & 7.6860 & 7.6878 & 7.6895 \\
 \textbf{.886}\markboth{0.886}{0.886} & 7.6913 & 7.6931 & 7.6948 & 7.6966 & 7.6984 & 7.7002 & 7.7019 & 7.7037 & 7.7055 & 7.7073 \\
\rowcolor{bg} \textbf{.887}\markboth{0.887}{0.887} & 7.7090 & 7.7108 & 7.7126 & 7.7144 & 7.7161 & 7.7179 & 7.7197 & 7.7215 & 7.7232 & 7.7250 \\
 \textbf{.888}\markboth{0.888}{0.888} & 7.7268 & 7.7286 & 7.7304 & 7.7321 & 7.7339 & 7.7357 & 7.7375 & 7.7393 & 7.7411 & 7.7428 \\
\rowcolor{bg} \textbf{.889}\markboth{0.889}{0.889} & 7.7446 & 7.7464 & 7.7482 & 7.7500 & 7.7518 & 7.7535 & 7.7553 & 7.7571 & 7.7589 & 7.7607 \\
 \textbf{.890}\markboth{0.890}{0.890} & 7.7625 & 7.7643 & 7.7660 & 7.7678 & 7.7696 & 7.7714 & 7.7732 & 7.7750 & 7.7768 & 7.7786 \\
\rowcolor{bg} \textbf{.891}\markboth{0.891}{0.891} & 7.7804 & 7.7822 & 7.7839 & 7.7857 & 7.7875 & 7.7893 & 7.7911 & 7.7929 & 7.7947 & 7.7965 \\
 \textbf{.892}\markboth{0.892}{0.892} & 7.7983 & 7.8001 & 7.8019 & 7.8037 & 7.8055 & 7.8073 & 7.8091 & 7.8109 & 7.8127 & 7.8145 \\
\rowcolor{bg} \textbf{.893}\markboth{0.893}{0.893} & 7.8163 & 7.8181 & 7.8199 & 7.8217 & 7.8235 & 7.8253 & 7.8271 & 7.8289 & 7.8307 & 7.8325 \\
 \textbf{.894}\markboth{0.894}{0.894} & 7.8343 & 7.8361 & 7.8379 & 7.8397 & 7.8415 & 7.8433 & 7.8451 & 7.8469 & 7.8487 & 7.8505 \\
\rowcolor{bg} \textbf{.895}\markboth{0.895}{0.895} & 7.8524 & 7.8542 & 7.8560 & 7.8578 & 7.8596 & 7.8614 & 7.8632 & 7.8650 & 7.8668 & 7.8686 \\
 \textbf{.896}\markboth{0.896}{0.896} & 7.8705 & 7.8723 & 7.8741 & 7.8759 & 7.8777 & 7.8795 & 7.8813 & 7.8832 & 7.8850 & 7.8868 \\
\rowcolor{bg} \textbf{.897}\markboth{0.897}{0.897} & 7.8886 & 7.8904 & 7.8922 & 7.8941 & 7.8959 & 7.8977 & 7.8995 & 7.9013 & 7.9031 & 7.9050 \\
 \textbf{.898}\markboth{0.898}{0.898} & 7.9068 & 7.9086 & 7.9104 & 7.9122 & 7.9141 & 7.9159 & 7.9177 & 7.9195 & 7.9214 & 7.9232 \\
\rowcolor{bg} \textbf{.899}\markboth{0.899}{0.899} & 7.9250 & 7.9268 & 7.9287 & 7.9305 & 7.9323 & 7.9341 & 7.9360 & 7.9378 & 7.9396 & 7.9415 \\
 \textcolor{blue}{\textbf{.900}}\markboth{0.900}{0.900} & 7.9433 & 7.9451 & 7.9469 & 7.9488 & 7.9506 & 7.9524 & 7.9543 & 7.9561 & 7.9579 & 7.9598 \\
\rowcolor{bg} \textbf{.901}\markboth{0.901}{0.901} & 7.9616 & 7.9634 & 7.9653 & 7.9671 & 7.9689 & 7.9708 & 7.9726 & 7.9744 & 7.9763 & 7.9781 \\
 \textbf{.902}\markboth{0.902}{0.902} & 7.9799 & 7.9818 & 7.9836 & 7.9855 & 7.9873 & 7.9891 & 7.9910 & 7.9928 & 7.9947 & 7.9965 \\
\rowcolor{bg} \textbf{.903}\markboth{0.903}{0.903} & 7.9983 & 8.0002 & 8.0020 & 8.0039 & 8.0057 & 8.0076 & 8.0094 & 8.0112 & 8.0131 & 8.0149 \\
 \textbf{.904}\markboth{0.904}{0.904} & 8.0168 & 8.0186 & 8.0205 & 8.0223 & 8.0242 & 8.0260 & 8.0279 & 8.0297 & 8.0316 & 8.0334 \\
\rowcolor{bg} \textbf{.905}\markboth{0.905}{0.905} & 8.0353 & 8.0371 & 8.0390 & 8.0408 & 8.0427 & 8.0445 & 8.0464 & 8.0482 & 8.0501 & 8.0519 \\
 \textbf{.906}\markboth{0.906}{0.906} & 8.0538 & 8.0556 & 8.0575 & 8.0593 & 8.0612 & 8.0631 & 8.0649 & 8.0668 & 8.0686 & 8.0705 \\
\rowcolor{bg} \textbf{.907}\markboth{0.907}{0.907} & 8.0724 & 8.0742 & 8.0761 & 8.0779 & 8.0798 & 8.0816 & 8.0835 & 8.0854 & 8.0872 & 8.0891 \\
 \textbf{.908}\markboth{0.908}{0.908} & 8.0910 & 8.0928 & 8.0947 & 8.0965 & 8.0984 & 8.1003 & 8.1021 & 8.1040 & 8.1059 & 8.1077 \\
\rowcolor{bg} \textbf{.909}\markboth{0.909}{0.909} & 8.1096 & 8.1115 & 8.1133 & 8.1152 & 8.1171 & 8.1190 & 8.1208 & 8.1227 & 8.1246 & 8.1264 \\
 \textbf{.910}\markboth{0.910}{0.910} & 8.1283 & 8.1302 & 8.1320 & 8.1339 & 8.1358 & 8.1377 & 8.1395 & 8.1414 & 8.1433 & 8.1452 \\
\rowcolor{bg} \textbf{.911}\markboth{0.911}{0.911} & 8.1470 & 8.1489 & 8.1508 & 8.1527 & 8.1546 & 8.1564 & 8.1583 & 8.1602 & 8.1621 & 8.1639 \\
 \textbf{.912}\markboth{0.912}{0.912} & 8.1658 & 8.1677 & 8.1696 & 8.1715 & 8.1733 & 8.1752 & 8.1771 & 8.1790 & 8.1809 & 8.1828 \\
\rowcolor{bg} \textbf{.913}\markboth{0.913}{0.913} & 8.1846 & 8.1865 & 8.1884 & 8.1903 & 8.1922 & 8.1941 & 8.1960 & 8.1979 & 8.1997 & 8.2016 \\
 \textbf{.914}\markboth{0.914}{0.914} & 8.2035 & 8.2054 & 8.2073 & 8.2092 & 8.2111 & 8.2130 & 8.2149 & 8.2167 & 8.2186 & 8.2205 \\
\rowcolor{bg} \textbf{.915}\markboth{0.915}{0.915} & 8.2224 & 8.2243 & 8.2262 & 8.2281 & 8.2300 & 8.2319 & 8.2338 & 8.2357 & 8.2376 & 8.2395 \\
 \textbf{.916}\markboth{0.916}{0.916} & 8.2414 & 8.2433 & 8.2452 & 8.2471 & 8.2490 & 8.2509 & 8.2528 & 8.2547 & 8.2566 & 8.2585 \\
\rowcolor{bg} \textbf{.917}\markboth{0.917}{0.917} & 8.2604 & 8.2623 & 8.2642 & 8.2661 & 8.2680 & 8.2699 & 8.2718 & 8.2737 & 8.2756 & 8.2775 \\
 \textbf{.918}\markboth{0.918}{0.918} & 8.2794 & 8.2813 & 8.2832 & 8.2851 & 8.2871 & 8.2890 & 8.2909 & 8.2928 & 8.2947 & 8.2966 \\
\rowcolor{bg} \textbf{.919}\markboth{0.919}{0.919} & 8.2985 & 8.3004 & 8.3023 & 8.3042 & 8.3062 & 8.3081 & 8.3100 & 8.3119 & 8.3138 & 8.3157 \\
 \textbf{.920}\markboth{0.920}{0.920} & 8.3176 & 8.3196 & 8.3215 & 8.3234 & 8.3253 & 8.3272 & 8.3291 & 8.3311 & 8.3330 & 8.3349 \\
\rowcolor{bg} \textbf{.921}\markboth{0.921}{0.921} & 8.3368 & 8.3387 & 8.3407 & 8.3426 & 8.3445 & 8.3464 & 8.3483 & 8.3503 & 8.3522 & 8.3541 \\
 \textbf{.922}\markboth{0.922}{0.922} & 8.3560 & 8.3580 & 8.3599 & 8.3618 & 8.3637 & 8.3657 & 8.3676 & 8.3695 & 8.3714 & 8.3734 \\
\rowcolor{bg} \textbf{.923}\markboth{0.923}{0.923} & 8.3753 & 8.3772 & 8.3792 & 8.3811 & 8.3830 & 8.3849 & 8.3869 & 8.3888 & 8.3907 & 8.3927 \\
 \textbf{.924}\markboth{0.924}{0.924} & 8.3946 & 8.3965 & 8.3985 & 8.4004 & 8.4023 & 8.4043 & 8.4062 & 8.4081 & 8.4101 & 8.4120 \\
\rowcolor{bg} \textbf{.925}\markboth{0.925}{0.925} & 8.4140 & 8.4159 & 8.4178 & 8.4198 & 8.4217 & 8.4236 & 8.4256 & 8.4275 & 8.4295 & 8.4314 \\
 \textbf{.926}\markboth{0.926}{0.926} & 8.4333 & 8.4353 & 8.4372 & 8.4392 & 8.4411 & 8.4431 & 8.4450 & 8.4470 & 8.4489 & 8.4508 \\
\rowcolor{bg} \textbf{.927}\markboth{0.927}{0.927} & 8.4528 & 8.4547 & 8.4567 & 8.4586 & 8.4606 & 8.4625 & 8.4645 & 8.4664 & 8.4684 & 8.4703 \\
 \textbf{.928}\markboth{0.928}{0.928} & 8.4723 & 8.4742 & 8.4762 & 8.4781 & 8.4801 & 8.4820 & 8.4840 & 8.4859 & 8.4879 & 8.4898 \\
\rowcolor{bg} \textbf{.929}\markboth{0.929}{0.929} & 8.4918 & 8.4938 & 8.4957 & 8.4977 & 8.4996 & 8.5016 & 8.5035 & 8.5055 & 8.5075 & 8.5094 \\
 \textbf{.930}\markboth{0.930}{0.930} & 8.5114 & 8.5133 & 8.5153 & 8.5173 & 8.5192 & 8.5212 & 8.5231 & 8.5251 & 8.5271 & 8.5290 \\
\rowcolor{bg} \textbf{.931}\markboth{0.931}{0.931} & 8.5310 & 8.5330 & 8.5349 & 8.5369 & 8.5389 & 8.5408 & 8.5428 & 8.5448 & 8.5467 & 8.5487 \\
 \textbf{.932}\markboth{0.932}{0.932} & 8.5507 & 8.5526 & 8.5546 & 8.5566 & 8.5585 & 8.5605 & 8.5625 & 8.5645 & 8.5664 & 8.5684 \\
\rowcolor{bg} \textbf{.933}\markboth{0.933}{0.933} & 8.5704 & 8.5724 & 8.5743 & 8.5763 & 8.5783 & 8.5803 & 8.5822 & 8.5842 & 8.5862 & 8.5882 \\
 \textbf{.934}\markboth{0.934}{0.934} & 8.5901 & 8.5921 & 8.5941 & 8.5961 & 8.5981 & 8.6000 & 8.6020 & 8.6040 & 8.6060 & 8.6080 \\
\rowcolor{bg} \textbf{.935}\markboth{0.935}{0.935} & 8.6099 & 8.6119 & 8.6139 & 8.6159 & 8.6179 & 8.6199 & 8.6218 & 8.6238 & 8.6258 & 8.6278 \\
 \textbf{.936}\markboth{0.936}{0.936} & 8.6298 & 8.6318 & 8.6338 & 8.6357 & 8.6377 & 8.6397 & 8.6417 & 8.6437 & 8.6457 & 8.6477 \\
\rowcolor{bg} \textbf{.937}\markboth{0.937}{0.937} & 8.6497 & 8.6517 & 8.6537 & 8.6557 & 8.6576 & 8.6596 & 8.6616 & 8.6636 & 8.6656 & 8.6676 \\
 \textbf{.938}\markboth{0.938}{0.938} & 8.6696 & 8.6716 & 8.6736 & 8.6756 & 8.6776 & 8.6796 & 8.6816 & 8.6836 & 8.6856 & 8.6876 \\
\rowcolor{bg} \textbf{.939}\markboth{0.939}{0.939} & 8.6896 & 8.6916 & 8.6936 & 8.6956 & 8.6976 & 8.6996 & 8.7016 & 8.7036 & 8.7056 & 8.7076 \\
 \textbf{.940}\markboth{0.940}{0.940} & 8.7096 & 8.7116 & 8.7136 & 8.7157 & 8.7177 & 8.7197 & 8.7217 & 8.7237 & 8.7257 & 8.7277 \\
\rowcolor{bg} \textbf{.941}\markboth{0.941}{0.941} & 8.7297 & 8.7317 & 8.7337 & 8.7357 & 8.7378 & 8.7398 & 8.7418 & 8.7438 & 8.7458 & 8.7478 \\
 \textbf{.942}\markboth{0.942}{0.942} & 8.7498 & 8.7519 & 8.7539 & 8.7559 & 8.7579 & 8.7599 & 8.7619 & 8.7640 & 8.7660 & 8.7680 \\
\rowcolor{bg} \textbf{.943}\markboth{0.943}{0.943} & 8.7700 & 8.7720 & 8.7740 & 8.7761 & 8.7781 & 8.7801 & 8.7821 & 8.7842 & 8.7862 & 8.7882 \\
 \textbf{.944}\markboth{0.944}{0.944} & 8.7902 & 8.7922 & 8.7943 & 8.7963 & 8.7983 & 8.8004 & 8.8024 & 8.8044 & 8.8064 & 8.8085 \\
\rowcolor{bg} \textbf{.945}\markboth{0.945}{0.945} & 8.8105 & 8.8125 & 8.8145 & 8.8166 & 8.8186 & 8.8206 & 8.8227 & 8.8247 & 8.8267 & 8.8288 \\
 \textbf{.946}\markboth{0.946}{0.946} & 8.8308 & 8.8328 & 8.8349 & 8.8369 & 8.8389 & 8.8410 & 8.8430 & 8.8450 & 8.8471 & 8.8491 \\
\rowcolor{bg} \textbf{.947}\markboth{0.947}{0.947} & 8.8512 & 8.8532 & 8.8552 & 8.8573 & 8.8593 & 8.8614 & 8.8634 & 8.8654 & 8.8675 & 8.8695 \\
 \textbf{.948}\markboth{0.948}{0.948} & 8.8716 & 8.8736 & 8.8756 & 8.8777 & 8.8797 & 8.8818 & 8.8838 & 8.8859 & 8.8879 & 8.8900 \\
\rowcolor{bg} \textbf{.949}\markboth{0.949}{0.949} & 8.8920 & 8.8941 & 8.8961 & 8.8982 & 8.9002 & 8.9023 & 8.9043 & 8.9064 & 8.9084 & 8.9105 \\
 \textbf{.950}\markboth{0.950}{0.950} & 8.9125 & 8.9146 & 8.9166 & 8.9187 & 8.9207 & 8.9228 & 8.9248 & 8.9269 & 8.9289 & 8.9310 \\
\rowcolor{bg} \textbf{.951}\markboth{0.951}{0.951} & 8.9331 & 8.9351 & 8.9372 & 8.9392 & 8.9413 & 8.9433 & 8.9454 & 8.9475 & 8.9495 & 8.9516 \\
 \textbf{.952}\markboth{0.952}{0.952} & 8.9536 & 8.9557 & 8.9578 & 8.9598 & 8.9619 & 8.9640 & 8.9660 & 8.9681 & 8.9702 & 8.9722 \\
\rowcolor{bg} \textbf{.953}\markboth{0.953}{0.953} & 8.9743 & 8.9764 & 8.9784 & 8.9805 & 8.9826 & 8.9846 & 8.9867 & 8.9888 & 8.9908 & 8.9929 \\
 \textbf{.954}\markboth{0.954}{0.954} & 8.9950 & 8.9970 & 8.9991 & 9.0012 & 9.0033 & 9.0053 & 9.0074 & 9.0095 & 9.0116 & 9.0136 \\
\rowcolor{bg} \textbf{.955}\markboth{0.955}{0.955} & 9.0157 & 9.0178 & 9.0199 & 9.0219 & 9.0240 & 9.0261 & 9.0282 & 9.0303 & 9.0323 & 9.0344 \\
 \textbf{.956}\markboth{0.956}{0.956} & 9.0365 & 9.0386 & 9.0407 & 9.0427 & 9.0448 & 9.0469 & 9.0490 & 9.0511 & 9.0532 & 9.0552 \\
\rowcolor{bg} \textbf{.957}\markboth{0.957}{0.957} & 9.0573 & 9.0594 & 9.0615 & 9.0636 & 9.0657 & 9.0678 & 9.0698 & 9.0719 & 9.0740 & 9.0761 \\
 \textbf{.958}\markboth{0.958}{0.958} & 9.0782 & 9.0803 & 9.0824 & 9.0845 & 9.0866 & 9.0887 & 9.0908 & 9.0928 & 9.0949 & 9.0970 \\
\rowcolor{bg} \textbf{.959}\markboth{0.959}{0.959} & 9.0991 & 9.1012 & 9.1033 & 9.1054 & 9.1075 & 9.1096 & 9.1117 & 9.1138 & 9.1159 & 9.1180 \\
 \textbf{.960}\markboth{0.960}{0.960} & 9.1201 & 9.1222 & 9.1243 & 9.1264 & 9.1285 & 9.1306 & 9.1327 & 9.1348 & 9.1369 & 9.1390 \\
\rowcolor{bg} \textbf{.961}\markboth{0.961}{0.961} & 9.1411 & 9.1432 & 9.1453 & 9.1474 & 9.1496 & 9.1517 & 9.1538 & 9.1559 & 9.1580 & 9.1601 \\
 \textbf{.962}\markboth{0.962}{0.962} & 9.1622 & 9.1643 & 9.1664 & 9.1685 & 9.1706 & 9.1728 & 9.1749 & 9.1770 & 9.1791 & 9.1812 \\
\rowcolor{bg} \textbf{.963}\markboth{0.963}{0.963} & 9.1833 & 9.1854 & 9.1876 & 9.1897 & 9.1918 & 9.1939 & 9.1960 & 9.1981 & 9.2003 & 9.2024 \\
 \textbf{.964}\markboth{0.964}{0.964} & 9.2045 & 9.2066 & 9.2087 & 9.2109 & 9.2130 & 9.2151 & 9.2172 & 9.2193 & 9.2215 & 9.2236 \\
\rowcolor{bg} \textbf{.965}\markboth{0.965}{0.965} & 9.2257 & 9.2278 & 9.2300 & 9.2321 & 9.2342 & 9.2363 & 9.2385 & 9.2406 & 9.2427 & 9.2449 \\
 \textbf{.966}\markboth{0.966}{0.966} & 9.2470 & 9.2491 & 9.2512 & 9.2534 & 9.2555 & 9.2576 & 9.2598 & 9.2619 & 9.2640 & 9.2662 \\
\rowcolor{bg} \textbf{.967}\markboth{0.967}{0.967} & 9.2683 & 9.2704 & 9.2726 & 9.2747 & 9.2768 & 9.2790 & 9.2811 & 9.2832 & 9.2854 & 9.2875 \\
 \textbf{.968}\markboth{0.968}{0.968} & 9.2897 & 9.2918 & 9.2939 & 9.2961 & 9.2982 & 9.3004 & 9.3025 & 9.3046 & 9.3068 & 9.3089 \\
\rowcolor{bg} \textbf{.969}\markboth{0.969}{0.969} & 9.3111 & 9.3132 & 9.3154 & 9.3175 & 9.3197 & 9.3218 & 9.3240 & 9.3261 & 9.3282 & 9.3304 \\
 \textbf{.970}\markboth{0.970}{0.970} & 9.3325 & 9.3347 & 9.3368 & 9.3390 & 9.3411 & 9.3433 & 9.3454 & 9.3476 & 9.3498 & 9.3519 \\
\rowcolor{bg} \textbf{.971}\markboth{0.971}{0.971} & 9.3541 & 9.3562 & 9.3584 & 9.3605 & 9.3627 & 9.3648 & 9.3670 & 9.3691 & 9.3713 & 9.3735 \\
 \textbf{.972}\markboth{0.972}{0.972} & 9.3756 & 9.3778 & 9.3799 & 9.3821 & 9.3843 & 9.3864 & 9.3886 & 9.3907 & 9.3929 & 9.3951 \\
\rowcolor{bg} \textbf{.973}\markboth{0.973}{0.973} & 9.3972 & 9.3994 & 9.4016 & 9.4037 & 9.4059 & 9.4081 & 9.4102 & 9.4124 & 9.4146 & 9.4167 \\
 \textbf{.974}\markboth{0.974}{0.974} & 9.4189 & 9.4211 & 9.4232 & 9.4254 & 9.4276 & 9.4297 & 9.4319 & 9.4341 & 9.4363 & 9.4384 \\
\rowcolor{bg} \textbf{.975}\markboth{0.975}{0.975} & 9.4406 & 9.4428 & 9.4450 & 9.4471 & 9.4493 & 9.4515 & 9.4537 & 9.4558 & 9.4580 & 9.4602 \\
 \textbf{.976}\markboth{0.976}{0.976} & 9.4624 & 9.4646 & 9.4667 & 9.4689 & 9.4711 & 9.4733 & 9.4755 & 9.4776 & 9.4798 & 9.4820 \\
\rowcolor{bg} \textbf{.977}\markboth{0.977}{0.977} & 9.4842 & 9.4864 & 9.4886 & 9.4907 & 9.4929 & 9.4951 & 9.4973 & 9.4995 & 9.5017 & 9.5039 \\
 \textbf{.978}\markboth{0.978}{0.978} & 9.5060 & 9.5082 & 9.5104 & 9.5126 & 9.5148 & 9.5170 & 9.5192 & 9.5214 & 9.5236 & 9.5258 \\
\rowcolor{bg} \textbf{.979}\markboth{0.979}{0.979} & 9.5280 & 9.5302 & 9.5324 & 9.5345 & 9.5367 & 9.5389 & 9.5411 & 9.5433 & 9.5455 & 9.5477 \\
 \textbf{.980}\markboth{0.980}{0.980} & 9.5499 & 9.5521 & 9.5543 & 9.5565 & 9.5587 & 9.5609 & 9.5631 & 9.5653 & 9.5675 & 9.5697 \\
\rowcolor{bg} \textbf{.981}\markboth{0.981}{0.981} & 9.5719 & 9.5741 & 9.5763 & 9.5786 & 9.5808 & 9.5830 & 9.5852 & 9.5874 & 9.5896 & 9.5918 \\
 \textbf{.982}\markboth{0.982}{0.982} & 9.5940 & 9.5962 & 9.5984 & 9.6006 & 9.6028 & 9.6051 & 9.6073 & 9.6095 & 9.6117 & 9.6139 \\
\rowcolor{bg} \textbf{.983}\markboth{0.983}{0.983} & 9.6161 & 9.6183 & 9.6206 & 9.6228 & 9.6250 & 9.6272 & 9.6294 & 9.6316 & 9.6339 & 9.6361 \\
 \textbf{.984}\markboth{0.984}{0.984} & 9.6383 & 9.6405 & 9.6427 & 9.6450 & 9.6472 & 9.6494 & 9.6516 & 9.6538 & 9.6561 & 9.6583 \\
\rowcolor{bg} \textbf{.985}\markboth{0.985}{0.985} & 9.6605 & 9.6627 & 9.6650 & 9.6672 & 9.6694 & 9.6716 & 9.6739 & 9.6761 & 9.6783 & 9.6805 \\
 \textbf{.986}\markboth{0.986}{0.986} & 9.6828 & 9.6850 & 9.6872 & 9.6895 & 9.6917 & 9.6939 & 9.6962 & 9.6984 & 9.7006 & 9.7029 \\
\rowcolor{bg} \textbf{.987}\markboth{0.987}{0.987} & 9.7051 & 9.7073 & 9.7096 & 9.7118 & 9.7140 & 9.7163 & 9.7185 & 9.7208 & 9.7230 & 9.7252 \\
 \textbf{.988}\markboth{0.988}{0.988} & 9.7275 & 9.7297 & 9.7320 & 9.7342 & 9.7364 & 9.7387 & 9.7409 & 9.7432 & 9.7454 & 9.7477 \\
\rowcolor{bg} \textbf{.989}\markboth{0.989}{0.989} & 9.7499 & 9.7521 & 9.7544 & 9.7566 & 9.7589 & 9.7611 & 9.7634 & 9.7656 & 9.7679 & 9.7701 \\
 \textbf{.990}\markboth{0.990}{0.990} & 9.7724 & 9.7746 & 9.7769 & 9.7791 & 9.7814 & 9.7836 & 9.7859 & 9.7881 & 9.7904 & 9.7926 \\
\rowcolor{bg} \textbf{.991}\markboth{0.991}{0.991} & 9.7949 & 9.7972 & 9.7994 & 9.8017 & 9.8039 & 9.8062 & 9.8084 & 9.8107 & 9.8130 & 9.8152 \\
 \textbf{.992}\markboth{0.992}{0.992} & 9.8175 & 9.8197 & 9.8220 & 9.8243 & 9.8265 & 9.8288 & 9.8311 & 9.8333 & 9.8356 & 9.8378 \\
\rowcolor{bg} \textbf{.993}\markboth{0.993}{0.993} & 9.8401 & 9.8424 & 9.8446 & 9.8469 & 9.8492 & 9.8514 & 9.8537 & 9.8560 & 9.8583 & 9.8605 \\
 \textbf{.994}\markboth{0.994}{0.994} & 9.8628 & 9.8651 & 9.8673 & 9.8696 & 9.8719 & 9.8742 & 9.8764 & 9.8787 & 9.8810 & 9.8833 \\
\rowcolor{bg} \textbf{.995}\markboth{0.995}{0.995} & 9.8855 & 9.8878 & 9.8901 & 9.8924 & 9.8946 & 9.8969 & 9.8992 & 9.9015 & 9.9038 & 9.9060 \\
 \textbf{.996}\markboth{0.996}{0.996} & 9.9083 & 9.9106 & 9.9129 & 9.9152 & 9.9174 & 9.9197 & 9.9220 & 9.9243 & 9.9266 & 9.9289 \\
\rowcolor{bg} \textbf{.997}\markboth{0.997}{0.997} & 9.9312 & 9.9334 & 9.9357 & 9.9380 & 9.9403 & 9.9426 & 9.9449 & 9.9472 & 9.9495 & 9.9518 \\
 \textbf{.998}\markboth{0.998}{0.998} & 9.9541 & 9.9563 & 9.9586 & 9.9609 & 9.9632 & 9.9655 & 9.9678 & 9.9701 & 9.9724 & 9.9747 \\
\rowcolor{bg} \textbf{.999}\markboth{0.999}{0.999} & 9.9770 & 9.9793 & 9.9816 & 9.9839 & 9.9862 & 9.9885 & 9.9908 & 9.9931 & 9.9954 & 9.9977 \\

\end{longtable}

\newpage
\markboth{}{}
\makeatletter
\def\@oddhead{\thepage \hfill \rightmark{} \leftmark}
\def\@evenhead{\rightmark{} \leftmark\hfill \thepage}
\makeatother

\section{How to use this book}

Tables of logarithms (``log tables'') are a quick way of simplifying
multiplication and division tasks when no calculator is available. This is due to the simple identities:

\begin{itemize}
\item $\log{(a \times b)} = \log{a} + \log{b}$
\item $\log{(a \div b)} = \log{a} - \log{b}$
\item $\log{a^b} = \log{(a)} \times b$
\item $\log{\sqrt[b]{a}} = \log{(a)} \div b$ 
\end{itemize}

To keep this book a reasonable size, logarithms are given for numbers in the range 1.000 -- 9.999. In order to work with larger or smaller numbers, it is necessary to convert them to scientific notation, $a \times 10^b$, where $1.000 \leq a < 9.999$. For example, 21.8 becomes  $2.18 \times 10^1$ and 0.00218 becomes $2.18 \times 10^{-3}$. It's important to keep track of the powers of ten, as in multiplication and division they control the position of the decimal point, and in roots they become involved in the table lookup process.

To look up a logarithm, the number is found in the left hand index row to two decimal places, and then the correct column is selected for the third decimal place. For example, to find the logarithm of 3.657:

\begin{longtable}[c]{|*{11}{c|}}
 & \textbf{0} & \textbf{1} & \textbf{2} & \textbf{3} & \textbf{4} & \textbf{5} & \textbf{6} & \textcolor{red}{\textbf{7}} & \textbf{8} & \textbf{9} \\
\hline \endhead
\hline \endfoot
%%%
 \textbf{3.64} & \multicolumn{10}{c|}{...} \\
\rowcolor{bg} \textcolor{red}{\textbf{3.65}} & .56229 & .56241 & .56253 & .56265 & .56277 & .56289 & .56301 & \textcolor{red}{\textit{.56312}} & .56324 & .56336 \\
 \textbf{3.66} & \multicolumn{10}{c|}{...} \\
\end{longtable}

Thus $\log{3.657} = 0.56312$. This is more correctly written $\log_{10}{3.657} = 0.56312$, but the decimal base will be omitted throughout.

To look up an anti-logarithm, a similar process is used. Antilogs are given to fewer decimal places that logarithms, again to keep the size of this book reasonable. To continue with our example, we will look up  $\log^{-1}{0.56312}$:


\begin{longtable}[c]{|*{11}{c|}}
 & \textbf{0} & \textcolor{red}{\textbf{1}} & \textbf{2} & \textbf{3} & \textbf{4} & \textbf{5} & \textbf{6} & \textbf{7} & \textbf{8} & \textbf{9} \\
\hline \endhead
\hline \endfoot
%%%
 \textbf{.562} & \multicolumn{10}{c|}{...} \\
\rowcolor{bg} \textcolor{red}{\textbf{.563}} & 3.6559 & \textcolor{red}{\textit{3.6568}} & 3.6576 & 3.6585 & 3.6593 & 3.6602 & 3.6610 & 3.6618 & 3.6627 & 3.6635 \\
 \textbf{.564} & \multicolumn{10}{c|}{...} \\
\end{longtable}

Note that we've had to round the lookup to 0.5631. It should surprise no-one that $\log^{-1}{0.5631} = 3.6568$, or our original example 3.657 when expressed to the same number of decimal places.

\subsection{Multiplication}

Example: $21.8 \times 36.57 = $?
\begin{enumerate}
\item Convert both numbers into scientific form: $2.18 \times 10^1 \times 3.657 \times 10^1$
\item Look up $\log{2.18}$. It's in the first column next to 2.18: \textbf{0.33846}
\item Look up $\log{3.657}$. It's in the eighth column next to 3.65: \textbf{0.56312}
\item Add the results: $0.33846 + 0.56312 = 0.90158$
\item Look up the anti-logarithm of 0.9016: \textbf{7.9726}
\item Multiply the result by the powers of ten left over: $7.9726 \times 10^{1+1} = 797.26$
\item Just by looking up logarithms and adding, $21.8 \times 36.57 \approx \textbf{797.26}$. Compare this with the calculated result, \textbf{797.226}.
\end{enumerate}

\subsection{Division}

Example: $23.27 \div 14.07 =$?
\begin{enumerate}
\item Convert both numbers into scientific form: $(2.327 \times 10^1) \div (1.407 \times 10^1)$
\item Look up $\log{2.327}$. It's in the eighth column next to 2.32: \textbf{0.36680}
\item Look up $\log{1.407}$. It's in the eighth column next to 1.40: \textbf{0.14829}
\item Subtract the results: $0.36680 - 0.14829 = 0.21851$
\item Look up the anti-logarithm of 0.2185: \textbf{1.6539}
\item Multiply the result by the powers of ten left over: $1.6539 \times 10^{1-1} = 1.6539$. (Since $n^0=1$.)
\item Just by looking up logarithms and subtracting, $23.27 \div 14.07 \approx \textbf{1.6539}$. Compare this with the calculated result, \textbf{1.654} to 4 significant figures.
\end{enumerate}


\subsection{Exponentiation}

Example: $8.232^3 = $?
\begin{enumerate}
\item Look up $\log{8.232}$. It's in the third column next to 8.23: \textbf{0.91551}
\item Multiply $0.91551 \times 3 = 2.74653$.
\item Split the result into integer (2) and fractional parts, and look up the anti-logarithm of the fraction, 0.7465: \textbf{5.5783}
\item Multiply the result by the integer part as powers of ten: $5.5783 \times 10^2 = 557.83$.
\item Just by looking up logarithms and multiplying, $8.232^3 \approx \textbf{557.83}$. Compare this with the calculated result, \textbf{557.8} to 4 significant figures.
\end{enumerate}

\subsection{Roots}

Example: $\sqrt[3]{28.317}= $?
\begin{enumerate}
\item Convert the number to scientific form: $2.8317 \times 10^1$, so the calculation becomes 
$\sqrt[3]{2.8317} \times \sqrt[3]{10}$
\item Look up $\log{2.832}$: \textbf{0.45209}
\item Divide $0.45209 \div 3 = 0.1507$.
\item We still have to deal with the $\sqrt[3]{10}$ term: $\log{10} = 1$, so $1 \div 3 = 0.3333$
\item Add the two results: $0.1507 + 0.3333 = 0.4840$
\item Look up the anti-logarithm of 0.4840: \textbf{3.0479}
\item So $\sqrt[3]{28.317} \approx 3.0479$. Compare this with the calculated result, \textbf{3.0480} to 5 significant figures.
\end{enumerate}
(If it is one's misfortune to have to work with ``imperial''
measurements, it will be quickly recognized that there are 28.317
litres [cubic decimetres] in a cubic foot, and 3.048 decimetres to the
foot.) 

\vfill

\end{document}  